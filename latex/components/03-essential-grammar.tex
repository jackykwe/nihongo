\documentclass[../nihongo-gakushuu-kyouzai.tex]{subfiles}
\graphicspath{ {../imgs/} }
\begin{document}
\onehalfspacing  % for 振り仮名

\setcounter{section}{2}
\section{Essential grammar} \label{sec:essential-grammar}

Now that we have an understanding of the basics, fundamentals, and foundations of Japanese grammar, this section provides specific grammar for practical situations.

\subsection{Verb stems}

Also known as the ます-stem in other texts (we refrain from that term as the stem is used in many more conjugations), stems can be obtained from verbs via the rules in Table~\ref{tbl:teineigo-forms}.

Verb stems are sometimes (not always!) the noun forms of the verbs, e.g.\ \ruby{休}{やす}み is the noun form of 休む.

We can suffix the stem with target particle に or directional particle へ, then follow with a motion verb (almost always 行く or くる). This <stem>\{に, へ\}\{行く, くる, \dots\} construct means ``to go/come do <stem>''. Use に to mean going/coming for the purpose of doing <stem>; use へ to put emphasis on the literal act of going/coming.

E.g.\ 明日、映画を\textbf{見に行く}。(Tomorrow, go to see movie.)

E.g.\ 昨日、友達が\textbf{遊びへきた}。(Yesterday, friend came towards a playing activity.; sounds weird)

E.g.\ 昨日、友達が\textbf{遊びにきた}。 (Yesterday, friend came to play.)

The expression 「楽しみにする」 (looking forward to) is formed from a similar grammar rule (though it's a special case and should be considered a set expression).

Compound verbs can be formed by appending verbs to stems. However, there aren't formulaic rules for these; memorise them as separate verbs in their own right.

E.g.\ 走り出す (break into a run)、切り\ruby{替}{か}える (switch to something else)、\ruby{付}{つ}け\ruby{加}{くわ}える (to add something by attaching it)、\ruby{言}{い}い出す (to start talking)、言いさす (to stop talking)、読み出す (read out data/retrieve)、読みさす (to leave half-read).


\subsection{Polite forms ます、です}

The Japanese covered thus far in Section~\ref{sec:basic-grammar} is fine for five-year-olds, but adults are expected to use \ruby{丁寧語}{てい|ねい|ご} (polite language), \ruby{尊敬語}{そん|けい|ご} (honorific language) and \ruby{謙譲語}{けん|じょう|ご} (humble language) where appropriate.

丁寧語 is used when speaking to people of higher social rank or people you're unfamiliar with. 尊敬語 and 謙譲語 are used in professional settings, and builds upon 丁寧語.

The only indicator of which language style is in use is sentence endings. It's not possible to tell whether someone is speaking in casual or polite speech until the sentence is finished.

In 丁寧語, use \textbf{「〜ます」} and its conjugations to make verbs polite, and use \textbf{「〜です」} for nouns and adjectives (note that the polite です is incompatible with the assertive だ). Detailed rules are in Table~\ref{tbl:teineigo-forms}. Note that ます and です must come at the end of a complete sentence, and never inside any embedded/relative clauses.

Polite verb examples:
\begin{itemize}
    \item (present positive) 明日、大学に\textbf{行きます}。 (Tomorrow, I go to university.)
    \item (past positive) 先週、ボブに\textbf{会いました}よ。 (Last week, I met Bob, you know.)
    \item (present negative) 面白くない映画は\textbf{見ません}。(I did not see uninteresting movies.)
    \item (past negative) 昼ご飯を\textbf{食べませんでした}ね。(Didn't eat lunch, right?)
\end{itemize}

Polite adjective examples:
\begin{itemize}
    \item (present positive) 子犬はとても\textbf{好きです}。(I like puppies very much/puppies like something very much.)
    \item (past positive) 先週に見た映画は、とても\textbf{面白かったです}。(The movie I saw last week was very interesting.)
    \item (present negative) その部屋はあまり\textbf{静かじゃないです}。 (The room is not very/really quiet.)
    \item (past negative) 昨日、時間が\textbf{なかったんです}。 (Yesterday, there was no time.; ある $\to$ \ul{なかった} (past negative) $\to$ なかった\ul{んだ} (explanatory) $\to$ なかったん\ul{です} (polite))
\end{itemize}

\begin{table}[h]
\centering
\resizebox{\textwidth}{!}{%
% Help: \multicolumn{2}{c}{}, \multirow{2}{*}{}, cmidrule(l){3-5}
\begin{tabular}{@{}crrrrrr@{}}
    \toprule
    \multirow{2}{*}{\textbf{Category}} & \multirow{2}{*}{\textbf{Dictionary form}} & \multirow{2}{*}{\textbf{Verb stem}} & \multicolumn{2}{c}{\textbf{Positive}} & \multicolumn{2}{c}{\textbf{Negative}} \\ \cmidrule(lr){4-5} \cmidrule(l){6-7}
    & & & \textbf{Present} & \textbf{Past} & \textbf{Present} & \textbf{Past} \\
    \midrule
    Noun/な-adjective & 「」 & &  「」\textgreen{です。} & 「」\textgreen{でした$\!\!^\dagger$。} & 「」じゃない\textgreen{です。} & 「」じゃなかった\textgreen{です。}\\
    (formal) & & & & & 「」じゃ\textblue{ありません。} & 「」じゃ\textblue{ありませんでした。} \\
    \midrule
    い-adjective & 「」い & &  「」い\textgreen{です。} & 「」かった\textgreen{です$\!\!^\ddagger$。} & 「」くない\textgreen{です。} & 「」くなかった\textgreen{です。} \\
    (formal) & & & & & 「」く\textblue{ありません。} & 「」く\textblue{ありませんでした。} \\
    \midrule
    \multirow{9}{*}{う-verb} & 「」す & 「」\textblue{し} & 「」\textblue{します。} & 「」\textblue{しました。} & 「」\textblue{しません。} & 「」\textblue{しませんでした。}\\
    & 「」く & 「」\textblue{き} & 「」\textblue{きます。} & 「」\textblue{きました。} & 「」\textblue{きません。} & 「」\textblue{きませんでした。}\\
    & 「」ぐ & 「」\textblue{ぎ} & 「」\textblue{ぎます。} & 「」\textblue{ぎました。} & 「」\textblue{ぎません。} & 「」\textblue{ぎませんでした。}\\
    & 「」む & 「」\textblue{み} & 「」\textblue{みます。} & 「」\textblue{みました。} & 「」\textblue{みません。} & 「」\textblue{みませんでした。}\\
    & 「」ね & 「」\textblue{に} & 「」\textblue{にます。} & 「」\textblue{にました。} & 「」\textblue{にません。} & 「」\textblue{にませんでした。}\\
    & 「」ぶ & 「」\textblue{び} & 「」\textblue{びます。} & 「」\textblue{びました。} & 「」\textblue{びません。} & 「」\textblue{びませんでした。}\\
    & 「」る & 「」\textblue{り} & 「」\textblue{ります。} & 「」\textblue{りました。} & 「」\textblue{りません。} & 「」\textblue{りませんでした。}\\
    & 「」つ & 「」\textblue{ち} & 「」\textblue{ちます。} & 「」\textblue{ちました。} & 「」\textblue{ちません。} & 「」\textblue{ちませんでした。}\\
    & 「」う & 「」\textblue{い} & 「」\textblue{います。} & 「」\textblue{いました。} & 「」\textblue{いません。} & 「」\textblue{いませんでした。}\\
    \midrule
    る-verb & 「」る & 「」  & 「」 \textblue{ます。} & 「」 \textblue{ました。} & 「」 \textblue{ません。} & 「」 \textblue{ませんでした}\\
    \midrule
    \multirow{3}{*}{Exception verb} & 「」する & 「」\textblue{し} & 「」\textblue{します。} & 「」\textblue{しました。} & 「」\textblue{しません。} & 「」\textblue{しませんでした}\\
    & くる & \textblue{き} & \textblue{きます。} & \textblue{きました。} & \textblue{きません。} & \textblue{きませんでした。} \\[0.5em]
    & \ruby{来}{く}る & \textblue{\ruby{来}{き}} & \textblue{\ruby{来}{き}ます。} & \textblue{\ruby{来}{き}ました。} & \textblue{\ruby{来}{き}ません。} & \textblue{\ruby{来}{き}ませんでした。} \\[0.5em]
    \bottomrule
\end{tabular}%
}
\caption{丁寧語 forms and their conjugations. \textgreen{green means additive} (without modifying the dictionary-form); \textblue{blue means substitutive} (modifies the dictionary-form); \textred{red means exception}. $^\dagger$: Past-positive nouns and な-adjectives do not use ``だったでした'' as that contains the assertive だ, which is incompatible with polite speech. Note however that this does not apply to い-adjectives ($^\ddagger$)! \textorange{Mnemonic: でした is a ``past tense'' marker, and appears in both past-positive nouns and adjectives, and past-negative verbs.}}
\label{tbl:teineigo-forms}
\end{table}

\subsubsection{です is NOT the polite form of だ, they are completely separate constructs}
This is a common misconception! です and だ are two fundamentally different concepts:
\begin{itemize}
    \item だ is declarative, whereas です is polite.

    E.g.\ そうだ is the declarative form of そう. そうです is the polite form of そう (see Table~\ref{tbl:teineigo-forms}), but not the polite form of そうだ!
    \item だ can be used both at the end of both complete sentences and relative clauses, whereas です can only be used at the end of complete sentences.

    E.g.\ そうだど思います。 \textred{(In an indirect quote, it is wrong to quote です: it must be changed to だ.)}

    E.g.\ 「はい、そうです」と答える。 (The only place where です can be in an embedded clause is in direct quotes.)
\end{itemize}

\subsection{Addressing people}
Take care to use the correct level of politeness.

\subsubsection{Referring to yourself}
To refer to yourself, use one of the following (in descending order of politeness):
\begin{itemize}
    \item \ruby{私}{わたくし}: used by both males and females, formal
    \item \ruby{私}{わたし}: used by both males and females, normal polite (you should use this most of the time)
    \item 僕: used by males, semi-polite to semi-casual
    \item 俺: used only by males, very casual, very rough
    \item あたし: used by females, cutesy and girly (most girls today use \ruby{私}{わたし} instead)
    \item your own name: used by children, feminine
    \item わし: used by elderly (males)
\end{itemize}

E.g.\ 私の\ruby{名前}{な|まえ}はジャッキーです。

\subsubsection{Referring to the listener (second-person)}
Even when directly addressing other people, you rarely use variants of ``you'', as constantly hammering the listener with ``you'' that comes across as accusatory and confrontational.

Instead, we use one of the following (in descending order of politeness):
\begin{itemize}
    \item <last name>: common
    \item <title>: common polite (e.g.\ \ruby{社長}{しゃ|ちょう} president/director, \ruby{課長}{か|ちょう} section manager, 先生 teacher/anyone with significant expertise including doctors)
    \item <last name><title>: common polite
    \item <last name>さん: common polite, in case no suitable title exists
    \item <first name>さん: common semi-polite
    \item <last name>くん: for males, casual/endearing, of equal or lower social position
    \item <last name>ちゃん: for females, casual/endearing, of equal or lower social position
    \item <first name>[\{くん, ちゃん\}]: only for people you're on first-name basis with
    \item <nothing>: common. In normal Japanese conversations, the topic/subject is commonly implied to be the listener. It's completely normal not to use anything at all, since you're directly addressing the listener!
    \item 君: casual, assuming/very close, used by males to address females, potentially rude
    \item あなた: rude if spoken, use only in contexts where you must refer to the audience (e.g.\ on a physical questionnaire)
    \item \textred{THOU SHALT NOT CROSS THIS LINE}
    \item あんた: rude, assuming/familiar, expresses annoyance
    \item お前/おめえ: rude, rough and coarse, used by males
    \item てめえ: very rude, sounds like you want to beat someone up (used exclusively in movies and comics)
    \item \ruby{貴様}{き|さま}: extremely rude, sounds like you want to take someone out (used exclusively in movies and comics)
\end{itemize}

\ruby[g]{貴方}{あなた} is an old-fashioned way for women to refer to their lover or husband, nowadays only used by married middle-aged women.

\subsubsection{Referring to the third person}
For people outside of your family, use one of these:
\begin{itemize}
    \item \ruby{彼}{かれ}: he
    \item \ruby{彼女}{かの|じょ}: she
    \item 彼/ボーイフレンド: boyfriend (prefer former)
    \item 彼女/ガールフレンド: girlfriend (prefer former)
\end{itemize}

When referring to family members, there are two levels of politeness: polite and casual. We only use the casual form when we are talking about our own family members to other people. When talking about the listener's family members or when talking directly to your family members, use the polite form.

The polite form comes before the casual form:
\begin{itemize}
    \item Parents: ご\ruby{両親}{りょう|しん} / \ruby{両親}{りょう|しん}
    \item Mother: お\ruby{母}{かあ}さん / \ruby{母}{はは}
    \item Father: お\ruby{父}{とう}さん / \ruby{父}{ちち}
    \item Wife: \ruby{奥}{おく}さん / \ruby{妻}{つま}
    \item Husband: ご\ruby{主人}{しゅ|じん} / \ruby{夫}{おっと} (don't drag the trailing お vowel)
    \item Older sister: お\ruby{姉}{ねえ}さん / \ruby{姉}{あね}
    \item Older brother: お\ruby{兄}{にい}さん / \ruby{兄}{あに}
    \item Younger sister: \ruby{妹}{いもうと}さん / \ruby{妹}{いもうと}
    \item Younger brother: \ruby{弟}{おとうと}さん / \ruby{弟}{おとうと}
    \item Son: \ruby{息子}{むす|こ}さん / \ruby{息子}{むす|こ}
    \item Daughter: \ruby{娘}{むすめ}さん / \ruby{娘}{むすめ}
\end{itemize}
Yes, ご主人\ruby{様}{さま} (master/husband) is an extension of ご主人 (husband).

\subsection{The question marker か}
These are common question words:
\begin{itemize}
    \item 誰: who
    \item 何: what
    \item いつ: when
    \item どこ: where
    \item なぜ/どうして/なんで: why

    なぜ is formal and forceful\\
    どうして is softer\\
    なんで is informal (\href{https://www.reddit.com/r/LearnJapanese/comments/nxxep9/difference_between_%E3%81%A9%E3%81%86%E3%81%97%E3%81%A6_%E3%81%AA%E3%82%93%E3%81%A7_and_%E3%81%AA%E3%81%9C/}{Source})
    \item どう/どうやって: how

    どう is more general (\href{https://ja.hinative.com/questions/161399}{Source})
    \item どれ/どちら/どっち: which

    どれ: three or more\\
    どちら: two\\
    どっち: two (informal; \href{https://www.tofugu.com/japanese-grammar/kore-sore-are-dore/}{Source})
\end{itemize}

\subsubsection{か in polite questions}
The purpose of か is to clearly mark a question in polite sentences. It's not strictly necessary, since polite sentences without a trailing か can be interpreted as a question using a rising trailing intonation during speech. However, it's commonly attached.

There is no need to use a question mark when か is used; the full-stop is used instead. Because か is polite, it is incompatible with the declarative だ.

E.g.\ お母さんはどこです\textbf{か。} \ruby{母}{はは}は買い物に行きますた。(Where is your mother? My mother went shopping.; 買い物 can refer to both purchased goods and the act of shopping)

E.g.\ イタリア\ruby{料理}{りょう|り}を食べに行きません\textbf{か。} すみません。ちょっと、お\ruby{腹}{なか}がいっぱいです。(Shall we eat Italian food? Sorry, my stomach is a little full.; いっぱい: full; Note that the question is phrased in the negative, see \S\ref{sec:positive-negative-questions})

\subsubsection{Positive and negative polite questions} \label{sec:positive-negative-questions}
\emph{See discussion at \href{https://ja.hinative.com/questions/16031339}{HiNative}.}

Positive polite questions are plain questions; negative questions have a nuance of suggestion/invitation.
\begin{itemize}
    \item 食べに行き\textbf{ます}か?: Are you going to eat?
    \item 食べに行き\textbf{ません}か?: Shall we go eat?
\end{itemize}

\subsubsection{か in casual questions: binary and sarcasm}
か has a slightly different purpose in casual speech. Casual questions usually either use explanatory 「の?」 or nothing at all, so か is not used here to craft questions. Instead, it is used specifically to:
\begin{itemize}
    \item question whether something is true or not

    E.g.\ こんなのを本当に食べる\textbf{か}? (This kind of thing, will they really eat?; こんなの is こんな + の (possession), where こんな means this type of)

    \item make rhetorical questions/express sarcasm

    E.g.\ そんなのは、ある\textbf{か}よ! (That kind of thing, do  I look like I would have something like that!?; そんなの is そんな + の (possession), where そんな means that type of)
\end{itemize}
Most actual questions use explanatory の or nothing at all apart from a rise in intonation.

E.g.\ こんなのを本当に食べる? (Something like this, are you really going to eat?)

E.g.\ そんなのは、ある\textbf{の}? (Do you have something like that?)

\subsubsection{か in embedded clauses: nothing to do with politeness}
This functions similar to direct quoting, and marks the questions in an embedded clause. The outer clause can then talk about the embedded question.

E.g.\ \textbf{昨日何を食べたか}忘れた。 (What I ate yesterday, I forgot.)

E.g.\ \textbf{彼は何を言ったか}分からない。 (What he said, I don't understand.)

E.g.\ \textbf{先生が学校に行ったか}教えない? (Whether the teacher went to school (binary question), would you please inform me (invitation)?)

To ask ``whether or not'' (binary question), we can use either <positive>か\textbf{どうか}, or <positive>か<negative>か.

E.g.\ 先生が学校に\textbf{行ったか行かなかった}か知らない。(Whether the teacher went to school or not, I don't know.)

E.g.\ 先生が学校に\textbf{行ったか\ul{どうか}}か知らない。(Whether the teacher went to school or not, I don't know.)


\subsubsection{Modifying question words with suffixes か、も、でも}
「」か refers to a particular existence (some\textasciitilde), 「」も refers to the universal (every\textasciitilde), and 「」でも refers to a non-particular existence (any\textasciitilde) (not to be confused with でも for ``but''). Question words and their variants are showed in Table~\ref{tbl:question-word-modifications}.

\begin{table}[h]
\centering
\resizebox{\textwidth}{!}{%
% Help: \multicolumn{2}{c}{}, \multirow{2}{*}{}, cmidrule(l){3-5}
\begin{tabular}{@{}rcrcrcrc@{}}
    \toprule
    \textbf{「」} & \textbf{Meaning} & \textbf{「」か} & \textbf{Meaning} & \textbf{「」も} & \textbf{Meaning} & \textbf{「」でも} & \textbf{Meaning} \\
    \midrule
    誰 & who & 誰か & someone & 誰も & everybody$^\dagger$ & 誰でも & anybody \\[0.5em]
    \ruby{何}{なに} & what & \ruby{何}{なに}か & something & \ruby{何}{なに}も & everything$^\ddagger$ & \ruby{何}{なん}でも & anything \\
    いつ & when & いつか & sometime & いつも & always & いつでも & anytime \\
    どこ & where & どこか & somewhere & どこ[に]も & everywhere & どこでも & anywhere \\
    なぜ & why & なぜか & some reason & & & & \\
    どう & how & どうか & somehow & & & & \\
    どれ & which (3 or more) & どれか & one from many & どれも & all & どれでも & any/whichever \\
    \bottomrule
\end{tabular}%
}
\caption{Question words and their modified variants. Treat these all as normal nouns. $^\dagger$: 誰も is usually used in negative sentences to mean nobody can do the verb, and to express the positive universal everybody we typically use \ruby{皆}{みんあ}[さん]; $^\ddagger$: 何も is used exclusively in negative sentences.}
\label{tbl:question-word-modifications}
\end{table}

To mean ``for some reason...'' you can say 「どういうわけか」.

「」か examples:
\begin{itemize}
    \item \textbf{誰か}がおいしいクッキーを全部食べた。 (Someone ate all the delicious cookies.)
    \item \textbf{誰が}\ruby{盗}{むす}んだのか、誰か知りませんか。 (Who stole it, doesn't anyone know?; 盗む: steal)
    \item \ruby{犯人}{はん|にん}を\textbf{どこか}で見ましたか。 (Did you see the criminal somewhere?)
    \item この\ruby{中}{なか}から\textbf{どれか}を\ruby{選}{えら}ぶの。 (You are to select a certain one from inside this.; から: from, 選ぶ: choose)
\end{itemize}

「」も examples:
\begin{itemize}
    \item この\ruby{質問}{しつ|もん}の\ruby{答}{こた}えは、\textbf{誰も}知ら\textbf{ない}。(The answer to this question, nobody knows.)
    \item 友達は\textbf{いつも}\ruby{遅}{おく}れる。 (Friend is always late.)
    \item ここにあるレストランは\textbf{どれも}おいしくない。 (All restaurants that are here are not tasty.)
    \item \ruby{今週末}{こん|しゅう|まつ}は、\textbf{どこにも}行かなかった。 (This weekend, went nowhere.; どこにも means ``target is nowhere'', も is grammatically the topic particle and should come after the target particle に, so \cancel{どこもに} is incorrect. Treat this as an exception.)
\end{itemize}

「」でも examples:
\begin{itemize}
    \item この質問の答えは、\textbf{誰でも}分かる。(The answer to this question, anyone understands.)
    \item 昼ご飯は、\textbf{どこでも}いいです。 (For lunch, anywhere is good.)
    \item あの人は、本当に\textbf{何でも}食べる。 (That person really eats anything.)
\end{itemize}

\subsection{Apologising}
\emph{Read a full article \href{https://www.clozemaster.com/blog/sorry-in-japanese/}{here}.}

Use one of the following (in descending order of politeness):
\begin{itemize}
    \item すみません: formal
    \item ごめんなさい: semi-formal
    \item ごめん[ね]: causal
    \item \ruby{悪}{わる}い: very casual, only for non-serious matters

    Can be used for past offence: \ruby{悪}{わる}かった。
\end{itemize}

\subsection{Compound sentences}

\subsubsection{て form}

The て-form of nouns, adjectives and verbs are used to form sequences of states (nouns/adjectives) or actions. The conjugation rules are in Table~\ref{tbl:te-form}.

\begin{table}[h]
\centering
\resizebox{\textwidth}{!}{%
% Help: \multicolumn{2}{c}{}, \multirow{2}{*}{}, cmidrule(l){3-5}
\begin{tabular}{@{}crrrl@{}}
    \toprule
    \textbf{Category} & \textbf{Dictionary form} &\textbf{Positive て-form} & \textbf{Negative て-form} & \textbf{Examples} \\
    \midrule
    Noun/な-adjective & 「」 & 「」\textgreen{で} & 「」\textgreen{じゃなくて} & 学生、友達、好き、\textred{嫌い} \\ \midrule
    \multirow{3}{*}{い-adjective} & 「」い & 「」\textblue{くて} & 「」\textblue{くなくて} & 嬉しい、気持ち悪い、難しい \\
    & いい & \textred{よ}\textblue{くて} & \textred{よ}くなくて &  \\
    & かっこいい & かっこ\textred{よ}\textblue{くて} & かっこ\textred{よ}くなくて &  \\ \midrule
    \multirow{10}{*}{う-verb} & 「」 す & 「」\textblue{して} & 「」\textblue{さなくて} & 話す、出す \\
    & 「」 く & 「」\textblue{いて} & 「」\textblue{かなくて} & 聞く、書く、描く \\
    & 行 く & 行\textblue{って} & 行\textblue{かなくて} &  \\
    & 「」 ぐ & 「」\textblue{いで} & 「」\textblue{がなくて} & 泳ぐ \\
    & 「」 む & 「」\textblue{んで} & 「」\textblue{まなくて} & 飲む、読む、\ruby{盗}{ぬす}む \\
    & 「」 ね & 「」\textblue{んで} & 「」\textblue{ななくて} & 死ね \\
    & 「」 ぶ & 「」\textblue{んで} & 「」\textblue{ばなくて} & 遊ぶ \\
    & 「」 る & 「」\textblue{って} & 「」\textblue{らなくて} & \textred{知る}、\textred{切る}、\textred{\ruby{帰}{かえ}る}、\textred{\ruby{走}{はし}る}、\ruby{降}{ふ}る \\
    & ある & あ\textblue{って} & \textred{なくて} &  \\
    & 「」 つ & 「」\textblue{って} & 「」\textblue{たなくて} & 待つ \\
    & 「」 う & 「」\textblue{って} & 「」\textblue{わなくて} & 買う、\ruby{会}{あ}う \\ \midrule
    る-verb & 「」 る &「」 \textblue{て} & 「」 \textblue{なくて} & いる、食べる、\ruby{出}{で}る、見る* \\ \midrule
    \multirow{3.5}{*}{Exception verb} & 「」する & 「」\textblue{して} & 「」\textblue{しなくて} & 勉強する \\
    & くる & \textred{き}\textblue{て} & \textred{こ}\textblue{なくて} &  \\[0.5em]
    & \ruby{来}{く}る & \textred{\ruby{来}{き}}\textblue{て} & \textred{\ruby{来}{こ}}\textblue{なくて} &  \\
    \bottomrule
\end{tabular}%
}
\caption{て-form conjugation rules, for nouns, adjectives and verbs. [] means optional; 「」 is a dictionary-form placeholer; \textgreen{green means additive} (without modifying the dictionary-form); \textblue{blue means substitutive} (modifies the dictionary-form); \textred{red means exception}.}
\label{tbl:te-form}
\end{table}


\end{document}
