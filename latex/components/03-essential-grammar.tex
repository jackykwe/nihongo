\documentclass[../nihongo-gakushuu-kyouzai.tex]{subfiles}
\graphicspath{ {../imgs/} }
\begin{document}
\onehalfspacing  % for 振り仮名

\setcounter{section}{2}
\section{Essential grammar} \label{sec:essential-grammar}

Now that we have an understanding of the basics, fundamentals, and foundations of Japanese grammar, this section provides specific grammar for practical situations.

\subsection{Verb stems}

Also known as the ます-stem in other texts (we refrain from that term as the stem is used in many more conjugations), stems can be obtained from verbs via the rules in Table~\ref{tbl:teineigo-forms}.

Verb stems are sometimes (not always!) the noun forms of the verbs, e.g.\ \ruby{休}{やす}み is the noun form of 休む.

We can suffix the stem with target particle に or directional particle へ, then follow with a motion verb (almost always 行く or くる). This <stem>\{に, へ\}\{行く, くる, \dots\} construct means ``to go/come do <stem>''. Use に to mean going/coming for the purpose of doing <stem>; use へ to put emphasis on the literal act of going/coming.

E.g.\ 明日、映画を\textbf{見に行く}。(Tomorrow, go to see movie.)

E.g.\ 昨日、友達が\textbf{遊びへきた}。(Yesterday, friend came towards a playing activity.; sounds weird)

E.g.\ 昨日、友達が\textbf{遊びにきた}。 (Yesterday, friend came to play.)

The expression 「楽しみにする」 (looking forward to) is formed from a similar grammar rule (though it's a special case and should be considered a set expression).

Compound verbs can be formed by appending verbs to stems. However, there aren't formulaic rules for these; memorise them as separate verbs in their own right.

E.g.\ 走り出す (break into a run)、切り\ruby{替}{か}える (switch to something else)、\ruby{付}{つ}け\ruby{加}{くわ}える (to add something by attaching it)、\ruby{言}{い}い出す (to start talking)、言いさす (to stop talking)、読み出す (read out data/retrieve)、読みさす (to leave half-read).


\subsection{Polite forms ます、です}

The Japanese covered thus far in Section~\ref{sec:basic-grammar} is fine for five-year-olds, but adults are expected to use \ruby{丁寧語}{てい|ねい|ご} (polite language), \ruby{尊敬語}{そん|けい|ご} (honorific language) and \ruby{謙譲語}{けん|じょう|ご} (humble language) where appropriate.

丁寧語 is used when speaking to people of higher social rank or people you're unfamiliar with. 尊敬語 and 謙譲語 are used in professional settings, and builds upon 丁寧語.

The only indicator of which language style is in use is sentence endings. It's not possible to tell whether someone is speaking in casual or polite speech until the sentence is finished.

In 丁寧語, use \textbf{「〜ます」} and its conjugations to make verbs polite, and use \textbf{「〜です」} for nouns and adjectives (note that the polite です is incompatible with the assertive だ). Detailed rules are in Table~\ref{tbl:teineigo-forms}. Note that ます and です must come at the end of a complete sentence, and never inside any embedded/relative clauses.

Polite verb examples:
\begin{itemize}
    \item (present positive) 明日、大学に\textbf{行きます}。 (Tomorrow, I go to university.)
    \item (past positive) 先週、ボブに\textbf{会いました}よ。 (Last week, I met Bob, you know.)
    \item (present negative) 面白くない映画は\textbf{見ません}。(I did not see uninteresting movies.)
    \item (past negative) 昼ご飯を\textbf{食べませんでした}ね。(Didn't eat lunch, right?)
\end{itemize}

Polite adjective examples:
\begin{itemize}
    \item (present positive) 子犬はとても\textbf{好きです}。(I like puppies very much/puppies like something very much.)
    \item (past positive) 先週に見た映画は、とても\textbf{面白かったです}。(The movie I saw last week was very interesting.)
    \item (present negative) その部屋はあまり\textbf{静かじゃないです}。 (The room is not very/really quiet.)
    \item (past negative) 昨日、時間が\textbf{なかったんです}。 (Yesterday, there was no time.; ある $\to$ \ul{なかった} (past negative) $\to$ なかった\ul{んだ} (explanatory) $\to$ なかったん\ul{です} (polite))
\end{itemize}

\begin{table}[h]
\centering
\resizebox{\textwidth}{!}{%
% Help: \multicolumn{2}{c}{}, \multirow{2}{*}{}, cmidrule(l){3-5}
\begin{tabular}{@{}crrrrrr@{}}
    \toprule
    \multirow{2}{*}{\textbf{Category}} & \multirow{2}{*}{\textbf{Dictionary form}} & \multirow{2}{*}{\textbf{Verb stem}} & \multicolumn{2}{c}{\textbf{Positive}} & \multicolumn{2}{c}{\textbf{Negative}} \\ \cmidrule(lr){4-5} \cmidrule(l){6-7}
    & & & \textbf{Present} & \textbf{Past} & \textbf{Present} & \textbf{Past} \\
    \midrule
    Noun/な-adjective & 「」 & &  「」\textgreen{です。} & 「」\textgreen{でした$\!\!^\dagger$。} & 「」じゃない\textgreen{です。} & 「」じゃなかった\textgreen{です。}\\
    (formal) & & & & & 「」じゃ\textblue{ありません。} & 「」じゃ\textblue{ありませんでした。} \\
    \midrule
    い-adjective & 「」い & &  「」い\textgreen{です。} & 「」かった\textgreen{です$\!\!^\ddagger$。} & 「」くない\textgreen{です。} & 「」くなかった\textgreen{です。} \\
    (formal) & & & & & 「」く\textblue{ありません。} & 「」く\textblue{ありませんでした。} \\
    \midrule
    \multirow{9}{*}{う-verb} & 「」す & 「」\textblue{し} & 「」\textblue{します。} & 「」\textblue{しました。} & 「」\textblue{しません。} & 「」\textblue{しませんでした。}\\
    & 「」く & 「」\textblue{き} & 「」\textblue{きます。} & 「」\textblue{きました。} & 「」\textblue{きません。} & 「」\textblue{きませんでした。}\\
    & 「」ぐ & 「」\textblue{ぎ} & 「」\textblue{ぎます。} & 「」\textblue{ぎました。} & 「」\textblue{ぎません。} & 「」\textblue{ぎませんでした。}\\
    & 「」む & 「」\textblue{み} & 「」\textblue{みます。} & 「」\textblue{みました。} & 「」\textblue{みません。} & 「」\textblue{みませんでした。}\\
    & 「」ね & 「」\textblue{に} & 「」\textblue{にます。} & 「」\textblue{にました。} & 「」\textblue{にません。} & 「」\textblue{にませんでした。}\\
    & 「」ぶ & 「」\textblue{び} & 「」\textblue{びます。} & 「」\textblue{びました。} & 「」\textblue{びません。} & 「」\textblue{びませんでした。}\\
    & 「」る & 「」\textblue{り} & 「」\textblue{ります。} & 「」\textblue{りました。} & 「」\textblue{りません。} & 「」\textblue{りませんでした。}\\
    & 「」つ & 「」\textblue{ち} & 「」\textblue{ちます。} & 「」\textblue{ちました。} & 「」\textblue{ちません。} & 「」\textblue{ちませんでした。}\\
    & 「」う & 「」\textblue{い} & 「」\textblue{います。} & 「」\textblue{いました。} & 「」\textblue{いません。} & 「」\textblue{いませんでした。}\\
    \midrule
    る-verb & 「」る & 「」  & 「」 \textblue{ます。} & 「」 \textblue{ました。} & 「」 \textblue{ません。} & 「」 \textblue{ませんでした}\\
    \midrule
    \multirow{3}{*}{Exception verb} & 「」する & 「」\textblue{し} & 「」\textblue{します。} & 「」\textblue{しました。} & 「」\textblue{しません。} & 「」\textblue{しませんでした}\\
    & くる & \textblue{き} & \textblue{きます。} & \textblue{きました。} & \textblue{きません。} & \textblue{きませんでした。} \\[0.5em]
    & \ruby{来}{く}る & \textblue{\ruby{来}{き}} & \textblue{\ruby{来}{き}ます。} & \textblue{\ruby{来}{き}ました。} & \textblue{\ruby{来}{き}ません。} & \textblue{\ruby{来}{き}ませんでした。} \\[0.5em]
    \bottomrule
\end{tabular}%
}
\caption{丁寧語 forms and their conjugations. \textgreen{green means additive} (without modifying the dictionary-form); \textblue{blue means substitutive} (modifies the dictionary-form); \textred{red means exception}. $^\dagger$: Past-positive nouns and な-adjectives do not use ``だったでした'' as that contains the assertive だ, which is incompatible with polite speech. Note however that this does not apply to い-adjectives ($^\ddagger$)! \textorange{Mnemonic: でした is a ``past tense'' marker, and appears in both past-positive nouns and adjectives, and past-negative verbs.}}
\label{tbl:teineigo-forms}
\end{table}

\subsubsection{です is NOT the polite form of だ, they are completely separate constructs}
This is a common misconception! です and だ are two fundamentally different concepts:
\begin{itemize}
    \item だ is declarative, whereas です is polite.

    E.g.\ そうだ is the declarative form of そう. そうです is the polite form of そう (see Table~\ref{tbl:teineigo-forms}), but not the polite form of そうだ!
    \item だ can be used both at the end of both complete sentences and relative clauses, whereas です can only be used at the end of complete sentences.

    E.g.\ そうだど思います。 \textred{(In an indirect quote, it is wrong to quote です: it must be changed to だ.)}

    E.g.\ 「はい、そうです」と答える。 (The only place where です can be in an embedded clause is in direct quotes.)
\end{itemize}

\subsection{Addressing people}



\end{document}
