\documentclass[../nihongo-gakushuu-kyouzai.tex]{subfiles}
\begin{document}
\appendix
\setcounter{section}{2}
\setcounter{subsection}{3}

\subsection{Adjectives}

\subsubsection{Emotions}
\begin{center}
\centering
\resizebox{\linewidth}{!}{%
% Help: \multicolumn{2}{c}{}, \multirow{2}{*}{}, cmidrule(l){3-5}
\begin{tabular}{@{}lcll@{}}
    \toprule
    \textbf{Descriptor} & \textbf{Cat.} & \textbf{Meaning} & \textbf{Notes} \\
    \toprule
    \ruby{嬉}{うれ}しい & い & happy/glad/delighted & \\
    \ruby{楽}{たの}しい & い & fun/enjoyable/happy & \\
    \ruby{欲}{ほ}しい & い & desired/wanted & \\
    \midrule
    \ruby{悲}{かな}しい & い & sad/miserable & \\
    \ruby{恥}{は}ずかしい & い & embarrassed/ashamed/humiliated & \\
    \ruby{懐}{なつ}かしい & い & nostalgic/fondly-remembered/missed & \\
    \ruby{寂}{さび}しい & い & lonely & \\
    \midrule
    \ruby{眠}{ねむ}い & い & sleepy/drowsy & \\
    \midrule
    \ruby{安心}{あん|しん} & な & relieved & \\
    \midrule
    \midrule
    \ruby{暖}{あたた}かい & い & pleasantly warm & \\
    \ruby{暑}{あつ}い & い & hot & \\
    \ruby{熱}{あつ}い & い & hot (to the touch) & \\
    \ruby{小寒}{こ|さむ}い & い & chilly/a little cold & \\
    \midrule
    \ruby{寒}{さむ}い & い & cold (weather) & \\
    % & & & \\
\bottomrule
\end{tabular}%
}
\captionof{table}{Adjectives: emotions.}
\label{tbl:appendix-vocab-adjectives-emotions}
\end{center}


\subsubsection{Consumption}
\begin{center}
\centering
\resizebox{\linewidth}{!}{%
% Help: \multicolumn{2}{c}{}, \multirow{2}{*}{}, cmidrule(l){3-5}
\begin{tabular}{@{}lcll@{}}
    \toprule
    \textbf{Descriptor} & \textbf{Cat.} & \textbf{Meaning} & \textbf{Notes} \\
    \toprule
    \ruby{使}{つか}いやすい & い & easy to use & \\
    \ruby{見}{み}やすい & い & easy to see & \\
    \ruby{読}{よ}みやすい & い & easy to read/legible & \\
    \ruby{飲}{の}みやすい & い & easy to drink/swallow & \\
    \midrule
    \midrule
    わかりやすい & い & easy to understand & \\
    \ruby{覚}{おぼ}えやすい & い & easy to learn/remember & \\
    \midrule
    \midrule
    \ruby{住}{す}みやすい & い & comfortable/convenient to live in (of a neighbourhood) & \\
    % & & & \\
\bottomrule
\end{tabular}%
}
\captionof{table}{Adjectives: consumption.}
\label{tbl:appendix-vocab-adjectives-consumption}
\end{center}


\subsubsection{Health}
\begin{center}
\centering
\resizebox{\linewidth}{!}{%
% Help: \multicolumn{2}{c}{}, \multirow{2}{*}{}, cmidrule(l){3-5}
\begin{tabular}{@{}lcll@{}}
    \toprule
    \textbf{Descriptor} & \textbf{Cat.} & \textbf{Meaning} & \textbf{Notes} \\
    \toprule
    \ruby{大丈夫}{だい|じょう|ぶ} & な & alright/problem-free/without fear & \\
    \ruby{健康}{けん|こう} & な & healthy/fit; wholesome & also a noun \\
    \midrule
    \ruby{元気}{げん|き} & な & lively/well/in good health & \\
    \ruby{病気}{びょう|き} & な & illness/disease/sickness & \\
    \midrule
    \midrule
    \ruby{不安}{ふ|あん} & な & anxious/uneasy/insecure & also a noun\\
    \midrule
    \midrule
    \ruby{暇}{ひま} & な & free/available & also a noun \\
    \ruby{忙}{いそが}しい & い & busy/occupied/hectic & \\
    \midrule
    \midrule
    \ruby{幸}{しあわ}せ & な & happy/blessed & also a noun \\
    \ruby{不幸}{ふ|しあわ}せ & な & unhappiness & also a noun \\
    % & & & \\
\bottomrule
\end{tabular}%
}
\captionof{table}{Adjectives: health.}
\label{tbl:appendix-vocab-adjectives-health}
\end{center}


\subsubsection{Colours}
\begin{center}
\centering
\resizebox{\linewidth}{!}{%
% Help: \multicolumn{2}{c}{}, \multirow{2}{*}{}, cmidrule(l){3-5}
\begin{tabular}{@{}lcll@{}}
    \toprule
    \textbf{Descriptor} & \textbf{Cat.} & \textbf{Meaning} & \textbf{Notes} \\
    \toprule
    \ruby{白}{しろ}い & い & white & \\
    % & & & \\
\bottomrule
\end{tabular}%
}
\captionof{table}{Adjectives: colours.}
\label{tbl:appendix-vocab-adjectives-colours}
\end{center}


\subsubsection{Agreeability}
\begin{center}
\centering
\resizebox{\linewidth}{!}{%
% Help: \multicolumn{2}{c}{}, \multirow{2}{*}{}, cmidrule(l){3-5}
\begin{tabular}{@{}lcll@{}}
    \toprule
    \textbf{Descriptor} & \textbf{Cat.} & \textbf{Meaning} & \textbf{Notes} \\
    \toprule
    ない & い & non-existent/not being there & (\ruby{無}{な}い) \\
    \midrule
    \midrule
    いい/\ruby{良}{よ}い/よい & い & good/nice/agreeable/OK & \href{https://salon.mainichi-kotoba.jp/archives/670}{[MK]}\\
    すごい & い & amazing/great/wonderful/terrific & (\ruby{凄}{すご}い) \\
    やばい & い & terrific/amazing/cool (``damn!'') & colloquial, slang \\
    \midrule
    \ruby{悪}{わる}い & い & bad/poor/undesirable/at fault & also an interjection \\
    まずい & い & bad taste/unpleasant/awful/problematic/unfavourable & \\
    だめ & な & not good/hopeless; cannot/not allowed & \\
    \ruby{嫌}{いや} & な & reluctant/disagreeable & \\
    やばい & い & awful/crazy/unhinged (``damn!'') & colloquial, slang \\
    \midrule
    \midrule
    \ruby{安全}{あん|ぜん} & な & safe/secure & also a noun \\
    \midrule
    \ruby{危険}{き|けん} & な & dangerous/hazardous & also a noun; \href{https://hinative.com/questions/16741337}{[HN]} \\
    \ruby{危}{あぶ}ない & い & dangerous/risky & also an interjection; \href{https://hinative.com/questions/16741337}{[HN]} \\
    やばい & い & dangerous/risky (``damn!'') & colloquial, slang \\
    \midrule
    \midrule
    \ruby{好}{す}き & な & likeable/favourite & \\
    \ruby{大好}{だい|す}き & な & strongly liked/loved & \\
    \midrule
    \ruby{嫌}{きら}い & \exception{な} & disliked/hated & \\
    \ruby{大嫌}{だい|きら}い & \exception{な} & strongly disliked/hated & \\
    \midrule
    \midrule
    \ruby{最高}{さい|こう} & な & best/finest; highest/maximum & \\
    \ruby{最良}{さい|りょう} & な & best/ideal & \\
    \ruby{高級}{こう|きゅう} & な & high class/calibre & \\
    \midrule
    \ruby{最低}{さい|てい} & な & worst/awful/nasty/disgusting; lowest/minimum & \\
    \ruby{最悪}{さい|あく} & な & worst (e.g.\ situation)) & \\
    \ruby{低級}{てい|きゅう} & な & low class/calibre; vulgar/cheap & \\
    \ruby{邪悪}{じゃ|あく} & な & evil/wicked & \\
    \midrule
    \midrule
    \ruby{当}{あ}たり\ruby{前}{まえ} & な & natural/obvious/common/ordinary/the norm & \\
    % & & & \\
\bottomrule
\end{tabular}%
}
\captionof{table}{Adjectives: agreeability.}
\label{tbl:appendix-vocab-adjectives-agreeability}
\end{center}


\subsubsection{Appearance and style}
\begin{center}
\centering
\resizebox{\linewidth}{!}{%
% Help: \multicolumn{2}{c}{}, \multirow{2}{*}{}, cmidrule(l){3-5}
\begin{tabular}{@{}lcll@{}}
    \toprule
    \textbf{Descriptor} & \textbf{Cat.} & \textbf{Meaning} & \textbf{Notes} \\
    \toprule
    かわいい & い & cute/adorable/charming/lovely/pretty & \\
    かっこいい/かっこ\ruby{良}{よ}い & い & cool/attractive/stylish & (\ruby{格好}{かっ|こ}いい/\ruby{格好良}{かっ|こ|よ}い) \\
    \ruby{綺麗}{き|れい} & な & pretty/beautiful/clean/tidy & \\
    \ruby{美}{うつく}しい & い & beautiful/pretty/lovely/sweet/pure (heart/friendship) & \\
    \ruby{素敵}{す|てき} & な & lovely/wonderful/fantastic/superb/nice/cool & \\
    \ruby{立派}{りっ|ぱ} & な & impressive/praiseworthy/splendid/handsome/well-rounded & \\
    \midrule
    \ruby{醜}{みにく}い & い & ugly/unattractive/unsightly/disgraceful/dishonourable & \\
    \midrule
    \midrule
    \ruby{新}{あたら}しい & い & new/novel/recent/latest/modern & \\
    \midrule
    \ruby{古}{ふる}い & い & old/antiquated/old-fashioned (of things, \textred{not people}) & \\
    \midrule
    \midrule
    \ruby{独自}{どく|じ} & な & characteristic/their own/unique/original/local & \href{https://dictionary.goo.ne.jp/thsrs/17037/meaning/m1u/}{[HN]} \\
    \ruby{特有}{とく|ゆう} & な & exclusive/characteristic/peculiar & \href{https://dictionary.goo.ne.jp/thsrs/17037/meaning/m1u/}{[HN]} \\
    \ruby{固有}{こ|ゆう} & な & inherent/characteristic/preculiar & \href{https://dictionary.goo.ne.jp/thsrs/17037/meaning/m1u/}{[HN]} \\
    \midrule
    \midrule
    \ruby{面白}{おも|しろ}い & い & interesting/fascinating/funny/entertaining & \\
    \midrule
    ダサい & い & lame/uncool & slang \\
    \ruby{寒}{さむ}い & い & lame/corny (joke) & \\
    \midrule
    \midrule
    バカ & な & stupid/foolish/ridiculous & (\ruby{馬鹿}{ば|か}) \\
    アホ & な & foolish/idiotic/simplistic & (\ruby{阿呆}{あ|ほ}) \\
    % & & & \\
\bottomrule
\end{tabular}%
}
\captionof{table}{Adjectives: appearance and style.}
\label{tbl:appendix-vocab-adjectives-appearance-and-style}
\end{center}


\subsubsection{Ability}
\begin{center}
\centering
\resizebox{\linewidth}{!}{%
% Help: \multicolumn{2}{c}{}, \multirow{2}{*}{}, cmidrule(l){3-5}
\begin{tabular}{@{}lcll@{}}
    \toprule
    \textbf{Descriptor} & \textbf{Cat.} & \textbf{Meaning} & \textbf{Notes} \\
    \toprule
    うまい & い & skilful/good & (\ruby{上手}{う|ま}い) \\
    \ruby{上手}{じょう|ず} & な & skilful/proficient/adept & \\
    \ruby{有能}{ゆう|のう} & な & capable/competent/efficient & \\
    \midrule
    \ruby{下手}{へ|た} & な & unskilful/poor/awkward & \\
    \ruby{苦手}{にが|て} & な & not very good at & \\
    \ruby{無能}{む|のう} & な & incapable/incompetent/inefficient & \\
    \midrule
    \midrule

    [お]やすい & い & easy & (\ruby{易}{やす}い) \\
    \ruby{簡単}{かん|たん} & な & easy/simple & \\
    \midrule
    \ruby{難}{むずか}しい & い & difficult/troublesome/impossible (euphemism) & \\
    \ruby{大変}{たい|へん} & な & difficult/challenging/serious/dreadful/terrible & also an adverb \\
    \ruby{無理}{む|り} & な & impossible/no way/unreasonable & \\
    % & & & \\
\bottomrule
\end{tabular}%
}
\captionof{table}{Adjectives: ability.}
\label{tbl:appendix-vocab-adjectives-ability}
\end{center}


\subsubsection{Personalities}
\begin{center}
\centering
\resizebox{\linewidth}{!}{%
% Help: \multicolumn{2}{c}{}, \multirow{2}{*}{}, cmidrule(l){3-5}
\begin{tabular}{@{}lcll@{}}
    \toprule
    \textbf{Descriptor} & \textbf{Cat.} & \textbf{Meaning} & \textbf{Notes} \\
    \toprule
    \ruby{優}{やさ}しい & い & kind/affectionate/gentle (character) & speech; \href{https://ja.hinative.com/question_summaries/112079}{[HN]} \\
    \ruby{親切}{しん|せつ} & な & kind/generous/gentle (action) & formal; \href{https://ja.hinative.com/question_summaries/112079}{[HN]} \\
    \ruby{心安}{こころ|やす}い & い & friendly/familiar/intimate & \\
    \midrule
    ひどい & い & cruel/heartless/harsh/very bad/awful & (\ruby{酷}{ひど}い) \\
    \midrule
    \midrule
    \ruby{静}{しず}か & な & quiet/silent/calm/peaceful & \\
    \ruby{冷静}{れい|せい} & な & calm/composed/serene & \\
    \ruby{気安}{き|やす}い & い & relaxed/familiar/friendly & \\
    \midrule
    \midrule
    \ruby{真面目}{ま|じ|め} & な & serious/sober/earnest/grave & \\
    hardworking, lazy... & & & \\
    \midrule
    \midrule
    \ruby{慎重}{しん|ちょう} & な & careful/cautious/prudent & \\
    \ruby{軽率}{けい|そつ} & な & careless/rash/hasty/imprudent & \\
    \midrule
    \midrule
    \ruby{有名}{ゆう|めい} & な & famous & \\
    \midrule
    \midrule
    \ruby{単純}{たん|じゅん} & な & simple/uncomplicated; simple-minded/naive & \\
    \ruby{騙}{だま}されやすい & い & gullible/naive & \\
    \midrule
    \ruby{複雑}{ふく|ざつ} & な & complex/complicated/intricate; mixed (feelings) & \\
    \ruby{受}{う}けやすい & い & susceptible/vulnerable/prone to & \\
    \ruby{感じ}{かん|じ}やすい & い & sensitive/susceptible & also: センシティブ \\
    \midrule
    \midrule
    \ruby{熱}{ね}しやすい & い & excitable & \\
    \midrule
    \ruby{飽}{あ}きやすい & い & easily bored/fickle/quick to lose interest & \\
    \ruby{疲}{つか}れやすい & い & easily fatigued & \\
    % & & & \\
\bottomrule
\end{tabular}%
}
\captionof{table}{Adjectives: personalities.}
\label{tbl:appendix-vocab-adjectives-personalities}
\end{center}


\subsubsection{Knowledge, truth and reality}
\begin{center}
\centering
\resizebox{\linewidth}{!}{%
% Help: \multicolumn{2}{c}{}, \multirow{2}{*}{}, cmidrule(l){3-5}
\begin{tabular}{@{}lcll@{}}
    \toprule
    \textbf{Descriptor} & \textbf{Cat.} & \textbf{Meaning} & \textbf{Notes} \\
    \toprule
    \ruby{本当}{ほう|とう} & な & real/true/genuine/authentic & \href{https://ja.hinative.com/questions/21280744}{[HN]} \\
    \ruby{正常}{せい|じょう} & な & normal & \\
    まじ/マジ & な & serious/not joking & abbreviation \\
    \midrule
    おかしい & い & laughable/ridiculous/strange/weird/suspicious & \\
    \ruby{異常}{い|じょう} & な & abnormal/strange & \\
    \ruby{不思議}{ふ|し|ぎ} & な & strange/mysterious & \\
    \ruby{信}{しん}じられない & い & unbelievable/incredible & \\
    \midrule
    \midrule
    \ruby{普通}{ふ|つう} & な & normal/ordinary/regular/usual/common & also an adverb \\
    \midrule
    tokubetsu & & & \\
    \midrule
    \midrule
    \ruby{正直}{しょう|じき} & な & honest/frank/candid & also an adverb \\
    \ruby{平等}{びょう|どう} & な & equal/impartial & also a noun \\
    \midrule
    untruthful & & & \\
    unequal & & & \\
    \midrule
    \midrule
    \ruby{既定}{き|てい} & な & fixed/decided/established & \\
    \midrule
    \ruby{未定}{み|てい} & な & not yet fixed/undecided/pending & \\
    % & & & \\
\bottomrule
\end{tabular}%
}
\captionof{table}{Adjectives: knowledge, truth and reality.}
\label{tbl:appendix-vocab-adjectives-knowledge-truth-and-reality}
\end{center}


\subsubsection{Courtesy}
\begin{center}
\centering
\resizebox{\linewidth}{!}{%
% Help: \multicolumn{2}{c}{}, \multirow{2}{*}{}, cmidrule(l){3-5}
\begin{tabular}{@{}lcll@{}}
    \toprule
    \textbf{Descriptor} & \textbf{Cat.} & \textbf{Meaning} & \textbf{Notes} \\
    \toprule
    teinei & & & \\
    \midrule
    \ruby{失礼}{しつ|れい} & な & discourteous/impolite & also a noun \\
    \ruby{無礼}{ぶ|れい} & な & rude/discourteous/insolent (stronger) & also a noun \\
    \midrule
    \midrule
    early & & & \\
    \midrule
    \ruby{遅}{おそ}い & い & slow/late (in the day)/late (behind time) & \\
    % & & & \\
\bottomrule
\end{tabular}%
}
\captionof{table}{Adjectives: courtesy.}
\label{tbl:appendix-vocab-adjectives-courtesy}
\end{center}


\subsubsection{Taste and texture}
\begin{center}
\centering
\resizebox{\linewidth}{!}{%
% Help: \multicolumn{2}{c}{}, \multirow{2}{*}{}, cmidrule(l){3-5}
\begin{tabular}{@{}lcll@{}}
    \toprule
    \textbf{Descriptor} & \textbf{Cat.} & \textbf{Meaning} & \textbf{Notes} \\
    \toprule
    おいしい & い & good-tasting/delicious/tasty & (\ruby[g]{美味}{おい}しい) \\
    うまい & い & delicious & (\ruby[g]{美味}{うま}い/\ruby{旨}{うま}い \href{https://business-textbooks.com/umai/}{[SKJnKKS]}) \\
    うめぇ & expression & delicious/skilled/good & colloquial \\
    \midrule
    \midrule
    ふわふわ & な & soft/fluffy/spongy & onomatopoeic; also an adverb \\
    % & & & \\
\bottomrule
\end{tabular}%
}
\captionof{table}{Adjectives: taste and texture.}
\label{tbl:appendix-vocab-adjectives-taste-and-texture}
\end{center}


\subsubsection{Amounts and sizes}
\begin{center}
\centering
\resizebox{\linewidth}{!}{%
% Help: \multicolumn{2}{c}{}, \multirow{2}{*}{}, cmidrule(l){3-5}
\begin{tabular}{@{}lcll@{}}
    \toprule
    \textbf{Descriptor} & \textbf{Cat.} & \textbf{Meaning} & \textbf{Notes} \\
    \toprule
    \ruby{大}{おお}きい & い & big/large/great & \\
    すごい & い & vast (in numbers)/to a great extent & (\ruby{凄}{すご}い) \\
    \midrule
    \ruby{小}{ちい}さい & い & small/little/tiny & \\
    \midrule
    \midrule
    \ruby{高}{たか}い & い & high/tall; expensive & \\
    \ruby{低}{ひく}い & い & low/short & \\
    \ruby{安}{やす}い & い & cheap & \\
    \midrule
    \midrule
    \ruby{大切}{たい|せつ} & な & important/significant; precious/cherished/beloved & also an adverb \\
    \midrule
    \midrule
    \ruby{大量}{たい|りょう} & な & massive quantity & \href{https://ja.hinative.com/questions/15390763}{[HN]} \\
    \ruby{多}{おお}い & い & many/large quantity of (esp.\ countable); frequent & [GMN] \\
    \ruby{多量}{た|りょう} & な & much/large amount of (esp.\ uncountable) & \href{https://ja.hinative.com/questions/15390763}{[HN]}, [GMN], \href{https://dictionary.goo.ne.jp/thsrs/14242/meaning/m0u/\%E3\%81\%9F\%E3\%81\%8F\%E3\%81\%95\%E3\%82\%93/}{[goo]} \\
    たくさん & な & a lot/lots/plenty/much/a great deal; enough/too much & (\ruby{沢山}{たく|さん}); also an adverb; \href{https://dictionary.goo.ne.jp/thsrs/14242/meaning/m0u/\%E3\%81\%9F\%E3\%81\%8F\%E3\%81\%95\%E3\%82\%93/}{[goo]} \\
    いっぱい & な & full/filled/overflowing & (\ruby{一杯}{いっ|ぱい}); also a noun and adverb; \href{https://dictionary.goo.ne.jp/thsrs/14242/meaning/m0u/\%E3\%81\%9F\%E3\%81\%8F\%E3\%81\%95\%E3\%82\%93/}{[goo]} \\
    \midrule
    \ruby{少量}{しょう|りょう} & な & small quantity & \\
    \ruby{少}{すく}ない & い & few/a little/scarce/insufficient; seldom & \\
    \midrule
    \midrule
    near/far & & & \\
    \midrule
    \midrule
    \ruby{久}{ひさ}しい & い & long (time that has passed)/old (story) & \\
    \ruby{久}{ひさ}しぶり & な & long time (since the last time) & \\
    \midrule
    \midrule
    \ruby{深刻}{しん|こく} & な & serious/severe/grave (of a crisis) & \\
    \ruby{必要}{ひつ|よう} & な & essential/necessary & also a noun \\
    % & & & \\
\bottomrule
\end{tabular}%
}
\captionof{table}{Adjectives: amounts and sizes.}
\label{tbl:appendix-vocab-adjectives-amounts-and-sizes}
\end{center}

\end{document}
