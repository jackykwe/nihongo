\documentclass[../nihongo-gakushuu-kyouzai.tex]{subfiles}
\begin{document}
\appendix
\setcounter{section}{2}
\setcounter{subsection}{3}

\subsection{\ruby{動詞}{どう|し} (verbs)} \label{appendix:verbs}

\subsubsection{Physical}
% Help: \SetCell[r=2,c=2]{c,m} <content>, \cmidrule[l]{3-4}
% Help: colspec: X[ratio, horizontal alignment] columns grow to fit width=\linewidth
%                  negative ratios: shrink to fit content and may not grow to full ratio
% Help: colspec: l/c/r columns do not grow
\longtabse[0.4]  % scale factor
{Verbs: physical.}  % caption
{tbl:appendix-vocab-verbs-physical}  % label
{}  % outer specification options
{
    colspec={X[-4,l]X[-1,c]X[12,l]X[-3,l]X[-4,l]X[-1,c]X[12,l]X[-3,l]},
    rowhead=2,
    % width=\linewidth,  % useful only with X columns
}  % inner specification options
{
    \toprule
    \SetCell[c=4]{c,m} \textbf{Transitive} & & & & \SetCell[c=4]{c,m} \textbf{Intransitive} & & & \\ \cmidrule[r]{1-4} \cmidrule[l]{5-8}
    \textbf{Action} & \textbf{Cat.} & \textbf{Meaning} & \textbf{Notes} & \textbf{Action} & \textbf{Cat.} & \textbf{Meaning} & \textbf{Notes} \\
    \midrule
    \viteq \ruby{触}{さわ}る & う & to touch/feel (intentional) & \href{https://dictionary.goo.ne.jp/thsrs/16231/meaning/m1u/}{[goo]} & \ruby{触}{さわ}る & う & to touch (intentional) & \href{https://dictionary.goo.ne.jp/thsrs/16231/meaning/m1u/}{[goo]} \\
    \viteq OにSを\ruby{触}{ふ}れる & る & to touch O using S (unintentional) \hl{???} & \href{https://dictionary.goo.ne.jp/thsrs/16231/meaning/m1u/}{[goo]} & \ruby{触}{ふ}れる & る & to touch/feel (unintentional); to touch/refer to a subject & \href{https://dictionary.goo.ne.jp/thsrs/16231/meaning/m1u/}{[goo]} \\
    ? & & & & \ruby{接}{せっ}する & E & to be close to/in contact (abstract; information/geographical border) & \href{https://dictionary.goo.ne.jp/thsrs/16231/meaning/m1u/}{[goo]} \\
    % & & & & & & & \\
    \midrule
    \midrule
    \ruby{持}{も}つ & う & to hold (in hand)/take/carry/possess; hold meeting & & - & & & \\
    % & & & & & & & \\
    \midrule
    \vit \ruby{横}{よこ}たえる & る & to lay down & & \ruby{横}{よこ}たわる & う & to lie down/stretch out & \\
    % & & & & & & & \\
    \midrule
    \midrule
    \vit \ruby{当}{あ}てる & る & to hit/put on/hold against & & \ruby{当}{あ}たる & う & to be hit/strike (e.g.\ a target/lottery/by an ailment) & \\
    \vit \ruby{打}{う}つ & う & to hit/strike/beat/punch (strong) & \href{https://ja.hinative.com/questions/3867085}{[HN]} & \ruby{打}{う}たれる & る & to be struck/beaten (strong) & \\
    ぶつ & う & to hit someone; strike/beat (stronger) & (\ruby{打}{ぶ}つ); \href{https://ja.hinative.com/questions/4651279\#answer-39822392}{[HN]} & ? & & & \\
    \vit ぶつける & る & to hit someone's head/crash into & (\ruby{打付}{ぶ|つ}ける); \href{https://ja.hinative.com/questions/18725588}{[HN]} & ぶつかる & う & to bump/crash into (large objects) & \href{https://ja.hinative.com/questions/94519\#answer-237544}{[HN]} \\
    ボッコボコにする & E & to severely beat up & & ? & & & \\
    % & & & & & & & \\
    \midrule
    \vit \ruby{漏}{も}らす & う & to let leak out (water/light/secret/information) & & \ruby{漏}{も}れる & る & to leak out/escape/shine through & \href{https://ja.hinative.com/questions/14216491}{[HN]} \\
    ? & & & & バレる & る & to leaked out/be exposed/be found out (of a secret/lie/improper behaviour) & slightly casual, \href{https://ja.hinative.com/questions/14216491}{[HN]} \\
    <v stem>\ruby{漏}{も}らす & う & to forget to do <v stem> & \suffix & - & & & \\
    % & & & & & & & \\
    \bottomrule
}


\subsubsection{Directions}
% Help: \SetCell[r=2,c=2]{c,m} <content>, \cmidrule[l]{3-4}
% Help: colspec: X[ratio, horizontal alignment] columns grow to fit width=\linewidth
%                  negative ratios: shrink to fit content and may not grow to full ratio
% Help: colspec: l/c/r columns do not grow
\longtabse[0.4]  % scale factor
{Verbs: directions.}  % caption
{tbl:appendix-vocab-verbs-directions}  % label
{}  % outer specification options
{
    colspec={X[-4,l]X[-1,c]X[12,l]X[-3,l]X[-4,l]X[-1,c]X[12,l]X[-3,l]},
    rowhead=2,
    % width=\linewidth,  % useful only with X columns
}  % inner specification options
{
    \toprule
    \SetCell[c=4]{c,m} \textbf{Transitive} & & & & \SetCell[c=4]{c,m} \textbf{Intransitive} & & & \\ \cmidrule[r]{1-4} \cmidrule[l]{5-8}
    \textbf{Action} & \textbf{Cat.} & \textbf{Meaning} & \textbf{Notes} & \textbf{Action} & \textbf{Cat.} & \textbf{Meaning} & \textbf{Notes} \\
    \midrule
    - & & & & \ruby{行}{い}く & う & to go/move through/proceed/reach (information/phase) & \\
    - & & & & <v te>て[い]く & う & {to <v te> and go (spatial\\to <v te> gradually/progressively into the future (temporal)} & (\ruby{行}{い}く); \aux, [formal] \\
    \ruby{持}{も}っていく & う & to take/bring/carry something along & & - & & & \\
    % & & & & & & & \\
    \midrule
    - & & & & \ruby{急行}{きゅう|こう}する & E & to hurry/rush to somewhere & \\
    - & & & & \ruby{緩行}{かん|こう}する & E & to go slowly to somewhere & \\
    % & & & & & & & \\
    \midrule
    \midrule
    - & & & & \ruby{来}{く}る & E & to come/approach/arrive & \\
    - & & & & <v te>てくる & E & {to <v te> and come back (spatial);\\to <v te> up to the present (temporal)} & (\ruby{来}{く}る); \aux, \href{https://japanese.stackexchange.com/a/43678}{[SE1]}, \href{https://japanese.stackexchange.com/q/48132}{[SE2]} \\
    \ruby{持}{も}ってくる & う & to take/bring/carry something over & & - & & & \\
    % & & & & & & & \\
    \midrule
    \midrule
    \vit \ruby{帰}{かえ}す & う & to send back/home (animate) & \href{https://ja.hinative.com/questions/23865042}{[HN]} & \ruby{帰}{かえ}る & \exception{う} & to return/go back/go home (animate) & \href{https://ja.hinative.com/questions/23865042}{[HN]} \\
    \vit \ruby{還}{かえ}す & う & to send back to origin (grander scale) & \href{https://ja.hinative.com/questions/23865042}{[HN]}, \href{https://kurashi-memocho.com/113.html}{[KRS]} & \ruby{還}{かえ}る & \exception{う} & to return back to origin (grander scale) & \href{https://ja.hinative.com/questions/23865042}{[HN]}, \href{https://kurashi-memocho.com/113.html}{[KRS]} \\
    \vit \ruby{返}{かえ}す & う & to return/put something back (inaminate) & \href{https://ja.hinative.com/questions/23865042}{[HN]} & \ruby{返}{かえ}る & \exception{う} & to return/go back (inaminate) & \href{https://ja.hinative.com/questions/23865042}{[HN]} \\
    % & & & & & & & \\
    \midrule
    \midrule
    ? & & & & \ruby{上}{のぼ}る & う & to ascend/go up/go upwards (focus on process) & \href{https://dictionary.goo.ne.jp/word/\%E4\%B8\%8A\%E3\%82\%8B/}{[goo]}\\
    ? & & & & \ruby{登}{のぼ}る & う & to go to a higher place & \href{https://dictionary.goo.ne.jp/word/\%E4\%B8\%8A\%E3\%82\%8B/}{[goo]} \\
    ? & & & & \ruby{昇}{のぼ}る & う & to rise (sun); be promoted in rank & \href{https://dictionary.goo.ne.jp/word/\%E4\%B8\%8A\%E3\%82\%8B/}{[goo]} \\
    \vit \ruby{上}{あ}げる & る & to raise; to do up (one's hair); to fly (kite)/launch (fireworks); to land (a boat); to show someone in/away; to enrol (school) & & \ruby{上}{あ}がる & う & to rise (focus on destination); to enter (from outside); to enrol/promote (school); to come ashore; to lift (rain); to spoil/die (e.g.\ battery) & \href{https://dictionary.goo.ne.jp/thsrs/15966/meaning/m1u/}{[goo]}, \href{https://hugkum.sho.jp/582833}{[HK]} \\
    \vit \ruby{乗}{の}せる & る & to pick up passenger/load goods & & \ruby{乗}{の}る & う & to board/embark & \\
    \ruby{拾}{ひろ}う & う & to pick up (item); book a taxi & & - & & & \\
    - & & & & \ruby{受}{う}かる & う & to pass an exam & \\
    \vit \ruby{起}{お}こす & う & to make upright/wake up & \href{https://dictionary.goo.ne.jp/word/\%E8\%B5\%B7\%E3\%81\%99/}{[goo]} & \ruby{起}{お}きる & る & to rise/wake up; to occur (esp.\ unfavourable incidents) & \\
    % & & & & & & & \\
    \midrule
    ? & & & & \ruby{下}{くだ}る & う & to descend/go down/go downwards (focus on process) & \\
    ? & & & & \ruby{下}{お}りる & る & to go to a lower place & \\
    ? & & & & \ruby{沈}{しず}む & う & to set (sun); be sunken/submerged & \\
    \vit \ruby{下}{お}ろす & う & to take down/bring down/lower & & \ruby{下}{さ}がる & う & to go downwards\emph{/step backwards} (focus on destination) & \href{https://ja.hinative.com/questions/7054838\#answer-36801861}{[HN]} \\
    \vit \ruby{降}{お}ろす & う & to drop off passenger/unload goods; oust & & \ruby{降}{お}りる & る & to alight/disembark & \\
    \vit \ruby{落}{お}とす & う & to drop/let fall; be defeated/rejected; download/copy & & \ruby{落}{お}ちる & る & to fall/drop/collapse/crash (focus on fall); be defeated/fail & \href{https://ja.hinative.com/questions/22550436}{[HN]} \\
    \vit \ruby{倒}{たお}す & う & to knock down/turn on its side/recline & & \ruby{倒}{たお}れる & る & to fall/collapse (focus on ground); to die/fall; to fold/go bankrupt (organisation) & \href{https://ja.hinative.com/questions/22550436}{[HN]} \\
    % & & & & & & & \\
    \midrule
    - & & & & \ruby{降}{ふ}る & う & to fall (rain/snow/ash); to beam down (sunlight/moonlight/luck/misfortune) & \\
    \midrule
    \midrule
    \vit \ruby{入}{い}れる & る & to put in/bring in/let in/insert/install (software) & & \ruby{入}{はい}る & \exception{う} & to enter/arrive/join/get into/fit into & \href{https://ja.hinative.com/questions/15301215}{[HN]} \\
    - & & & & \ruby{立}{た}ち\ruby{入}{い}る & \exception{う} & to trespass/intrude/interfere/pry/delve deeper & \\
    % & & & & & & & \\
    \midrule
    \vit \ruby{出}{だ}す & う & to take out/get out/publish/send (letter)/produce & CN 出~ & \ruby{出}{で}る & る & to exit/leave/come out/flow/appear/answer (phone/door) & incl.\ CN 出~ usages \\
    <v stem>\ruby{出}{だ}す & う & to begin to <v stem>; to <v stem> out (e.g.\ jump out, carry out) & \suffix & - & & & \\
    % This line is occupied, do not delete %%%%%%%%%%%%%%%%%%%%%%%%%%%%%%%%%%%%%%% ^many (but there are uses of 出す that don't have CN equivalents), e.g. 出去、出来、出游、出现、出售、出火、出炉
    \vit \ruby{抜}{ぬ}く & う & to pull out/extract/unplug; omit/skip/pass & & \ruby{抜}{ぬ}ける & る & to come out/fall out/be extracted; be omitted/missing & \\
    \vit \ruby{抜}{ぬ}け\ruby{出}{だ}す & う & to slip out/sneak away/break free & & \ruby{抜}{ぬ}け\ruby{出}{で}る & る & to slip out/steal out & \\
    % & & & & & & & \\
    \midrule
    \midrule
    \vit \ruby{減}{へ}らす & う & to decrease & & \ruby{減}{へ}る & \exception{う} & to decrease & \\
    \vit \ruby{増}{ふ}やす & う & to increase & & \ruby{増}{ふ}える & る & to increase & \\
    % & & & & & & & \\
    \midrule
    \midrule
    \vit \ruby{開}{あ}ける & る & to open (business/general; revealing vacant space) & \href{https://www.tofugu.com/japanese/akeru-aku-hirakeru-hiraku/}{[TFG]} & \ruby{開}{あ}く & う & to open (business/general; revealing vacant space) & \href{https://dictionary.goo.ne.jp/thsrs/16355/meaning/m0u/}{[goo]}, \href{https://ja.hinative.com/question_summaries/350008}{[HN]}, \href{https://www.tofugu.com/japanese/akeru-aku-hirakeru-hiraku/}{[TFG]} \\
    \viteq \ruby{開}{ひら}く & う & to open (focus on non-linear unfolding movement) & \href{https://dictionary.goo.ne.jp/thsrs/16355/meaning/m0u/}{[goo]}, \href{https://ja.hinative.com/question_summaries/350008}{[HN]}, \href{https://www.tofugu.com/japanese/akeru-aku-hirakeru-hiraku/}{[TFG]} & \ruby{開}{ひら}く & う & to open (focus on non-linear unfolding movement) & \href{https://dictionary.goo.ne.jp/thsrs/16355/meaning/m0u/}{[goo]}, \href{https://ja.hinative.com/question_summaries/350008}{[HN]}, \href{https://www.tofugu.com/japanese/akeru-aku-hirakeru-hiraku/}{[TFG]} \\
    \ruby{空}{あ}ける & る & to clear/empty out/make space; leave/be temporarily away & - & & & \\
    - & & & & \ruby{明}{あ}ける & る & to dawn/begin (day/new year); leave schedule free/make time for & \\
    - & & & & \ruby{開}{ひら}ける & る & to unfold (figurative/formal; view/development/path forward) & \href{https://www.tofugu.com/japanese/akeru-aku-hirakeru-hiraku/}{[TFG]} \\
    % & & & & & & & \\
    \midrule
    \viteq \ruby{閉}{と}じる & う & to close/shut (business/general/non-linear folding) & \href{https://dictionary.goo.ne.jp/thsrs/16377/meaning/m1u/}{[goo]}, \href{https://japanese.stackexchange.com/a/32676}{[SE]} & \ruby{閉}{と}じる & う & to close/shut (business/general/non-linear folding) & \href{https://dictionary.goo.ne.jp/thsrs/16377/meaning/m1u/}{[goo]}, \href{https://japanese.stackexchange.com/a/32676}{[SE]} \\
    \vit \ruby{閉}{し}める & る & to close/shut (focus on linear movement) & \href{https://dictionary.goo.ne.jp/thsrs/16377/meaning/m1u/}{[goo]}, \href{https://japanese.stackexchange.com/a/32676}{[SE]} & \ruby{閉}{し}まる & う & to close/shut (focus on linear movement) & \href{https://japanese.stackexchange.com/a/32676}{[SE]}\\
    % & & & & & & & \\
    \midrule
    \midrule
    & & & & \ruby{昇進}{しょう|しん}する & E & to promote/rise in rank (workplace) & \\
    % & & & & & & & \\
    \bottomrule
}


\subsubsection{Clothing}
% \href{https://japanesewithkanako.com/how-to-say-wear-in-japanese/}{???}
% \href{https://www.linguajunkie.com/japanese/to-wear-in-japanese}{???}
% \href{https://www.youtube.com/watch?v=w_uJJyGkzZo}{???}
% \href{https://www.facebook.com/watch/?v=1437814793073414}{???}

% Help: \SetCell[r=2,c=2]{c,m} <content>, \cmidrule[l]{3-4}
% Help: colspec: X[ratio, horizontal alignment] columns grow to fit width=\linewidth
%                  negative ratios: shrink to fit content and may not grow to full ratio
% Help: colspec: l/c/r columns do not grow
\longtabse[0.4]  % scale factor
{Verbs: clothing.}  % caption
{tbl:appendix-vocab-verbs-clothing}  % label
{}  % outer specification options
{
    colspec={X[-4,l]X[-1,c]X[12,l]X[-3,l]X[-4,l]X[-1,c]X[12,l]X[-3,l]},
    rowhead=2,
    % width=\linewidth,  % useful only with X columns
}  % inner specification options
{
    \toprule
    \SetCell[c=4]{c,m} \textbf{Transitive} & & & & \SetCell[c=4]{c,m} \textbf{Intransitive} & & & \\ \cmidrule[r]{1-4} \cmidrule[l]{5-8}
    \textbf{Action} & \textbf{Cat.} & \textbf{Meaning} & \textbf{Notes} & \textbf{Action} & \textbf{Cat.} & \textbf{Meaning} & \textbf{Notes} \\
    \midrule
    つける 付ける & & & & & & & \\
    % & & & & & & & \\
    \bottomrule
}


\subsubsection{Emotions}
% Help: \SetCell[r=2,c=2]{c,m} <content>, \cmidrule[l]{3-4}
% Help: colspec: X[ratio, horizontal alignment] columns grow to fit width=\linewidth
%                  negative ratios: shrink to fit content and may not grow to full ratio
% Help: colspec: l/c/r columns do not grow
\longtabse[0.4]  % scale factor
{Verbs: emotions.}  % caption
{tbl:appendix-vocab-verbs-emotions}  % label
{}  % outer specification options
{
    colspec={X[-4,l]X[-1,c]X[12,l]X[-3,l]X[-4,l]X[-1,c]X[12,l]X[-3,l]},
    rowhead=2,
    % width=\linewidth,  % useful only with X columns
}  % inner specification options
{
    \toprule
    \SetCell[c=4]{c,m} \textbf{Transitive} & & & & \SetCell[c=4]{c,m} \textbf{Intransitive} & & & \\ \cmidrule[r]{1-4} \cmidrule[l]{5-8}
    \textbf{Action} & \textbf{Cat.} & \textbf{Meaning} & \textbf{Notes} & \textbf{Action} & \textbf{Cat.} & \textbf{Meaning} & \textbf{Notes} \\
    \midrule
    - & & & & \ruby{泣}{な}く & う & to cry & \\
    \midrule
    \midrule
    - & & & & キレる & る & to snap/flip/get angry/lose one's temper & \\
    \midrule
    \midrule
    - & & & & びっくりする & E & to be surprised/frightened/startled & \\
    \midrule
    \midrule
    \ruby{大切}{たい|せつ}にする & E & to cherish/treasure & & - & & & \\
    \midrule
    \midrule
    - & & & & ドキドキする & E & to beat fast (heart)/throb/pound/palpitate & also an adverb \\
    % & & & & & & & \\
    \bottomrule
}


\subsubsection{Production}
To use \ruby{続}{つづ}ける as an auxiliary verb, suffix it to the stem of the main verb (e.g.\ 「\ruby{歌}{うた}い\ruby{続}{つづ}ける」 means to continue to sing, and \ruby{歌}{うた}い is the stem of \ruby{歌}{うた}う).
% Help: \SetCell[r=2,c=2]{c,m} <content>, \cmidrule[l]{3-4}
% Help: colspec: X[ratio, horizontal alignment] columns grow to fit width=\linewidth
%                  negative ratios: shrink to fit content and may not grow to full ratio
% Help: colspec: l/c/r columns do not grow
\longtabse[0.4]  % scale factor
{Verbs: production.}  % caption
{tbl:appendix-vocab-verbs-production}  % label
{}  % outer specification options
{
    colspec={X[-4,l]X[-1,c]X[12,l]X[-3,l]X[-4,l]X[-1,c]X[12,l]X[-3,l]},
    rowhead=2,
    % width=\linewidth,  % useful only with X columns
}  % inner specification options
{
    \toprule
    \SetCell[c=4]{c,m} \textbf{Transitive} & & & & \SetCell[c=4]{c,m} \textbf{Intransitive} & & & \\ \cmidrule[r]{1-4} \cmidrule[l]{5-8}
    \textbf{Action} & \textbf{Cat.} & \textbf{Meaning} & \textbf{Notes} & \textbf{Action} & \textbf{Cat.} & \textbf{Meaning} & \textbf{Notes} \\
    \midrule
    \ruby{作}{つく}る & う & to make/prepare (food)/grow (agriculture)/cultivate (people) & \href{https://dictionary.goo.ne.jp/word/\%E4\%BD\%9C\%E3\%82\%8B}{[goo]} & - & & & \\
    \ruby{造}{つく}る & う & to construct (large-scale buildings, manufacturing) & \href{https://dictionary.goo.ne.jp/word/\%E4\%BD\%9C\%E3\%82\%8B}{[goo]} & - & & & \\
    \ruby{創}{つく}る & う & to create/compose (artistic)/start a business & \href{https://dictionary.goo.ne.jp/word/\%E4\%BD\%9C\%E3\%82\%8B}{[goo]} & - & & & \\
    % & & & & & & & \\
    \midrule
    \vit つける & る & to affix/attach/join/apply; to assign; to tail/watch & (\ruby{付}{つ}ける); also in Table~\ref{tbl:appendix-vocab-verbs-clothing} & \ruby{付}{つ}く & つく & to come with/be provided/attached; to stain/scar/dye & \\
    % & & & & & & & \\
    \midrule
    \vit \ruby{加}{くわ}える & る & to add/include/sum (objects/people/concepts); to increase (heat/influence/speed) & & \ruby{加}{くわ}わる & う & to participate/be added to (objects/people/concepts); to increase (heat/influence/speed) & \\
    \ruby{付}{つ}け\ruby{加}{くわ}える & る & to add on/supplement/append & & - & & & \\
    % & & & & & & & \\
    \midrule
    \midrule
    \vit \ruby{始}{はじ}める & る & to start/begin/initiate & & \ruby{始}{はじ}まる & う & to start/begin & \\
    \viteq スタートする & E & to start/begin & & スタートする & E & to start/begin & \\
    - & & & & \ruby{出発}{しゅっ|ぱつ}する & E & to depart/leave/set off & \\
    % & & & & & & & \\
    \midrule
    \vit \ruby{続}{つづ}ける & る & to continue & & \ruby{続}{つづ}く & う & to continue & \\
    <v stem>\ruby{続}{つづ}ける & る & to continue <v stem> & \aux & & & & \\
    \vit \ruby{終}{お}える & る & to finish & & \ruby{終}{お}わる & う & to end/finish & \\
    <v stem>\ruby{終}{お}える & る & to finish <v stem> & \aux & & & & \\
    % & & & & & & & \\
    \midrule
    \midrule
    \ruby{書}{か}く & う & to write & & ? & & & \\
    \ruby{描}{か}く & う & to draw/paint & & ? & & & \\
    \ruby{描}{えが}く & う & to imagine; to depict (abstract concept) & & ? & & & \\
    % & & & & & & & \\
    \midrule
    \midrule
    \ruby{話}{はな}す & う & to talk/speak & & - & & & \\
    - & & & & しゃべる & \exception{う} & to chat/chatter/talk & (\ruby{喋}{しゃべ}る) \\
    % & & & & & & & \\
    \midrule
    \vit \ruby{歌}{うた}う & う & to sing & & \ruby{歌}{うた}う & う & to sing \\
    % & & & & & & & \\
    \midrule
    - & & & & \ruby{踊}{おど}る & う & to dance (a hopping dance) & \\
    - & & & & \ruby{小躍}{こ|おど}りする & E & to dance for joy & \\
    % & & & & & & & \\
    \midrule
    \midrule
    - & & & & \ruby{歩}{ある}く & う & to walk & \\
    - & & & & ぶらぶらする & E & to walk leisurely/aimlessly & onomatopoeic, also an adverb \\
    - & & & & \ruby{走}{はし}る & \exception{う} & to run; drive (vehicle); flash (lightning); wind (road) & \\
    - & & & & \ruby{走}{はし}り\ruby{出}{だ}す & う & to start running/break into a run & \\
    % & & & & & & & \\
    \midrule
    - & & & & \ruby{泳}{およ}ぐ & う & to swim/weave through a crowd & \\
    % & & & & & & & \\
    \midrule
    \midrule
    \vit \ruby{取}{と}る & う & to take (notes/break/time)/obtain/pass/obtain & & \ruby{取}{と}れる & る & to come off (button/handle/lid) & \\
    とる & う & to have/take/consume (a meal/vitamins) & (\ruby{摂}{と}る) & & & & \\
    % & & & & & & & \\
    \midrule
    \vit \ruby{撮}{と}る & う & to take a photograph & & \ruby{撮}{と}れる & る & to be taken (photograph) & \\
    \vit \ruby{録}{と}る & う & to record an audio or video & & \ruby{録}{と}れる & る & to be recorded/caught on tape (audio or video) & \\
    \vit \ruby{捕}{と}る & う & to catch an object/capture an animal & & \ruby{捕}{と}れる & る & to be caught (object)/captured (animal) & \\
    \vit \ruby{採}{と}る & う & to adopt (method/proposal); to collect/gather (flowers/plants) & & \ruby{採}{と}れる & る & to be collected/gathered (flowers/plants) & \\
    \vit \ruby{摘}{つ}む & う & to pick/pluck (flowers); to nip/snip/cut/trim & & つまむ & う & to pick up (with chopsticks/tweezers)/pinch/hold & \\
    \vit \ruby{集}{あつ}める & る & to collect/assemble/gather (collectibles/people/information) & & \ruby{集}{あつ}まる & う & to assemble/gather/collect & \\
    % & & & & & & & \\
    \midrule
    \midrule
    \vit \ruby{切}{き}る & \exception{う} & to cut/open (sealed); turn off (lights/appliance); hang up; (conversation); shuffle/discard (cards/tiles); punch (ticket) & & \ruby{切}{き}れる & る & to break/snap; run out/stop working/expire; be disconnected; be shuffled (cards/tiles); run out (stock); break up & \\
    <v stem>\ruby{切}{き}る & \exception{う} & \hl{to be able to} do <v stem> completely & \aux & & & & \\
    % & & & & & & & \\
    \midrule
    \midrule
    - & & & & \ruby{満開}{まん|かい}する & E & to be in full bloom (esp.\ of cherry blossom) & \\
    % & & & & & & & \\
    \bottomrule
}


\subsubsection{Consumption}
% Help: \SetCell[r=2,c=2]{c,m} <content>, \cmidrule[l]{3-4}
% Help: colspec: X[ratio, horizontal alignment] columns grow to fit width=\linewidth
%                  negative ratios: shrink to fit content and may not grow to full ratio
% Help: colspec: l/c/r columns do not grow
\longtabse[0.4]  % scale factor
{Verbs: consumption.}  % caption
{tbl:appendix-vocab-verbs-consumption}  % label
{}  % outer specification options
{
    colspec={X[-4,l]X[-1,c]X[12,l]X[-3,l]X[-4,l]X[-1,c]X[12,l]X[-3,l]},
    rowhead=2,
    % width=\linewidth,  % useful only with X columns
}  % inner specification options
{
    \toprule
    \SetCell[c=4]{c,m} \textbf{Transitive} & & & & \SetCell[c=4]{c,m} \textbf{Intransitive} & & & \\ \cmidrule[r]{1-4} \cmidrule[l]{5-8}
    \textbf{Action} & \textbf{Cat.} & \textbf{Meaning} & \textbf{Notes} & \textbf{Action} & \textbf{Cat.} & \textbf{Meaning} & \textbf{Notes} \\
    \midrule
    \vit \ruby{見}{み}る & る & to see/observe & & \ruby{見}{み}える & る & to be seen/visible & \\
    \vit \ruby{見}{み}つける & る & to find/discover/detect & & \ruby{見}{み}つかる & う & to be found/discovered & \\
    \vit バラす & う & to expose/disclose/leak a secret & colloquial & バレる & る & to be exposed/found out/leak a secret & \\
    % & & & & & & & \\
    \midrule
    \midrule
    \vit \ruby{聞}{き}く & う & to hear & & \ruby{聞}{き}こえる & る & to be heard/audible & \\
    \ruby{聴}{き}く & う & to listen attentively (music) & & ? & & & \\
    % & & & & & & & \\
    \midrule
    \midrule
    \ruby{食}{た}べる & る & to eat & & - & & & \\
    \ruby{食}{た}べすぎる & る & to overeat & & - & & & \\
    % & & & & & & & \\
    \midrule
    \ruby{飲}{の}む & う & to drink/swallow/take medicine & & - & & & \\
    \ruby{呑}{の}む & う & to gulp/swallow whole & & - & & & \\
    % & & & & & & & \\
    \midrule
    \midrule
    かぶる & う & to put on (head)/be covered with/shoulder responsibility & & - & & & \\
    \ruby{着}{き}る & る & to wear (upper body) & & - & & & \\
    \ruby{履}{は}く & う & to put on (lower body: pants, shoes) & & - & & & \\
    % & & & & & & & \\
    \midrule
    \midrule
    \vit つける & る & to turn on/switch on/light up (appliance/fire) & (\ruby{点}{つ}ける) & つく & う & to come on/ignite/be turned on/lit (appliance/fire) & (\ruby{点}{つ}く) \\
    % & & & & & & & \\
    \midrule
    \vit \ruby{消}{け}す & う & to erase/delete/rid; turn off/extinguish (appliance/fire) & & \ruby{消}{き}える & る & to disappear/vanish; go out/be turned off/extinguished (appliance/fire) & \\
    \ruby{削除}{さく|じょ}する & E & to delete/erase/eliminate & & - & & & \\
    % & & & & & & & \\
    \midrule
    \midrule
    \ruby{遊}{あそ}ぶ & う & to play & & & & & \\
    \ruby{遊}{あそ}ばす & う & to entertain/amuse someone & & - & & & \\
    % & & & & & & & \\
    \midrule
    \midrule
    \ruby{買}{か}う & う & to buy & & - & & & \\
    % & & & & & & & \\
    \midrule
    \midrule
    ロックする & E & to lock & & - & & & \\
    % & & & & & & & \\
    \bottomrule
}


\subsubsection{Interaction}
% Help: \SetCell[r=2,c=2]{c,m} <content>, \cmidrule[l]{3-4}
% Help: colspec: X[ratio, horizontal alignment] columns grow to fit width=\linewidth
%                  negative ratios: shrink to fit content and may not grow to full ratio
% Help: colspec: l/c/r columns do not grow
\longtabse[0.4]  % scale factor
{Verbs: interaction.}  % caption
{tbl:appendix-vocab-verbs-interaction}  % label
{}  % outer specification options
{
    colspec={X[-4,l]X[-1,c]X[12,l]X[-3,l]X[-4,l]X[-1,c]X[12,l]X[-3,l]},
    rowhead=2,
    % width=\linewidth,  % useful only with X columns
}  % inner specification options
{
    \toprule
    \SetCell[c=4]{c,m} \textbf{Transitive} & & & & \SetCell[c=4]{c,m} \textbf{Intransitive} & & & \\ \cmidrule[r]{1-4} \cmidrule[l]{5-8}
    \textbf{Action} & \textbf{Cat.} & \textbf{Meaning} & \textbf{Notes} & \textbf{Action} & \textbf{Cat.} & \textbf{Meaning} & \textbf{Notes} \\
    \midrule
    - & & & & ある & う & to exist/have (inaminate) & (\ruby{有}{あ}る) \\
    - & & & & いる & る & to exist (animate) & (\ruby{居}{い}る) \\
    - & & & & <v te>ている & る & progressive state of action/being & (\ruby{居}{い}る); \aux \\
    % & & & & & & & \\
    \midrule
    - & & & & なる & う & to become/get/attain/reach/turn into/be completed & usu.\ 「〜になる」 \\
    & & & & ご<noun>になる & う & to do <noun> & \aux, honorific \\
    & & & & お<v stem>になる & う & to do <v stem> & \aux, honorific \\
    % & & & & & & & \\
    \midrule
    \ruby{置}{お}く & う & to leave behind/put/place & & - & & & \\
    <v te>ておく & う & to do <v te> in advance in preparation for something & & - & & & \\
    % & & & & & & & \\
    \midrule
    \midrule
    - & & & & いる & る & to need/want & (\ruby{要}{い}る) \\
    % & & & & & & & \\
    \midrule
    \midrule
    \vit やめる & る & to stop/end/quit/cancel/abandon/refrain & (\ruby{止}{や}める) & やむ & う & to stop/cease/be over & (\ruby{止}{や}む)\\
    \ruby{辞}{や}める & る & to resign/retire/quit a job & & - & & & \\
    % & & & & & & & \\
    \midrule
    \midrule
    - & & & & \ruby{会}{あ}う & う & to meet/encounter & \\
    - & & & & \ruby{逢}{あ}う & う & to meet/encounter (close friends/romantic) & \href{https://ja.hinative.com/questions/22148235}{[HN]} \\
    - & & & & \ruby{遭}{あ}う & う & to have an undesired meeting/experience/accident & \\
    % & & & & & & & \\
    \midrule
    \midrule
    \viteq \ruby{待}{ま}つ & う & to wait & & \ruby{待}{ま}つ & う & to wait & \\
    % & & & & & & & \\
    \midrule
    \midrule
    \ruby{一緒}{いっ|しょ}にする & E & to do together/unite/mix & & & & & \\
    % The following line break is necessary to prevent interpretation of the next [

    [<with list>と] \ruby{一緒}{いっ|しょ}になる & う & to rendezvous/join/meet together/get married with & \htc & & & \\  % Hard To Categorise: neither strictly transitive nor strictly intransitive?
    % & & & & & & & \\
    \midrule
    \midrule
    \vit \ruby{感謝}{かん|しゃ}する & E & to thank & & \ruby{感謝}{かん|しゃ}する & E & to be grateful/thankful & \\
    % & & & & & & & \\
    \midrule
    \midrule
    ? & & & & (ご)\ruby{注意}{ちゅう|い}する & E & to pay attention/remind/caution & \\

    & & & & \hl{All WOSURU family} & & & \\
    % & & & & & & & \\
    \midrule
    \midrule
    \ruby{登録}{とう|ろく}する & E & to be entered into a register/to register/subscribe (YouTube) & \href{https://dictionary.goo.ne.jp/word/\%e7\%99\%bb\%e9\%8c\%b2/}{[goo]} & & & & \\
    - & & & & \ruby{入会}{にゅう|かい}する & E & to enrol/admit into a club/society/mailing list & \href{https://ja.hinative.com/questions/22502664}{[HN]} \\
    - & & & & \ruby{退会}{たい|かい}する & E & to withdraw/resign from a club/society/mailing list & \\
    - & & & & \ruby{加入}{か|にゅう}する & E & to join a group/project & \href{https://ja.hinative.com/questions/22502664}{[HN]} \\
    % & & & & & & & \\
    \midrule
    \midrule
    ? & & & & \ruby{電話}{でん|わ}する & E & to call (phone call) & \\
    % & & & & & & & \\
    \midrule
    \midrule
    \viteq \ruby{告白}{こく|はく}する & E & to confess to a crime/wrongdoing & & \ruby{告白}{こく|はく}する & E & to confess one's romantic feelings & \\
    % & & & & & & & \\
    \midrule
    \midrule
    ノートする & E & to note down & & - & & & \\
    % & & & & & & & \\
    \bottomrule
}


\subsubsection{Health}
% Help: \SetCell[r=2,c=2]{c,m} <content>, \cmidrule[l]{3-4}
% Help: colspec: X[ratio, horizontal alignment] columns grow to fit width=\linewidth
%                  negative ratios: shrink to fit content and may not grow to full ratio
% Help: colspec: l/c/r columns do not grow
\longtabse[0.4]  % scale factor
{Verbs: health.}  % caption
{tbl:appendix-vocab-verbs-health}  % label
{}  % outer specification options
{
    colspec={X[-4,l]X[-1,c]X[12,l]X[-3,l]X[-4,l]X[-1,c]X[12,l]X[-3,l]},
    rowhead=2,
    % width=\linewidth,  % useful only with X columns
}  % inner specification options
{
    \toprule
    \SetCell[c=4]{c,m} \textbf{Transitive} & & & & \SetCell[c=4]{c,m} \textbf{Intransitive} & & & \\ \cmidrule[r]{1-4} \cmidrule[l]{5-8}
    \textbf{Action} & \textbf{Cat.} & \textbf{Meaning} & \textbf{Notes} & \textbf{Action} & \textbf{Cat.} & \textbf{Meaning} & \textbf{Notes} \\
    \midrule
    & & & & \ruby{寝}{ね}る & る & to lie down/go to bed & \\
    - & & & & \ruby{睡}{ねむ}る & う & to sleep/rest in peace (euphemism) & \\
    % & & & & & & & \\
    \midrule
    \midrule
    & & & & \ruby{生}{い}きる & る & to live/come to life/make a living & \\
    % & & & & & & & \\
    \midrule
    & & & & \ruby{死}{し}ぬ & う & to die/pass away & \\
    % & & & & & & & \\
    \bottomrule
}


\subsubsection{Ability}
% Help: \SetCell[r=2,c=2]{c,m} <content>, \cmidrule[l]{3-4}
% Help: colspec: X[ratio, horizontal alignment] columns grow to fit width=\linewidth
%                  negative ratios: shrink to fit content and may not grow to full ratio
% Help: colspec: l/c/r columns do not grow
\longtabse[0.4]  % scale factor
{Verbs: ability.}  % caption
{tbl:appendix-vocab-verbs-ability}  % label
{}  % outer specification options
{
    colspec={X[-4,l]X[-1,c]X[12,l]X[-3,l]X[-4,l]X[-1,c]X[12,l]X[-3,l]},
    rowhead=2,
    % width=\linewidth,  % useful only with X columns
}  % inner specification options
{
    \toprule
    \SetCell[c=4]{c,m} \textbf{Transitive} & & & & \SetCell[c=4]{c,m} \textbf{Intransitive} & & & \\ \cmidrule[r]{1-4} \cmidrule[l]{5-8}
    \textbf{Action} & \textbf{Cat.} & \textbf{Meaning} & \textbf{Notes} & \textbf{Action} & \textbf{Cat.} & \textbf{Meaning} & \textbf{Notes} \\
    \midrule
    - & & & & できる & る & to be able to do  & (\ruby{出来}{で|き}る) \\
    % & & & & & & & \\
    \bottomrule
}


\subsubsection{Education and correctness}
\hl{\href{https://japanese.stackexchange.com/questions/56962/standard-mathematical-operations-expressed-in-japanese}{MATH}}

% Help: \SetCell[r=2,c=2]{c,m} <content>, \cmidrule[l]{3-4}
% Help: colspec: X[ratio, horizontal alignment] columns grow to fit width=\linewidth
%                  negative ratios: shrink to fit content and may not grow to full ratio
% Help: colspec: l/c/r columns do not grow
\longtabse[0.4]  % scale factor
{Verbs: education and correctness.}  % caption
{tbl:appendix-vocab-verbs-education-and-correctness}  % label
{}  % outer specification options
{
    colspec={X[-4,l]X[-1,c]X[12,l]X[-3,l]X[-4,l]X[-1,c]X[12,l]X[-3,l]},
    rowhead=2,
    % width=\linewidth,  % useful only with X columns
}  % inner specification options
{
    \toprule
    \SetCell[c=4]{c,m} \textbf{Transitive} & & & & \SetCell[c=4]{c,m} \textbf{Intransitive} & & & \\ \cmidrule[r]{1-4} \cmidrule[l]{5-8}
    \textbf{Action} & \textbf{Cat.} & \textbf{Meaning} & \textbf{Notes} & \textbf{Action} & \textbf{Cat.} & \textbf{Meaning} & \textbf{Notes} \\
    \midrule
    \ruby{訊}{き}く & う & to ask/enquire & & ? & & & \\
    \vit \ruby{質問}{しつ|もん}する & E & to ask a question & & \ruby{質問}{しつ|もん}[を]する & E & to ask a question & \\
    % & & & & & & & \\
    \midrule
    \midrule
    \ruby{教}{おし}える & る & to teach/inform & & - & & & \\
    \ruby{説明}{せつ|めい}する & E & to explain/describe & & - & & & \\
    - & & & & \ruby{答}{こた}える & る & to answer/reply & \\
    % & & & & & & & \\
    \midrule
    \midrule
    & & & & わかる & う & to understand/comprehend & also an interjection \\
    % & & & & & & & \\
    \midrule
    \midrule
    \ruby{習}{なら}う & う & to take lessons/learn/be trained (under a teacher) & & - & & & \\
    \ruby{練習}{れん|しゅう}する & E & to practise/train/drill & & - & & & \\
    % & & & & & & & \\
    \midrule
    \vit \ruby{勉強}{べん|きょう}する & E & to study & & \ruby{勉強}{べん|きょう}する & E & to work hard &  \\
    - & & & & \ruby{無理}{む|り}する & E & to work/try too hard & \\
    \ruby{自習}{じ|しゅう}する & E & to self-study & & - & & & \\
    \midrule
    - & & & & \ruby{頑張}{がん|ば}る & う & to persevere/keep at it/hang on/do one's best & \\
    % & & & & & & & \\
    \midrule
    \midrule
    \vit \ruby{違}{ちが}える & る & to make a mistake & \href{https://ja.hinative.com/questions/10453376}{[HN]} & \ruby{違}{ちが}う & う & to differ/be different; to be wrong/mistaken & \\
    % & & & & & & & \\
    \bottomrule
}


\subsubsection{Knowledge, truth and reality}
% Help: \SetCell[r=2,c=2]{c,m} <content>, \cmidrule[l]{3-4}
% Help: colspec: X[ratio, horizontal alignment] columns grow to fit width=\linewidth
%                  negative ratios: shrink to fit content and may not grow to full ratio
% Help: colspec: l/c/r columns do not grow
\longtabse[0.4]  % scale factor
{Verbs: knowledge, truth and reality.}  % caption
{tbl:appendix-vocab-verbs-knowledge-truth-and-reality}  % label
{}  % outer specification options
{
    colspec={X[-4,l]X[-1,c]X[12,l]X[-3,l]X[-4,l]X[-1,c]X[12,l]X[-3,l]},
    rowhead=2,
    % width=\linewidth,  % useful only with X columns
}  % inner specification options
{
    \toprule
    \SetCell[c=4]{c,m} \textbf{Transitive} & & & & \SetCell[c=4]{c,m} \textbf{Intransitive} & & & \\ \cmidrule[r]{1-4} \cmidrule[l]{5-8}
    \textbf{Action} & \textbf{Cat.} & \textbf{Meaning} & \textbf{Notes} & \textbf{Action} & \textbf{Cat.} & \textbf{Meaning} & \textbf{Notes} \\
    \midrule
    \ruby{思}{おも}う & う & to think/believe/judge/imagine/recall & & - & & & \\
    \ruby{考}{かん}える & る & to consider/think over/reflect on & & - & & & \\
    % & & & & & & & \\
    \midrule
    \midrule
    \ruby{信}{しん}じる & る & to believe/trust/have faith in & & - & & & \\
    \ruby{信用}{しん|よう}する & E & to trust (information/source; past) & \href{https://japanese.stackexchange.com/q/24275}{[SE]} & - & & & \\
    \ruby{信頼}{しん|らい}する & E & to trust (a person/organisation; future) & \href{https://japanese.stackexchange.com/q/24275}{[SE]} & - & & & \\
    % & & & & & & & \\
    \midrule
    \midrule
    - & & & & \ruby{気}{き}づく & う & to notice/realise/sense (internal/somehow nuance) & (\ruby{気付}{き|づ}く); \href{https://ja.hinative.com/questions/23322375}{[HN]} \\
    - & & & & \ruby{気}{き}がつく & う & to notice/realise/sense; regain consciousness & (\ruby{気}{き}が\ruby{付}{つ}く); \href{https://ja.hinative.com/questions/23322375}{[HN]} \\
    % & & & & & & & \\
    \midrule
    \vit \ruby{知}{し}る & \exception{う} & to know/be familiar with & & \ruby{知}{し}れる & る & to be known/understood/discovered & \\
    \ruby{覚}{おぼ}える & る & to remember/memorise/learn by heart & & - & & & \\
    \ruby{思}{おも}い\ruby{出}{だ}す & う & to recall/remember/recollect & & - & & & \\
    % & & & & & & & \\
    \midrule
    \ruby{忘}{わす}れる & る & to forget & & - & & & \\
    % & & & & & & & \\
    \midrule
    \midrule
    \ruby{証明}{しょう|めい}する & E & to prove/testify & & - & & & \\
    % & & & & & & & \\
    \bottomrule
}


\subsubsection{Conflict and resolution}
% Help: \SetCell[r=2,c=2]{c,m} <content>, \cmidrule[l]{3-4}
% Help: colspec: X[ratio, horizontal alignment] columns grow to fit width=\linewidth
%                  negative ratios: shrink to fit content and may not grow to full ratio
% Help: colspec: l/c/r columns do not grow
\longtabse[0.4]  % scale factor
{Verbs: conflict and resolution.}  % caption
{tbl:appendix-vocab-verbs-conflict-and-resolution}  % label
{}  % outer specification options
{
    colspec={X[-4,l]X[-1,c]X[12,l]X[-3,l]X[-4,l]X[-1,c]X[12,l]X[-3,l]},
    rowhead=2,
    % width=\linewidth,  % useful only with X columns
}  % inner specification options
{
    \toprule
    \SetCell[c=4]{c,m} \textbf{Transitive} & & & & \SetCell[c=4]{c,m} \textbf{Intransitive} & & & \\ \cmidrule[r]{1-4} \cmidrule[l]{5-8}
    \textbf{Action} & \textbf{Cat.} & \textbf{Meaning} & \textbf{Notes} & \textbf{Action} & \textbf{Cat.} & \textbf{Meaning} & \textbf{Notes} \\
    \midrule
    - & & & & \ruby{勝}{か}つ & う & to win (personal) & \\
    - & & & & \ruby{勝利}{しょう|り}する & E & to win (larger scale) & \\
    % & & & & & & & \\
    \midrule
    - & & & & \ruby{負}{ま}ける & る & to lose (personal)/succumb to/give in to & \\
    - & & & & \ruby{敗北}{はい|ぼく}する & E & to lose (larger scale) & \\
    % & & & & & & & \\
    \midrule
    \midrule
    \ruby{馬鹿}{ば|か}にする & E & to make fun of/look down on/make light of & & - & & & \\
    % & & & & & & & \\
    \midrule
    \ruby{許}{ゆる}す & う & to allow/approve; forgive/excuse & & - & & & \\
    \ruby{気}{き}にする & E & (negative nuance) to mind/care/worry about something & & \ruby{気}{き}にする & E & (negative nuance) to mind/care/worry about something & \\
    % & & & & & & & \\
    \midrule
    \midrule
    \ruby{殺}{ころ}す & う & to kill; to suppress/destroy/stifle (talent/feelings/yawn/laugh) & & - & & & \\
    % & & & & & & & \\
    \bottomrule
}


\subsubsection{Change}
% Help: \SetCell[r=2,c=2]{c,m} <content>, \cmidrule[l]{3-4}
% Help: colspec: X[ratio, horizontal alignment] columns grow to fit width=\linewidth
%                  negative ratios: shrink to fit content and may not grow to full ratio
% Help: colspec: l/c/r columns do not grow
\longtabse[0.4]  % scale factor
{Verbs: change.}  % caption
{tbl:appendix-vocab-verbs-change}  % label
{}  % outer specification options
{
    colspec={X[-4,l]X[-1,c]X[12,l]X[-3,l]X[-4,l]X[-1,c]X[12,l]X[-3,l]},
    rowhead=2,
    % width=\linewidth,  % useful only with X columns
}  % inner specification options
{
    \toprule
    \SetCell[c=4]{c,m} \textbf{Transitive} & & & & \SetCell[c=4]{c,m} \textbf{Intransitive} & & & \\ \cmidrule[r]{1-4} \cmidrule[l]{5-8}
    \textbf{Action} & \textbf{Cat.} & \textbf{Meaning} & \textbf{Notes} & \textbf{Action} & \textbf{Cat.} & \textbf{Meaning} & \textbf{Notes} \\
    \midrule
    \vit \ruby{変}{か}える & る & to alter/transform/convert/vary/change & & \ruby{変}{か}わる & う & to transform/change; to move to (new place) & \\
    \vit \ruby{換}{か}える & る & to exchange & & \ruby{換}{か}わる & る & to switch/be exchanged/change places & \\
    \vit \ruby{替}{か}える & る & to replace & & \ruby{替}{か}わる & る & to relieve/replace & \\
    \vit \ruby{代}{か}える & る & to substitute & & \ruby{代}{か}わる & る & to substitute/hand over (telephone) & \\
    % & & & & & & & \\
    \midrule
    \viteq \ruby{着替}{き|か}える & る & to change one's clothes & & \ruby{着替}{き|か}える & る & to change one's clothes & \\
    % & & & & & & & \\
    \midrule
    \midrule
    ? & & & & \ruby{声変}{こえ|が}わりする & E & to break voice & \\
    \bottomrule
}

\hl{SUFFIXES AND AUXES}

\end{document}
