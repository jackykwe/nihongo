\documentclass[../nihongo-gakushuu-kyouzai.tex]{subfiles}
\graphicspath{ {../imgs/} }
\begin{document}
\onehalfspacing  % for 振り仮名

\section{Basic grammar}
Some preliminary notes:
\begin{itemize}
    \item The articles ``the'' and ``a'' do not exist in Japanese.
    \item Japanese does not distinguish between a future action and a general statement (e.g. ``I
    will go to the store'' vs. ``I go to the store'').
\end{itemize}

\subsection{Building clauses and sentences}
\emph{Read the main article on \href{https://www.tofugu.com/japanese-grammar/sentences-and-clauses/}{Tofugu}.}

\textbf{Essential clause elements} are necessary parts of any clause. In 日本語, the only essential clause element is the \textbf{predicate}, which is the information about the subject (which is often omitted if clear from context). In contrast, in English, both subject and predicate are essential to every valid clause.

\textbf{Non-essential clause elements} add complexity to the conveyed meaning. There are multiple:
\begin{itemize}
    \item An \textbf{object} is the element of a clause acted upon by a transitive verb\footnote{A transitive verb is a verb that entails one or more transitive objects, for example, ``enjoys'' in Amadeus enjoys music. This contrasts with intransitive verbs, which do not entail transitive objects, for example, ``arose'' in Beatrice arose. \href{https://en.wikipedia.org/wiki/Transitive_verb}{(Source)}}. It is suffixed by を, the object marker particle.

    \emph{A \textbf{direct object} is the person or thing that directly receives the action or effect of the verb. It answers the question ``what'' or ``whom.'' An \textbf{indirect object} answers the question ``for what,'' ``of what,'' ``to what,'' ``for whom,'' ``of whom,'' or ``to whom''and accompanies a direct object.} \href{https://www.britannica.com/dictionary/eb/qa/Direct-and-Indirect-Objects}{(Source)}

    E.g.\ お\textbf{\ruby{寿司}{す|し}}を\ruby{作}{つく}る。
    \item A \textbf{subject} is the entity that controls the verb in a clause. It is suffixed by が, the subject marker particle.

    E.g.\ \textbf{お父さん}がお寿司を作る。
    \item A \textbf{topic}. This is NOT to be confused with a subject. In English. It is suffixed by は, the topic-binding particle. For further details, see Section~\ref{sec:topics-and-subjects}.

    E.g.\ \textbf{\ruby{毎週}{まい|しゅう}\ruby{日曜日}{にち|よう|び}}はお父さんがお寿司を作る。
    \item An \textbf{adverbial} provides information about the circumstances surrounding a sentence, such as the who, what, when, where, why and how. It is suffixed by the に and で particles.

    E.g.\ 毎週日曜日はお父さんが\textbf{\ruby{台所}{だい|どころ}}でお寿司を作る。(台所 is kitchen.)

    \item A sentence final particle is adds nuance to the sentence. の adds an explanatory nuance. よ adds a conversational nuance. ね adds \hl{???}

    E.g.\ 毎週日曜日はお父さんが台所でお寿司を作る\textbf{の}。
\end{itemize}

Japanese is primarily an SOV language, but this can be switched up in order to emphasise information at the beginning of sentences, and de-emphasise information tucked away at the end of sentences. This is possible due to the case-marking particles.

Complex sentences can be formed by linking clauses using \textbf{conjunctive particles} (e.g.\ から (therefore), ので, けど, なら), which makes the connection (e.g.\ the therefore relationship) more explicit, or the \textbf{conjugation} of verbs and い-adjectives (e.g.\ the て form), which de-emphasises the connection. Apart from linking clauses, \textbf{embedding clauses} is also possible, typically used via \textbf{direct quotation} (using optional 「」 marks, or 〜と\ruby{言}{い}いました (said)。), \textbf{indirect quotation} (using 〜と\ruby{思}{おも}う; the と particle acts like a spoken quotation mark) and \textbf{noun modification by adjective-clauses}.

\subsection{Topics vs. subjects: は、が particles} \label{sec:topics-and-subjects}
(Read discussion on \href{https://www.reddit.com/r/LearnJapanese/comments/jt49jj/please_stop_thinking_in_terms_of_\%E3\%81\%AF_vs_\%E3\%81\%8C/}{Reddit}, and Tofugu pages for \href{https://www.tofugu.com/japanese-grammar/sentences-and-clauses/}{clauses and sentences}, \href{https://www.tofugu.com/japanese-grammar/particle-wa/}{は}, \href{https://www.tofugu.com/japanese-grammar/particle-ga/}{が}, and \href{https://www.tofugu.com/japanese/wa-and-ga/}{their differences}.)

\subsubsection{Subject as part-of-speech, topic as meta-concept}
The subject and topic of a sentence is hard to distinguish from an English perspective, since in English, the subject is also the topic by default. However, in Japanese, they are not necessarily the same. Whereas the subject can change from clause to clause, the topic can remain the same across numerous clauses (spanning a phrase, sentence or even paragraph). Note that the topic is not a grammatical part-of-speech! The topic is the theme of discourse.

\subsubsection{は particle: the topic marker}
は roughly translates to ``as for'' in English, but is used far more often than ``as for''.

は \textbf{casts focus on a topic}, and implicitly conveys the idea that other potential topics are cast aside. This strength of this \textbf{implicit contrast} depends on context and usage, specifically how unusual it is to see は used in place of case-marking particles. Here is an illustration of the implicit contrast:
\begin{itemize}
    \item コーヒーを\ruby{飲}{の}みますか? The を particle singles out coffee as the object (which the verb 飲み acts upon) of the question.
    \item コーヒーは飲みますか? The は particle shines the spotlight on coffee, but implies the existence of other drinks. The speaker may be talking about other drinks, then moved the conversation to coffee. The speaker may want to offer coffee but imply the availability of other options.
\end{itemize}

The topic is \ul{always} something already in the listener's consciousness. Therefore, in clauses containing は, emphasis is placed on the new information following the particle は.




Here are its various purposes:
\begin{description}
    \item[は as a topic marker] Suffixed to a noun-phrase which is the intended topic.

    Examples (topic is bolded):
    \begin{itemize}
        \item \textbf{日本}は\ruby{面白}{おも|しろ}い。
        \item \textbf{これ}は何ですか? (なに is known from context; ``this'')
        \item \textbf{\ruby{夏}{なつ}}は日本へ\ruby{行}{い}くつもりです。(つもり: plan/intention; the speaker assumed that the listener doesn't know that they were planning to go to Japan.)

        \textbf{日本へ行くの}はいつですか?(の turns the verb into a noun-phrase; いつ: when)
    \end{itemize}
    \item[は for contrasting two topics] This happens when two (or more) topics are used in the same sentence.

    E.g.\ \textbf{東京}は\ruby{物価}{ぶっ|か}ガ\ruby{高}{たか}いけど、\textbf{\ruby[g]{田舎}{いなか}}は物価が\ruby{安}{やす}い。(物価: cost of living; 田舎: countryside; 安い: cheap).
    \item[は in middle of negative い-adjectives to add implicit/explicit nuance] The added nuance is like that of ``while''/``although''/``that's not the whole story'', and the clause containing はXい is typically followed by a clause adding continuation (though not compulsory, if the nuance's content is implied).

    E.g.\ 日本語は\ruby{難}{むずか}しくない。\\
    E.g.\ 日本語は難しく\textbf{は}ない[けど、ただ時間がかかる]。(かかる: take a resource) \\
    E.g.\ 日本語は難しく\textbf{は}ありませんか、時間がかかります。(formal language)\\
    E.g.\ 日本語は難しく\textbf{は}あるけど、面白い。(ある: exist, opposite of ない)
    \item[は in middle of negative nouns and な-adjectives to add implicit/explicit nuance] Similar effect to the above.

    E.g.\ \ruby{有名}{ゆう|めい}で\textbf{は}ない[か、\ruby{人気}{にん|き}はある]。(有名: famous; 人気: popular/well-liked)



\end{description}



\subsubsection{が particle}
Note that は is NOT a case-marking particle, whose job is to mark the grammatical role an element plays in a sentence. Instead, it binds a sentence to some known context. \emph{``は tells us nothing about the grammatical role the item it marks plays in a sentence, it only establishes that item as the context under which the rest of the sentence holds true. So any word or phrase can become the topic, \ul{regardless} of whatever grammatical role it plays otherwise. So the topic can be the subject, or the topic can be the direct object, or the topic can be the indirect object, or an adverb, etc. \ul{The topic can be anything.}''}




\subsection{Basic grammatical structures}
Conjugation is the change in \textbf{verb} form to fit contexts.

Basic vocabulary:
\begin{itemize}
    \item うん、ううん: informal yes/no
    \item でも: but
\end{itemize}

\subsubsection{Expressing state-of-being}
There is no ``to be'' (is, are, was, were, am) in Japanese.
\begin{itemize}
    \item だ: declarative/assertive present state-of-being, suffixed to nouns and な-adjectives only.

    An ``assertive'' marker. Using it without a communicative particle like よ or ね would sound standoff-ish and abrupt (i.e.\ rude) in spoken Japanese. A sentence ending with だ wouldn't sound like something that's used in an actual conversation, and is typically used inside of indirect quotations (when paraphrasing what someone else said).
    % Source: bottom of https://www.tofugu.com/japanese-grammar/sentences-and-clauses/

    % E.g.\ \ruby{元気}{げん|き}? 元気[だ]。

    % E.g.\ 学生だ。
    \item じゃない: negative present state-of-being, suffixed to nouns and な-adjectives only.

    % E.g.\ \ruby{元気}{げん|き}? 元気じゃない。

    % E.g.\ 学生じゃない。
    \item だった: past state-of-being, suffixed to nouns and な-adjectives only.
    \item じゃなかった: negative past state-of-being, suffixed to nouns and な-adjectives only.
    % E.g.\ \ruby{元気}{げん|き}? 元気[だ]。

    % E.g.\ 学生だ。
\end{itemize}

E.g.\ \ruby{元気}{げん|き}?元気[だ]。 (Greeting and response)

E.g.\ 元気じゃない / 元気だった / 元気じゃなかった。

\subsubsection{Particles} \label{sec:particles}
For は and が, see Section~\ref{sec:topics-and-subjects}.
\begin{itemize}
    \item は: introductory topic particle ``as for/about'', suffixed to the topic you're introducing.

    E.g.\ アリスは学生?うん、学生。

    E.g.\ 今日は試験だ。ジョンは?ジョンは明日。
    \item も: inclusive topic particle (``also''), suffixed to the topic you're including.

    E.g.\ アリスは学生?うん、トムも学生。\\
    E.g.\ アリスは学生?うん、でもトムは学生じゃない。\textred{(it is incorrect to use も here, as it is not inclusive with the positive state-of-being!)}\\
    E.g.\ アリスは学生?ううん、トムも学生じゃない。
    \item が: identifier particle ``the one'', suffixed to a question or the identified. Used when the topic is unknown, and you are either asking for what the topic is, or identifying what the topic is.

    E.g.\ 誰が学生? 私が学生。(Who the one student? Me the one student.)\\
\end{itemize}

\subsubsection{Adjectives}
Adjectives modify a noun that comes after it, and are connected to nouns using the
\begin{itemize}
    \item な-adjectives: act like nouns and use the same particle rules as in Section~\ref{sec:particles}. Use な to directly modify the noun that comes after な, only in the present state-of-being case.

    E.g.\ \ruby{静}{しず}かな人、きれいな人。

    E.g.\ 友達は\ruby{親切}{しん|せつ}。(Friend is kind.) 友達は親切な人だ。(Friend is kind person).

    E.g.\ ボブは何が好き[じゃない/だった/じゃなかった]? (What does/doesn't/did/didn't Bob like?) ボブは魚が好き[だ/じゃない/だった/じゃなかった]。(Bob likes/doesn't like/liked/didn't like fish.)

    E.g.\ 魚が[好き\textbf{な}/好きじゃない/好きだった/好きじゃなかった]人。(Person that likes/does not like/liked/did not like fish.) The entire clause before 人 is an adjective.

    E.g.\ 魚が好きじゃない人は、肉が好きだ。(Person who doesn't like fish likes meat.)\\
    E.g.\ 魚が好きな人は、野菜も好きだ。(Person who likes fish also likes wild vegetables.)

    \item い-adjectives:
\end{itemize}
\end{document}
