\documentclass[../nihongo-gakushuu-kyouzai.tex]{subfiles}
\graphicspath{ {../imgs/} }
\begin{document}
\onehalfspacing  % for 振り仮名

\section{Basic grammar}
Some preliminary notes:
\begin{itemize}
    \item The articles ``the'' and ``a'' do not exist in Japanese.
    \item Japanese does not distinguish between a future action and a general statement (e.g. ``I
    will go to the store'' vs. ``I go to the store'').
\end{itemize}

\subsection{Building clauses and sentences}
\emph{Read the main article on \href{https://www.tofugu.com/japanese-grammar/sentences-and-clauses/}{Tofugu}.}

\textbf{Essential clause elements} are necessary parts of any clause. In 日本語, the only essential clause element is the \textbf{predicate}, which is the information about the subject (which is often omitted if clear from context). \ul{This can be just a verb!} In contrast, in English, both subject and predicate are essential to every valid clause.

\textbf{Non-essential clause elements} add complexity to the conveyed meaning. There are multiple:
\begin{itemize}
    \item An \textbf{object} is the element of a clause acted upon by a transitive verb\footnote{A transitive verb is a verb that entails one or more transitive objects, for example, ``enjoys'' in Amadeus enjoys music. This contrasts with intransitive verbs, which do not entail transitive objects, for example, ``arose'' in Beatrice arose. \href{https://en.wikipedia.org/wiki/Transitive_verb}{(Source)}}. It is suffixed by を, the object marker particle.

    \emph{A \textbf{direct object} is the person or thing that directly receives the action or effect of the verb. It answers the question ``what'' or ``whom.'' An \textbf{indirect object} answers the question ``for what,'' ``of what,'' ``to what,'' ``for whom,'' ``of whom,'' or ``to whom''and accompanies a direct object.} \href{https://www.britannica.com/dictionary/eb/qa/Direct-and-Indirect-Objects}{(Source)}

    E.g.\ お\textbf{\ruby{寿司}{す|し}}を\ruby{作}{つく}る。
    \item A \textbf{subject} is the entity that controls the verb in a clause. It is suffixed by が, the subject marker particle.

    E.g.\ \textbf{お父さん}がお寿司を作る。
    \item A \textbf{topic}. This is NOT to be confused with a subject. In English. It is suffixed by は, the topic-binding particle. For further details, see Section~\ref{sec:topics-and-subjects}.

    E.g.\ \textbf{\ruby{毎週}{まい|しゅう}\ruby{日曜日}{にち|よう|び}}はお父さんがお寿司を作る。
    \item An \textbf{adverbial} provides information about the circumstances surrounding a sentence, such as the who, what, when, where, why and how. It is suffixed by the に and で particles.

    E.g.\ 毎週日曜日はお父さんが\textbf{\ruby{台所}{だい|どころ}}でお寿司を作る。(台所 is kitchen.)

    \item A sentence final particle is adds nuance to the sentence. の adds an explanatory nuance. よ adds a conversational nuance. ね adds \hl{???}

    E.g.\ 毎週日曜日はお父さんが台所でお寿司を作る\textbf{の}。
\end{itemize}

Japanese is primarily an SOV language, but this can be switched up in order to emphasise information at the beginning of sentences, and de-emphasise information tucked away at the end of sentences. This is possible due to the case-marking particles.

Complex sentences can be formed by linking clauses using \textbf{conjunctive particles} (e.g.\ から (therefore), ので, けど, なら), which makes the connection (e.g.\ the therefore relationship) more explicit, or the \textbf{conjugation} of verbs and い-adjectives (e.g.\ the て form), which de-emphasises the connection. Apart from linking clauses, \textbf{embedding clauses} is also possible, typically used via \textbf{direct quotation} (using optional 「」 marks, or 〜と\ruby{言}{い}いました (said)。), \textbf{indirect quotation} (using 〜と\ruby{思}{おも}う; the と particle acts like a spoken quotation mark) and \textbf{noun modification by adjective-clauses}.

\subsection{Topics vs. subjects: は、が particles} \label{sec:topics-and-subjects}
(Read discussion on \href{https://www.reddit.com/r/LearnJapanese/comments/jt49jj/please_stop_thinking_in_terms_of_\%E3\%81\%AF_vs_\%E3\%81\%8C/}{Reddit}, and Tofugu pages for \href{https://www.tofugu.com/japanese-grammar/sentences-and-clauses/}{clauses and sentences}, \href{https://www.tofugu.com/japanese-grammar/particle-wa/}{は}, \href{https://www.tofugu.com/japanese-grammar/particle-ga/}{が}, and \href{https://www.tofugu.com/japanese/wa-and-ga/}{their differences}.)

\subsubsection{Subject as part-of-speech, topic as meta-concept}
The subject and topic of a sentence is hard to distinguish from an English perspective, since in English, the subject is also the topic by default. However, in Japanese, they are not necessarily the same. Whereas the subject can change from clause to clause, the topic can remain the same across numerous clauses (spanning a phrase, sentence or even paragraph). Note that the topic is not a grammatical part-of-speech! The topic is the theme of discourse.

\subsubsection{は particle: the topic marker}
は roughly translates to ``as for'' in English, but is used far more often than ``as for''.

は \textbf{casts \ul{focus}/spotlight on a topic}, and implicitly conveys the idea that other potential topics are cast aside. This strength of this \textbf{implicit contrast} depends on context and usage, specifically how unusual it is to see は used in place of case-marking particles. Here is an illustration of the implicit contrast:
\begin{itemize}
    \item コーヒーを\ruby{飲}{の}みますか? The を particle singles out coffee as the object (which the verb 飲み acts upon) of the question.
    \item コーヒーは飲みますか? The は particle shines the spotlight on coffee, but implies the existence of other drinks. The speaker may be talking about other drinks, then moved the conversation to coffee. The speaker may want to offer coffee but imply the availability of other options.
\end{itemize}

The topic is \ul{always} something already in the listener's consciousness. Therefore, in clauses containing は, emphasis is placed on the new information following the particle は.




Here are its various purposes:
\begin{description}
    \item[は as a topic marker] Suffixed to a noun-phrase which is the intended topic.

    Examples (topic is bolded):
    \begin{itemize}
        \item \textbf{日本}は\ruby{面白}{おも|しろ}い。
        \item \textbf{これ}は何ですか? (なに is known from context; ``this'')
        \item \textbf{\ruby{夏}{なつ}}は日本へ\ruby{行}{い}くつもりです。(つもり: plan/intention; the speaker assumed that the listener doesn't know that they were planning to go to Japan.)

        \textbf{日本へ行くの}はいつですか?(の turns the verb into a noun-phrase; いつ: when)
    \end{itemize}
    \item[は for contrasting two topics] This happens when two (or more) topics are used in the same sentence.

    Examples (topic is bolded):
    \begin{itemize}
        \item E.g.\ \textbf{東京}は\ruby{物価}{ぶっ|か}が\ruby{高}{たか}いけど、\textbf{\ruby[g]{田舎}{いなか}}は物価が\ruby{安}{やす}い。(物価: cost of living; 田舎: countryside; 安い: cheap).
        \item E.g.\ \textbf{お肉}は食べないけど、\textbf{魚}は食べるよ。
    \end{itemize}

    If the effect of contrast isn't required, then が is usually used instead:

    \begin{itemize}
        \item E.g.\ 東京が\ruby{物価}{ぶっ|か}が\ruby{高}{たか}い。
        \item E.g.\ お肉は食べない。
    \end{itemize}

    \item[は in middle of negative い-adjectives to add implicit/explicit nuance/spotlight] The added nuance is like that of ``while''/``although''/``that's not the whole story'', and the clause containing はXい is typically followed by a clause adding continuation (though not compulsory, if the nuance's content is implied).

    E.g.\ 日本語は\ruby{難}{むずか}しくない。\\
    E.g.\ 日本語は難しく\textbf{は}ない[けど、ただ時間がかかる]。(かかる: take a resource) \\
    E.g.\ 日本語は難しく\textbf{は}ありませんか、時間がかかります。(formal language)\\
    E.g.\ 日本語は難しく\textbf{は}あるけど、面白い。(ある: exist, opposite of ない)
    \item[は in middle of negative nouns and な-adjectives to add implicit/explicit nuance/spotlight] Similar effect to the above.

    E.g.\ \ruby{有名}{ゆう|めい}\textbf{では}ない[が、\ruby{人気}{にん|き}はある]。(有名: famous; 人気: popularity)

    E.g.\ 先生\textbf{では}ありませんが、\ruby{説明}{せつ|めい}が\ruby{上手}{じょう|ず}な人です。

    \textred{人気 is a noun that is typically used as 人気がある or 人気がない. 人気じゃない is possible, means the same thing, but is colloquial.}

    These are examples of the \textbf{compound particle} では.
    \item[は incompatible with question words, but commonly used in answers] は cannot be suffixed to question words like 誰, 何 and どこ (except special circumstances). The topic is always something that is already in the listener's consciousness: it isn't unknown! Instead, question words are suffixed by が, を, etc.\ depending on their role in the sentence.

    It is however common to answer such questions は to establish the now known topic.

    E.g.\ 誰がいましたか? (Who was there?)

    E.g.\ だれか\ruby{山田}{やま|だ}さんですか?山田さんはあの人です。
    \item[は suffix in compound particles to add implicit/explicit nuance/spotlight] When forming compound particles, は cannot be suffixed to も, が or を.
    \begin{itemize}
        \item は can either ``replace'' に or become には.

        E.g.\ 日本に\ruby{行}{い}ったことがある。(I've been to Japan; neutral statement.)\\
        E.g.\ 日本\textbf{[に]は}行ったことがある。(I've been to Japan; + implicit comparison with other countries, perhaps I've never been to those.) (は is more casual than には.)

        E.g.\ruby{台所}{だい|どころ}で\ruby{犬}{いぬ}が\ruby{吠}{ほ}えています。(The dog is barking in the kitchen; neutral statement.)\\
        E.g.\ 台所\textbf{では}犬が吠えています。 (The dog is barking in the kitchen; + implicit comparison with other locations.)

        E.g.\ \ruby{弟}{おとうと}と\ruby{映画館}{えい|が|かん}に行きました。(I went to the movies with my younger brother; neutral statement.)\\
        E.g.\ 弟\textbf{とは}に行きました。(I went to the movies with my younger brother; with focus cast on brother, and implicit nuance that we don't know who else I did or didn't go with.)
    \end{itemize}
    \item[は to convey hesitation] Usually prefixed to ね to form the compound particle はね in such scenarios.

    E.g.\ お寿司は[ね]...
    \item[は for changing scenes] The speaker can intentionally break the conversation's storyline or momentum to emphasise something. This is done by repeating and re-shining the spotlight on the topic, even though it's already obvious to the listener.

    E.g.\ \ruby{昨日}{き|のう}は\textbf{お兄ちゃんが}勉強を教えてくれた。まず、英語の勉強を教えてくれて、それから\ruby{国語}{こく|ご}の勉強だった。でも、\ruby{算数}{さん|すう}の\ruby{宿題}{しゅく|だい}をしている\ruby{途中}{と|ちゅう}に、\textbf{お兄ちゃんは}ゲームを始めた。(Yesterday, my big brother helped me study. To start with, he helped me with English, and then with Japanese. But he started playing a game while we were in the middle of doing my math homework; focus/topic shifted from brother teaching to brother playing game.)

    \item[Softening implicit comparison with comma 、(in writing only)] This is commonly done with time phrases such as \ruby{来週}{らい|しゅう}、\ruby{先月}{せん|げつ}、\ruby{今年}{こ|とし}.%\ruby{先週}{せん|しゅう}\ruby{今週}{こん|しゅう}、\ruby{来週}{らい|しゅう}、\ruby{先月}{せん|げつ}、\ruby{今月}{こん|げつ}、\ruby{来月}{らい|げつ}、\ruby{先年}\ruby{今年}{} (this year):

    E.g.\ \ruby{今年}{こ|とし}、日本に行くんです。

    E.g.\ 日本語、面白い。

    E.g.\ 少し、食べました。(I ate a little.)


    % 今月(、本月): This month (is September, etc.)\\
    % 当月: Current month ('s rent, etc.)
\end{description}

Note that は is NOT a case-marking particle, whose job is to mark the grammatical role an element plays in a sentence. Instead, it binds a sentence to some known context. \emph{``は tells us nothing about the grammatical role the item it marks plays in a sentence, it only establishes that item as the context under which the rest of the sentence holds true. So any word or phrase can become the topic, \ul{regardless} of whatever grammatical role it plays otherwise. So the topic can be the subject, or the topic can be the direct object, or the topic can be the indirect object, or an adverb, etc. \ul{The topic can be anything.}''}

Also note that は cannot be used in adjective phrases if there is no contrast involved:
\begin{itemize}
    \item ジェニー[の/が/は]\ruby{落}{お}ちた\ruby{学校}{がっ|こう}に私は\ruby{受}{う}かった。(I was accepted by the school that Jenny failed to get into.)
    \item ジェニー[の/が]落ちた学校にわたしも落ちた。(は is illegal here because there is no contrast!)
\end{itemize}

\subsubsection{が particle}
\textred{Pre-requisite knowledge: Verbs and their basic conjugations, as in Table~\ref{tbl:grammar-conjugation-summary}.}

が is simply the grammatical subject marker particle. が is suffixed to noun phrases. It ``points a finger at'' the subject of a sentence, and implicitly puts emphasis on that subject (making it clear nothing else is the subject).

E.g.\ あの犬[が]\ruby{吠}{ほ}えた。(That dog barked.)

E.g.\ 誰か[が]いる。(Someone's here.)

E.g.\ この\ruby{納豆}{なっ|とう}[が]おいしい。(This natto is delicious.)


The subject is often omitted if clear. In fact, inclusion of the subject in cases where it's usually omitted brings about a kind of emphasis (e.g.\ ``Did our dog do something to you? \emph{Your} dog barked.'').

Sometimes, が itself is omitted instead, especially in spoken context. There is no change to the level of emphasis. This is up to personal preference. \ul{However}, if the clause carries the meaning of singling out a particular member from a crowd, then が cannot be omitted: ジェニー\textred{が}\ruby{犯人}{はん|にん}だ。


\subsection{Prelude: Basic grammatical structures}
Conjugation is the change in \textbf{verb} or \textbf{い-adjective} form to fit contexts.

Basic vocabulary:
\begin{itemize}
    \item うん、ううん: informal yes/no
    \item でも: but
\end{itemize}

\subsection{Expressing state-of-being}
There is no ``to be'' (is, are, was, were, am) in Japanese.
\begin{itemize}
    \item だ: declarative/assertive present state-of-being, suffixed to nouns and な-adjectives only.

    An ``assertive'' marker. Using it without a communicative particle like よ or ね would sound standoff-ish and abrupt (i.e.\ rude) in spoken Japanese. A sentence ending with だ wouldn't sound like something that's used in an actual conversation, and is only natural in select cases:
    \begin{itemize}
        \item inside of indirect quotations (when paraphrasing what someone else said).

        E.g.\ 人間は\ruby{火星}{か|せい}に\ruby{住}{す}めないと\ruby{思}{おも}う。(I don't think that humans can live on Mars.)

        E.g.\ \ruby{近所}{きん|じょ}の人から\ruby{桜}{さくら}が\ruby{満開}{まん|かい}\textbf{だ}と教えてもらった。(A neighbour told me that the cherry blossoms were in full bloom.)
        \item when you find or notice something

        E.g.\ あ、[雪/\ruby{雨}{あめ}/\ruby{虹}{にじ}]\textbf{だ}!

        E.g.\ あ、\ruby{電話}{でん|わ}\textbf{だ}。

        \item when you feel strongly about something

        E.g.\ 有名なアイドルのこと写真\ruby{撮}{と}ったん\textbf{だ}!(I took a picture with a famous idol! 撮る: to take a picture, variant of \ruby{取}{と}る.)

        E.g.\ 本当\textbf{だ}!

        E.g.\ はあ、明日は漢字のテスト\textbf{だ}。
    \end{itemize}
    % Source: bottom of https://www.tofugu.com/japanese-grammar/sentences-and-clauses/
    % Source: https://www.instagram.com/reel/C_VNHEPyjru/

    % E.g.\ \ruby{元気}{げん|き}? 元気[だ]。

    % E.g.\ 学生だ。
    \item じゃない: negative present state-of-being, suffixed to nouns and な-adjectives only.

    % E.g.\ \ruby{元気}{げん|き}? 元気じゃない。

    % E.g.\ 学生じゃない。
    \item だった: past state-of-being, suffixed to nouns and な-adjectives only.
    \item じゃなかった: negative past state-of-being, suffixed to nouns and な-adjectives only.
    % E.g.\ \ruby{元気}{げん|き}? 元気[だ]。

    % E.g.\ 学生だ。
\end{itemize}

E.g.\ \ruby{元気}{げん|き}?元気[だ]。 (Greeting and response)

E.g.\ 元気じゃない / 元気だった / 元気じゃなかった。

\subsection{Starter particles は、も、が} \label{sec:particles}
For は and が, see Section~\ref{sec:topics-and-subjects}.
\begin{itemize}
    \item は: introductory topic particle ``as for/about'', suffixed to the topic you're introducing.

    E.g.\ アリスは学生?うん、学生。

    E.g.\ 今日は試験だ。ジョンは?ジョンは明日。
    \item も: inclusive topic particle (``also''), suffixed to the topic you're including.

    E.g.\ アリスは学生?うん、トムも学生。\\
    E.g.\ アリスは学生?うん、でもトムは学生じゃない。\textred{(it is incorrect to use も here, as it is not inclusive with the positive state-of-being!)}\\
    E.g.\ アリスは学生?ううん、トムも学生じゃない。
    \item が: identifier particle ``the one'', suffixed to a question or the identified. Used when the topic is unknown, and you are either asking for what the topic is, or identifying what the topic is.

    E.g.\ 誰が学生? 私が学生。(Who the one student? Me the one student.)\\
\end{itemize}

\subsection{Adjectives}
Adjectives (adjective phrases) modify a noun that comes after it.
\begin{itemize}
    \item な-adjectives: act like nouns and use the same particle rules as in Section~\ref{sec:particles}. Use な to directly modify the noun that comes after な, only in the present positive case.

    % \begin{itemize}
    %     \item Present positive: <な-adjective>\textbf{な}<noun>
    %     \item Present negative: <な-adjective>じゃない<noun>
    %     \item Past positive: <な-adjective>だった<noun>
    %     \item Past negative: <な-adjective>じゃなかった<noun>
    % \end{itemize}

    E.g.\ \ruby{静}{しず}かな人、きれいな人。

    E.g.\ 友達は\ruby{親切}{しん|せつ}。(Friend is kind.) 友達は親切な人だ。(Friend is kind person).

    E.g.\ ボブは何が好き[じゃない/だった/じゃなかった]? (What does/doesn't/did/didn't Bob like?) ボブは魚が好き[だ/じゃない/だった/じゃなかった]。(Bob likes/doesn't like/liked/didn't like fish.)

    E.g.\ 魚が[好き\textbf{な}/好きじゃない/好きだった/好きじゃなかった]人。(Person that likes/does not like/liked/did not like fish.) The entire clause before 人 is an adjective.

    E.g.\ 魚が好きじゃない人は、肉が好きだ。(Person who doesn't like fish likes meat.)\\
    E.g.\ 魚が好きな人は、野菜も好きだ。(Person who likes fish also likes wild vegetables.)

    \item い-adjectives: always end with 平仮名「い」that is \ul{not} part of a 漢字 word's pronunciation: it must literally be a 平仮名「い」. As examples, 好き、きれい(綺麗)、きらい(嫌い)are all not い-adjectives; they are な-adjectives. \textred{嫌い being a な-adjective has to do with 嫌い being derived from the verb 嫌う. \hl{REVISIT FUTURE}}


    Do not attach the だ suffix to い-adjectives, just as you don't use だ with じゃない。な is not applicable for い-adjectives: there is no need for any attachments.

    % \begin{itemize}
    %     \item Present positive: <い-adjective>い <noun>
    %     \item Present negative: <い-adjective>くない <noun>
    %     \item Past positive: <い-adjective>かった <noun>
    %     \item Past negative: <い-adjective>くなかった <noun>
    % \end{itemize}

    E.g.\ \ruby{値段}{ね|だん}が\ruby{高}{たか}いレストランは\textbf{あまり}好き\textbf{じゃない}。(I \textbf{don't really} like expensive restaurants.)\footnote{\textbf{あまり} is typically used as an intensifier modifier for \emph{negative} adjectives: ``don't really, 25-50\%'. It can also be used as a modifier for positive adjectives, ``excessive'', but in those cases it must be trailing: 食べるあまり、\ruby{悲}{かな}しさのあまり。(MT)}

    Regarding conjugation, one い-adjective family is an exception: いい. Historically, the word for good changed over time from \ruby{良}{よ}い to いい, but conjugations are still take 良い as the base. Same applies to \ruby{格好}{かっ|こ}いい, which takes 格好良い as the base. %Therefore:
    % \begin{itemize}
    %     \item Present positive: いい/かっこいい <noun>
    %     \item Present negative: 良くない/かっこよくない <noun>
    %     \item Past positive: 良かった/かっこよかった <noun>
    %     \item Past negative: 良くなかった/かっこよくなかった <noun>
    % \end{itemize}
\end{itemize}

Table~\ref{tbl:adjective-conjugations} shows the conjugations and usage syntax for な- and い- adjectives.

\begin{table}[h]
\centering
% \resizebox{\textwidth}{!}{%
% Help: \multicolumn{2}{c}{}, \multirow{2}{*}{}, cmidrule(l){3-5}
\begin{tabular}{@{}rrr@{}}
    \toprule
    & \textbf{Positive} & \textbf{Negative} \\ \midrule
    \multirow{4}{*}{\textbf{Present}} & <な-adjective>な<noun> & <な-adjective>じゃない<noun> \\
    & <い-adjective>い <noun> & <い-adjective>くない <noun> \\
    & いい <noun> & よくない <noun> \\
    & かっこいい <noun> & かっこよくない <noun> \\
    \midrule
    \multirow{4}{*}{\textbf{Past}} &  <な-adjective>だった<noun> &  <な-adjective>じゃなかった<noun> \\
    & <い-adjective>かった <noun> & <い-adjective>くなかった <noun> \\
    & よかった <noun> & よくなかった <noun> \\
    & かっこよかった <noun> & かっこよくなかった <noun> \\ \bottomrule
\end{tabular}%
% }
\caption{Adjective conjugations. I purposely use よかった instead of 良かった (and their variants) here to show that you can use eihter; it's up to personal preference. I personally prefer the use of 漢字 (i.e.\ 良かった) because it can be more specific and it's easier to read (in that at a glance, it's easier to derive meaning from 漢字's widely varying word shapes, compared to a sea of 平仮名 characters). \textred{Note however that 良い is usually written in 仮名 alone when used as part of conjugations, as in this table.}}
\label{tbl:adjective-conjugations}
\end{table}

\subsection{Verbs}
Verbs always come at the end of clauses.

Verbs are categorised into three groups, as shown in Table~\ref{tbl:verb-classification}. Be flexible: all these terms are used in different textbooks and dictionaries. Thankfully, they are easy to remember, along with the observation that there are way more Group I verbs than Group II verbs, and there are only two Group III verbs (or up to a dozen, depending on how you count them).\footnote{Mnemonic I'm using: Group I is the most superior; 五段 is superior to 一段; う comes before る in the ローマ字 alphabet.}

\begin{table}[h]
\centering
% \resizebox{\textwidth}{!}{%
% Help: \multicolumn{2}{c}{}, \multirow{2}{*}{}, cmidrule(l){3-5}
\begin{tabular}{@{}clll@{}}
    \toprule
    & \textbf{Group 1} & \textbf{Group 2} & \textbf{Group 3}\\ \midrule
    \multirow{5}{*}{\textbf{Synonyms}} & Pentagrade verb & Monograde verb & Irregular verb \\[0.5em]
    & \ruby{五段動詞}{ご|だん|どう|し} & \ruby{一段動詞}{いち|だん|どう|し} & \ruby{不規則動詞}{ふ|き|そく|どう|し}* \\
    & Godan verb & Ichidan verb & Special class \\
    & Group I verb & Group II verb & Group III verb \\
    & う-verb & る-verb & Exception verb \\ \bottomrule
\end{tabular}%
% }
\caption{Verb classifications. *There isn't a Japanese term for exception verbs; する and 来る are the only members of the  不規則動詞 subclass.}
\label{tbl:verb-classification}
\end{table}

In a nutshell, \textbf{る-verbs is the class of \emph{almost all} \ul{-iru/-eru} verbs.} Exceptions include 帰る, which is a う-verb.

Conjugations for verbs are the most complicated among all parts-of-speech, and are shown in Table~\ref{tbl:grammar-conjugation-summary}.

\subsection{Conjugation summary}
Table~\ref{tbl:grammar-conjugation-summary} shows all the conjugation rules we've seen so far.

\begin{table}[h]
\centering
\resizebox{\textwidth}{!}{%
% Help: \multicolumn{2}{c}{}, \multirow{2}{*}{}, cmidrule(l){3-5}
\begin{tabular}{@{}crrrrl@{}}
    \toprule
    \multirow{2}{*}{\textbf{Category}} & \multicolumn{2}{c}{\textbf{Positive}} & \multicolumn{2}{c}{\textbf{Negative}} & \multirow{2}{*}{\textbf{Examples}} \\ \cmidrule(lr){2-3} \cmidrule(l){4-5}
    & \textbf{Present} & \textbf{Past} & \textbf{Present} & \textbf{Past} \\ \midrule
    Noun/な-adjective & 「」\textgreen{[だ]} & 「」\textgreen{だった} & 「」\textgreen{じゃない} & 「」\textgreen{じゃなかった} & 学生、友達 \\ \midrule
    な-adjective & 「」\textgreen{な} & 「」\textgreen{だった} & 「」\textgreen{じゃない} & 「」\textgreen{じゃなかった} & 綺麗、好き、\textred{嫌い}  \\ \midrule
    \multirow{3}{*}{い-adjective} & 「」い & 「」\textblue{かった} & 「」\textblue{く[は]ない} & 「」\textblue{く[は]なかった} & 高い、難しい、おいしい \\
    & いい & \textred{よ}\textblue{かった} & \textred{よ}\textblue{く[は]ない} & \textred{よ}\textblue{く[は]なかった} & \\
    & かっこいい & かっこ\textred{よ}\textblue{かった} & かっこ\textred{よ}\textblue{く[は]ない} & かっこ\textred{よ}\textblue{く[は]なかった} & \\ \midrule
    \multirow{10}{*}{う-verb} & 「」 す & 「」\textblue{した} & 「」\textblue{さない} & 「」\textblue{さなかった} & 話す \\
    & 「」 く & 「」\textblue{いた} & 「」\textblue{かない} & 「」\textblue{かなかった} & 聞く、書く、描く \\
    & 行 く & 行\textblue{った} & 行\textblue{かない} & 行\textblue{かなかった} & \\
    & 「」 ぐ & 「」\textblue{いだ} & 「」\textblue{がない} & 「」\textblue{がなかった} & 泳ぐ \\
    & 「」 む & 「」\textblue{んだ} & 「」\textblue{まない} & 「」\textblue{まなかった} & 飲む \\
    & 「」 ね & 「」\textblue{んだ} & 「」\textblue{なない} & 「」\textblue{ななかった} & 死ね \\
    & 「」 ぶ & 「」\textblue{んだ} & 「」\textblue{ばない} & 「」\textblue{ばなかった} & 遊ぶ \\
    & 「」 る & 「」\textblue{っだ} & 「」\textblue{らない} & 「」\textblue{らなかった} & ある、\textred{\ruby{帰}{かえ}る} \\
    & 「」 つ & 「」\textblue{っだ} & 「」\textblue{たない} & 「」\textblue{たなかった} & 待つ \\
    & 「」 う & 「」\textblue{っだ} & 「」\textblue{わない} & 「」\textblue{わなかった} & 買う、\ruby{会}{あ}う \\ \midrule
    る-verb & 「」 る &「」 \textblue{た} & 「」 \textblue{ない} & 「」 \textblue{なかった} & いる、食べる、\ruby{出}{で}る、見る* \\ \midrule
    % \multirow{3.85}{*}{Exception verb} & 「」する & 「」した & 「」しない & 「」しなかった & \\[0.5em]
    \multirow{3.5}{*}{Exception verb} & 「」する & 「」\textblue{した} & 「」\textblue{しない} & 「」\textblue{しなかった} & 勉強する \\[0.5em]
    & \ruby{来}{く}る & \textred{\ruby{来}{き}}\textblue{た} & \textred{\ruby{来}{こ}}\textblue{ない} & \textred{\ruby{来}{こ}}\textblue{なかった} & \\
    & くる & \textred{き}\textblue{た} & \textred{こ}\textblue{ない} & \textred{こ}\textblue{なかった} & \\ \bottomrule
\end{tabular}%
}
\caption{Basic conjugation rules, for nouns, adjectives and verbs. [] means optional; 「」 is a dictionary-form placeholer; \textgreen{green means additive} (without modifying the dictionary-form); \textblue{blue means substitutive} (modifies the dictionary-form); \textred{red means exception}.}
\label{tbl:grammar-conjugation-summary}
\end{table}

\end{document}
