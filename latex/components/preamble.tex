%%%%%%%%%%
% LAYOUT %
%%%%%%%%%%
\usepackage[a4paper,top=2cm,bottom=2cm]{geometry}
\usepackage[colorlinks]{hyperref}
% \usepackage[allbordercolors={1 0 0}]{hyperref}
% \usepackage[hidelinks]{hyperref}
\usepackage{lastpage}  % ``n of m'' page numbering
% \usepackage{pdflscape}  % to allow some pages to be landscaped, with \begin{landscape}
\usepackage{parskip}
\usepackage{caption}  % for manual captions within multicol

% Rename sections for \documentclass{ltjarticle}
\renewcommand{\figurename}{Figure} % ltjarticle overrides this to 図 (と) ... <not known yet>
\renewcommand{\tablename}{Table} % ltjarticle overrides this to 表 (ひょう)
\renewcommand{\contentsname}{Contents} % ltjarticle overrides this to 表 (ひょう)


%%%%%%%%%%%%%%%%%%%
% FILE MANAGEMENT %
%%%%%%%%%%%%%%%%%%%
\usepackage{subfiles}
\usepackage{zref-xr} % for references to labels across subfiles


%%%%%%%%
% FONT %
%%%%%%%%
\usepackage[normalem]{ulem} % strikethrough
\usepackage[soul]{lua-ul} % for highlighting via \hl and underline via \ul. LuaLaTeX version of soul courtesy of https://tex.stackexchange.com/a/586053
\usepackage{luacolor}  % LuaLaTeX version of soul courtesy of https://tex.stackexchange.com/a/586053


%%%%%%%%%%%
% FIGURES %
%%%%%%%%%%%
\usepackage{float}


%%%%%%%%%%%
% COLOURS %
%%%%%%%%%%%
% \usepackage{xcolor}
\usepackage{ninecolors}
\NineColors{saturation=high}
\newcommand{\textorange}[1]{\textcolor{brown7}{#1}}
\newcommand{\textred}[1]{\textcolor{red5}{#1}}
% \newcommand{\exception}[1]{\textcolor{red5}{\textbf{\hl{#1}}}}
% \newcommand{\exceptiong}[1]{\textcolor{red5}{\textbf{\phantom{あ}\hl{#1}\phantom{あ}}}}  % g for guard
% \newcommand{\exception}[1]{\textcolor{red5}{\textbf{#1}}}
% \newcommand{\exceptiong}[1]{\textcolor{red5}{\textbf{\phantom{あ}#1\phantom{あ}}}}  % g for guard
% \usepackage{microtype}  % required for \textls, hacky solution for furigana
\newcommand{\textwhite}[1]{\textcolor{white}{#1}}
\newcommand{\textblue}[1]{\textcolor{blue5}{#1}}
\newcommand{\textgreen}[1]{\textcolor{green5}{#1}}
\newcommand{\textpurple}[1]{\textcolor{violet5}{#1}}
\newcommand{\textlightgrey}[1]{\textcolor{lightgray}{#1}}
\newcommand{\textgrey}[1]{\textcolor{gray}{#1}}
\newcommand{\textblack}[1]{\textcolor{black}{#1}}


%%%%%%%%%%
% TABLES %
%%%%%%%%%%
% General advice for tabulars: wrapping them with \resizebox is BAD practice because it results in inconsistent font sizes across tables! https://tex.stackexchange.com/a/600094
\usepackage{tabularray}
\usepackage[math=fp]{datatool-base}  % LaTeX arithmetic courtesy of https://www.dickimaw-books.com/latex/admin/html/arithmetic.shtml
\usepackage{scalefnt}
\UseTblrLibrary{booktabs}
\DefTblrTemplate{contfoot-text}{default}{\emph{continued on next page\dots}}
\DefTblrTemplate{conthead-text}{default}{(continued)}

% When developing, use \longtabsea to see the table (only maximum one will be rendered, any duplicate <\longtabsea>s will show a warning). When done, revert to \longtabse.
% Definition of longtabse (e for enhanced)
\ifcsname currentdatestring\endcsname
    % DOCKER MODE (see Dockerfile)

    % [#1] is the scale factor
    % #2 is caption
    % #3 is the label
    % #4 is the inner specificaiton
    % #5 is the table body
    \newcommand{\longtabse}[5][1]{
        \dtlmul{\newheavyrulewidth}{.08}{#1}  % default is .08em
        \dtlmul{\newlightrulewidth}{.05}{#1}  % default is .05em
        \dtlmul{\newcmidrulewidth}{.03}{#1}  % default is .03em
        \dtlmul{\newbelowrulesep}{.65}{#1}  % default is .65ex
        \dtlmul{\newaboverulesep}{.4}{#1}  % default is .4ex, fine-tuned to .65ex
        \dtlmul{\newdefaultaddspace}{.5}{#1}  % default is .5em
        \dtlmul{\newrulesep}{2}{#1}  % default is 2pt
        \dtlmul{\newstretch}{1}{#1}  % default is 1
        \dtlmul{\newabovesep}{2}{#1}  % default is 2pt
        \dtlmul{\newbelowsep}{2}{#1}  % default is 2pt
        \dtlmul{\newrowsep}{2}{#1}  % default is 2pt
        \dtlmul{\newleftsep}{6}{#1}  % default is 6pt
        \dtlmul{\newrightsep}{6}{#1}  % default is 6pt
        \dtlmul{\newcolsep}{6}{#1}  % default is 6pt
        \dtlmul{\newinversescale}{1}{#1}  % helper value for \adjustbox's scale factor
        \dtlmul{\newcelltoppadding}{1.55}{#1}  % helper value for \adjustbox's margin
        \dtlmul{\newcellbottompadding}{1.4}{#1}  % helper value for \adjustbox's margin

        \heavyrulewidth=\newheavyrulewidth em
        \lightrulewidth=\newlightrulewidth em
        \cmidrulewidth=\newcmidrulewidth em
        \belowrulesep=\newbelowrulesep ex
        \aboverulesep=\newaboverulesep ex
        \defaultaddspace=\newdefaultaddspace em
        \begin{longtabs}[
            caption={#2},
            label={#3},
            expand=\viteq\vit,
        ]{
            rows={valign=m},
            cells={font=\scalefont{\newinversescale}},
            vspan=even,
            rulesep=\newrulesep pt,
            stretch=\newstretch,
            abovesep=\newabovesep pt,
            belowsep=\newbelowsep pt,
            rowsep=\newrowsep pt,
            leftsep=\newleftsep pt,
            rightsep=\newrightsep pt,
            colsep=\newcolsep pt,
            #4,
        }
            #5
        \end{longtabs}
    }
\else
    % LOCAL MODE
    \newcommand{\longtabse}[5][1]{
        \begin{center}
            \captionof{table}{#2}
            \label{#3}
            \vspace{1em}
            \textlightgrey{[DEVELOP MODE] \textbf{COMPILATION SKIPPED} (use \texttt{\textbackslash{}longtabsea} to show)}
            \vspace{1em}
        \end{center}
    }
\fi
% Definition of longtabsea (e for enhanced; a for active) is identical to the above
\ifcsname currentdatestring\endcsname
    % DOCKER MODE (see Dockerfile)
    \newcommand{\longtabsea}[5][1]{
        \dtlmul{\newheavyrulewidth}{.08}{#1}  % default is .08em
        \dtlmul{\newlightrulewidth}{.05}{#1}  % default is .05em
        \dtlmul{\newcmidrulewidth}{.03}{#1}  % default is .03em
        \dtlmul{\newbelowrulesep}{.65}{#1}  % default is .65ex
        \dtlmul{\newaboverulesep}{.4}{#1}  % default is .4ex, fine-tuned to .65ex
        \dtlmul{\newdefaultaddspace}{.5}{#1}  % default is .5em
        \dtlmul{\newrulesep}{2}{#1}  % default is 2pt
        \dtlmul{\newstretch}{1}{#1}  % default is 1
        \dtlmul{\newabovesep}{2}{#1}  % default is 2pt
        \dtlmul{\newbelowsep}{2}{#1}  % default is 2pt
        \dtlmul{\newrowsep}{2}{#1}  % default is 2pt
        \dtlmul{\newleftsep}{6}{#1}  % default is 6pt
        \dtlmul{\newrightsep}{6}{#1}  % default is 6pt
        \dtlmul{\newcolsep}{6}{#1}  % default is 6pt
        \dtlmul{\newinversescale}{1}{#1}  % helper value for \adjustbox's scale factor
        \dtlmul{\newcelltoppadding}{1.55}{#1}  % helper value for \adjustbox's margin
        \dtlmul{\newcellbottompadding}{1.4}{#1}  % helper value for \adjustbox's margin

        \heavyrulewidth=\newheavyrulewidth em
        \lightrulewidth=\newlightrulewidth em
        \cmidrulewidth=\newcmidrulewidth em
        \belowrulesep=\newbelowrulesep ex
        \aboverulesep=\newaboverulesep ex
        \defaultaddspace=\newdefaultaddspace em
        \begin{longtabs}[
            caption={#2},
            label={#3},
            expand=\viteq\vit,
        ]{
            rows={valign=m},
            cells={font=\scalefont{\newinversescale}},
            vspan=even,
            rulesep=\newrulesep pt,
            stretch=\newstretch,
            abovesep=\newabovesep pt,
            belowsep=\newbelowsep pt,
            rowsep=\newrowsep pt,
            leftsep=\newleftsep pt,
            rightsep=\newrightsep pt,
            colsep=\newcolsep pt,
            #4,
        }
            #5
        \end{longtabs}
    }
\else
    % LOCAL MODE
    \newcommand{\longtabsea}[5][1]{
        \dtlmul{\newheavyrulewidth}{.08}{#1}  % default is .08em
        \dtlmul{\newlightrulewidth}{.05}{#1}  % default is .05em
        \dtlmul{\newcmidrulewidth}{.03}{#1}  % default is .03em
        \dtlmul{\newbelowrulesep}{.65}{#1}  % default is .65ex
        \dtlmul{\newaboverulesep}{.4}{#1}  % default is .4ex, fine-tuned to .65ex
        \dtlmul{\newdefaultaddspace}{.5}{#1}  % default is .5em
        \dtlmul{\newrulesep}{2}{#1}  % default is 2pt
        \dtlmul{\newstretch}{1}{#1}  % default is 1
        \dtlmul{\newabovesep}{2}{#1}  % default is 2pt
        \dtlmul{\newbelowsep}{2}{#1}  % default is 2pt
        \dtlmul{\newrowsep}{2}{#1}  % default is 2pt
        \dtlmul{\newleftsep}{6}{#1}  % default is 6pt
        \dtlmul{\newrightsep}{6}{#1}  % default is 6pt
        \dtlmul{\newcolsep}{6}{#1}  % default is 6pt
        \dtlmul{\newinversescale}{1}{#1}  % helper value for \adjustbox's scale factor
        \dtlmul{\newcelltoppadding}{1.55}{#1}  % helper value for \adjustbox's margin
        \dtlmul{\newcellbottompadding}{1.4}{#1}  % helper value for \adjustbox's margin

        \heavyrulewidth=\newheavyrulewidth em
        \lightrulewidth=\newlightrulewidth em
        \cmidrulewidth=\newcmidrulewidth em
        \belowrulesep=\newbelowrulesep ex
        \aboverulesep=\newaboverulesep ex
        \defaultaddspace=\newdefaultaddspace em
        \begin{longtabs}[
            caption={#2},
            label={#3},
            expand=\viteq\vit,
        ]{
            rows={valign=m},
            cells={font=\scalefont{\newinversescale}},
            vspan=even,
            rulesep=\newrulesep pt,
            stretch=\newstretch,
            abovesep=\newabovesep pt,
            belowsep=\newbelowsep pt,
            rowsep=\newrowsep pt,
            leftsep=\newleftsep pt,
            rightsep=\newrightsep pt,
            colsep=\newcolsep pt,
            #4,
        }
            #5
        \end{longtabs}
        \renewcommand{\longtabsea}[5][1]{
            \begin{center}
                \captionof{table}{#2}
                \label{#3}
                \vspace{1em}
                \textlightgrey{[DEVELOP MODE]} \textred{\textbf{DUPLICATE \texttt{\textbackslash{}longtabsea}}}
                \vspace{1em}
            \end{center}
        }
    }
\fi

\newcommand{\multirc}[2]{\SetCell[r=#1\relax,c=#2\relax]{c,m}}
\newcommand{\aux}{\textsc{aux}}
\newcommand{\prefix}{\textsc{prefix}}
\newcommand{\suffix}{\textsc{suffix}}
\newcommand{\conjunction}{\textsc{conjunction}}
\newcommand{\htc}{\textsc{\textbf{HTC}}}
\newcommand{\viteq}{\SetRow{cyan!10}}
\newcommand{\vit}{\SetRow{gray!10}}
\newcommand{\exception}[1]{\textred{\textbf{\hl{#1}}}}


%%%%%%%%%%%%%%%%%%%%%%%%%%
% BASIC LATEX STRUCTURES %
%%%%%%%%%%%%%%%%%%%%%%%%%%
\usepackage{enumitem} % For custom enumerate
% Description labels/references coutesy of https://tex.stackexchange.com/a/1248
\makeatletter
\let\orgdescriptionlabel\descriptionlabel
\renewcommand*{\descriptionlabel}[1]{%
    \let\orglabel\label
    \let\label\@gobble
    \phantomsection
    \edef\@currentlabel{#1\unskip}%
    %\edef\@currentlabelname{#1}%
    \let\label\orglabel
    \orgdescriptionlabel{#1}%
}
\makeatother


%%%%%%%%%%%%%%%%%
% MISCELLANEOUS %
%%%%%%%%%%%%%%%%%
\usepackage{pxrubrica}  % Support for 振り仮名
\rubysetup{j}  % Jukugo ruby by default, use | to separate word boundaries if available
\usepackage{siunitx}
\usepackage{nicefrac}
\usepackage{amssymb}  % for \lessapprox
\usepackage{cancel}
\usepackage{centernot}
\ifcsname currentdatestring\endcsname  % if \currentdatestring is defined (in Docker mode; passed from CLI: see Dockerfile)
    \usepackage[regular]{newcomputermodern}  % required for scalefnt to work properly and scale character horizontal width; Latin Modern isn't good enough, need New Computer Modern.
\fi


%%%%%%%%%%%
% HEADING %
%%%%%%%%%%%
\usepackage{fancyhdr}
\pagestyle{fancy}
\fancyhf{}
\lhead{\textbf{ジャッキー・カン}\\kung.jwe@gmail.com}
\rhead{\textbf{日本語\ruby{学習}{がく|しゅう}\ruby{教材}{きょう|ざい}}}
% \rhead{\textbf{日本語\ruby{学習}{ガク|シュウ}\ruby{教材}{キョウ|ザイ}}}
% Date formatting courtesy of https://tex.stackexchange.com/a/638289
\usepackage{datetime2}
\DTMnewstyle{myformat}{
    \renewcommand*\DTMdisplay[9]{
        \DTMtwodigits{##1}\DTMtwodigits{##2}\DTMtwodigits{##3} \DTMtwodigits{##5}\DTMtwodigits{##6}\DTMtwodigits{##7}
    }
}{}{}{}
\ifcsname currentdatestring\endcsname
    % DOCKER MODE (see Dockerfile)
    \chead{\textlightgrey{\currentdatestring}}
\else
    % LOCAL MODE
    \chead{\textlightgrey{\DTMsetstyle{myformat} [DEVELOP MODE] \DTMnow}}
\fi
\setlength{\headheight}{28pt}  % Adjust according to compilation warnings
\cfoot{Page \thepage\ of \pageref{LastPage}}
\renewcommand{\headrulewidth}{0.5pt}
\renewcommand{\footrulewidth}{0.5pt}
