\documentclass[../nihongo-gakushuu-kyouzai-grammar.tex]{subfiles}
\begin{document}
\appendix
\setcounter{section}{0}
\section{Grammar mega summary}

\subsection{Conjugation rules summary} \label{appendix:conjugation-rules-summary}
\subsubsection{Nouns}

% Help: \SetCell[r=2,c=2]{c,m} <content>, \cmidrule[l]{3-4}
% Help: colspec: X[ratio, horizontal alignment] columns grow to fit width=\linewidth
%                  negative ratios: shrink to fit content and may not grow to full ratio
% Help: colspec: l/c/r columns do not grow
\longtabse[0.75]  % scale factor
{Noun conjugation rules.}  % caption
{tbl:appendix-noun-conjugations}  % label
{}  % outer specification options
{
    colspec={X[-1,c]X[-1,c]X[-2,l]X[-2,l]},
    rowhead=1,
    % width=\linewidth,  % useful only with X columns
}  % inner specification options
{
    \toprule
    \textbf{Purpose} & \textbf{Tense} & \textbf{Casual schema} & \textbf{Polite schema} \\
    \midrule
    \SetCell[r=8]{c,m} State-of-being & Present-positive & <noun>[だ] & <noun>です。 \\*
    & Present-negative & <noun>じゃない & {<noun>じゃないです。\\<noun>じゃありません。} \\*
    & Past-positive & <noun>だった & <noun>でした。 \\*
    & Past-negative & <noun>じゃなかった & {<noun>じゃなかったです。\\<noun>じゃありませんでした。} \\*
    & て-positive & <noun>で & - \\*
    & て-negative & <noun>じゃなくて & - \\*
    & Conditional-positive & <noun>であれば & - \\*
    & Conditional-negative & <noun>じゃなければ & - \\
    \bottomrule
}

\textorange{In \textgreen{丁寧語}, 「〜した」indicates the past-tense.}

As alternative polite forms:
\begin{itemize}
    \item 「〜ないです。」can be replaced with「〜ありまえん。」,
    \item 「〜なかったです。」 can be replaced with 「〜ありませんでした。」 .
\end{itemize}

\subsubsection{な-adjectives}

% Help: \SetCell[r=2,c=2]{c,m} <content>, \cmidrule[l]{3-4}
% Help: colspec: X[ratio, horizontal alignment] columns grow to fit width=\linewidth
%                  negative ratios: shrink to fit content and may not grow to full ratio
% Help: colspec: l/c/r columns do not grow
\longtabse[0.75]  % scale factor
{な-adjective conjugation rules.}  % caption
{tbl:appendix-な-adjective-conjugations}  % label
{}  % outer specification options
{
    colspec={X[-1,c]X[-1,c]X[-2,l]X[-2,l]},
    rowhead=1,
    % width=\linewidth,  % useful only with X columns
}  % inner specification options
{
    \toprule
    \textbf{Purpose} & \textbf{Tense} & \textbf{Casual schema} & \textbf{Polite schema} \\
    \midrule
    \SetCell[r=8]{c,m} {State-of-being\\(same as nouns)} & Present-positive & <na-adj>[だ] & <na-adj>です。 \\*
    & Present-negative & <na-adj>じゃない & {<na-adj>じゃないです。\\<na-adj>じゃありません。} \\*
    & Past-positive & <na-adj>だった & <na-adj>でした。 \\*
    & Past-negative    & <na-adj>じゃなかった & {<na-adj>じゃなかったです。\\<na-adj>じゃありませんでした。} \\*
    & て-positive & <noun>で & - \\*
    & て-negative & <noun>じゃなくて & - \\*
    & Conditional-positive & <na-adj>であれば & - \\
    & Conditional-negative & <na-adj>じゃなければ & - \\
    \midrule
    \SetCell[r=4]{c,m} Noun modifier & Present-positive & <na-adj>\textbf{な}<noun> & - \\*
    & Present-negative & <na-adj>じゃない<noun> & - \\*
    & Past-positive & <na-adj>だった<noun> & - \\*
    & Past-negative & <na-adj>じゃなかった<noun> & - \\
    \midrule
    Adverb & - & <na-adj>に & - \\
    \bottomrule
}


\subsubsection{い-adjectives}
All い-adjectives end with 〜い that is \ul{not} part of the 漢字's pronunciation.

% Help: \SetCell[r=2,c=2]{c,m} <content>, \cmidrule[l]{3-4}
% Help: colspec: X[ratio, horizontal alignment] columns grow to fit width=\linewidth
%                  negative ratios: shrink to fit content and may not grow to full ratio
% Help: colspec: l/c/r columns do not grow
\longtabse[0.75]  % scale factor
{い-adjective conjugation rules.}  % caption
{tbl:appendix-い-adjective-conjugations}  % label
{}  % outer specification options
{
    colspec={X[-1,c]X[-1,c]X[-2,l]X[-2,l]},
    rowhead=1,
    % width=\linewidth,  % useful only with X columns
}  % inner specification options
{
    \toprule
    \textbf{Purpose} & \textbf{Tense} & \textbf{Casual schema} & \textbf{Polite schema} \\
    \midrule
    \SetCell[r=8]{c,m} State-of-being & Present-positive & <i-adj root>い & <i-adj root>いです。 \\*
    & Present-negative & <i-adj root>\textbf{く}ない & {<i-adj root>\textbf{く}ないです。\\<i-adj root>\textbf{く}ありません。} \\*
    & Past-positive & <i-adj root>\textbf{か}った & <i-adj root>\textbf{か}ったです。 \\*
    & Past-negative & <i-adj root>\textbf{く}なかった & {<i-adj root>\textbf{く}なかったです。\\<i-adj root>\textbf{く}ありませんでした。} \\*
    & て-positive & <i-adj root>\textbf{く}て & - \\*
    & て-negative & <i-adj root>\textbf{く}なくて & - \\*
    & Conditional-positive & <i-adj root>ければ & - \\
    & Conditional-negative & <i-adj root>\textbf{く}なければ & - \\
    % & & & \\
    \midrule
    \SetCell[r=4]{c,m} Noun modifier & Present-positive & <i-adj root>い<noun> & - \\*
    & Present-negative & <i-adj root>\textbf{く}ない<noun> & - \\*
    & Past-positive & <i-adj root>\textbf{か}った<noun> & - \\*
    & Past-negative & <i-adj root>\textbf{く}なかった<noun> & - \\ \midrule
    Adverb & - & <i-adj root>\textbf{く} & - \\
    % & & & \\
    \bottomrule
}


\color{red}
Exceptions:
\begin{description}
    \item[い-adjectives ending with 「〜いい」] When in any form other than present-positive, the root changes from 「〜い」 to 「〜よ」.
\end{description}
\color{black}


\subsubsection{Verbs}
See Table~\ref{tbl:verb-classification} for a summary of the three categories. In a nutshell, る-verbs is the class of \emph{almost all} \ul{-iru/-eru} verbs; all other verbs are う-verbs. Exception verbs are する and 来る. \textorange{Mnemonic: Group I is the most superior; 五段 is superior to 一段; う comes before る in the 平仮名 alphabet chart.}

% Help: \SetCell[r=2,c=2]{c,m} <content>, \cmidrule[l]{3-4}
% Help: colspec: X[ratio, horizontal alignment] columns grow to fit width=\linewidth
%                  negative ratios: shrink to fit content and may not grow to full ratio
% Help: colspec: l/c/r columns do not grow
\longtabse[0.75]  % scale factor
{Verb conjugation rules.}  % caption
{tbl:appendix-verb-conjugations}  % label
{}  % outer specification options
{
    colspec={X[1,c]X[2,c]X[2,l]X[2,l]X[2,l]X[2,l]},
    rowhead=1,
    % width=\linewidth,  % useful only with X columns
}  % inner specification options
{
    \toprule
    \textbf{Class} & \textbf{Tense} & \SetCell[c=2]{c,m} \textbf{Casual schema} & & \SetCell[c=2]{c,m} \textbf{Polite schema} & \\
    \midrule
    \SetCell[r=34]{c,m} う verb & Present-positive & \SetCell[c=2]{l,m} <$*$-end root>○ & & \SetCell[c=2]{l,m} <$*$-end root>\ruby{○}{〜い}ます。 & \\* \cmidrule[l]{2-6}
    & \SetCell[r=3]{c,m} Negative & \SetCell[c=2]{l,m} <$*\setminus$\{う\}-end root>\ruby{○}{〜あ}\textbf{な}い & & \SetCell[r=3,c=2]{l,m} <$*$-end root>\ruby{○}{〜い}ません。 & \\*
    & & \SetCell[c=2]{l,m} <う-end root>わ\textbf{な}い & & & \\*
    & & \SetCell[c=2]{l,m} \ruby[<g>]{\textbf{な}い}{ある\to} & & & \\* \cmidrule[l]{2-6}
    & \SetCell[r=5]{c,m} Past & \SetCell[c=2]{l,m} <す-end root>した & & \SetCell[r=5,c=2]{l,m} <$*$-end root>\ruby{○}{〜い}ました。 & \\*
    & & \SetCell[c=2]{l,m} <く-end root>いた & & \SetCell[c=2]{l,m} & \\*
    & & \SetCell[c=2]{l,m} <ぐ-end root>いだ & & \SetCell[c=2]{l,m} & \\*
    & & \SetCell[c=2]{l,m} <\{む,ぬ,ぶ\}-end root>んだ & & \SetCell[c=2]{l,m} & \\*
    & & \SetCell[c=2]{l,m} <\{る,つ,う\}-end root>った & & \SetCell[c=2]{l,m} & \\* \cmidrule[l]{2-6}
    & \SetCell[r=3]{c,m} Past-negative & \SetCell[c=2]{l,m} {<$*\setminus$\{う\}-end root>\ruby{○}{〜あ}\textbf{な}かった} & & \SetCell[r=3,c=2]{l,m} <$*$-end root>\ruby{○}{〜い}ませんでした。 & \\*
    & & \SetCell[c=2]{l,m} <う-end root>わ\textbf{な}かった & & \SetCell[c=2]{l,m} & \\*
    & & \SetCell[c=2]{l,m} \ruby[<g>]{\textbf{な}かった}{ある\to} & & \SetCell[c=2]{l,m} & \\* \cmidrule[l]{2-6}
    & (Stem) & \SetCell[c=2]{l,m} - & & \SetCell[c=2]{l,m} <$*$-end root>\ruby{○}{〜い} & \\* \cmidrule[l]{2-6}
    & \SetCell[r=5]{c,m} Te & \SetCell[c=2]{l,m} <す-end root>して & & \SetCell[r=5,c=2]{l,m} - & \\*
    & & \SetCell[c=2]{l,m} <く-end root>いて & & \SetCell[c=2]{l,m} \\*
    & & \SetCell[c=2]{l,m} <ぐ-end root>いで & & \SetCell[c=2]{l,m} \\*
    & & \SetCell[c=2]{l,m} <\{む,ぬ,ぶ\}-end root>んで & & \SetCell[c=2]{l,m} \\*
    & & \SetCell[c=2]{l,m} <\{る,つ,う\}-end root>って & & \SetCell[c=2]{l,m} \\* \cmidrule[l]{2-6}
    & \SetCell[r=3]{c,m} Te-negative & \SetCell[c=2]{l,m} <$*\setminus$\{う\}-end root>\ruby{○}{〜あ}\textbf{な}くて & & \SetCell[r=3,c=2]{l,m} - & \\*
    & & \SetCell[c=2]{l,m} <う-end root>わ\textbf{な}くて & & \SetCell[c=2]{l,m} \\*
    & & \SetCell[c=2]{l,m} \ruby[<g>]{\textbf{な}くて}{ある\to} & & \SetCell[c=2]{l,m} \\* \cmidrule[l]{2-6}
    & {Progressive\\(conjugate 〜いる)} & \SetCell[c=2]{l,m} <v te>いる & & \SetCell[c=2]{l,m} - & \\* \cmidrule[l]{2-6}
    & {Potential\\(conjugate 〜る)} & \SetCell[c=2]{l,m} {<$*$-end root>\ruby{○}{〜え}る\\$\cdots$} & & \SetCell[c=2]{l,m} {<$*$-end root>\ruby{○}{〜え}ます。\\$\cdots$} \\* \cmidrule[l]{2-6}
    & Conditional & \SetCell[c=2]{l,m} <$*$-end root>\ruby{○}{〜え}ば & & \SetCell[c=2]{l,m} - & \\* \cmidrule[l]{2-6}
    & \SetCell[r=3]{c,m} Conditional-negative & \SetCell[c=2]{l,m} <$*\setminus$\{う\}-end root>\ruby{○}{〜あ}なければ & & \SetCell[c=2]{l,m} - & \\*
    & & \SetCell[c=2]{l,m} <\{う\}-end root>わなければ & & \SetCell[c=2]{l,m} - & \\*
    & & \SetCell[c=2]{l,m} \ruby[g]{なければ}{ない\to} & & \SetCell[c=2]{l,m} - & \\* \cmidrule[l]{2-6}
    & {Desiderative\\(conjugate i-adj)} & \SetCell[c=2]{l,m} {<$*$-end root>\ruby{○}{〜い}たい\\$\cdots$} & & \SetCell[c=2]{l,m} - & \\* \cmidrule[l]{2-6}
    & Volitional & \SetCell[c=2]{l,m} <$*$-end root>\ruby{○}{〜お}う & & \SetCell[c=2]{l,m} <$*$-end root>\ruby{○}{〜い}ましょう。 & \\* \cmidrule[l]{2-6}
    & Imperative & \SetCell[c=2]{l,m} <$*$-end root>\ruby{○}{〜え} & & \SetCell[c=2]{l,m} - & \\*
    & Imperative-negative & \SetCell[c=2]{l,m} <$*$-end root>○な & & \SetCell[c=2]{l,m} - & \\* \cmidrule[l]{2-6}
    & {Causative\\(conjugate 〜る\\\textlightgrey{/conjugate 〜す})} & \SetCell[c=2]{l,m} {<$*$-end root>\ruby{○}{〜あ}せる\\$\cdots$\\\textlightgrey{<$*$-end root>\ruby{○}{〜あ}す}\\\textlightgrey{$\cdots$}} & & \SetCell[c=2]{l,m} {<$*$-end root>\ruby{○}{〜あ}せます。\\$\cdots$\\\textlightgrey{<$*$-end root>\ruby{○}{〜あ}します。}\\\textlightgrey{$\cdots$}} & \\* \cmidrule[l]{2-6}
    & {Passive\\(conjugate 〜る)} & \SetCell[c=2]{l,m} {<$*$-end root>\ruby{○}{〜あ}れる\\$\cdots$} & & \SetCell[c=2]{l,m} {<$*$-end root>\ruby{○}{〜あ}れます。\\$\cdots$} & \\* \cmidrule[l]{2-6}
    & {Causative-passive\\(conjugate 〜る)} & \SetCell[c=2]{l,m} {<$*$-end root>\ruby{○}{〜あ}せられる\\$\cdots$\\\textlightgrey{<$*\setminus$\{す\}-end root>\ruby{○}{〜あ}される}\\\textlightgrey{$\cdots$}} & & \SetCell[c=2]{l,m} {<$*$-end root>\ruby{○}{〜あ}せられます。\\$\cdots$\\\textlightgrey{<$*\setminus$\{す\}-end root>\ruby{○}{〜あ}されます。}\\\textlightgrey{$\cdots$}} & \\
    % & & & & & \\
    \midrule
    \SetCell[r=18]{c,m} る verb & Dictionary & \SetCell[c=2]{l,m} {<る-end root>る} & & \SetCell[c=2]{l,m} {<る-end root>ます。} \\*
    & Negative & \SetCell[c=2]{l,m} <る-end root>ない & & \SetCell[c=2]{l,m} <る-end root>ません。 \\*
    & Past & \SetCell[c=2]{l,m} <る-end root>た & & \SetCell[c=2]{l,m} <る-end root>ました。 \\*
    & Past-negative & \SetCell[c=2]{l,m} <る-end root>なかった & & \SetCell[c=2]{l,m} <る-end root>ませんでした。 \\*
    & (Stem) & \SetCell[c=2]{l,m} - & & \SetCell[c=2]{l,m} <る-end root> & \\*
    & Te & \SetCell[c=2]{l,m} <る-end root>て & & \SetCell[c=2]{l,m} - & \\*
    & Te-negative & \SetCell[c=2]{l,m} <る-end root>なくて & & \SetCell[c=2]{l,m} - & \\*
    & {Progressive\\(conjugate 〜いる)} & \SetCell[c=2]{l,m} {<v te>いる\\$\cdots$} & & \SetCell[c=2]{l,m} {<v te>います。\\$\cdots$} \\*
    & {Potential\\(conjugate 〜る)} & \SetCell[c=2]{l,m} {<る-end root>られる\\$\cdots$} & & \SetCell[c=2]{l,m} {<る-end root>られます。\\$\cdots$} \\*
    & Conditional & \SetCell[c=2]{l,m} <る-end root>れば & & \SetCell[c=2]{l,m} - & \\*
    & Conditional-negative & \SetCell[c=2]{l,m} <る-end root>なければ & & \SetCell[c=2]{l,m} - & \\*
    & {Desiderative\\(conjugate i-adj)} & \SetCell[c=2]{l,m} {<る-end root>たい\\$\cdots$} & & \SetCell[c=2]{l,m} - & \\*
    & Volitional & \SetCell[c=2]{l,m} <る-end root>よう & & \SetCell[c=2]{l,m} <る-end root>ましょう。 & \\*
    & Imperative & \SetCell[c=2]{l,m} <る-end root>ろ & & \SetCell[c=2]{l,m} - & \\*
    & Imperative-negative & \SetCell[c=2]{l,m} <る-end root>るな & & \SetCell[c=2]{l,m} - & \\*
    & {Causative\\(conjugate 〜る\\\textlightgrey{/conjugate 〜す})} & \SetCell[c=2]{l,m} {<る-end root>させる\\$\cdots$\\\textlightgrey{<る-end root>さす}\\\textlightgrey{$\cdots$}} & & \SetCell[c=2]{l,m} {<る-end root>させます。\\$\cdots$\\\textlightgrey{<る-end root>さします。}\\\textlightgrey{$\cdots$}} & \\*
    & {Passive\\(conjugate 〜る)} & \SetCell[c=2]{l,m} {<る-end root>られる\\$\cdots$} & & \SetCell[c=2]{l,m} {<る-end root>られます。\\$\cdots$} & \\*
    & {Causative-passive\\(conjugate 〜る)} & \SetCell[c=2]{l,m} {<る-end root>させられる\\$\cdots$} & & \SetCell[c=2]{l,m} {<る-end root>させられます。\\$\cdots$} & \\
    % & & & & & \\
    \midrule
    \SetCell[r=18]{c,m} exception verb & Dictionary & 〜する & \ruby{来}{く}る & 〜します。 & \ruby{来}{き}ます。 \\*
    & Negative & 〜しない & \ruby{来}{こ}ない & 〜しません。 & \ruby{来}{き}ません。 \\*
    & Past & 〜した & \ruby{来}{き}た & 〜しました。 & \ruby{来}{き}ました。 \\*
    & Past-negative & 〜しなかった & \ruby{来}{こ}なかった & 〜しませんでした。 & \ruby{来}{き}ませんでした。 \\*
    & (Stem) & \SetCell[c=2]{l,m} & - & 〜し & \ruby{来}{き} \\*
    & Te & 〜して & \ruby{来}{き}て & - & - \\*
    & Te-negative & 〜しなくて & \ruby{来}{こ}なくて & - & - \\*
    & {Progressive\\(conjugate 〜いる)} & {<v te>いる\\$\cdots$} & {<v te>いる\\$\cdots$} & {<v te>います。\\$\cdots$} & {<v te>います。\\$\cdots$} \\*
    & {Potential\\(conjugate 〜る)} & {〜できる\\$\cdots$} & {\ruby{来}{こ}られる\\$\cdots$} & {〜できます。\\$\cdots$} & {\ruby{来}{こ}られます。\\$\cdots$} \\*
    & Conditional & 〜すれば & \ruby{来}{く}れば & - & - \\*
    & Conditional-negative & 〜しなければ & \ruby{来}{こ}なければ & - & - \\*
    & {Desiderative\\(conjugate i-adj)} & {〜したい\\$\cdots$} & {\ruby{来}{き}たい\\$\cdots$} & - & - \\*
    & Volitional & 〜しよう & \ruby{来}{こ}よう & 〜しましょう。 & \ruby{来}{き}ましょう。 \\*
    & Imperative & 〜しろ & \ruby{来}{こ}い & - & - \\*
    & Imperative-negative & 〜するな & \ruby{来}{く}るな & - & - \\*
    & {Causative\\(conjugate 〜る\\\textlightgrey{/conjugate 〜す})} & {〜させる\\$\cdots$\\\textlightgrey{〜さす}\\\textlightgrey{$\cdots$}} & {\ruby{来}{こ}させる\\$\cdots$\\\textlightgrey{\ruby{来}{こ}さす}\\\textlightgrey{$\cdots$}} & {〜させます。\\$\cdots$\\\textlightgrey{〜さします。}\\\textlightgrey{$\cdots$}} & {\ruby{来}{こ}させます。\\$\cdots$\\\textlightgrey{\ruby{来}{こ}さします。}\\\textlightgrey{$\cdots$}} \\*
    & {Passive\\(conjugate 〜る)} & {〜される\\$\cdots$} & {\ruby{来}{こ}られる\\$\cdots$} & {〜られます。\\$\cdots$} & {\ruby{来}{こ}られます。\\$\cdots$} \\*
    & {Causative-passive\\(conjugate 〜る)} & {〜させられる\\$\cdots$} & {\ruby{来}{こ}させられる\\$\cdots$} & {〜させられます。\\$\cdots$} & {\ruby{来}{こ}させられます。\\$\cdots$} \\
    % & & & & & \\
    \bottomrule
}

\color{orange}
Basic rules/observations better expressed via prose:
\begin{itemize}
    \item All present-negative forms end with 「〜ない」.
    \item The conditional form for all verbs is the same: change the last sound to \ruby{◯}{〜え} and attach ば.
    \item The conditional-negative forms all end in 「〜なければ」, which is obtained from replacing い in 「〜ない」 with ければ. This replacement rule also applies for い-adjectives, replacing the trailing 「〜い」 for 「〜ければ」.
    \item The volitional form for exception verbs kind of follow the rules of る verbs: drop る and replace with よう, but additionally there's also a ``hint of past tense'' in there, explaining the こ sound in \ruby{来}{こ}よう.
    \item The imperative-negative form for all verbs is the same, just attach な to the dictionary form.
\end{itemize}

\color{red}
Exceptions:
\begin{itemize}
    \item \ruby{行}{い}く's past-positive form is 行った, not ``行いた''. Only 行く uses 〜った; all other 〜く verbs still use 〜いた.
    \item ある's present-negative form is ない, not ``あらない''.
    \item くれる's imperative form is くれ, not ``くれろ''.
\end{itemize}

\color{black}

The following table is a condensed version, showing where the rules come from.


% Help: \SetCell[r=2,c=2]{c,m} <content>, \cmidrule[l]{3-4}
% Help: colspec: X[ratio, horizontal alignment] columns grow to fit width=\linewidth
%                  negative ratios: shrink to fit content and may not grow to full ratio
% Help: colspec: l/c/r columns do not grow
\longtabse[0.75]  % scale factor
{Condensed verb conjugation rules. <v negative fragment>/<vnf> refers to <v negative> but dropping the trailing 「い」 character; <v past fragment>/<vpf> refers to <v past> but dropping the trailing 「た/だ」 character.}  % caption
{tbl:appendix-condensed-verb-conjugations}  % label
{}  % outer specification options
{
    colspec={X[1,c]X[2,c]X[2,l]X[2,l]X[2,l]X[2,l]},
    rowhead=1,
    % width=\linewidth,  % useful only with X columns
}  % inner specification options
{
    \toprule
    \textbf{Class} & \textbf{Tense} & \SetCell[c=2]{c,m} \textbf{Casual schema} & & \SetCell[c=2]{c,m} \textbf{Polite schema} & \\
    \midrule
    \SetCell[r=21]{c,m} う verb & Dictionary & \SetCell[c=2]{l,m} <$*$-end root>○ & & \SetCell[c=2]{l,m} <stem>ます。 & \\* \cmidrule[l]{2-6}
    & \SetCell[r=3]{c,m} Negative & \SetCell[c=2]{l,m} <$*\setminus$\{う\}-end root>\ruby{○}{〜あ}\textbf{な}い & & \SetCell[r=3,c=2]{l,m} <stem>ません。 & \\*
    & & \SetCell[c=2]{l,m} <う-end root>わ\textbf{な}い & & & \\*
    & & \SetCell[c=2]{l,m} \ruby[<g>]{\textbf{な}い}{ある\to} & & & \\* \cmidrule[l]{2-6}
    & \SetCell[r=5]{c,m} Past & \SetCell[c=2]{l,m} <す-end root>した & & \SetCell[r=5,c=2]{l,m} <stem>ました。 & \\*
    & & \SetCell[c=2]{l,m} <く-end root>いた & & \SetCell[c=2]{l,m} & \\*
    & & \SetCell[c=2]{l,m} <ぐ-end root>いだ & & \SetCell[c=2]{l,m} & \\*
    & & \SetCell[c=2]{l,m} <\{む,ぬ,ぶ\}-end root>んだ & & \SetCell[c=2]{l,m} & \\*
    & & \SetCell[c=2]{l,m} <\{る,つ,う\}-end root>った & & \SetCell[c=2]{l,m} & \\* \cmidrule[l]{2-6}
    & Past-negative & \SetCell[c=2]{l,m} <v negative fragment>かった & & \SetCell[c=2]{l,m} <stem>ませんでした。 & \\* \cmidrule[l]{2-6}
    & (Stem) & \SetCell[c=2]{l,m} - & & \SetCell[c=2]{l,m} <$*$-end root>\ruby{○}{〜い} & \\* \cmidrule[l]{2-6}
    & Te & \SetCell[c=2]{l,m} <v past fragment>て & & \SetCell[c=2]{l,m} - & \\*
    & Te-negative & \SetCell[c=2]{l,m} <v negative fragment>くて & & \SetCell[c=2]{l,m} - & \\* \cmidrule[l]{2-6}
    & {Progressive\\(conjugate 〜いる)} & \SetCell[c=2]{l,m} <v te>いる & & \SetCell[c=2]{l,m} - & \\* \cmidrule[l]{2-6}
    & {Potential\\(conjugate 〜る)} & \SetCell[c=2]{l,m} {<$*$-end root>\ruby{○}{〜え}る\\$\cdots$} & & \SetCell[c=2]{l,m} {<$*$-end root>\ruby{○}{〜え}ます。\\$\cdots$} \\* \cmidrule[l]{2-6}
    & Conditional & \SetCell[c=2]{l,m} <$*$-end root>\ruby{○}{〜え}ば & & \SetCell[c=2]{l,m} - & \\*
    & Conditional-negative & \SetCell[c=2]{l,m} <v negative fragment>ければ & & \SetCell[c=2]{l,m} - & \\* \cmidrule[l]{2-6}
    & {Desiderative\\(conjugate i-adj)} & \SetCell[c=2]{l,m} {<stem>たい\\$\cdots$} & & \SetCell[c=2]{l,m} - & \\* \cmidrule[l]{2-6}
    & Volitional & \SetCell[c=2]{l,m} <$*$-end root>\ruby{○}{〜お}う & & \SetCell[c=2]{l,m} <$*$-end root>\ruby{○}{〜い}ましょう。 & \\* \cmidrule[l]{2-6}
    & Imperative & \SetCell[c=2]{l,m} <$*$-end root>\ruby{○}{〜え} & & \SetCell[c=2]{l,m} - & \\*
    & Imperative-negative & \SetCell[c=2]{l,m} <v dict>な & & \SetCell[c=2]{l,m} - & \\
    % & & & & & \\
    \midrule
    \SetCell[r=15]{c,m} る verb & Dictionary & \SetCell[c=2]{l,m} {<る-end root>る} & & \SetCell[c=2]{l,m} <stem>ます。 \\*
    & Negative & \SetCell[c=2]{l,m} <る-end root>ない & & \SetCell[c=2]{l,m} <stem>ません。 \\*
    & Past & \SetCell[c=2]{l,m} <る-end root>た & & \SetCell[c=2]{l,m} <stem>ました。 \\*
    & Past-negative & \SetCell[c=2]{l,m} <v negative fragment>かった & & \SetCell[c=2]{l,m} <stem>ませんでした。 \\*
    & (Stem) & \SetCell[c=2]{l,m} - & & \SetCell[c=2]{l,m} <る-end root> & \\*
    & Te & \SetCell[c=2]{l,m} <v past fragment>て & & \SetCell[c=2]{l,m} - & \\*
    & Te-negative & \SetCell[c=2]{l,m} <v negative fragment>くて & & \SetCell[c=2]{l,m} - & \\*
    & {Progressive\\(conjugate 〜いる)} & \SetCell[c=2]{l,m} {<v te>いる\\$\cdots$} & & \SetCell[c=2]{l,m} {<v te>います。\\$\cdots$} \\*
    & {Potential\\(conjugate 〜る)} & \SetCell[c=2]{l,m} {<る-end root>られる\\$\cdots$} & & \SetCell[c=2]{l,m} {<る-end root>られます。\\$\cdots$} \\*
    & Conditional & \SetCell[c=2]{l,m} <る-end root>れば & & \SetCell[c=2]{l,m} - & \\*
    & Conditional-negative & \SetCell[c=2]{l,m} <v negative fragment>ければ & & \SetCell[c=2]{l,m} - & \\*
    & {Desiderative\\(conjugate i-adj)} & \SetCell[c=2]{l,m} {<stem>たい\\$\cdots$} & & \SetCell[c=2]{l,m} - & \\*
    & Volitional & \SetCell[c=2]{l,m} <る-end root>よう & & \SetCell[c=2]{l,m} <る-end root>ましょう。 & \\*
    & Imperative & \SetCell[c=2]{l,m} <る-end root>ろ & & \SetCell[c=2]{l,m} - & \\*
    & Imperative-negative & \SetCell[c=2]{l,m} <v dict>な & & \SetCell[c=2]{l,m} - & \\
    % & & & & & \\
    \midrule
    \SetCell[r=15]{c,m} exception verb & Dictionary & 〜する & \ruby{来}{く}る & 〜します。 & \ruby{来}{き}ます。 \\*
    & Negative & 〜しない & \ruby{来}{こ}ない & 〜しません。 & \ruby{来}{き}ません。 \\*
    & Past & 〜した & \ruby{来}{き}た & 〜しました。 & \ruby{来}{き}ました。 \\*
    & Past-negative & 〜<vnf>かった & <vnf>かった & 〜しませんでした。 & \ruby{来}{き}ませんでした。 \\*
    & (Stem) & \SetCell[c=2]{l,m} & - & 〜し & \ruby{来}{き} \\*
    & Te & <vpf>て & <vpf>て & - & - \\*
    & Te-negative & <vnf>くて & <vnf>くて & - & - \\*
    & {Progressive\\(conjugate 〜いる)} & {<v te>いる\\$\cdots$} & {<v te>いる\\$\cdots$} & {<v te>います。\\$\cdots$} & {<v te>います。\\$\cdots$} \\*
    & {Potential\\(conjugate 〜る)} & {〜できる\\$\cdots$} & {\ruby{来}{こ}られる\\$\cdots$} & {〜できます。\\$\cdots$} & {\ruby{来}{こ}られます。\\$\cdots$} \\*
    & Conditional & 〜すれば & \ruby{来}{く}れば & - & - \\*
    & Conditional-negative & <vnf>ければ & <vnf>ければ & - & - \\*
    & {Desiderative\\(conjugate i-adj)} & {<stem>たい\\$\cdots$} & {<stem>たい\\$\cdots$} & - & - \\*
    & Volitional & 〜しよう & \ruby{来}{こ}よう & 〜しましょう。 & \ruby{来}{き}ましょう。 \\*
    & Imperative & 〜しろ & \ruby{来}{こ}い & - & - \\*
    & Imperative-negative & <v dict>な & <v dict>な & - & - \\
    % & & & & & \\
    \bottomrule
}


\subsection{Particle and schema summary}

% Help: \SetCell[r=2,c=2]{c,m} <content>, \cmidrule[l]{3-4}
% Help: colspec: X[ratio, horizontal alignment] columns grow to fit width=\linewidth
%                  negative ratios: shrink to fit content and may not grow to full ratio
% Help: colspec: l/c/r columns do not grow
\longtabse[0.75]  % scale factor
{All particles seen so far.}  % caption
{tbl:particle-summary}  % label
{}  % outer specification options
{
    colspec={X[-1,l]X[-1,l]X[1,l]X[-1,l]},
    rowhead=1,
    rows={valign=h},
    % width=\linewidth,  % useful only with X columns
}  % inner specification options
{
    \toprule
    \textbf{Particle} & \textbf{Particle name/purpose} & \textbf{Schemae} & \textbf{Sections} \\
    \midrule
    は & introductory topic marker & <main/new topic>は & \S\ref{sec:topic-marker}, \S\ref{sec:particles} \\
    も & inclusive topic marker & <inclusive topic>も & \S\ref{sec:particles} \\
    % & & & \\
    \midrule
    が & subject marker & <subj>が & \S\ref{sec:particles} \\
    を & direct object marker & <obj>を<v transitive> & \S\ref{sec:verb-particles} \\
    & location-traversed marker & <location>を<motion v (intransitive OK)> & \S\ref{sec:verb-particles} \\
    に & target marker & <target>に\textlightgrey{[は/も]}<v> & \S\ref{sec:verb-particles} \\
    & location-target marker & <location>に\textlightgrey{[は/も]}<v> & \S\ref{sec:verb-particles} \\
    & time-target marker & <time>[に\textlightgrey{[は/も]}]<v> & \S\ref{sec:verb-particles} \\
    へ & direction marker & <direction>へ\textlightgrey{[は/も]}<v> & \S\ref{sec:verb-particles} \\
    で & context marker & <by-way-of (where/what/how) context>で\textlightgrey{[は/も]} & \S\ref{sec:verb-particles} \\
    から & from-marker & <from>から & \S\ref{sec:verb-particles} \\
    まで & to-marker & <to>まで & \S\ref{sec:verb-particles} \\
    % & & & \\
    \midrule
    と & together-with marker & <nn>と<v> & \S\ref{sec:noun-related-particles} \\
    % & & & \\
    \midrule
    \midrule
    と & noun exclusive listing connector & (<nn>と)*<nn> & \S\ref{sec:noun-related-particles} \\
    どか/や & noun vague listing connector & (<nn>\{とか/や\})*<nn> & \S\ref{sec:noun-related-particles} \\
    し & reason vague listing connector & \textred{($*$)} (<reason>し)*<reason> & \S\ref{sec:reason-vague-listing-connector} \\
    たり[する] & adj/verb vague listing marker & (<past adj/v>り、)*<past adj/v>りする & \S\ref{sec:adj-verb-vague-listing-construct} \\
    の & complaining listing marker & \{<nn/na-adj>だの/<v>の\}* & \S\ref{sec:no-the-nominaliser} \\
    % & & & \\
    \midrule
    と & quote marker & {「<quote>」と<v>\\<quote>と<v>} & \S\ref{sec:building-clauses-and-sentences}, \S\ref{sec:direct-quotation}, \S\ref{sec:indirect-quotation} \\
    って & quote marker (abbreviation) & <quote>って & \S\ref{sec:building-clauses-and-sentences}, \S\ref{sec:direct-quotation}, \S\ref{sec:indirect-quotation} \\
    って & generic verb (abbreviation) & <quote>って & \S\ref{sec:referring-to-relative-clause} \\
    て & generic verb (abbreviation) & て<v>、<sentence> & \S\ref{sec:referring-to-relative-clause} \\
    % & & & \\
    \midrule
    \midrule
    の & label marker & <label nn>の[<labelled nn>] & \S\ref{sec:noun-related-particles} \\
    & direction-label marker & <direction>への[<labelled nn>] & \S\ref{sec:no-the-label-marker} \\
    & from-label marker & <from>からの[<labelled nn>] & \S\ref{sec:no-the-label-marker} \\
    & only-label marker & <only>だけの[<labelled nn>] & \S\ref{sec:no-the-label-marker} \\
    & quote-label marker & <quote>との[<labelled nn>] & \S\ref{sec:no-the-label-marker} \\
    & regarding-label marker & <regarding>についての[<labelled nn>] & \S\ref{sec:no-the-label-marker} \\
    % & & & \\
    \midrule
    の & subjective label marker & <na-adj label>の[<labelled nn>] & \S\ref{sec:no-adjectives} \\
    (な) & objective label marker & <na-adj label>な<labelled> & \S\ref{sec:no-adjectives} \\
    & mood marker & <nn>な気分 & \S\ref{sec:no-adjectives} \\
    % & & & \\
    \midrule
    \midrule
    の& nominaliser/generic noun & \textred{($*$)} <adj-phrase/v-phrase>\textlightgrey{\{}の\textlightgrey{,物,こと\}} & \S\ref{sec:noun-related-particles} \\
    & enthusiastic extraordinary & {<positive adj>の、<negative adj>の\\<adj>のなんのって} & \S\ref{sec:no-the-nominaliser} \\
    の/ん & explanatory ender & \textred{($*$)} <sentence>\{の[だ/です],んだ,んです\} & \S\ref{sec:noun-related-particles} \\
    ので/んで & non-causal explanation/reason marker & {\textred{($*$)} <reason>\{ので/んで\}<result>\\な\{ので/んで\}<result>\\\textred{($*$)} <reason>\{の[だ/です]/んだ/んです\}} & \S\ref{sec:causation-reasoning-particles} \\
    から & direct cause marker & {\textred{($*$)} <direct cause>から<result>\\だから<result>\\\textred{($*$)} <direct cause>から [です]} & \S\ref{sec:causation-reasoning-particles} \\
    んだった & just-remembered marker & <v dict>んだった & \S\ref{sec:noun-related-particles} \\
    んじゃない & prohibition marker (slang) & <v dict>ん\{じゃない/じゃありません\} & \S\ref{sec:noun-related-particles} \\
    んじゃなかった & feeling regret marker (slang) & <v dict>んじゃなかった & \S\ref{sec:noun-related-particles} \\
    と & expected consequence marker & \textred{($*$)} <predicate>と<statement> & \S\ref{sec:expected-consequence-conditionals} \\
    % & & & \\
    \midrule
    % & & & \\
    \midrule
    ね & seeking agreement ender & <sentence>ね & \S\ref{sec:sentence-ending-particles} \\
    よ & presenting new information ender & <sentence>よ & \S\ref{sec:sentence-ending-particles} \\
    のに & despite marker & \textred{($*$)} <despite>のに、<sentence> & \S\ref{sec:despite-marker-particle} \\
    \SetCell[r=2]{l,m} {けど/が\\\textlightgrey{けれど}\\\textlightgrey{けれども}} & general connector & \textred{($*$)} <s1>\{けど/が\}<s2> & \S\ref{sec:general-and-contradiction-connector-particles} \\
    & contradiction connector & \textred{($*$)} <s1>\{けど/が\}<contradicting s2> & \S\ref{sec:general-and-contradiction-connector-particles} \\
    % & & & \\
    \midrule
    の & casual question ender & <sentence>の & \S\ref{sec:noun-related-particles} \\
    か & polite question ender & <sentence>か & \S\ref{sec:ka-in-polite-questions} \\
    & casual binary/sarcastic question ender & <sentence>か & \S\ref{sec:ka-in-polite-questions} \\
    & whether-or-not question marker & {<positive v>か<negative v>か\\<positive v>かどうか} & \S\ref{sec:ka-question-embedded-clauses} \\
    % & & & \\
    \midrule
    でも & But &  &  \\
    ばかり & only? approximately? <v te>just (time)? &  &  \\
    くらい & approximately/about/around & & \\
    より & than & & \\
    しか & nothing but/no more than & & \\
    % & & & \\
    \bottomrule
}

\color{red}
($*$) Important notes:
\begin{itemize}
    \item For から::direct cause marker, if a non-conjugated na-adj/noun is used at the end of <direct cause>, だから must be used instead of から for disambiguating with the from-marker (e.g.\ 友達\textbf{だ}から).

    For けど/が::general connector and けど/が::contradiction connector, if a non-conjugated na-adj/noun is used at the end of <s1>, だけど/だが must be used instead (e.g.\ 友達\textbf{だ}\{けど/が\}).

    For し::reason vague listing connector, if a non-conjugated na-adj/noun is used at the end of <reason>, だし must be used instead (e.g.\ 友達\textbf{だ}し).
    \item For の::nominaliser/generic noun, if a non-conjugated na-adj is used as the <adj-phrase>, the following な particle must be used for disambiguating with the label marker (e.g.\ 静か\textbf{な}の\dots).

    For の::explanatory ender:
    \begin{itemize}
        \item if a non-conjugated na-adj/noun is used at the end of <sentence>, the following な particle must be used for disambiguating with the label marker (e.g.\ 緊張\textbf{な}のです。).
        \item polite form (〜ますの for verbs and 〜ですの for い-adjectives) cannot be used; used 〜のです/〜んです instead.
    \end{itemize}

    For ので::non-causal explanation/reason marker, if a non-conjugated na-adj/noun is used at the end of <reason>, なので is used instead for disambiguating with the label marker (e.g.\ 静か\textbf{な}ので).

    For のに::despite marker, if a non-conjugated na-adj/noun is used as the end of <despite>, the following な particle must be used (e.g.\ 静か\textbf{な}のに).
\end{itemize}
\color{black}

\hl{Also (see takoboto): かも、じゃん}


\longtabse[0.6]  % scale factor
{Other schemae seen so far. <v te fragment> refers to <v te> but dropping the trailing 「て」 character; <v negative fragment> refers to <v negative> but dropping the trailing 「い」 character.}  % caption
{tbl:schema-summary}  % label
{}  % outer specification options
{
    colspec={X[-1,l]X[-1,l]X[1,l]X[-1,l]},
    rowhead=1,
    rows={valign=h},
    % width=\linewidth,  % useful only with X columns
}  % inner specification options
{
    \toprule
    \textbf{Schema (short)} & \textbf{Schema name/purpose} & \textbf{Schemae} & \textbf{Sections} \\
    \midrule
    に行く/に来る & to go/come and then do & <v stem>に行く/に来る & \S\ref{sec:verb-stems} \\
    へ行く/へ来る & to go/come for the purpose of doing & <v stem>へ行く/へ来る & \S\ref{sec:verb-stems} \\
    てある & resultant state (implicit preparation) & <v te>ある & \S\ref{sec:resultant-state-tearu} \\
    ておく/とく & completed action (explicit preparation) & <v te>おく/<v te fragment>とく & \S\ref{sec:auxiliary-verb-teoku} \\
    ていく & spatial/temporal do and go & <v te>いく & \S\ref{sec:auxiliary-verbs-teiku-tekiru} \\
    てくる & spatial/temporal do and come & <v te>くる & \S\ref{sec:auxiliary-verbs-teiku-tekiru} \\
    ことができる & given opportunity to do & <v>\{こと/の\}ができる & \S\ref{sec:special-cases-mirareru-kikeru}\\
    あり\ruby{得}{え}る & potential to exist & <nn>\{は/も/etc.\}ありえる & \S\ref{sec:potential-to-exist-arieru}\\
    にする & to make become & {<na-adj>にする\\<i-adj root>くする} & \S\ref{sec:auxiliary-verbs-nisuru-ninaru} \\
    & to decide on & <nn>にする & \S\ref{sec:auxiliary-verbs-nisuru-ninaru} \\
    ことにする & to decide to do & <v>ことにする & \S\ref{sec:auxiliary-verbs-nisuru-ninaru} \\
    ようにする & to try to do & <v>ようにする & \S\ref{sec:auxiliary-verbs-nisuru-ninaru} \\
    になる & to become & {<na-adj>になる\\<i-adj root>くなる\\<nn>になる} & \S\ref{sec:auxiliary-verbs-nisuru-ninaru} \\
    ことになる & to have been arranged/decided to do & <v>ことになる & \S\ref{sec:auxiliary-verbs-nisuru-ninaru} \\
    ようになる & to change state of doing & <v>ようになる & \S\ref{sec:auxiliary-verbs-nisuru-ninaru} \\
    & to change state of feasibility & <v potential>ようになる & \S\ref{sec:auxiliary-verbs-nisuru-ninaru} \\
    なら[ば] & contextual conditional & <context>なら[ば]、<statement> & \S\ref{sec:contextual-conditionals}; \aux \\
    - & prohibition & {<v te>は\{だめ/いけない/ならない\}\\<v te fragment>\{ちゃ/じゃ\}\{だめ/いけない/ならない\}} & \S\ref{sec:prohibition} \\
    - & requirement & {<v te-negative>は\{だめ/いけない/ならない\}\\<v negative fragment>くちゃ\\<v negative>と\{だめ/いけない/ならない\}\\<v negative>と\\<v conditional-negative>\{だめ/いけない/ならない\}\\<v negative fragment>きゃ} & \S\ref{sec:requirement} \\
    - & permission & {<v te>も\{いい/大丈夫/構わない\}\\<v te>いい} & \S\ref{sec:permission} \\
    - & suggestion & <v conditional>/<v past conditional>どう & \S\ref{sec:suggestions} \\
    という & definition connector & <definition>という<thing> & \S\ref{sec:definition-toiu} \\
    というか & re-definition connector & <to be redefined>というか、<redefinition> & \S\ref{sec:rephrasing-and-refining-definitions-with-toiuka-connector} \\
    という\{の/こと\} & generic verb & <quote>という\{の/こと\} & \S\ref{sec:referring-to-relative-clause} \\
    てみる & trialling something & <v te>みる & \S\ref{sec:trialling-something-out-temiru} \\
    とする & attempting something & <v volitional>と\{する/<v>\} & \S\ref{sec:attempting-something-volitional-to} \\
    てあげる & giving a favour (first person) & <v te>あげる & \S\ref{sec:giving} \\
    てくれる & giving a favour (second person) & <v te>くれる & \S\ref{sec:giving} \\
    てもらう & receiving a favour & <v te>もらう & \S\ref{sec:receiving} \\
    てくれる/てもらえう & soliciting a favour & <v te>\{くれる/もらえる\} & \S\ref{sec:soliciting-favours-kureru-moraeru} \\
    てください。/て & honorific request & {<v te>ください。\\<v te>} & \S\ref{sec:honorific-requests} \\
    てちょうだい。 & casual request & <v te>ちょうだい。 & \S\ref{sec:casual-requests} \\
    なさい/な & polite firm request & {<v stem>なさい\\<v stem>な} & \S\ref{sec:polite-firm-requests} \\
    \bottomrule
}

\end{document}
