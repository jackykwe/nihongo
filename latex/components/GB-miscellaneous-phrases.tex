\documentclass[../nihongo-gakushuu-kyouzai-grammar.tex]{subfiles}
\begin{document}
\appendix
\setcounter{section}{1}
\section{Phrases}

\begin{itemize}
    % https://www.youtube.com/watch?v=ZlwcPM4zRlQ
    \item はぁ!?
    \item バカにしてる? (Are you making fun of me?) してないしてない。
    \item フンッ! (Hmph/blow)
    \item \ruby{朝}{あさ}\ruby{早}{はや}いです。(Early in the morning.)
    \item \ruby{爽快}{そう|かい}な\ruby{朝}{あさ}!総会の\ruby{目覚}{め|ざ}め! (Starting a new day full of energy!)
    \item まだ\ruby{寝}{ね}たいの! (I don't want to get up already!) ういちゃん\ruby{朝}{あさ}だよ!(Wakey, wakey, Ui-chan!) やだ! (No!)
    \item まだ\ruby{6時}{ろく|じ}だよ?(It's only 6 AM.) もう6時だよ! (It's already 6 AM!)
    \item 私たち(からしたら)\ruby{寝}{ね}る\ruby{時間}{じ|かん}だろ? (This is when we normally get into bed.)
    \item ということで!(And here we are!)
    \item みんな聞きましたね? (You heard her, everyone.)

    \item Conversation: %https://www.youtube.com/shorts/PS3okUw-JgU

    男子: もしもし?

    みゆきさん: もしもし。

    男子: お\ruby{疲}{つか}れ。(Glad you could make it.)

    みゆきさん: お疲れ。

    男子: ちょっとさ、あの、\ruby{1つ}{ひ|とつ}、あのちょと、聞きたいことあって。(So um, there is one thing I'd like to ask you.)

    みゆきさん: うん。 (Yeah?)

    男子: ちょっとあの、\ruby{今度}{こん|ど}さ、二人でさ、\ruby{飯}{めし}でもいかね? (Would you like to go out for a meal together sometime?; 今度: soon; 飯: meal)

    みゆきさん: ご飯?

    男子: そうそうそう。

    男子: \ruby{気}{き}になって、みゆきさんのこと。(I've been interested about you, Miyuki.)

    みゆきさん: そうなの? (Is that so?)

    男子: そう。

    みゆきさん: うん、ごはん、いいよ。行こ。(Yeah, dinner's good, let's go!)

    男子: マジ? (Really?)

    みゆきさん: うん、\ruby{本当}{ほん|とう}にいいよいいよ。

    男子: えじゃちょっとさ、\ruby{詳}{くわ}しい\ruby{予定}{よ|てい}さまた\ruby{後}{あと}で、あの\ruby{話}{はな}そう。

    みゆきさん: わかった。

    男子: ありがと。

    みゆきさん: いやこちらこそありがと。(No thank YOU.)

    男子: じゃね、おつかれ。

    みゆきさん: うん、おつかれ。
\end{itemize}

\end{document}
