\documentclass[../nihongo-gakushuu-kyouzai-vocabulary.tex]{subfiles}
\begin{document}
\appendix
\setcounter{section}{2}

\section{\ruby{形容詞}{けい|よう|し} (adjectives)}


\subsection{Emotions}
% Help: \SetCell[r=2,c=2]{c,m} <content>, \cmidrule[l]{3-4}
% Help: colspec: X[ratio, horizontal alignment] columns grow to fit width=\linewidth
%                  negative ratios: shrink to fit content and may not grow to full ratio
% Help: colspec: l/c/r columns do not grow
\longtabse[0.75]  % scale factor
{Adjectives: emotions.}  % caption
{tbl:appendix-vocab-adjectives-emotions}  % label
{}  % outer specification options
{
    colspec={X[-3,l]X[-1,c]X[3,l]X[-3,l]},
    rowhead=1,
    % width=\linewidth,  % useful only with X columns
}  % inner specification options
{
    \toprule
    \textbf{Descriptor} & \textbf{Cat.} & \textbf{Meaning} & \textbf{Notes} \\
    \midrule
    \ruby{感動的}{かん|どう|てき} & な & moving/touching/stirring & \\
    % & & & \\
    \midrule
    \ruby{嬉}{うれ}しい & い & happy/glad/delighted & \\
    \ruby{楽}{たの}しい & い & fun/enjoyable/happy & \\
    \ruby{欲}{ほ}しい & い & desired/wanted & \\
    <v te>ほしい & い & I want you to do <v>/I want <v> done & \aux \\
    イケイケ & な & eager/enthusiastic/excited/spirited/energetic & (\ruby{行}{い}け\ruby{行}{い}け) \\
    めでたい & い & happy/auspicious/joyous & (\ruby{目出度}{め|で|た}い); also in Table~\ref{tbl:appendix-vocab-adjectives-personalities} \\
    おめでたい & い & for congratulations/worthy of celebration/auspicious (occasion/ending/etc.) & (お\ruby{目出度}{め|で|た}い); polite; also in Table~\ref{tbl:appendix-vocab-adjectives-personalities} \\
    % & & & \\
    \midrule
    \ruby{大切}{たい|せつ} & な & important/significant; precious/cherished/beloved & also an verb, adverb \\
    \ruby{愛}{いと}しい & い & lovely/dear/beloved/darling/dearest & \\
    \ruby{惜}{お}しい & い & precious/dear/valuable & also in Table~\ref{tbl:appendix-vocab-adjectives-agreeability} \\
    % & & & \\
    \midrule
    \ruby{悲}{かな}しい & い & sad/miserable & \\
    \ruby{恥}{は}ずかしい & い & embarrassed/ashamed/humiliated & \\
    \ruby{懐}{なつ}かしい & い & nostalgic/fondly-remembered/missed & \\
    \ruby{寂}{さび}しい & い & lonely & \\
    \ruby{済}{す}まない & い & remorseful/sorry/apologetic/conscience-stricken & also an interjection, also in Table~\ref{tbl:appendix-vocab-adjectives-agreeability} \\
    \ruby{気}{き}の\ruby{毒}{どく} & な & feel bad/sorry/regretful/regret & also in Table~\ref{tbl:appendix-vocab-adjectives-appearance-and-style} \\
    \ruby{興味}{きょう|み}ない & い & uninterested in/having no interested in & \\
    % & & & \\
    \midrule
    \ruby{惜}{お}しい & い & regrettable/disappointing/unfortunate/a pity & also in Table~\ref{tbl:appendix-vocab-adjectives-agreeability} \\
    \ruby{残念}{ざん|ねん} & な & regrettable/unfortunate/disappointing/vexing & \\
    \ruby{悔}{くや}しい & い & frustrated/annoyed/bitterly disappointed (over failure/injustice) & \\
    \ruby{羨}{うらや}ましい & い & envious/jealous; enviable (position) & \\
    % & & & \\
    \midrule
    \ruby{苦}{くる}しい & い & psychologically painful/distressing/stressful; difficult/struggling (circumstances) & \\
    つらい & い & painful/heart-breaking/difficult (emotionally); tough/hard/harsh (situations) & (\ruby{辛}{つら}い $\neq$ \ruby{辛}{から}い) \\
    \ruby{痛}{いた}い & い & painful/sore; [cringy/embarrasing] & [slang] \\
    きな\ruby{臭}{くさ}い & い & tense/strained/``smelling of gunpowder'' & also in Table~\ref{tbl:appendix-vocab-adjectives-taste-and-texture} \\
    % & & & \\
    \midrule
    \ruby{眠}{ねむ}い & い & sleepy/drowsy & \\
    % & & & \\
    \midrule
    \ruby{安心}{あん|しん} & な & relieved & \\
    % & & & \\
    \midrule
    \midrule
    \ruby{暖}{あたた}かい & い & pleasantly warm & \\
    \ruby{暑}{あつ}い & い & hot & \\
    \ruby{寒}{さむ}い & い & cold (weather) & also in Table~\ref{tbl:appendix-vocab-adjectives-appearance-and-style} \\
    \ruby{小寒}{こ|さむ}い & い & chilly/a little cold & \\
    % & & & \\
    \midrule
    \ruby{熱}{あつ}い & い & hot (to the touch); emotionally passionate/zealous/enthusiastic; hot-tempered & also in Table~\ref{tbl:appendix-vocab-adjectives-appearance-and-style} \\
    \ruby{冷}{つめ}たい & い & cold/chilly/icy/freezing (to the touch); emotionally cold/unfriendly/distant & \\
    つれない & い & cold/unsympathetic/heartless/unfriendly & \\
    \ruby{疎}{うと}い & い & distant/aloof/estranged & also in Table~\ref{tbl:appendix-vocab-adjectives-ability} \\
    \ruby{心}{こころ}の\ruby{貧}{まず}しい & い & ungenerous/poor in spirit/with no great feelings & \\
    % & & & \\
    \bottomrule
}


\subsection{Production}
% Help: \SetCell[r=2,c=2]{c,m} <content>, \cmidrule[l]{3-4}
% Help: colspec: X[ratio, horizontal alignment] columns grow to fit width=\linewidth
%                  negative ratios: shrink to fit content and may not grow to full ratio
% Help: colspec: l/c/r columns do not grow
\longtabse[0.75]  % scale factor
{Adjectives: production.}  % caption
{tbl:appendix-vocab-adjectives-production}  % label
{}  % outer specification options
{
    colspec={X[-3,l]X[-1,c]X[3,l]X[-3,l]},
    rowhead=1,
    % width=\linewidth,  % useful only with X columns
}  % inner specification options
{
    \toprule
    \textbf{Descriptor} & \textbf{Cat.} & \textbf{Meaning} & \textbf{Notes} \\
    \midrule
    \ruby{自動的}{じ|どう|てき} & な & automatic & \\
    % & & & \\
    \bottomrule
}


\subsection{Consumption}
% Help: \SetCell[r=2,c=2]{c,m} <content>, \cmidrule[l]{3-4}
% Help: colspec: X[ratio, horizontal alignment] columns grow to fit width=\linewidth
%                  negative ratios: shrink to fit content and may not grow to full ratio
% Help: colspec: l/c/r columns do not grow
\longtabse[0.75]  % scale factor
{Adjectives: consumption.}  % caption
{tbl:appendix-vocab-adjectives-consumption}  % label
{}  % outer specification options
{
    colspec={X[-3,l]X[-1,c]X[3,l]X[-3,l]},
    rowhead=1,
    % width=\linewidth,  % useful only with X columns
}  % inner specification options
{
    \toprule
    \textbf{Descriptor} & \textbf{Cat.} & \textbf{Meaning} & \textbf{Notes} \\
    \midrule
    \ruby{使}{つか}いやすい & い & easy to use & (\ruby{使}{つか}い\ruby{易}{やす}い) \\
    \ruby{見}{み}やすい & い & easy to see & (\ruby{見}{み}\ruby{易}{やす}い) \\
    \ruby{読}{よ}みやすい & い & easy to read/legible & (\ruby{読}{よ}み\ruby{易}{やす}い) \\
    \ruby{飲}{の}みやすい & い & easy to drink/swallow & (\ruby{飲}{の}み\ruby{易}{やす}い) \\
    % & & & \\
    \midrule
    \midrule
    わかりやすい & い & easy to understand & (\ruby{分}{わ}かり\ruby{易}{やす}い) \\
    \ruby{覚}{おぼ}えやすい & い & easy to learn/remember & (\ruby{覚}{おぼ}え\ruby{易}{やす}い) \\
    % & & & \\
    \midrule
    \midrule
    \ruby{住}{す}みやすい & い & comfortable/convenient to live in (of a neighbourhood) & (\ruby{住}{す}み\ruby{易}{やす}い) \\
    % & & & \\
    \bottomrule
}


\subsection{Health}
% Help: \SetCell[r=2,c=2]{c,m} <content>, \cmidrule[l]{3-4}
% Help: colspec: X[ratio, horizontal alignment] columns grow to fit width=\linewidth
%                  negative ratios: shrink to fit content and may not grow to full ratio
% Help: colspec: l/c/r columns do not grow
\longtabse[0.75]  % scale factor
{Adjectives: health.}  % caption
{tbl:appendix-vocab-adjectives-health}  % label
{}  % outer specification options
{
    colspec={X[-3,l]X[-1,c]X[3,l]X[-3,l]},
    rowhead=1,
    % width=\linewidth,  % useful only with X columns
}  % inner specification options
{
    \toprule
    \textbf{Descriptor} & \textbf{Cat.} & \textbf{Meaning} & \textbf{Notes} \\
    \midrule
    \ruby{大丈夫}{だい|じょう|ぶ} & な & alright/problem-free/without fear & \\
    \ruby{健康}{けん|こう} & な & healthy/fit; wholesome & also a noun \\
    % & & & \\
    \midrule
    \ruby{無意識}{む|い|しき} & の & unconscious/involuntary/unintentional & \\
    % & & & \\
    \midrule
    \ruby{精神的}{せい|しん|てき} & な & mental/spiritual/emotional & \\
    \ruby{元気}{げん|き} & な & lively/well/in good health & \\
    \ruby{平気}{へい|き} & な & all right/fine/OK & slang; also in Table~\ref{tbl:appendix-vocab-adjectives-personalities} \\
    \ruby{病気}{びょう|き} & な & illness/disease/sickness & \\
    % & & & \\
    \midrule
    \midrule
    \ruby{不安}{ふ|あん} & な & anxious/uneasy/insecure & also a noun\\
    % & & & \\
    \midrule
    \midrule
    \ruby{暇}{ひま} & な & free/available/not busy/unoccupied/idle & also a noun; \href{https://jisho.org/word/\%E6\%9A\%87}{[JS]} \\
    \ruby{忙}{いそが}しい & い & busy/occupied/hectic & \\
    \ruby{大忙}{おお|いそが}し & な & very busy & \\
    % & & & \\
    \midrule
    \midrule
    \ruby{幸}{しあわ}せ & な & happy/blessed & also a noun \\
    \ruby{不幸}{ふ|こう} & な & unhappy/sorrowful/misfortunate/disastrous (short term/long term) & \href{https://dictionary.goo.ne.jp/thsrs/5311/meaning/m0u/}{[HN]} \\
    \ruby{不幸}{ふ|しあわ}せ & な & unhappy/misfortunate/unlucky (long term) & also a noun; \href{https://dictionary.goo.ne.jp/thsrs/5311/meaning/m0u/}{[HN]} \\
    % & & & \\
    \bottomrule
}


\subsection{Colours}
\emph{Read the main article on \href{https://cotoacademy.com/colors-japanese-use-japanese-color-words/}{CTA}.}

Only four colours were recognised as basic colours in ancient Japan: red, blue, white, black. Blue and green used to be both referred to as \ruby{青}{あお}い.

% Help: \SetCell[r=2,c=2]{c,m} <content>, \cmidrule[l]{3-4}
% Help: colspec: X[ratio, horizontal alignment] columns grow to fit width=\linewidth
%                  negative ratios: shrink to fit content and may not grow to full ratio
% Help: colspec: l/c/r columns do not grow
\longtabse[0.75]  % scale factor
{Adjectives: colours.}  % caption
{tbl:appendix-vocab-adjectives-colours}  % label
{}  % outer specification options
{
    colspec={X[-3,l]X[-1,c]X[3,l]X[-3,l]},
    rowhead=1,
    % width=\linewidth,  % useful only with X columns
}  % inner specification options
{
    \toprule
    \textbf{Descriptor} & \textbf{Cat.} & \textbf{Meaning} & \textbf{Notes} \\
    \midrule
    \ruby{赤}{あか}い & い & red & \\
    \ruby{青}{あお}い & い & blue; green (for fruits/vegetables/traffic lights); inexperienced & \\
    \ruby{黒}{くろ}い & い & black & also in Table~\ref{tbl:appendix-vocab-adjectives-agreeability} \\
    \ruby{白}{しろ}い & い & white & \\
    \midrule
    \midrule
    \ruby{黄色}{き|いろ}い & い & yellow & ``黄い'' is invalid \\
    \ruby{茶色}{ちゃ|いろ}い & い & brown & ``茶い'' is invalid \\
    \ruby{茶色}{ちゃ|いろ}っぽい & い & brownish & \\
    % & & & \\
    \bottomrule
}


\subsection{Agreeability}
% Help: \SetCell[r=2,c=2]{c,m} <content>, \cmidrule[l]{3-4}
% Help: colspec: X[ratio, horizontal alignment] columns grow to fit width=\linewidth
%                  negative ratios: shrink to fit content and may not grow to full ratio
% Help: colspec: l/c/r columns do not grow
\longtabse[0.75]  % scale factor
{Adjectives: agreeability.}  % caption
{tbl:appendix-vocab-adjectives-agreeability}  % label
{}  % outer specification options
{
    colspec={X[-3,l]X[-1,c]X[3,l]X[-3,l]},
    rowhead=1,
    % width=\linewidth,  % useful only with X columns
}  % inner specification options
{
    \toprule
    \textbf{Descriptor} & \textbf{Cat.} & \textbf{Meaning} & \textbf{Notes} \\
    \midrule
    ない & い & non-existent/not being there & (\ruby{無}{な}い) \\
    % & & & \\
    \midrule
    \midrule
    いい/\ruby{良}{よ}い/よい & い & good/nice/agreeable/OK & \href{https://salon.mainichi-kotoba.jp/archives/670}{[MK]}\\
    よろしい & い & good/OK/all right/fine/very well/will do/may/can & (\ruby{宜}{よろ}しい); honorific \\
    \ruby{気持}{き|も}ちいい & い & pleasant/good feeling & also an expression \\
    \ruby{結構}{けっ|こう} & な & nice/splendid/lovely/wonderful/fine & also in Table~\ref{tbl:appendix-vocab-adjectives-amounts-and-sizes} \\
    すごい & い & amazing/great/wonderful/terrific & (\ruby{凄}{すご}い) \\
    \ruby{素晴}{す|ば}らしい & い & wonderful/splendid/manigifcent & \\
    \ruby{素敵}{す|てき} & な & lovely/wonderful/fantastic/superb/nice/cool & \\
    \ruby{偉}{えら}い & い &  excellent/remarkable &  also in Table~\ref{tbl:appendix-vocab-adjectives-appearance-and-style} \\
    やばい & い & terrific/amazing/cool (``damn!'') & colloquial, slang; also: やべー \\
    \ruby{惜}{お}しい & い & too good for/deserving better & also in Table~\ref{tbl:appendix-vocab-adjectives-emotions} \\
    \ruby{問題}{もん|だい}ない & い & unobjectionable & also an expression \\
    % & & & \\
    \midrule
    \ruby{悪}{わる}い & い & bad/poor/undesirable/at fault (says served) & also an interjection, \href{https://oshiete.goo.ne.jp/qa/3191614.html}{[goo]} \\
    まずい & い & bad taste/unpleasant/awful/problematic/unfavourable (says server) & \href{https://oshiete.goo.ne.jp/qa/3191614.html}{[goo]} \\
    \ruby{気}{き}まずい & い & awkward/embarrassing/uneasy & \\
    だめ/ダメ & な & not good/hopeless; cannot/not allowed & ダメ is informal, \href{https://ja.hinative.com/questions/19206672}{[HN]} \\
    \ruby{嫌}{いや} & な & reluctant/disagreeable & \\
    いやらしい & い & unpleasant/disagreeable/nasty & (\ruby{嫌}{いや}らしい) \\
    \ruby{不快}{ふ|かい} & な & unpleasant/displeasing/discomforting & \\
    \ruby{気持}{き|も}ち\ruby{悪}{わる}い & い & disgusting/gross/revolting/unpleasant/bad feeling & \\
    キモい & い & disgusting/gross (abbreviation) & slang \\
    \ruby{気色悪}{き|しょく|わる}い & い & weird/disgusting/sickening & \\
    きしょい & い & gross/disgusting/sickening (abbreviation of \ruby{気色悪}{き|しょく|わる}い) & slang; also: きしょー \\
    やばい & い & awful/crazy/unhinged (``damn!'') & colloquial, slang; also: やべー \\
    \ruby{罰当}{ばち|あ}たり & の & cursed/damned/accursed & \\
    ありえない & い & unthinkable/ridiculous/absurd & (あり\ruby{得}{え}ない); also in Table~\ref{appendix-vocab-tbl:appendix-vocab-adjectives-ability} \\
    とんでもない & い & unthinkable/unexpected/absurd/outrageous/preposterous/terrible & also an interjection \\
    % & & & \\
    \midrule
    \midrule
    \ruby{好}{す}き & な & likeable/favourite & \\
    \ruby{大好}{だい|す}き & な & strongly liked/loved & \\
    \ruby{興味深}{きょう|み|ぶか}い & い & very interesting/of great interest & \\
    % & & & \\
    \midrule
    \ruby{嫌}{きら}い & \exception{な} & disliked/hated & \\
    \ruby{大嫌}{だい|きら}い & \exception{な} & strongly disliked/hated & \\
    % & & & \\
    \midrule
    \midrule
    \ruby{安全}{あん|ぜん} & な & safe/secure & also a noun \\
    セーフ & な & acceptable/fine/OK; in time (for) & \\
    % & & & \\
    \midrule
    \ruby{怖}{こわ}い & い & scary/frightening/eerie/dreadful & \\
    \ruby{恐}{おそ}ろしい & い & dreadful/terrifying/frightening/terrible; starling/surprising & \\
    \ruby{危険}{き|けん} & な & dangerous/hazardous & also a noun; \href{https://hinative.com/questions/16741337}{[HN]} \\
    \ruby{危}{あぶ}ない & い & dangerous/risky & also an interjection; \href{https://hinative.com/questions/16741337}{[HN]} \\
    やばい & い & dangerous/risky (``damn!'') & colloquial, slang; also: やべー \\
    % & & & \\
    \midrule
    \midrule
    \ruby{最高}{さい|こう} & な・の & best/finest; highest/maximum & \\
    \ruby{最良}{さい|りょう} & の & best/ideal & \\
    \ruby{高級}{こう|きゅう} & な & high class/calibre & \\
    % & & & \\
    \midrule
    \ruby{最低}{さい|てい} & な・の & worst/awful/nasty/disgusting; lowest/minimum & \\
    \ruby{最悪}{さい|あく} & の & worst (e.g.\ situation) & \\
    \ruby{低級}{てい|きゅう} & な & low class/calibre; vulgar/cheap & \\
    \ruby{邪悪}{じゃ|あく} & な & evil/wicked & \\
    すまない & い & inexcusable/unjustifiable/unpardonable & (\ruby{済}{す}まない); also an interjection, also in Table~\ref{tbl:appendix-vocab-adjectives-emotions} \\
    % & & & \\
    \midrule
    \midrule
    \ruby{当}{あ}たり\ruby{前}{まえ} & の & natural/obvious/common/ordinary/the norm & \\
    \ruby{当然}{とう|ぜん} & の & natural/right/proper/just/appropriate & also an adverb \\
    \ruby{相当}{そう|とう} & の & appropriate/suitable/befitting/proportionate & also a verb, adverb; also in Table~\ref{tbl:appendix-vocab-adjectives-amounts-and-sizes} \\
    % & & & \\
    \midrule
    しょうがない & い & there's no other way/can't be helped; hopeless/anoying/troublesome/awful & \\
    \ruby{仕方}{し|かた}ない & い & there's no other way/can't be helped; hopeless/anoying/troublesome/awful & \\
    <nn/adj/v te>てしょうがない & い & cannot help but <nn/adj/v te> & \\
    <nn/adj/v te>て\ruby{仕方}{し|かた}ない & い & cannot help but <nn/adj/v te> & \\
    <nn/adj/v te>てもしょうがない & い & it's no use <nn/adj/v te>/useless/no good/insufficient/not enough & \\
    <nn/adj/v te>ても\ruby{仕方}{し|かた}ない & い & it's no use <nn/adj/v te>/useless/no good/insufficient/not enough & \\
    % & & & \\
    \midrule
    \ruby{禁}{きん}じ\ruby{得}{え}ない & い & cannot help/hold back/suppress (laughing/feeling sympathy/tears/anger/etc.) & \\
    % & & & \\
    \bottomrule
}


\subsection{Appearance and style}
% Help: \SetCell[r=2,c=2]{c,m} <content>, \cmidrule[l]{3-4}
% Help: colspec: X[ratio, horizontal alignment] columns grow to fit width=\linewidth
%                  negative ratios: shrink to fit content and may not grow to full ratio
% Help: colspec: l/c/r columns do not grow
\longtabse[0.75]  % scale factor
{Adjectives: appearance and style.}  % caption
{tbl:appendix-vocab-adjectives-appearance-and-style}  % label
{}  % outer specification options
{
    colspec={X[-3,l]X[-1,c]X[3,l]X[-3,l]},
    rowhead=1,
    % width=\linewidth,  % useful only with X columns
}  % inner specification options
{
    \toprule
    \textbf{Descriptor} & \textbf{Cat.} & \textbf{Meaning} & \textbf{Notes} \\
    \midrule
    <adj/v stem>そう & な & having the appearance of/seeming that & \aux \\
    <nn>っぽい/ぽい & い & <nn> -ish/-like & \suffix \\
    <nn/v>みたい & な & -like/sort of/similar to/resembling & \suffix \\
    % & & & \\
    \midrule
    こんな\ruby{風}{ふう}/こういう\ruby{風}{ふう} & な & this kind (closer to speaker) & (こう\ruby{言}{い}う\ruby{風}{ふう}) \\
    こんな\ruby{様}{よう}/こういう\ruby{様}{よう} & な & this kind (closer to speaker) & (こう\ruby{言}{い}う\ruby{様}{よう}); formal \\
    そんな\ruby{風}{ふう}/そういう\ruby{風}{ふう} & な & this kind (closer to listener) & (そう\ruby{言}{い}う\ruby{風}{ふう}) \\
    そんな\ruby{様}{よう}/そういう\ruby{様}{よう} & な & this kind (closer to listener) & (そう\ruby{言}{い}う\ruby{様}{よう}); formal \\
    あんな\ruby{風}{ふう}/ああいう\ruby{風}{ふう} & な & this kind (distant) & (ああ\ruby{言}{い}う\ruby{風}{ふう}) \\
    あんな\ruby{様}{よう}/ああいう\ruby{様}{よう} & な & this kind (distant) & (ああ\ruby{言}{い}う\ruby{様}{よう}); formal \\
    % & & & \\
    \midrule
    \midrule
    かわいい & い & cute/adorable/charming/lovely/pretty & \\
    かっこいい/かっこ\ruby{良}{よ}い & い & cool/attractive/stylish & (\ruby{格好}{かっ|こ}いい/\ruby{格好良}{かっ|こ|よ}い) \\
    \ruby{綺麗}{き|れい} & な & pretty/beautiful & \\
    \ruby{美}{うつく}しい & い & beautiful/pretty/lovely/sweet/pure (heart/friendship) & \\
    \ruby{魅力的}{み|りょく|てき} & な & charming/fascinating/attractive & \\
    \ruby{清楚}{せい|そ} & な & neat and clean/tidy/trim & \\
    \ruby{立派}{りっ|ぱ} & な & impressive/praiseworthy/splendid/handsome/well-rounded & \\
    \ruby{背}{せ}が\ruby{高}{たか}い & い & tall (of a person) & \\
    % & & & \\
    \midrule
    \ruby{完璧}{かん|ぺき} & な & perfect/complete/flawless & \\
    \ruby{完全}{かん|ぜん} & の & perfect/complete \\
    % & & & \\
    \midrule
    かっこ\ruby{悪}{わ}い & い & unattractive/ugly/unstylish/uncool & \\
    \ruby{醜}{みにく}い & い & ugly/unattractive/unsightly/disgraceful/dishonourable & \\
    \ruby{背}{せ}が\ruby{低}{ひく}い & い & short (of a person) & \\
    % & & & \\
    \midrule
    \midrule
    ムキムキ & な & muscular/brawny & also an adverb \\
    % & & & \\
    \midrule
    \midrule
    \ruby{色白}{いろ|じろ} & の & fair-skinned/ light-complexioned & \\
    \ruby{色黒}{いろ|ぐろ} & の & dark-skinned & \\
    % & & & \\
    \midrule
    \midrule
    きれい & な & clean/tidy & (\ruby{綺麗}{き|れい}) \\
    \ruby{新鮮}{しん|せん} & な & fresh & \\
    %fragrant  & & & \\
    % & & & \\
    \midrule
    \ruby{汚}{きたな}い & い & dirty/filthy/messy/untity/vulgar & \\
    \ruby{臭}{くさ}い & い & smelly/stinking & also in Table~\ref{tbl:appendix-vocab-adjectives-agreeability} \\
    <nn/adj/v>\ruby{臭}{くさ}い & い & smelling of/appearing like & \suffix \\
    ボコボコ & な & holey/full of holes/dents & \\
    % & & & \\
    \midrule
    \midrule
    \ruby{面白}{おも|しろ}い & い & interesting/fascinating/funny/entertaining & \\
    おもろい & い & interesting/fascinating/funny/intriguing & slang \\
    \ruby{珍}{めずら}しい & い & rare/uncommon/unusual/curious/new/fine/precious & \\
    \ruby{熱}{あつ}い & い & hot topic/of interest & also in Table~\ref{tbl:appendix-vocab-adjectives-emotions} \\
    % & & & \\
    \midrule
    つまらない & dull/uninteresting/boring/tedious; insignificant/worthless; useless/pointless/disappointing & (\ruby{詰}{つま}らない) \\
    ダサい & い & lame/uncool & slang \\
    \ruby{寒}{さむ}い & い & lame/corny (joke) & also in Table~\ref{tbl:appendix-vocab-adjectives-emotions} \\
    % & & & \\
    \midrule
    \midrule
    \ruby{裕福}{ゆう|ふく} & な & wealthy/rich/affluent/well-off & \\
    \ruby{豊}{ゆた}か & な & rich/abundant/plentiful/ample & \href{https://taigigo.jitenon.jp/word/p11942}{[TGG]} \\
    % & & & \\
    \midrule
    \ruby{貧乏}{びん|ぼう} & な & poor/poverty-stricken & \href{https://dictionary.goo.ne.jp/thsrs/6586/meaning/m0u}{[goo]} \\
    \ruby{貧}{まず}しい & い & lacking inner richness/poor/needy & \href{https://taigigo.jitenon.jp/word/p11942}{[TGG]}; \href{https://dictionary.goo.ne.jp/thsrs/6586/meaning/m0u}{[goo]} \\
    \ruby{乏}{とぼ}しい & い & shortage/scarce/limited/meagre & \href{https://taigigo.jitenon.jp/word/p11942}{[TGG]}; \href{https://dictionary.goo.ne.jp/thsrs/6586/meaning/m0u}{[goo]} \\
    % & & & \\
    \midrule
    \midrule
    \ruby{偉}{えら}い & い & respected/great/famous/celebrated/distinguished & also in Table~\ref{tbl:appendix-vocab-adjectives-agreeability} \\
    % & & & \\
    \midrule
    かわいそう & な & pitiful/pathetic & (\ruby{可哀想}{か|わい|そう}/\ruby{可愛}{か|わい}そう) \\
    \ruby{気}{き}の\ruby{毒}{どく} & な & pitiful/unfortunate/poor/miserable & also in Table~\ref{tbl:appendix-vocab-adjectives-emotions} \\
    % & & & \\
    \midrule
    \midrule
    \ruby{新}{あたら}しい & い & new/novel/recent/latest/modern & \\
    \ruby{若}{わか}い & い & young/youthful; immature & \href{https://ja.hinative.com/questions/14498}{[HN]} \\
    % & & & \\
    \midrule
    \ruby{古}{ふる}い & い & old/antiquated/old-fashioned (of things, \textred{not people}) & \\
    % & & & \\
    \midrule
    \midrule
    \ruby{独特}{どく|とく} & の & peculiar/unique/characteristic & \href{https://dictionary.goo.ne.jp/thsrs/17037/meaning/m1u/}{[HN]} \\
    \ruby{独自}{どく|じ} & の & characteristic/their own/unique/original/local & \href{https://dictionary.goo.ne.jp/thsrs/17037/meaning/m1u/}{[HN]} \\
    \ruby{特有}{とく|ゆう} & の & exclusive/characteristic/peculiar & \href{https://dictionary.goo.ne.jp/thsrs/17037/meaning/m1u/}{[HN]} \\
    \ruby{固有}{こ|ゆう} & の & inherent/characteristic/perculiar & \href{https://dictionary.goo.ne.jp/thsrs/17037/meaning/m1u/}{[HN]} \\
    % & & & \\
    \midrule
    \ruby{普通}{ふ|つう} & の & normal/ordinary/regular/usual/common & also an adverb \\
    \ruby{一般的}{いっ|ぱん|てき} & な & general/popular/common/typical & \\
    % & & & \\
    \midrule
    \ruby{特別}{とく|べつ} & な & special/particular/extraordinary/exceptional & also an adverb, \href{https://dictionary.goo.ne.jp/thsrs/17027/meaning/m0u/}{[goo]} \\
    \ruby{特殊}{とく|しゅ} & ば & special/particular/peculiar/unique & \href{https://dictionary.goo.ne.jp/thsrs/17027/meaning/m0u/}{[goo]} \\
    % & & & \\
    \bottomrule
}


\subsection{Ability}
% Help: \SetCell[r=2,c=2]{c,m} <content>, \cmidrule[l]{3-4}
% Help: colspec: X[ratio, horizontal alignment] columns grow to fit width=\linewidth
%                  negative ratios: shrink to fit content and may not grow to full ratio
% Help: colspec: l/c/r columns do not grow
\longtabse[0.75]  % scale factor
{Adjectives: ability.}  % caption
{tbl:appendix-vocab-adjectives-ability}  % label
{}  % outer specification options
{
    colspec={X[-3,l]X[-1,c]X[3,l]X[-3,l]},
    rowhead=1,
    % width=\linewidth,  % useful only with X columns
}  % inner specification options
{
    \toprule
    \textbf{Descriptor} & \textbf{Cat.} & \textbf{Meaning} & \textbf{Notes} \\
    \midrule
    うまい & い & skilful/good & (\ruby{上手}{う|ま}い) \\
    \ruby{上手}{じょう|ず} & な & skilful/proficient/adept & \\
    \ruby{有能}{ゆう|のう} & な & capable/competent/efficient & \\
    \ruby{強}{つよ}い & い & strong/dependable; competent/skilled; rigid/solid & \\
    \ruby{詳}{くわ}しい & い & detailed/full & \\
    に\ruby{詳}{くわ}しい & い & knowledgeable/well-informed/familiar about & \\
    \ruby{流暢}{りゅう|ちょう} & な & fluent (in a language) & \\
    ペラペラ & な & fluent (speaking a foreign language) & also an adverb \\
    ネイティブ & な & native & also a noun \\
    \ruby{強大}{きょう|だい} & な & mighty/powerful & \\
    \ruby{天才的}{てん|さい|てき} & な & talented/gifted/prodigious/virtuoso/masterful & \\
    \ruby{頭}{あたま}がいい & い & bright/intelligent/clever/smart & \\
    \ruby{賢}{かしこ}い & い & wise/clever/smart  & \\
    % & & & \\
    \midrule
    \ruby{下手}{へ|た} & な & unskilful/poor/awkward & \\
    ヘタクソ & な & unskilled/clumsy/lousy/poor/awkward/shitty & (\ruby{下手糞}{へ|た|くそ}); derogatory \\
    \ruby{苦手}{にが|て} & な & not very good at & \\
    \ruby{無能}{む|のう} & の & incapable/incompetent/inefficient & \\
    \ruby{弱}{よわ}い & い & weak/frail/tender; unskilled & \\
    に\ruby{疎}{うと}い & い & ignorant/ill-informed/unfamiliar about & also in Table~\ref{tbl:appendix-vocab-adjectives-emotions} \\
    \ruby{弱小}{じゃく|しょう} & な & puniness; youth & \\
    \ruby{頭}{あたま}が\ruby{悪}{わる}い & い & slow/weak-headed/dumb & \\
    バカ & な & stupid/foolish/ridiculous & (\ruby{馬鹿}{ば|か}) \\
    アホ & な & foolish/idiotic/simplistic & (\ruby{阿呆}{あ|ほ}) \\
    \ruby{忘}{わす}れっぽい & い & forgetful & \\
    % & & & \\
    \midrule
    \midrule

    [お]やすい & い & easy & (\ruby{易}{やす}い) \\
    <v masu>やすい & い & easy to/likely to/have a tendency to <v masu> & (\ruby{易}{やす}い); \suffix \\
    \ruby{簡単}{かん|たん} & な & easy/simple & \\
    % & & & \\
    \midrule
    \ruby{難}{むずか}しい & い & difficult/troublesome/impossible (euphemism) & \\
    \ruby{大変}{たい|へん} & な & difficult/challenging & also an adverb, also in Table~\ref{tbl:appendix-vocab-adjectives-amounts-and-sizes} \\
    \ruby{無理}{む|り} & な & impossible/no way/unreasonable & \\
    ありえない & い & impossible & (あり\ruby{得}{え}ない); also in Table~\ref{appendix-vocab-tbl:appendix-vocab-adjectives-agreeability} \\
    % & & & \\
    \midrule
    めんどくさい & い & troublesome/bothersome/tiresome & (\ruby{面倒臭}{めん|ど|くさ}い) \\
    めんどい & い & troublesome/bothersome & (\ruby{面倒}{めん|どう}い); slang \\
    % & & & \\
    \midrule
    \midrule
    \ruby{否}{いな}めない & い & undeniable/cannot deny & \\
    % & & & \\
    \bottomrule
}


\subsection{Personalities}
% Help: \SetCell[r=2,c=2]{c,m} <content>, \cmidrule[l]{3-4}
% Help: colspec: X[ratio, horizontal alignment] columns grow to fit width=\linewidth
%                  negative ratios: shrink to fit content and may not grow to full ratio
% Help: colspec: l/c/r columns do not grow
\longtabse[0.75]  % scale factor
{Adjectives: personalities.}  % caption
{tbl:appendix-vocab-adjectives-personalities}  % label
{}  % outer specification options
{
    colspec={X[-3,l]X[-1,c]X[3,l]X[-3,l]},
    rowhead=1,
    % width=\linewidth,  % useful only with X columns
}  % inner specification options
{
    \toprule
    \textbf{Descriptor} & \textbf{Cat.} & \textbf{Meaning} & \textbf{Notes} \\
    \midrule
    \ruby{優}{やさ}しい & い & kind/affectionate/gentle (character) & speech; \href{https://ja.hinative.com/question_summaries/112079}{[HN]} \\
    \ruby{親切}{しん|せつ} & な & kind/generous/gentle (action) & formal; \href{https://ja.hinative.com/question_summaries/112079}{[HN]} \\
    \ruby{心安}{こころ|やす}い & い & friendly/familiar/intimate & \\
    \ruby{懐}{なつ}っこい & い & amiable/affable/likeable & \\
    \ruby{人懐}{ひと|なつ}っこい & い & friendly/amiable/affable/sociable/loving company; (of animals) talking kindly to people & \\
    % & & & \\
    \midrule
    \ruby{素直}{す|なお} & な & frank/upfront/candid/direct/honest (about one's feelings/thoughts) & \\
    % & & & \\
    \midrule
    ひどい & い & cruel/heartless/harsh/very bad/awful & (\ruby{酷}{ひど}い) \\
    \ruby{残酷}{ざん|こく} & な & cruel/brutal/ruthless/merciless/inhumane & \\
    % & & & \\
    \midrule
    \midrule
    \ruby{静}{しず}か & な & quiet/silent/calm/peaceful & \\
    \ruby{冷静}{れい|せい} & な & composed/calm/serene & \\
    \ruby{穏}{おだ}やか & な & peaceful/gentle/calm/mild/quiet & \\
    \ruby{気安}{き|やす}い & い & relaxed/familiar/friendly & \\
    \ruby{平気}{へい|き} & な & cool/calm/composed/unconcerned/nonchalant/unmoved/indifferent & also in Table~\ref{tbl:appendix-vocab-adjectives-health} \\
    % & & & \\
    \midrule
    \ruby{騒}{さわ}がしい & い & noisy/boisterous & \\
    うるさい & い & noisy/loud; annoying/persistent & (\ruby{煩}{うるさ}い) \\
    うざい & い & noisy/loud & slang \\
    % & & & \\
    \midrule
    \midrule
    \ruby{真面目}{ま|じ|め} & な & serious/sober/earnest/grave & \\
    hardworking, lazy... & & & \\
    % & & & \\
    \midrule
    \midrule
    おかしい & い & crazy/eccentric & (\ruby[g]{可笑}{おか}しい); also in Table~\ref{tbl:appendix-vocab-adjectives-knowledge-truth-and-reality} \\
    % & & & \\
    \midrule
    \midrule
    \ruby{慎重}{しん|ちょう} & な & careful/cautious/prudent & \\
    % & & & \\
    \midrule
    \ruby{軽率}{けい|そつ} & な & careless/rash/hasty/imprudent & \\
    \ruby{気}{き}にしない & い & not caring/not giving a damn & \\
    \ruby{屁}{へ}とも\ruby{思}{おも}わない & い & not giving a damn/not caring a bit & idiomatic \\
    いい\ruby{加減}{か|げん} & な & irresponsible/perfunctory/careless & also an adverb, also in Table~\ref{tbl:appendix-vocab-adjectives-amounts-and-sizes} \\
    % & & & \\
    \midrule
    \midrule
    \ruby{有名}{ゆう|めい} & な & famous & \\
    % & & & \\
    \midrule
    \midrule
    \ruby{単純}{たん|じゅん} & な & simple/uncomplicated; simple-minded/naive & \\

    [お]めでたい & い & naive/too good-natured/gullible/foolish/simple & ([お]\ruby{目出度}{め|で|た}い); also in Table~\ref{tbl:appendix-vocab-adjectives-emotions} \\
    \ruby{騙}{だま}されやすい & い & gullible/naive & (\ruby{騙}{だま}され\ruby{易}{やす}い) \\
    % & & & \\
    \midrule
    \ruby{複雑}{ふく|ざつ} & な & complex/complicated/intricate; mixed (feelings) & \\
    \ruby{受}{う}けやすい & い & susceptible/vulnerable/prone to & (\ruby{受}{う}け\ruby{易}{やす}い) \\
    \ruby{感じ}{かん|じ}やすい & い & sensitive/susceptible & (\ruby{感じ}{かん|じ}\ruby{易}{やす}い); also: センシティブ \\
    % & & & \\
    \midrule
    \midrule
    \ruby{熱}{ね}しやすい & い & excitable & (\ruby{熱}{ね}し\ruby{易}{やす}い) \\
    % & & & \\
    \midrule
    \ruby{飽}{あ}きやすい & い & easily bored/fickle/quick to lose interest & (\ruby{飽}{あ}き\ruby{易}{やす}い) \\
    \ruby{疲}{つか}れやすい & い & easily fatigued & (\ruby{疲}{つか}れ\ruby{易}{やす}い) \\
    % & & & \\
    \bottomrule
}


\subsection{Education and correctness}
% Help: \SetCell[r=2,c=2]{c,m} <content>, \cmidrule[l]{3-4}
% Help: colspec: X[ratio, horizontal alignment] columns grow to fit width=\linewidth
%                  negative ratios: shrink to fit content and may not grow to full ratio
% Help: colspec: l/c/r columns do not grow
\longtabse[0.75]  % scale factor
{Adjectives: education and correctness.}  % caption
{tbl:appendix-vocab-adjectives-education-and-correctness}  % label
{}  % outer specification options
{
    colspec={X[-3,l]X[-1,c]X[3,l]X[-3,l]},
    rowhead=1,
    % width=\linewidth,  % useful only with X columns
}  % inner specification options
{
    \toprule
    \textbf{Descriptor} & \textbf{Cat.} & \textbf{Meaning} & \textbf{Notes} \\
    \midrule
    \ruby{正}{ただ}しい & い & right/correct; proper/lawful & \\
    % & & & \\
    \bottomrule
}


\subsection{Knowledge, truth and reality}
% Help: \SetCell[r=2,c=2]{c,m} <content>, \cmidrule[l]{3-4}
% Help: colspec: X[ratio, horizontal alignment] columns grow to fit width=\linewidth
%                  negative ratios: shrink to fit content and may not grow to full ratio
% Help: colspec: l/c/r columns do not grow
\longtabse[0.75]  % scale factor
{Adjectives: knowledge, truth and reality.}  % caption
{tbl:appendix-vocab-adjectives-knowledge-truth-and-reality}  % label
{}  % outer specification options
{
    colspec={X[-3,l]X[-1,c]X[3,l]X[-3,l]},
    rowhead=1,
    % width=\linewidth,  % useful only with X columns
}  % inner specification options
{
    \toprule
    \textbf{Descriptor} & \textbf{Cat.} & \textbf{Meaning} & \textbf{Notes} \\
    \midrule
    \ruby{明}{あき}らか & な & obvious/clear/evident/definite & \href{https://ja.hinative.com/questions/4623371}{[HN]} \\
    \ruby{不明}{ふ|めい} & な unknown/unidentified/obscure/ambiguous/uncertain/unclear & \href{https://ja.hinative.com/questions/4623371}{[HN]} \\
    \ruby{紛}{まぎ}らわしい & い & ambiguous/equivocal/misleading/easy mixed up/confusing & \\
    \ruby{明確}{めい|かく} & な & clear/precise/definite/distinct & \href{https://ja.hinative.com/questions/4623371}{[HN]} \\
    \ruby{不明確}{ふ|めい|かく} & な & inaccurate/indistinct/imprecise/unclear/indecisive & \href{https://ja.hinative.com/questions/4623371}{[HN]} \\
    \ruby{曖昧}{あい|まい} & な & vague/ambiguous & \href{https://ja.hinative.com/questions/4623371}{[HN]} \\
    \ruby{確}{たし}か & な & certain/sure/definite; reliable/trustworthy/accurate/correct & \\
    \ruby{不確}{ふ|たし}か & な & uncertain/unclear/indefinite & \\
    % & & & \\
    \midrule
    \midrule
    \ruby{本当}{ほう|とう} & の & real/true/genuine/authentic & \href{https://ja.hinative.com/questions/21280744}{[HN]} \\
    \ruby{正常}{せい|じょう} & な & normal & \\
    まじ/マジ & な & serious/not joking & abbreviation \\
    がち & な & serious/earnest/honest/real/legit & slang \\
    % & & & \\
    \midrule
    おかしい & い & laughable/ridiculous/strange/weird/suspicious & (\ruby[g]{可笑}{おか}しい); also in Table~\ref{tbl:appendix-vocab-adjectives-personalities} \\
    \ruby{怪}{あや}しい & い & suspicious/dubious/dodgy; ominous (weather) & \\
    \ruby{変}{へん} & な & strange/odd/peculiar/weird/eccentric/funny/suspicious/fishy; unexpected & \\
    \ruby{黒}{くろ}い & い & suspicious/shady/evil & also in Table~\ref{tbl:appendix-vocab-adjectives-colours} \\
    \ruby{臭}{くさ}い & い & suspicious & also in Table~\ref{tbl:appendix-vocab-adjectives-appearance-and-style} \\
    \ruby{異常}{い|じょう} & な & abnormal/strange & \\
    \ruby{非常}{ひ|じょう} & な & extreme/great/extraordinary/unusual & also a noun \\
    \ruby{不思議}{ふ|し|ぎ} & な & strange/mysterious & \\
    \ruby{信}{しん}じられない & い & unbelievable/incredible & \\
    % & & & \\
    \midrule
    \midrule
    \ruby{正直}{しょう|じき} & な & honest/frank/candid & also an adverb \\
    \ruby{平等}{びょう|どう} & の & equal/impartial & also a noun \\
    % & & & \\
    \midrule
    untruthful & & & \\
    unequal & & & \\
    % & & & \\
    \bottomrule
}


\subsection{Courtesy}
% Help: \SetCell[r=2,c=2]{c,m} <content>, \cmidrule[l]{3-4}
% Help: colspec: X[ratio, horizontal alignment] columns grow to fit width=\linewidth
%                  negative ratios: shrink to fit content and may not grow to full ratio
% Help: colspec: l/c/r columns do not grow
\longtabse[0.75]  % scale factor
{Adjectives: courtesy.}  % caption
{tbl:appendix-vocab-adjectives-courtesy}  % label
{}  % outer specification options
{
    colspec={X[-3,l]X[-1,c]X[3,l]X[-3,l]},
    rowhead=1,
    % width=\linewidth,  % useful only with X columns
}  % inner specification options
{
    \toprule
    \textbf{Descriptor} & \textbf{Cat.} & \textbf{Meaning} & \textbf{Notes} \\
    \midrule

    [ご]\ruby{丁寧}{てい|ねい} & な & polite/courteous/civil; conscientious/thorough/careful & [honorific] \\
    % & & & \\
    \midrule
    \ruby{失礼}{しつ|れい} & な & discourteous/impolite & also a noun \\
    \ruby{無礼}{ぶ|れい} & な & rude/discourteous/insolent (stronger) & also a noun \\
    % & & & \\
    \midrule
    \midrule
    \ruby{速}{はや}い & い & fast & \\
    \ruby{高速}{こう|そく} & な & high-speed/rapid/express & also a noun \\
    \ruby{急速}{きゅう|そく} & な & rapid (progress) & \\
    \ruby{早}{はや}い & い & early/too early & \\
    % & & & \\
    \midrule
    \ruby{遅}{おそ}い & い & slow/late (in the day)/late (behind time) & \\
    \ruby{重}{おも}い & い & slow/sluggish/laggy & also in Table~\ref{tbl:appendix-vocab-adjectives-amounts-and-sizes} \\
    % & & & \\
    \bottomrule
}


\subsection{Conflict and resolution}
% Help: \SetCell[r=2,c=2]{c,m} <content>, \cmidrule[l]{3-4}
% Help: colspec: X[ratio, horizontal alignment] columns grow to fit width=\linewidth
%                  negative ratios: shrink to fit content and may not grow to full ratio
% Help: colspec: l/c/r columns do not grow
\longtabse[0.75]  % scale factor
{Adjectives: conflict and resolution.}  % caption
{tbl:appendix-vocab-adjectives-conflict-and-resolution}  % label
{}  % outer specification options
{
    colspec={X[-3,l]X[-1,c]X[3,l]X[-3,l]},
    rowhead=1,
    % width=\linewidth,  % useful only with X columns
}  % inner specification options
{
    \toprule
    \textbf{Descriptor} & \textbf{Cat.} & \textbf{Meaning} & \textbf{Notes} \\
    \midrule
    \ruby{不正}{ふ|せい} & な & unjust/unfair/dishonest/illegal & also a noun \\
    % & & & \\
    \bottomrule
}


\subsection{Taste and texture}
\emph{Read the main article for the five basic flavours on \href{https://cotoacademy.com/useful-words-describe-food-japanese-illustrated-guide/}{CTA}.}

\hl{More \href{https://cotoacademy.com/useful-words-describe-food-japanese-illustrated-guide/}{here}}

% Help: \SetCell[r=2,c=2]{c,m} <content>, \cmidrule[l]{3-4}
% Help: colspec: X[ratio, horizontal alignment] columns grow to fit width=\linewidth
%                  negative ratios: shrink to fit content and may not grow to full ratio
% Help: colspec: l/c/r columns do not grow
\longtabse[0.75]  % scale factor
{Adjectives: taste and texture.}  % caption
{tbl:appendix-vocab-adjectives-taste-and-texture}  % label
{}  % outer specification options
{
    colspec={X[-3,l]X[-1,c]X[3,l]X[-3,l]},
    rowhead=1,
    % width=\linewidth,  % useful only with X columns
}  % inner specification options
{
    \toprule
    \textbf{Descriptor} & \textbf{Cat.} & \textbf{Meaning} & \textbf{Notes} \\
    \midrule
    おいしい & い & good-tasting/delicious/tasty & (\ruby[g]{美味}{おい}しい) \\
    うまい & い & delicious & (\ruby[g]{美味}{うま}い/\ruby{旨}{うま}い \href{https://business-textbooks.com/umai/}{[SKJnKKS]}) \\
    うめぇ/うめえ/うめー & expression & delicious/skilled/good & colloquial \\
    % & & & \\
    \midrule
    \midrule
    \ruby{酸}{す}っぱい & い & sour & \\
    \ruby{甘}{あま}い & い & sweet & \\
    \ruby{苦}{にが}い & い & bitter & \\
    \ruby{辛}{から}い & い & spicy & \\
    \ruby{塩}{しょ}っぱい & い & salty & \\
    % & & & \\
    \midrule
    \midrule
    ふわふわ & の & soft/fluffy/spongy & \onomatopoeic; also an adverb \\
    % & & & \\
    \midrule
    \midrule
    きな\ruby{臭}{くさ}い & い & smelling burnt/scorched & also in Table~\ref{tbl:appendix-vocab-adjectives-emotions} \\
    % & & & \\
    \bottomrule
}


\subsection{Amounts and sizes}
% Help: \SetCell[r=2,c=2]{c,m} <content>, \cmidrule[l]{3-4}
% Help: colspec: X[ratio, horizontal alignment] columns grow to fit width=\linewidth
%                  negative ratios: shrink to fit content and may not grow to full ratio
% Help: colspec: l/c/r columns do not grow
\longtabse[0.75]  % scale factor
{Adjectives: amounts and sizes.}  % caption
{tbl:appendix-vocab-adjectives-amounts-and-sizes}  % label
{}  % outer specification options
{
    colspec={X[-3,l]X[-1,c]X[3,l]X[-3,l]},
    rowhead=1,
    % width=\linewidth,  % useful only with X columns
}  % inner specification options
{
    \toprule
    \textbf{Descriptor} & \textbf{Cat.} & \textbf{Meaning} & \textbf{Notes} \\
    \midrule
    でかい & い & huge/big/gargantuan & slang \\
    \ruby{大}{おお}きい & い & big/large/great & \\
    すごい & い & vast (in numbers)/to a great extent & (\ruby{凄}{すご}い) \\
    いい\ruby{加減}{か|げん} & な & already enough (experssing wish for something to end), see いい\ruby{加減}{か|げん}にする & also an adverb, also in Table~\ref{tbl:appendix-vocab-adjectives-personalities} \\
    \ruby{大}{だい}〜 & な & large/great/huge/major/important/serious/severe & \prefix \\
    \ruby{相当}{そう|とう} & な & considerable/substantial & \\
    % & & & \\
    \midrule
    \ruby{小}{ちい}さい & い & small/little/tiny & \\
    ちいちゃい & い & small/little/tiny & (\ruby{小}{ちい}ちゃい); slang \\
    ちっちゃい & い & tiny/wee & (\ruby{小}{ち}っちゃい); slang \\
    % & & & \\
    \midrule
    \midrule
    \ruby{高}{たか}い & い & high/tall; expensive & \\
    \ruby{低}{ひく}い & い & low/short & \\
    \ruby{安}{やす}い & い & cheap & \\
    % & & & \\
    \midrule
    \midrule
    \ruby{深}{ふか}い & い & deep; profound; dense/thick; close (relationship); intense/strong; late & \\
    \ruby{浅}{あさ}い & い & shallow/superficial; slight (wound); light (sleep); pale (colour); inadequate (knowledge); early/young (e.g.\ night/season) & \\
    \ruby{関係}{かん|けい}ない & い & unrelated/irrelevant & \\
    % & & & \\
    \midrule
    \midrule
    \ruby{重}{おも}い & い & heavy (weight/feeling) & also in Table~\ref{tbl:appendix-vocab-adjectives-courtesy} \\
    \ruby{軽}{かる}い & い & light (weight/feeling) & also in Table~\ref{tbl:appendix-vocab-adjectives-courtesy} \\
    % & & & \\
    \midrule
    \ruby{太}{ふと}い & い & thick; deep/sonorous (of voice) & \\
    \ruby{細}{ほそ}い & い & thin/slender; thin/sparse (of voice) & \\
    スリム & な & slim & \\
    % & & & \\
    \midrule
    \midrule
    \ruby{短}{みじか}い & い & short/brief (length) (spacial/temporal/detail) & \\
    \ruby{長}{なが}い & い & long (length) (spacial/temporal) & \\
    \ruby{末永}{すえ|なが}い & い & everlasting/permanent/very long/many years of & \\
    % & & & \\
    \midrule
    \ruby{近}{ちか}い & い & near (distance) (spacial/temporal/relationship/similarity) & \\
    \ruby{遠}{とお}い & い & far (distance) (spacial/temporal/relationship/similarity) & \\
    % & & & \\
    \midrule
    \midrule
    \ruby{広}{ひろ}い & い & wide/spacious/vast & \\\
    \ruby{狭}{せま}い & い & narrow/confined/cramped & \\
    % & & & \\
    \midrule
    \midrule
    \ruby{大量}{たい|りょう} & の & massive quantity & \href{https://ja.hinative.com/questions/15390763}{[HN]} \\
    \ruby{多}{おお}い & い & many/large quantity of (esp.\ countable); frequent & [GMN] \\
    \ruby{多量}{た|りょう} & の & much/large amount of (esp.\ uncountable) & \href{https://ja.hinative.com/questions/15390763}{[HN]}, [GMN], \href{https://dictionary.goo.ne.jp/thsrs/14242/meaning/m0u/\%E3\%81\%9F\%E3\%81\%8F\%E3\%81\%95\%E3\%82\%93/}{[goo]} \\
    \ruby{十分}{じゅう|ぶん} & な & enough/sufficient/plenty/adequate/satisfactory & also an adverb \\
    たくさん & の & a lot/lots/plenty/much/a great deal; enough/too much & (\ruby{沢山}{たく|さん}); also an adverb; \href{https://dictionary.goo.ne.jp/thsrs/14242/meaning/m0u/\%E3\%81\%9F\%E3\%81\%8F\%E3\%81\%95\%E3\%82\%93/}{[goo]} \\
    いっぱい & の & full/filled/overflowing & (\ruby{一杯}{いっ|ぱい}); also a noun and adverb; \href{https://dictionary.goo.ne.jp/thsrs/14242/meaning/m0u/\%E3\%81\%9F\%E3\%81\%8F\%E3\%81\%95\%E3\%82\%93/}{[goo]} \\
    % & & & \\
    \midrule
    \ruby{必要}{ひつ|よう} & な & essential/necessary & also a noun \\
    \ruby{必要}{ひつ|よう}ない & い & unnecessary/not needed & \\
    \ruby{不必要}{ふ|ひつ|よう} & な & unnecessary/needless & \\
    \ruby{不要}{ふ|よう} & の & unnecessary/unneeded & \\
    \ruby{結構}{けっ|こう} & な & not needing any more (``I'm fine/no thank you'')/sufficient/enough & also in Table~\ref{tbl:appendix-vocab-adjectives-agreeability} \\
    \ruby{不用}{ふ|よう} & の & disused/unused & \\
    % & & & \\
    \midrule
    \ruby{少量}{しょう|りょう} & の & small quantity & \\
    \ruby{少}{すく}ない & い & few/a little/scarce/insufficient; seldom & \\
    % & & & \\
    \midrule
    \midrule
    \ruby{久}{ひさ}しい & い & long (time that has passed)/old (story) & \\
    \ruby{久}{ひさ}しぶり & の & long time (since the last time) & \\
    % & & & \\
    \midrule
    \midrule
    \ruby{深刻}{しん|こく} & な & serious/severe/grave (of a crisis) & \\
    \ruby{重}{おも}い & い & serious/severe/critical (punishment/illness) & also in Table~\ref{tbl:appendix-vocab-adjectives-courtesy} \\
    \ruby{軽}{かる}い & い & non-serious/minor/unimportant/trivial (punishment/illness) & also in Table~\ref{tbl:appendix-vocab-adjectives-courtesy} \\
    % & & & \\
    \ruby{大変}{たい|へん} & な & serious/dreadful/terrible & also an adverb, also in Table~\ref{tbl:appendix-vocab-adjectives-ability} \\
    % & & & \\
    \bottomrule
}

\subsection{Change}
% Help: \SetCell[r=2,c=2]{c,m} <content>, \cmidrule[l]{3-4}
% Help: colspec: X[ratio, horizontal alignment] columns grow to fit width=\linewidth
%                  negative ratios: shrink to fit content and may not grow to full ratio
% Help: colspec: l/c/r columns do not grow
\longtabse[0.75]  % scale factor
{Adjectives: change.}  % caption
{tbl:appendix-vocab-adjectives-change}  % label
{}  % outer specification options
{
    colspec={X[-3,l]X[-1,c]X[3,l]X[-3,l]},
    rowhead=1,
    % width=\linewidth,  % useful only with X columns
}  % inner specification options
{
    \toprule
    \textbf{Descriptor} & \textbf{Cat.} & \textbf{Meaning} & \textbf{Notes} \\
    \midrule
    \ruby{変}{か}わらない & い & constant/invariant & \\
    % & & & \\
    \bottomrule
}


\end{document}
