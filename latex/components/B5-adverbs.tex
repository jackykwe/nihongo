\documentclass[../nihongo-gakushuu-kyouzai.tex]{subfiles}
\begin{document}
\appendix
\setcounter{section}{2}
\setcounter{subsection}{4}

\subsection{\ruby{副詞}{ふく|し}と\ruby{接続詞}{せつ|ぞく|し} (adverbs and conjunctions)}
Adverbs modify both verbs and adjectives. They may also modify entire noun phrases or sentences.


\subsubsection{Grammatical}
% Help: \SetCell[r=2,c=2]{c,m} <content>, \cmidrule[l]{3-4}
% Help: colspec: X[ratio, horizontal alignment] columns grow to fit width=\linewidth
%                  negative ratios: shrink to fit content and may not grow to full ratio
% Help: colspec: l/c/r columns do not grow
\longtabse[0.75]  % scale factor
{Adverbs: grammatical.}  % caption
{tbl:appendix-vocab-adverbs-grammatical}  % label
{}  % outer specification options
{
    colspec={X[-3,l]X[3,l]X[-3,l]},
    rowhead=1,
    % width=\linewidth,  % useful only with X columns
}  % inner specification options
{
    \toprule
    \textbf{Modifier} & \textbf{Meaning} & \textbf{Notes} \\
    \midrule
    \ruby{例}{たと}えば & for example/for instance & \\
    つまり & in short/in other words & (\ruby{詰}{つ}まり) \\
    \midrule
    まず & firstly & (\ruby{先}{ま}ず) \\
    \ruby{次}{つ}いで & secondly/next/subsequently & also a \conjunction \\
    {CだからE\\{}[CですからE]} & therefore (to speaker, E is a natural consequence of C; strong expression of speaker's attitude) & {\conjunction; also an expression; \href{https://www.youtube.com/watch?v=DSYc2BQrJEY}{[MCJ]}\\{}[polite]} \\
    CそれでE & (objective そ) therefore (general cause-and-effect; E must have certainly happened (either past or now)) & \conjunction; \href{https://dictionary.goo.ne.jp/thsrs/16809/meaning/m0u/}{[goo]}, \href{https://www.youtube.com/watch?v=DSYc2BQrJEY}{[MCJ]} \\
    CそこでE & (objective そ) therefore (C is problem/situation, E is action taken to solve/improve/advance) & \conjunction; \href{https://dictionary.goo.ne.jp/thsrs/16809/meaning/m0u/}{[goo]}, \href{https://www.youtube.com/watch?v=DSYc2BQrJEY}{[MCJ]} \\
    CするとE & thereupon (E happens \emph{immediately} after C) & \conjunction; \href{https://www.youtube.com/watch?v=DSYc2BQrJEY}{[MCJ]} \\
    % & & \\
    \bottomrule
}


\subsubsection{Intensity modifiers}
% Help: \SetCell[r=2,c=2]{c,m} <content>, \cmidrule[l]{3-4}
% Help: colspec: X[ratio, horizontal alignment] columns grow to fit width=\linewidth
%                  negative ratios: shrink to fit content and may not grow to full ratio
% Help: colspec: l/c/r columns do not grow
\longtabse[0.75]  % scale factor
{Adverbs: intensity modifiers.}  % caption
{tbl:appendix-vocab-adverbs-intensity}  % label
{}  % outer specification options
{
    colspec={X[-3,l]X[3,l]X[-3,l]},
    rowhead=1,
    % width=\linewidth,  % useful only with X columns
}  % inner specification options
{
    \toprule
    \textbf{Modifier} & \textbf{Meaning} & \textbf{Notes} \\
    \midrule
    \ruby{全然}{ぜん|ぜん}<negative> & not at all & \\
    % & & \\
    \midrule
    とても<negative> & not at all/simply cannot & \\
    あまり<negative> & not very & (\ruby{余}{あま}り); slightly formal \href{https://hinative.com/questions/19606346}{[HN1]}, \href{https://ja.hinative.com/questions/19223174}{[HN2]} \\
    \ruby{別}{べつ}に<negative> & not particularly (nuance: not interested) & slightly informal, can be rude; \href{https://hinative.com/questions/19606346}{[HN1]}, \href{https://ja.hinative.com/questions/19223174}{[HN2]} \\
    % & & \\
    \midrule
    % <so>のあまり<verb>& so much <so> that you <verb> & 嬉しさのあまり\ruby{泣}{な}いた。\\
    \ruby{少}{すこ}し & somewhat/slightly/a little & \\
    ちょっと & a bit/slightly/somewhat/quite; just a minute & \\
    % & & \\
    \midrule
    かなり & quite/considerably/pretty & \\
    \ruby{大}{だい}〜 & large/big/great/severe & \prefix. \htc; technically な-adj/noun \\
    すごく & very/immensely/awfully & (\ruby{凄}{すご}く) \\
    \ruby{大変}{たい|へん} & very/greatly/terribly/awfully & also an adjective \\
    そりゃ & very/extremely & \\
    とても & very/exceedingly/awfully & \\
    % & & \\
    \midrule
    \ruby{全然}{ぜん|ぜん} & extremely/very & e.g.\ 「\ruby{全然}{ぜん|ぜん}いいよ」 \\
    \ruby{全部}{ぜん|ぶ} & entirely/wholly/altogether & also a noun \\
    % & & \\
    \midrule
    \midrule
    たくさん & a lot/lots/plenty/much/a great deal; enough/too much & (\ruby{沢山}{たく|さん}); also an adjective \\
    いっぱい & fully/as much as possible; a lot/many; all of & (\ruby{一杯}{いっ|ぱい}); also a noun and adjective \\
    % & & \\
    \midrule
    \midrule
    もしや & possibly/perhaps/by some chance & (\ruby{若}{も}しや) \\
    もしかし & maybe/perhaps/by some chance & (\ruby{若}{も}しかし) \\
    \ruby{多分}{た|ぶん} & probably/perhaps & \\
    まず & probably/most likely/almost certainly & (\ruby{先}{ま}ず) \\
    % & & \\
    \midrule
    \ruby{確}{たし}かに & certainly/for sure/indeed/really & \\
    % & & \\
    \bottomrule
}


\subsubsection{Time}
% Help: \SetCell[r=2,c=2]{c,m} <content>, \cmidrule[l]{3-4}
% Help: colspec: X[ratio, horizontal alignment] columns grow to fit width=\linewidth
%                  negative ratios: shrink to fit content and may not grow to full ratio
% Help: colspec: l/c/r columns do not grow
\longtabse[0.75]  % scale factor
{Adverbs: time.}  % caption
{tbl:appendix-vocab-adverbs-time}  % label
{}  % outer specification options
{
    colspec={X[-3,l]X[3,l]X[-3,l]},
    rowhead=1,
    % width=\linewidth,  % useful only with X columns
}  % inner specification options
{
    \toprule
    \textbf{Modifier} & \textbf{Meaning} & \textbf{Notes} \\
    \midrule
    さっさと & immediately/without delay/hurriedly/quickly & \\
    \ruby{早}{はや}く & early/soon/quickly/swiftly/rapidly & \\
    % & & \\
    \midrule
    \ruby{遅}{おそ}く & late/slowly & \\
    % & & \\
    \midrule
    \midrule
    もう & already; not any more/longer; again/another & again/another: used with counting 1 \\
    % & & \\
    \midrule
    ずっと & the whole time/continuously; much (more); (by) far & \\
    いつも & always & (\ruby[g]{何時}{いつ}も) \\
    % & & \\
    \midrule
    \midrule
    これから & from now on/in the future; from here & also a noun \\
    % & & \\
    \bottomrule
}


\subsubsection{Attitude}
% Help: \SetCell[r=2,c=2]{c,m} <content>, \cmidrule[l]{3-4}
% Help: colspec: X[ratio, horizontal alignment] columns grow to fit width=\linewidth
%                  negative ratios: shrink to fit content and may not grow to full ratio
% Help: colspec: l/c/r columns do not grow
\longtabse[0.75]  % scale factor
{Adverbs: attitude.}  % caption
{tbl:appendix-vocab-adverbs-attitude}  % label
{}  % outer specification options
{
    colspec={X[-3,l]X[3,l]X[-3,l]},
    rowhead=1,
    % width=\linewidth,  % useful only with X columns
}  % inner specification options
{
    \toprule
    \textbf{Modifier} & \textbf{Meaning} & \textbf{Notes} \\
    \midrule
    ぶらぶら & (walking) leisurely/aimlessly & onomatopoeic, also a verb \\
    % & & \\
    \midrule
    ちゃんと & diligently/seriously/earnestly; properly/perfectly/exactly/regularly; quickly & onomatopoeic \\
    \ruby{大切}{たい|せつ}に & carefully/with great care & \\
    % & & \\
    \midrule
    \midrule
    \ruby{普通}{ふ|つう}に & normally/ordinarily/usually/generally/commonly & \\
    % & & \\
    \midrule
    \midrule
    \ruby{本当}{ほん|とう}に/\ruby{本当}{ほん|と}に & really/truly & \\
    \ruby{正直}{しょう|じき} & honestly/frankly & also an adjective \\
    % & & \\
    \midrule
    \midrule
    \ruby{別}{べつ}に & separately/additionally/extra & \\
    % & & \\
    \bottomrule
}


\subsubsection{Emotions}
% Help: \SetCell[r=2,c=2]{c,m} <content>, \cmidrule[l]{3-4}
% Help: colspec: X[ratio, horizontal alignment] columns grow to fit width=\linewidth
%                  negative ratios: shrink to fit content and may not grow to full ratio
% Help: colspec: l/c/r columns do not grow
\longtabse[0.75]  % scale factor
{Adverbs: emotions.}  % caption
{tbl:appendix-vocab-adverbs-emotions}  % label
{}  % outer specification options
{
    colspec={X[-3,l]X[3,l]X[-3,l]},
    rowhead=1,
    % width=\linewidth,  % useful only with X columns
}  % inner specification options
{
    \toprule
    \textbf{Modifier} & \textbf{Meaning} & \textbf{Notes} \\
    \midrule
    ドキドキ & thump-thump/bang-bang/pit-a-pat/pitter-patter & onomatopoeic; also a verb \\
    % & & \\
    \bottomrule
}


\subsubsection{Appearance and style}
% Help: \SetCell[r=2,c=2]{c,m} <content>, \cmidrule[l]{3-4}
% Help: colspec: X[ratio, horizontal alignment] columns grow to fit width=\linewidth
%                  negative ratios: shrink to fit content and may not grow to full ratio
% Help: colspec: l/c/r columns do not grow
\longtabse[0.75]  % scale factor
{Adverbs: appearance and style.}  % caption
{tbl:appendix-vocab-adverbs-appearance-and-style}  % label
{}  % outer specification options
{
    colspec={X[-3,l]X[3,l]X[-3,l]},
    rowhead=1,
    % width=\linewidth,  % useful only with X columns
}  % inner specification options
{
    \toprule
    \textbf{Modifier} & \textbf{Meaning} & \textbf{Notes} \\
    \midrule
    こう & in this way (closer to speaker) & \\
    そう & in that way (closer to listener) & also an interjection \\
    ああ & in that way (distant) & \\
    こんあふうに & approximately in this way (closer to speaker) & (こんな\ruby{風}{ふう}に) \\
    そんあふうに & approximately in that way (closer to listener) & (そんな\ruby{風}{ふう}に) \\
    あんあふうに & approximately in that way (distant) & (あんな\ruby{風}{ふう}に) \\
    % & & \\
    \bottomrule
}


\subsubsection{Interaction}
% Help: \SetCell[r=2,c=2]{c,m} <content>, \cmidrule[l]{3-4}
% Help: colspec: X[ratio, horizontal alignment] columns grow to fit width=\linewidth
%                  negative ratios: shrink to fit content and may not grow to full ratio
% Help: colspec: l/c/r columns do not grow
\longtabse[0.75]  % scale factor
{Adverbs: interaction.}  % caption
{tbl:appendix-vocab-adverbs-interaction}  % label
{}  % outer specification options
{
    colspec={X[-3,l]X[3,l]X[-3,l]},
    rowhead=1,
    % width=\linewidth,  % useful only with X columns
}  % inner specification options
{
    \toprule
    \textbf{Modifier} & \textbf{Meaning} & \textbf{Notes} \\
    \midrule
    \ruby{久}{ひさ}しぶりに & for the first time in a while/after a long time & \\
    % & & \\
    \midrule
    \ruby{一緒}{いっ|しょ} & together/at the same time; identical & \\
    <with>と\ruby{一緒}{いっ|しょ}に<verb> & together with & \\
    % & & \\
    \bottomrule
}


\subsubsection{Taste and texture}
% Help: \SetCell[r=2,c=2]{c,m} <content>, \cmidrule[l]{3-4}
% Help: colspec: X[ratio, horizontal alignment] columns grow to fit width=\linewidth
%                  negative ratios: shrink to fit content and may not grow to full ratio
% Help: colspec: l/c/r columns do not grow
\longtabse[0.75]  % scale factor
{Adverbs: taste and texture.}  % caption
{tbl:appendix-vocab-adverbs-taste-and-texture}  % label
{}  % outer specification options
{
    colspec={X[-3,l]X[3,l]X[-3,l]},
    rowhead=1,
    % width=\linewidth,  % useful only with X columns
}  % inner specification options
{
    \toprule
    \textbf{Modifier} & \textbf{Meaning} & \textbf{Notes} \\
    \midrule
    ふわふわ & lightly/buoyantly & onomatopoeic, also an adjective \\
    % & & \\
    \midrule
    \midrule
    ぷにぷに & squishy/springy/bouncy (chubby when used on person) & onomatopoeic \\
    がりがり & hard/crunchy (of muscles, when used on person) & onomatopoeic \\
    % & & \\
    \bottomrule
}


\subsubsection{Amounts and sizes}
% Help: \SetCell[r=2,c=2]{c,m} <content>, \cmidrule[l]{3-4}
% Help: colspec: X[ratio, horizontal alignment] columns grow to fit width=\linewidth
%                  negative ratios: shrink to fit content and may not grow to full ratio
% Help: colspec: l/c/r columns do not grow
\longtabse[0.75]  % scale factor
{Adverbs: amounts and sizes.}  % caption
{tbl:appendix-vocab-adverbs-amounts-and-sizes}  % label
{}  % outer specification options
{
    colspec={X[-3,l]X[3,l]X[-3,l]},
    rowhead=1,
    % width=\linewidth,  % useful only with X columns
}  % inner specification options
{
    \toprule
    \textbf{Modifier} & \textbf{Meaning} & \textbf{Notes} \\
    \midrule
    どんなに & to what extent/amount & \\
    こんなに & to this extent/amount & \\
    そんなに & to that extent/amount & \\
    あんなに & to that extent/amount (distant memory) & \\
    % & & \\
    \bottomrule
}


\end{document}
