\documentclass[../nihongo-gakushuu-kyouzai.tex]{subfiles}
\begin{document}
\appendix
\setcounter{section}{1}
\section{Vocabulary mega summary}
\tableofcontents


\subsection{Interjections and expressions}
\subsubsection{Greetings}
\begin{center}
\resizebox{\linewidth}{!}{%
% Help: \multicolumn{2}{c}{}, \multirow{2}{*}{}, cmidrule(l){3-5}
\begin{tabular}{@{}lll@{}}
    \toprule
    \textbf{Interjection} & \textbf{Meaning} & \textbf{Notes} \\ \midrule
    おはよう & good morning & \\
    おはようございます & good morning & polite \\
    お\ruby{休}{やす}み[なさい] & good night & \\
    ようこそ & welcome & \\
    いらっしゃいませ & welcome (in shops and restaurants) & \\
    ごちそうさま & thank you for the meal & \\
    ごちそうさまでした & thank you for the meal & polite \\
    もしもし & hello (on phone)/excuse me (calling out to someone) & \\
    お\ruby{待}{ま}たせしました & thank you for waiting/sorry to have kept you waiting & polite \\
    \ruby{失礼}{しつ|れい}します & goodbye/excuse me (while I take my leave) & polite \\
    \ruby{失礼}{しつ|れい}しました & I'm sorry/excuse me/my apologies & polite \\
    またね & bye/see you later & \\
    お\ruby{久}{ひさ}しぶり & long time no see & polite \\
    \ruby{行}{い}ってくる & I'm off/see you later & \\
    \ruby{行}{い}ってきます & I'm off/see you later & \\
    ただいま & I'm home/I'm back & \\
    お\ruby{帰}{かえ}り[なさい] & welcome home & \\
    & & \\
    % & & \\
    \bottomrule
\end{tabular}%
}
\captionof{table}{Interections and expressions: greetings. \hl{TO ORGANISE}}
\label{tbl:appendix-vocab-interjections-and-expressions-greetings}
\end{center}

\subsubsection{Exclamations}
\begin{center}
\resizebox{\linewidth}{!}{%
% Help: \multicolumn{2}{c}{}, \multirow{2}{*}{}, cmidrule(l){3-5}
\begin{tabular}{@{}lll@{}}
    \toprule
    \textbf{Interjection} & \textbf{Meaning} & \textbf{Notes} \\ \midrule
    はい & yes/that is correct/I'm here/pardon? & \\
    うん & yes/yeah/mhmm & \\
    \ruby{大丈夫}{だい|じょう|ぶ} & no thanks/I'm good & \\
    いえ/いいえ & no & \\
    ううん/うーん & um/well/no & \\
    どうも & thanks (abbreviation) & \\
    どうもありがと & thank you very much & \\
    \ruby{何}{なに}か & what (are you trying to say/do you mean)? & \\
    \ruby{助}{たす}けて & help! & \\
    \ruby{失礼}{しつ|れい}しました & I'm sorry/excuse me/my apologies & polite \\
    お\ruby{願}{ねが}いします & please & polite (humble) \\
    どういたしまして & you're welcome/don't mention it/not at all/my pleasure & \\
    なんでよ & why? why not? what's wrong? & \\
    やだ/ヤダ & no way/not a chance & feminine/childish \\
    もう & jeez/come on & \\
    \ruby{気}{き}にするな & don't worry about it/nevermind & \\
    こちらこそ & it is I who should say so & \\
    わかる & I know/I think so too & \\
    わかる? & Do you know?/do you think so too? & \\
    \ruby{悪}{わる}い & my bad/sorry & also an adjective \\
    \ruby{危}{あぶ}ない & watch out!/look out!/be careful! \\
    & & \\
    & & \\
    % & & \\
    \bottomrule
\end{tabular}%
}
\captionof{table}{Interections and expressions: exclamations. \hl{TO ORGANISE}}
\label{tbl:appendix-vocab-interjections-and-expressions-exclamations}
\end{center}


\subsubsection{Sentence builders}
\begin{center}
\resizebox{\linewidth}{!}{%
% Help: \multicolumn{2}{c}{}, \multirow{2}{*}{}, cmidrule(l){3-5}
\begin{tabular}{@{}lll@{}}
    \toprule
    \textbf{Interjection} & \textbf{Meaning} & \textbf{Notes} \\ \midrule
    <basis>からするど & judging from/on the basis of/from the point of view of & \\
    <basis>からしたら & judging from/on the basis of/from the point of view of & \\
    <basis>からすれば & judging from/on the basis of/from the point of view of & \\
    & & \\
    & & \\
    & & \\
    & & \\
    % & & \\
    \bottomrule
\end{tabular}%
}
\captionof{table}{Interections and expressions: sentence builders.}
\label{tbl:appendix-vocab-interjections-and-expressions-sentence-builders}
\end{center}


\begin{multicols}{2}
[
\subsection{Nouns}
]


\subsubsection{Counting (generic)}
\begin{center}
\resizebox{\linewidth}{!}{%
% Help: \multicolumn{2}{c}{}, \multirow{2}{*}{}, cmidrule(l){3-5}
\begin{tabular}{@{}lll@{}}
    \toprule
    \textbf{Name} & \textbf{Meaning} & \textbf{Notes} \\ \midrule
    ichi ni san... & & \\
    \ruby{一}{ひと}つ/\ruby{1}{ひと}つ & one & \\
    \ruby{二}{ふた}つ/\ruby{2}{ふた}つ & two & \\
    \ruby{三}{みっ}つ/\ruby{3}{みっ}つ & three & \\
    & & \\
    \ruby{全部}{ぜん|ぶ} & all/entire/whole/altogether & also an adverb \\
    & & \\
    % & & \\
    \bottomrule
\end{tabular}%
}
\captionof{table}{Nouns: counting (generic).}
\label{tbl:appendix-vocab-nouns-counting-generic}
\end{center}



\subsubsection{Counting people}
\begin{center}
\resizebox{\linewidth}{!}{%
% Help: \multicolumn{2}{c}{}, \multirow{2}{*}{}, cmidrule(l){3-5}
\begin{tabular}{@{}lll@{}}
    \toprule
    \textbf{Name} & \textbf{Meaning} & \textbf{Notes} \\ \midrule
    \ruby{一人}{ひと|り}/\ruby{1人}{ひと|り} & One person; being alone/single & \\
    & & \\
    \ruby{一緒}{いっ|しょ} & together & \\
    % & & \\
    \bottomrule
\end{tabular}%
}
\captionof{table}{Nouns: counting people.}
\label{tbl:appendix-vocab-nouns-counting-people}
\end{center}


\subsubsection{Counting age}
\begin{center}
\resizebox{\linewidth}{!}{%
% Help: \multicolumn{2}{c}{}, \multirow{2}{*}{}, cmidrule(l){3-5}
\begin{tabular}{@{}lll@{}}
    \toprule
    \textbf{Name} & \textbf{Meaning} & \textbf{Notes} \\ \midrule
    & & \\
    & & \\
    % & & \\
    \bottomrule
\end{tabular}%
}
\captionof{table}{Nouns: counting age.}
\label{tbl:appendix-vocab-nouns-counting-age}
\end{center}


\subsubsection{Counting thin/flat things}
\begin{center}
\resizebox{\linewidth}{!}{%
% Help: \multicolumn{2}{c}{}, \multirow{2}{*}{}, cmidrule(l){3-5}
\begin{tabular}{@{}lll@{}}
    \toprule
    \textbf{Name} & \textbf{Meaning} & \textbf{Notes} \\ \midrule
    & & \\
    & & \\
    % & & \\
    \bottomrule
\end{tabular}%
}
\captionof{table}{Nouns: counting thin/flat things.}
\label{tbl:appendix-vocab-nouns-counting-thin-flat-things}
\end{center}


\subsubsection{Counting thin long things (bottles)}
\begin{center}
\resizebox{\linewidth}{!}{%
% Help: \multicolumn{2}{c}{}, \multirow{2}{*}{}, cmidrule(l){3-5}
\begin{tabular}{@{}lll@{}}
    \toprule
    \textbf{Name} & \textbf{Meaning} & \textbf{Notes} \\ \midrule
    & & \\
    & & \\
    % & & \\
    \bottomrule
\end{tabular}%
}
\captionof{table}{Nouns: counting thin long things (bottles).}
\label{tbl:appendix-vocab-nouns-counting-thin-long-things-bottles}
\end{center}


\subsubsection{Counting drinks (cups/glasses)}
\begin{center}
\resizebox{\linewidth}{!}{%
% Help: \multicolumn{2}{c}{}, \multirow{2}{*}{}, cmidrule(l){3-5}
\begin{tabular}{@{}lll@{}}
    \toprule
    \textbf{Name} & \textbf{Meaning} & \textbf{Notes} \\ \midrule
    & & \\
    & & \\
    % & & \\
    \bottomrule
\end{tabular}%
}
\captionof{table}{Nouns: counting drinks (cups/glasses).}
\label{tbl:appendix-vocab-nouns-counting-drinks-cups-glasses}
\end{center}


\subsubsection{Counting machines/vehicles}
\begin{center}
\resizebox{\linewidth}{!}{%
% Help: \multicolumn{2}{c}{}, \multirow{2}{*}{}, cmidrule(l){3-5}
\begin{tabular}{@{}lll@{}}
    \toprule
    \textbf{Name} & \textbf{Meaning} & \textbf{Notes} \\ \midrule
    & & \\
    & & \\
    % & & \\
    \bottomrule
\end{tabular}%
}
\captionof{table}{Nouns: counting machines/vehicles.}
\label{tbl:appendix-vocab-nouns-counting-machines-vehicles}
\end{center}


\subsubsection{Counting books}
\begin{center}
\resizebox{\linewidth}{!}{%
% Help: \multicolumn{2}{c}{}, \multirow{2}{*}{}, cmidrule(l){3-5}
\begin{tabular}{@{}lll@{}}
    \toprule
    \textbf{Name} & \textbf{Meaning} & \textbf{Notes} \\ \midrule
    & & \\
    & & \\
    % & & \\
    \bottomrule
\end{tabular}%
}
\captionof{table}{Nouns: counting books.}
\label{tbl:appendix-vocab-nouns-counting-books}
\end{center}


\subsubsection{Counting clothes}
\begin{center}
\resizebox{\linewidth}{!}{%
% Help: \multicolumn{2}{c}{}, \multirow{2}{*}{}, cmidrule(l){3-5}
\begin{tabular}{@{}lll@{}}
    \toprule
    \textbf{Name} & \textbf{Meaning} & \textbf{Notes} \\ \midrule
    & & \\
    & & \\
    % & & \\
    \bottomrule
\end{tabular}%
}
\captionof{table}{Nouns: counting clothes.}
\label{tbl:appendix-vocab-nouns-counting-clothes}
\end{center}


\subsubsection{Counting small things}
\begin{center}
\resizebox{\linewidth}{!}{%
% Help: \multicolumn{2}{c}{}, \multirow{2}{*}{}, cmidrule(l){3-5}
\begin{tabular}{@{}lll@{}}
    \toprule
    \textbf{Name} & \textbf{Meaning} & \textbf{Notes} \\ \midrule
    \ruby{個}{こ} & Counter for small things & \\
    \ruby{個々}{こ|こ} & individual & e.g.\ 「そのクラスの個々のメンバー\dots」, like CN's 个个\\
    & & \\
    & & \\
    % & & \\
    \bottomrule
\end{tabular}%
}
\captionof{table}{Nouns: counting small things.}
\label{tbl:appendix-vocab-nouns-counting-small-things}
\end{center}


\subsubsection{Counting shoes and socks}
\begin{center}
\resizebox{\linewidth}{!}{%
% Help: \multicolumn{2}{c}{}, \multirow{2}{*}{}, cmidrule(l){3-5}
\begin{tabular}{@{}lll@{}}
    \toprule
    \textbf{Name} & \textbf{Meaning} & \textbf{Notes} \\ \midrule
    & & \\
    & & \\
    % & & \\
    \bottomrule
\end{tabular}%
}
\captionof{table}{Nouns: counting shoes and socks.}
\label{tbl:appendix-vocab-nouns-counting-shoes-and-socks}
\end{center}


\subsubsection{Counting houses}
\begin{center}
\resizebox{\linewidth}{!}{%
% Help: \multicolumn{2}{c}{}, \multirow{2}{*}{}, cmidrule(l){3-5}
\begin{tabular}{@{}lll@{}}
    \toprule
    \textbf{Name} & \textbf{Meaning} & \textbf{Notes} \\ \midrule
    & & \\
    & & \\
    % & & \\
    \bottomrule
\end{tabular}%
}
\captionof{table}{Nouns: counting houses.}
\label{tbl:appendix-vocab-nouns-counting-houses}
\end{center}



\subsubsection{Counting floors}
\begin{center}
\resizebox{\linewidth}{!}{%
% Help: \multicolumn{2}{c}{}, \multirow{2}{*}{}, cmidrule(l){3-5}
\begin{tabular}{@{}lll@{}}
    \toprule
    \textbf{Name} & \textbf{Meaning} & \textbf{Notes} \\ \midrule
    & & \\
    & & \\
    % & & \\
    \bottomrule
\end{tabular}%
}
\captionof{table}{Nouns: counting floors.}
\label{tbl:appendix-vocab-nouns-counting-floors}
\end{center}



\subsubsection{Counting small animals}
\begin{center}
\resizebox{\linewidth}{!}{%
% Help: \multicolumn{2}{c}{}, \multirow{2}{*}{}, cmidrule(l){3-5}
\begin{tabular}{@{}lll@{}}
    \toprule
    \textbf{Name} & \textbf{Meaning} & \textbf{Notes} \\ \midrule
    & & \\
    & & \\
    % & & \\
    \bottomrule
\end{tabular}%
}
\captionof{table}{Nouns: counting small animals.}
\label{tbl:appendix-vocab-nouns-counting-small-animals}
\end{center}


\subsubsection{Counting big animals}
\begin{center}
\resizebox{\linewidth}{!}{%
% Help: \multicolumn{2}{c}{}, \multirow{2}{*}{}, cmidrule(l){3-5}
\begin{tabular}{@{}lll@{}}
    \toprule
    \textbf{Name} & \textbf{Meaning} & \textbf{Notes} \\ \midrule
    & & \\
    & & \\
    % & & \\
    \bottomrule
\end{tabular}%
}
\captionof{table}{Nouns: counting big animals.}
\label{tbl:appendix-vocab-nouns-counting-big-animals}
\end{center}



\subsubsection{Counting hours}
\begin{center}
\resizebox{\linewidth}{!}{%
% Help: \multicolumn{2}{c}{}, \multirow{2}{*}{}, cmidrule(l){3-5}
\begin{tabular}{@{}lll@{}}
    \toprule
    \textbf{Name} & \textbf{Meaning} & \textbf{Notes} \\ \midrule
    & & \\
    & & \\
    % & & \\
    \bottomrule
\end{tabular}%
}
\captionof{table}{Nouns: counting hours.}
\label{tbl:appendix-vocab-nouns-counting-hours}
\end{center}


\subsubsection{Counting minutes}
\begin{center}
\resizebox{\linewidth}{!}{%
% Help: \multicolumn{2}{c}{}, \multirow{2}{*}{}, cmidrule(l){3-5}
\begin{tabular}{@{}lll@{}}
    \toprule
    \textbf{Name} & \textbf{Meaning} & \textbf{Notes} \\ \midrule
    & & \\
    & & \\
    % & & \\
    \bottomrule
\end{tabular}%
}
\captionof{table}{Nouns: counting minutes.}
\label{tbl:appendix-vocab-nouns-counting-minutes}
\end{center}


\subsubsection{Counting seconds}
\begin{center}
\resizebox{\linewidth}{!}{%
% Help: \multicolumn{2}{c}{}, \multirow{2}{*}{}, cmidrule(l){3-5}
\begin{tabular}{@{}lll@{}}
    \toprule
    \textbf{Name} & \textbf{Meaning} & \textbf{Notes} \\ \midrule
    & & \\
    & & \\
    % & & \\
    \bottomrule
\end{tabular}%
}
\captionof{table}{Nouns: counting seconds.}
\label{tbl:appendix-vocab-nouns-counting-seconds}
\end{center}


\subsubsection{Counting positions}
\begin{center}
\resizebox{\linewidth}{!}{%
% Help: \multicolumn{2}{c}{}, \multirow{2}{*}{}, cmidrule(l){3-5}
\begin{tabular}{@{}lll@{}}
    \toprule
    \textbf{Name} & \textbf{Meaning} & \textbf{Notes} \\ \midrule
    & & \\
    & & \\
    % & & \\
    \bottomrule
\end{tabular}%
}
\captionof{table}{Nouns: counting positions.}
\label{tbl:appendix-vocab-nouns-counting-positions}
\end{center}


\subsubsection{Counting occurrences}
\begin{center}
\resizebox{\linewidth}{!}{%
% Help: \multicolumn{2}{c}{}, \multirow{2}{*}{}, cmidrule(l){3-5}
\begin{tabular}{@{}lll@{}}
    \toprule
    \textbf{Name} & \textbf{Meaning} & \textbf{Notes} \\ \midrule
    & & \\
    & & \\
    % & & \\
    \bottomrule
\end{tabular}%
}
\captionof{table}{Nouns: counting occurrences.}
\label{tbl:appendix-vocab-nouns-counting-occurrences}
\end{center}


\subsubsection{Calendar months and days of a week}
\begin{center}
\resizebox{\linewidth}{!}{%
% Help: \multicolumn{2}{c}{}, \multirow{2}{*}{}, cmidrule(l){3-5}
\begin{tabular}{@{}lll@{}}
    \toprule
    \textbf{Name} & \textbf{Meaning} & \textbf{Notes} \\ \midrule
    & & \\
    & & \\
    % & & \\
    \bottomrule
\end{tabular}%
}
\captionof{table}{Nouns: Calendar months and days of a week.}
\label{tbl:appendix-vocab-nouns-calendar-months-and-days-of-a-week}
\end{center}


\subsubsection{Calendar days}
\begin{center}
\resizebox{\linewidth}{!}{%
% Help: \multicolumn{2}{c}{}, \multirow{2}{*}{}, cmidrule(l){3-5}
\begin{tabular}{@{}lll@{}}
    \toprule
    \textbf{Name} & \textbf{Meaning} & \textbf{Notes} \\ \midrule
    & & \\
    & & \\
    % & & \\
    \bottomrule
\end{tabular}%
}
\captionof{table}{Nouns: calendar days.}
\label{tbl:appendix-vocab-nouns-calendar-days}
\end{center}


\subsubsection{Directions}
\begin{center}
\resizebox{\linewidth}{!}{%
% Help: \multicolumn{2}{c}{}, \multirow{2}{*}{}, cmidrule(l){3-5}
\begin{tabular}{@{}lll@{}}
    \toprule
    \textbf{Name} & \textbf{Meaning} & \textbf{Notes} \\ \midrule
    \ruby{上}{うえ} & up & \\
    \ruby{下}{した} & down & \\
    \ruby{左}{ひだり} & left & \\
    \ruby{右}{みぎ} & right & \\
    \ruby{上}{のぼ}り & upwards/upbound/ascent & \\
    \ruby{下}{くだ}り & downwards/downbound/descent & \\
    \ruby{後}{うし}ろ & back/behind/rear (physical) & \href{https://ja.hinative.com/questions/4072942}{[HN]} \\
    \ruby{後}{あと} & behind (esp.\ abstract)/after/later & \href{https://ja.hinative.com/questions/4072942}{[HN]} \\
    & & \\
    \ruby{遠回}{とお|まわ}り & detour/roundabout way & \\
    \ruby{昇進}{しょう|しん} & promotion/rise in rank & e.g.\ workplace: \ruby{昇進人事}{しょう|しん|じん|じ} \\
    \ruby{格上}{かく|あ}げ & status upgrade/promotion & e.g.\ friendship status \\
    \ruby{降格}{こう|かく} & demotion/drop in rank & e.g.\ workplace: \ruby{降格人事}{こう|かく|じん|じ} \\
    \ruby{格下}{かく|さ}げ & status downgrade/demotion & e.g.\ friendship status  \\
    & & \\
    & & \\
    & & \\
    % & & \\
    \bottomrule
\end{tabular}%
}
\captionof{table}{Nouns: directions.}
\label{tbl:appendix-vocab-nouns-directions}
\end{center}

\subsubsection{Places}
\begin{center}
\resizebox{\linewidth}{!}{%
% Help: \multicolumn{2}{c}{}, \multirow{2}{*}{}, cmidrule(l){3-5}
\begin{tabular}{@{}lll@{}}
    \toprule
    \textbf{Name} & \textbf{Meaning} & \textbf{Notes} \\ \midrule
    \ruby{町}{ちょう}/\ruby{町}{まち} & street/neighbourhood & \href{https://ja.hinative.com/questions/17979737}{[HN1]}, \href{https://hinative.com/questions/20251204}{[HN2]} \\
    \ruby{町内会}{ちょう|ない|かい} & neighbourhood association & \\
    \ruby{図書館}{と|しょ|かん} & library & \\
    \ruby{扉}{とびら} & door/gate/opening & also: ドア \\
    \ruby{橋}{はし} & bridge & \\
    \ruby{山}{やま} & mountain/hill & \\
    \ruby{山々}{やま|やま} & mountains/hills & \\
    ビル & (multi-floor) building & \\
    レストラン & restaurant (esp.\ Western) & \\
    \ruby{高速道路}{こう|そく|どう|ろ} & highway/expressway & \\
    \ruby{家}{いえ} & house (physical entity) & neutral, \href{https://japanese.stackexchange.com/questions/3726/what-is-the-difference-between-\%E3\%81\%84\%E3\%81\%88-and-\%E3\%81\%86\%E3\%81\%A1}{[SE]} \\
    うち & one's home (of speaker, by default) & \href{https://japanese.stackexchange.com/questions/3726/what-is-the-difference-between-\%E3\%81\%84\%E3\%81\%88-and-\%E3\%81\%86\%E3\%81\%A1}{[SE]}, also a pronoun \\
    \ruby{部屋}{へ|や} & room & \\
    \ruby{学校}{がっ|こう} & school & \\
    & & \\
    & & \\
    % & & \\
    \bottomrule
\end{tabular}%
}
\captionof{table}{Nouns: places.}
\label{tbl:appendix-vocab-nouns-places}
\end{center}

\subsubsection{Furniture}
\begin{center}
\resizebox{\linewidth}{!}{%
% Help: \multicolumn{2}{c}{}, \multirow{2}{*}{}, cmidrule(l){3-5}
\begin{tabular}{@{}lll@{}}
    \toprule
    \textbf{Name} & \textbf{Meaning} & \textbf{Notes} \\ \midrule
    \ruby{椅子}{い|す} & chair/stool & \\
    \ruby{台所}{だい|どころ} & kitchen & \\
    & & \\
    % & & \\
    \bottomrule
\end{tabular}%
}
\captionof{table}{Nouns: furniture.}
\label{tbl:appendix-vocab-nouns-furniture}
\end{center}

\subsubsection{Organisms}
\begin{center}
\resizebox{\linewidth}{!}{%
% Help: \multicolumn{2}{c}{}, \multirow{2}{*}{}, cmidrule(l){3-5}
\begin{tabular}{@{}lll@{}}
    \toprule
    \textbf{Name} & \textbf{Meaning} & \textbf{Notes} \\ \midrule
    \ruby{魚}{さかな} & fish & \\
    \ruby{猫}{ねこ} & cat & \\
    \ruby{人間}{にん|げん} & human being/humankind & \\
    & & \\
    & & \\
    % & & \\
    \bottomrule
\end{tabular}%
}
\captionof{table}{Nouns: organisms.}
\label{tbl:appendix-vocab-nouns-organisms}
\end{center}


\subsubsection{Food}
\begin{center}
\resizebox{\linewidth}{!}{%
% Help: \multicolumn{2}{c}{}, \multirow{2}{*}{}, cmidrule(l){3-5}
\begin{tabular}{@{}lll@{}}
    \toprule
    \textbf{Name} & \textbf{Meaning} & \textbf{Notes} \\ \midrule
    \ruby{卵}{たまご} & eggs/egg/roe & \\
    \ruby{肉}{にく} & meat & \\
    \ruby{食}{た}べ\ruby{物}{もの} & food & \\
    \ruby{丼}{どんぶり}/\ruby{丼}{どん} & porcelain bowl/meat served over rice & \\
    ご\ruby{飯}{はん} & cooked rice/meal & \\
    そば & buckwheat/buckwheat noodles & \\
    \ruby{野菜}{や|さい} & vegetable & also: ベジタブル \\
    \ruby{朝}{あさ}ご\ruby{飯}{はん} & breakfast & \\
    \ruby{昼}{ひる}ご\ruby{飯}{はん} & lunch & \\
    \ruby{晩}{ばん}ご\ruby{飯}{はん} & dinner & \\
    \ruby{抹茶}{まっ|ちゃ} & matcha, powdered green tea & \\
    \ruby{飲}{の}み & the act of drinking & \\
    \ruby{飲}{の}み\ruby{物}{もの} & beverage & \\
    ジュース & soft drink (usually fruit-based)/sweet drink/juice & \\
    \ruby{果物}{くだ|もの} & fruit & \\
    & & \\
    % & & \\
    \bottomrule
\end{tabular}%
}
\captionof{table}{Nouns: food.}
\label{tbl:appendix-vocab-nouns-food}
\end{center}


\subsubsection{Small objects: stationery}
\begin{center}
\resizebox{\linewidth}{!}{%
% Help: \multicolumn{2}{c}{}, \multirow{2}{*}{}, cmidrule(l){3-5}
\begin{tabular}{@{}lll@{}}
    \toprule
    \textbf{Name} & \textbf{Meaning} & \textbf{Notes} \\ \midrule
    \ruby{書}{しょ} & book/document & \\
    \ruby{辞書}{じ|しょ} & dictionary & \\
    \ruby{本}{ほん} & book/volume/script & \\
    \ruby{帳}{ちょう} & book/register & \\
    \ruby{紙}{かみ} & paper & \\
    & & \\
    & & \\
    % & & \\
    \bottomrule
\end{tabular}%
}
\captionof{table}{Nouns: small objects: stationery.}
\label{tbl:appendix-vocab-nouns-small-objects-stationery}
\end{center}

\subsubsection{Date}
\begin{center}
\resizebox{\linewidth}{!}{%
% Help: \multicolumn{2}{c}{}, \multirow{2}{*}{}, cmidrule(l){3-5}
\begin{tabular}{@{}lll@{}}
    \toprule
    \textbf{Name} & \textbf{Meaning} & \textbf{Notes} \\ \midrule
    \ruby{日}{ひ}にち & (referring to) the date of an event & 「<event>の日にち」 \\
    \ruby{毎日}{まい|にち} & every day & also an adverb, \href{https://ja.hinative.com/questions/24476486}{[HN]} \\
    \ruby{日々}{ひ|び} & day after day & also an adverb, \href{https://ja.hinative.com/questions/24476486}{[HN]} \\
    & & \\
    \ruby{先日}{せん|じつ} & the other day/a few days ago & also an adverb \\
    \ruby[g]{昨日}{きのう} & yesterday (from now) & also an adverb \\
    \ruby[g]{今日}{きょう} & today (now) & also an adverb \\
    \ruby[g]{明日}{あした} & tomorrow (from now) & also an adverb \\
    \ruby{前日}{ぜん|じつ} & the day before (an event) & also an adverb \\
    \ruby{当日}{とう|じつ} & the day (of an event) & also an adverb \\
    \ruby{翌日}{よく|じつ} & the day after (an event) & also an adverb \\
    \ruby{翌々日}{よく|よく|じつ} & two days later (an event) & also an adverb \\ \midrule
    \ruby{先週}{せん|しゅう} & last week (from now) & also an adverb, \href{https://ja.hinative.com/questions/15897169}{[HN]} \\
    \ruby{今週}{こん|しゅう} & this week (now) & also an adverb \\
    \ruby{来週}{らい|しゅう} & next week (from now) & also an adverb, \href{https://www.goodcross.com/words/22234-2020}{[GC]} \\
    \ruby{次週}{じ|しゅう} & next week (recurring event, e.g.\ TV) & also an adverb, \href{https://www.goodcross.com/words/22234-2020}{[GC]} \\
    \ruby{前週}{ぜん|しゅう} & the week before (an event) & also an adverb, \href{https://ja.hinative.com/questions/15897169}{[HN]} \\
    \ruby{当週}{とう|しゅう} & the week (of an event) & also an adverb \\
    \ruby{翌週}{よく|しゅう} & the week after (an event) & also an adverb, \href{https://www.goodcross.com/words/22234-2020}{[GC]} \\
    \ruby{翌々週}{よく|よく|しゅう} & two weeks after (an event) & also an adverb, \href{https://www.goodcross.com/words/22234-2020}{[GC]} \\ \midrule
    \ruby{先月}{せん|げつ} & last month (from now) & also an adverb \\
    \ruby{今月}{こん|げつ} & this month (now) & also an adverb \\
    \ruby{来月}{らい|げつ} & next month (from now) & also an adverb \\
    \ruby{前月}{ぜん|げつ} & the month before (an event) & also an adverb \\
    \ruby{当月}{とう|げつ} & the month (of an event) & also an adverb \\
    \ruby{翌月}{よく|げつ} & the month after (an event) & also an adverb \\
    \ruby{翌々月}{よく|よく|げつ} & two months after (an event) & also an adverb \\
    & & \\
    & & \\
    \ruby{誕生日}{たん|じょう|び} & birthday & \\
    \ruby{春}{はる} & spring & \\
    \ruby{夏}{なつ} & summer & \\
    \ruby{秋}{あき} & autumn & \\
    \ruby{冬}{ふゆ} & winter & \\
    & & \\
    % & & \\
    \bottomrule
\end{tabular}%
}
\captionof{table}{Nouns: date.}
\label{tbl:appendix-vocab-nouns-date}
\end{center}

\subsubsection{Time}
\begin{center}
\resizebox{\linewidth}{!}{%
% Help: \multicolumn{2}{c}{}, \multirow{2}{*}{}, cmidrule(l){3-5}
\begin{tabular}{@{}lll@{}}
    \toprule
    \textbf{Name} & \textbf{Meaning} & \textbf{Notes} \\ \midrule
    \ruby{時}{とき} & time/moment & e.g.\ 「16\ruby{歳}{さい}の\ruby{時}{とき}私は\dots」\\
    \ruby{刻}{とき} & (referring to) time of day & \\
    \ruby{秋}{とき} & important time & also: \ruby{秋}{あき} \\
    \ruby{時々}{とき|どき} & sometimes/occasionally & \\
    \ruby{朝}{あさ} & morning & also an adverb, \href{https://ja.hinative.com/questions/18618956}{[HN]} \\
    \ruby{午前}{ご|ぜん} & before noon/ante meridian (a.m.) & also an adverb, \href{https://ja.hinative.com/questions/18618956}{[HN]} \\
    \ruby{昼}{ひる} & noon & also an adverb\\
    \ruby{午後}{ご|ご} & afternoon/after noon/post meridian (p.m.) & also an adverb \\
    \ruby{夕方}{ゆう|がた} & evening/dusk & also an adverb \\
    \ruby{晩}{ばん} & evening/night & also an adverb, \href{https://ja.hinative.com/questions/13398881}{[HN]}, \href{https://dictionary.goo.ne.jp/thsrs/12925/meaning/m1u/}{[goo]} \\
    \ruby{夜}{よる} & evening/night (slightly formal) & also an adverb, \href{https://ja.hinative.com/questions/13398881}{[HN]}, \href{https://dictionary.goo.ne.jp/thsrs/12925/meaning/m1u/}{[goo]} \\
    & & \\
    \ruby{朝日}{あさ|ひ} & morning sun/rising sun (the event) & \href{https://ja.hinative.com/questions/20406767}{[HN]} \\
    \ruby{日}{ひ}の\ruby{出}{で} & sunrise (the moment it rises) & \href{https://ja.hinative.com/questions/20406767}{[HN]} \\
    \ruby{夕日}{ゆう|ひ} & evening sun/setting sun (the event) & \href{https://ja.hinative.com/questions/20210983\#answer-47252259}{[HN]} \\
    \ruby{日}{ひ}の\ruby{入}{い}り & sunset (the moment it sets) & \href{https://ja.hinative.com/questions/20210983\#answer-47252259}{[HN]} \\
    & & \\
    & & \\
    \ruby[g]{今朝}{けさ} & this morning & \\
    \ruby[g]{今日}{きょう}の\ruby{午後}{ご|ご} & this afternoon & \\
    \ruby{今晩}{こん|ばん} & this evening/tonight & \\
    \ruby{今夜}{こん|や} & this evening/tonight (slightly formal) & \\
    & & \\
    & & \\
    \bottomrule
\end{tabular}%
}
\captionof{table}{Nouns: time.}
\label{tbl:appendix-vocab-nouns-time}
\end{center}



\subsubsection{Pronouns and question words}
Gramatically, pronouns are used in place of nouns and noun phrases.
\begin{center}
\resizebox{\linewidth}{!}{%
% Help: \multicolumn{2}{c}{}, \multirow{2}{*}{}, cmidrule(l){3-5}
\begin{tabular}{@{}lll@{}}
    \toprule
    \textbf{Name} & \textbf{Meaning} & \textbf{Notes} \\ \midrule
    \ruby{何}{なに} & what & \\
    \ruby{何}{なに}か & something & also an interjection \\
    \ruby{誰}{だれ} & who & \\
    いつ & when & \\
    どこ & where & \\
    \ruby{何時}{なん|じ} & which hour (of day) & \\
    なぜ & why & direct/formal/rude, \href{https://ja.hinative.com/questions/21654599\#answer-50366344}{[HN]}, \href{https://japanese.stackexchange.com/a/2703}{[SE]} \\
    どうして & why/how/by what means & informal, \href{https://ja.hinative.com/questions/21654599\#answer-50366344}{[HN]}, \href{https://japanese.stackexchange.com/a/2703}{[SE]} \\
    どうしてです & why/how/by what means & semi-formal, \href{https://ja.hinative.com/questions/21654599\#answer-50366344}{[HN]} \\
    なんで & why & informal, speech, \href{https://ja.hinative.com/questions/21654599\#answer-50366344}{[HN]}, \href{https://japanese.stackexchange.com/a/2703}{[SE]} \\
    どう & how/in what way/how about & \\
    & & \\
    \ruby{私}{わたくし} & I/me & formal, \href{https://ja.hinative.com/questions/21654599\#answer-50366344}{[HN]}, \href{https://japanese.stackexchange.com/a/2703}{[SE]} \\
    \ruby{私}{あたし} & I/me & feminine, less common \\
    うち & I/me & feminine, familiar (Kanji is \ruby{内}{うち}), also a place \\
    \ruby{私}{わたし} & I/me & slightly formal/distant \\
    \ruby{僕}{ぼく} & I/me & masculine, distant \\
    \ruby{自分}{じ|ぶん} & Myself/oneself/himself/herself/I/me & distant \\
    \ruby{俺}{おれ} & I/me & masculine, familiar \\
    \ruby{私}{わたし}たち & we/us & \\
    & & \\
    こちら & this one/way/direction (here, closer to speaker) & \\
    これ & this one/way/direction (here, closer to speaker) & slightly informal \\
    そちら & that one/way/direction (there, closer to listener) & \\
    それ & that one/way/direction (there, closer to listener) & slightly informal \\
    あちら & that one/way/direction (there, distant) & \\
    あれ & that one/way/direction (there, distant) & slightly informal \\
    & & \\
    あなた & you & rude/distant \\
    あなたたち & you (plural) & rude/distant \\
    お\ruby{前}{まえ} & you & rude \\
    お\ruby{前}{まえ}ら & you (plural) & rude \\
    & & \\
    \ruby{彼}{かれ} & he/him & also a noun \\
    \ruby{彼女}{かの|じょ} & she/her & also a noun \\
    % & & \\
    \bottomrule
\end{tabular}%
}
\captionof{table}{Nouns: pronouns and question words.}
\label{tbl:appendix-vocab-nouns-pronouns-and-question-words}
\end{center}

\subsubsection{Roles and occupations}
\begin{center}
\resizebox{\linewidth}{!}{%
% Help: \multicolumn{2}{c}{}, \multirow{2}{*}{}, cmidrule(l){3-5}
\begin{tabular}{@{}lll@{}}
    \toprule
    \textbf{Name} & \textbf{Meaning} & \textbf{Notes} \\ \midrule
    \ruby{女}{おんあ} & female/woman & \\
    \ruby{女子}{じょ|し} & woman/girl & \\
    \ruby{女子高生}{じょ|し|こう|せい} & female high-school student & \\
    \ruby{男}{おとこ} & make/man & \\
    \ruby{男子}{だん|し} & man/boy & \\
    \ruby{男子高生}{だん|し|こう|せい} & male high-school student & \\
    \ruby{学生}{がく|せい} & student & \\
    \ruby{小学生}{しょう|がく|せい} & elementary/primary school student& \\
    \ruby{中学生}{ちゅう|がく|せい} & junior high/middle school student & \\
    \ruby{高校生}{こう|こう|せい} & high school student & \\
    \ruby{大学生}{だい|がく|せい} & univeristy student & \\
    \ruby{初学者}{しょ|がく|しゃ} & beginner & \\
    \ruby{将軍}{しょう|ぐん} & general (military, historical) & \\
    \ruby{店長}{\textbf{て}ん|ちょう} & shop manager & \\
    \ruby{友達}{とも|だち} & friend & \\
    \ruby{友人}{ゆう|じん} & friend & formal \\
    \ruby{問題児}{もん|だい|じ} & problem child & \\
    \ruby{有名人}{ゆう|めい|じん} & famous person/celebrity/public figure & \\
    \ruby{彼}{かれ} & boyfriend & also a pronoun \\
    \ruby{彼女}{かの|じょ} & girlfriend & also a pronoun \\
    \ruby{医者}{い|しゃ} & doctor/physician & \\
    & & \\
    % & & \\
    \bottomrule
\end{tabular}%
}
\captionof{table}{Nouns: roles and occupations.}
\label{tbl:appendix-vocab-nouns-roles-and-occupations}
\end{center}


\subsubsection{Family}
\begin{center}
\resizebox{\linewidth}{!}{%
% Help: \multicolumn{2}{c}{}, \multirow{2}{*}{}, cmidrule(l){3-5}
\begin{tabular}{@{}lll@{}}
    \toprule
    \textbf{Name} & \textbf{Meaning} & \textbf{Notes} \\ \midrule
    \ruby{親子}{おや|こ} & parent and child & \\
    \ruby{弟}{おとうと} & younger brother & \\
    & & \\
    & & \\
    & & \\
    % & & \\
    \bottomrule
\end{tabular}%
}
\captionof{table}{Nouns: family.}
\label{tbl:appendix-vocab-nouns-family}
\end{center}


\subsubsection{Body parts}
\begin{center}
\resizebox{\linewidth}{!}{%
% Help: \multicolumn{2}{c}{}, \multirow{2}{*}{}, cmidrule(l){3-5}
\begin{tabular}{@{}lll@{}}
    \toprule
    \textbf{Name} & \textbf{Meaning} & \textbf{Notes} \\ \midrule
    \ruby{髪}{かみ} & hair (on the head) & also: ヘア \\
    & & \\
    & & \\
    % & & \\
    \bottomrule
\end{tabular}%
}
\captionof{table}{Nouns: body parts.}
\label{tbl:appendix-vocab-nouns-body-parts}
\end{center}


\subsubsection{Clothing}
\begin{center}
\resizebox{\linewidth}{!}{%
% Help: \multicolumn{2}{c}{}, \multirow{2}{*}{}, cmidrule(l){3-5}
\begin{tabular}{@{}lll@{}}
    \toprule
    \textbf{Name} & \textbf{Meaning} & \textbf{Notes} \\ \midrule
    \ruby{制服}{せい|ふく} & uniform & \\
    & & \\
    & & \\
    & & \\
    % & & \\
    \bottomrule
\end{tabular}%
}
\captionof{table}{Nouns: clothing.}
\label{tbl:appendix-vocab-nouns-clothing}
\end{center}


\subsubsection{Emotions}
\begin{center}
\resizebox{\linewidth}{!}{%
% Help: \multicolumn{2}{c}{}, \multirow{2}{*}{}, cmidrule(l){3-5}
\begin{tabular}{@{}lll@{}}
    \toprule
    \textbf{Name} & \textbf{Meaning} & \textbf{Notes} \\ \midrule
    \ruby{涙}{なみだ} & tears & \\
    \ruby{一人}{ひと|り}ぼっち & aloneness/loneliness/solitude & \\
    \ruby{安心感}{あん|しん|かん} & sense of security & \\
    & & \\
    % & & \\
    \bottomrule
\end{tabular}%
}
\captionof{table}{Nouns: emotions.}
\label{tbl:appendix-vocab-nouns-emotions}
\end{center}


\subsubsection{Production}
\begin{center}
\resizebox{\linewidth}{!}{%
% Help: \multicolumn{2}{c}{}, \multirow{2}{*}{}, cmidrule(l){3-5}
\begin{tabular}{@{}lll@{}}
    \toprule
    \textbf{Name} & \textbf{Meaning} & \textbf{Notes} \\ \midrule
    \ruby{話}{はなし} & speech/conversation/topic/subject & \\
    \ruby{歌}{うた} & song/singing & \\
    \ruby{踊}{おど}り & dance & \\
    \ruby{切符}{きっ|ぷ} & ticket & \\
    \ruby{小躍}{こ|おど}り & dancing/jumping for joy & \\
    \ruby{結果}{けっ|か} & result/outcome/consequence & \\
    \ruby{雑誌}{ざっ|し} & journal/magazine & \\
    \ruby{音楽}{おん|がく} & music & \\
    \ruby{歌}{うた} & singing/song & \\
    \ruby{曲}{きょく} & piece/composition/song/track & \\
    \ruby{作}{つく}り & the making/production/components of & \\
    つもり & plan/intention; assumption; estimation & \\
    \ruby{勉強}{べん|きょう} & study/diligence/hard work & \\
    \ruby{経済}{けい|ざい} & economy & \\
    \ruby{地理}{ち|り} & geography & \\
    \ruby{歴史}{れき|し} & history & \\
    \ruby{映画}{えい|が} & movie/film/motion picture & \\
    \ruby{写真}{しゃ|しん} & photograph/photo/picture/snapshot & \\
    & & \\
    % & & \\
    \bottomrule
\end{tabular}%
}
\captionof{table}{Nouns: production.}
\label{tbl:appendix-vocab-nouns-production]}
\end{center}


\subsubsection{Consumption}
\begin{center}
\resizebox{\linewidth}{!}{%
% Help: \multicolumn{2}{c}{}, \multirow{2}{*}{}, cmidrule(l){3-5}
\begin{tabular}{@{}lll@{}}
    \toprule
    \textbf{Name} & \textbf{Meaning} & \textbf{Notes} \\ \midrule
    かばん & bag/briefcase/basket & \\
    \ruby{袋}{ふくろ} & bag/sack/pouch & \\
    \ruby{試験}{し|けん} & examination/test & also: テスト \\
    ビル & bill/invoice & \\
    \ruby{値段}{ね|だん} & price/cost & \\

    [お]\ruby{金}{かね} & money & [polite] \\
    & & \\
    & & \\
    & & \\
    % & & \\
    \bottomrule
\end{tabular}%
}
\captionof{table}{Nouns: consumption.}
\label{tbl:appendix-vocab-nouns-consumption}
\end{center}



\subsubsection{Interaction}
\begin{center}
\resizebox{\linewidth}{!}{%
% Help: \multicolumn{2}{c}{}, \multirow{2}{*}{}, cmidrule(l){3-5}
\begin{tabular}{@{}lll@{}}
    \toprule
    \textbf{Name} & \textbf{Meaning} & \textbf{Notes} \\ \midrule
    \ruby{思}{おも}い & thought & \\
    \ruby{感謝}{かん|しゃ} & thanks/gratitude/appreciation & \\
    \ruby{問題}{もん|だい} & problem/question & \\
    \ruby{質問}{しつ|もん} & question/enquiry & \\
    \ruby{説明}{せつ|めい} & explanation & \\
    \ruby{情報}{じょう|ほう} & information/news/intelligence & \\

    [お]\ruby{願}{ねが}い & request/favour/wish/desire/hope & [polite] \\
    〜\ruby{願}{ねがい} & written application & \suffix \\
    \ruby{内緒}{ない|しょ} & secret (in/out-group, personal level) & \href{https://ja.hinative.com/questions/6644230}{[HN]} \\
    \ruby{秘密}{ひ|みつ} & secret (official/corporate/country) & childish; \href{https://ja.hinative.com/questions/6644230}{[HN]} \\
    \ruby{答}{こた}え & answer/reply/response & \\
    & & \\
    いくら & how much (price) & \\
    メールアドレス & email address & \\
    \ruby{登録}{とう|ろく} & presence in register/records; registration & \href{https://dictionary.goo.ne.jp/word/\%e7\%99\%bb\%e9\%8c\%b2/}{[goo]} \\
    \ruby{入会}{にゅう|かい} & enrolment/admission into a club/society/mailing list & \href{https://ja.hinative.com/questions/22502664}{[HN]} \\
    \ruby{退会}{たい|かい} & withdrawal/resignation from a club/society/mailing list & \\
    \ruby{加入}{か|にゅう} & becoming a member of (e.g.\ a group/project) & \href{https://ja.hinative.com/questions/22502664}{[HN]} \\
    \ruby{電話}{でん|わ} & phone/phone call & \\
    パソコン & personal computer (PC) & \\
    \ruby{宿題}{しゅく|だい} & homework/assignment & \\
    \ruby{勝}{か}ち & win/victory (personal) & \\
    \ruby{勝利}{しょう|り} & win/victory (larger scale) & \\
    \ruby{負}{ま}け & loss/defeat (personal) & \\
    \ruby{敗北}{はい|ぼく} & loss/defeat (larger scale) & \\
    \ruby{練習}{れん|しゅう} & practice/train/drill & \\
    \ruby{自習}{じ|しゅう} & self-study & \\
    & & \\
    % & & \\
    \bottomrule
\end{tabular}%
}
\captionof{table}{Nouns: interaction.}
\label{tbl:appendix-vocab-nouns-interaction}
\end{center}


\subsubsection{Health}
\begin{center}
\resizebox{\linewidth}{!}{%
% Help: \multicolumn{2}{c}{}, \multirow{2}{*}{}, cmidrule(l){3-5}
\begin{tabular}{@{}lll@{}}
    \toprule
    \textbf{Name} & \textbf{Meaning} & \textbf{Notes} \\ \midrule
    \ruby{寝袋}{ね|ぶくろ} & sleeping bag & \\
    \ruby{休}{やす}み & rest/vacation & \\
    \ruby{春休}{はる|やす}み & spring break/vacation & \\
    \ruby{夏休}{なつ|やす}み & summer vacation & \\
    \ruby{秋休}{あき|やす}み & autumn break/vacation & also: \ruby{秋}{とき} \\
    \ruby{冬休}{ふゆ|やす}み & winter vacation & \\
    \ruby{月見}{つき|み} & Japanese equivalent of CN's mid-autumn festival (\ruby{同}{おな}じ\ruby{日}{ひ}) & \\
    \ruby{一酸化炭素中毒}{いっ|さん|か|たん|そ|ちゅう|どく} & carbon monoxide poisoning & \\
    \ruby{暇}{ひま} & free time/time off/leisure & also an adjective \\
    & & \\
    & & \\
    % & & \\
    \bottomrule
\end{tabular}%
}
\captionof{table}{Nouns: health.}
\label{tbl:appendix-vocab-nouns-health}
\end{center}



\subsubsection{Positive traits, strengths}
\begin{center}
\centering
\resizebox{\linewidth}{!}{%
% Help: \multicolumn{2}{c}{}, \multirow{2}{*}{}, cmidrule(l){3-5}
\begin{tabular}{@{}lll@{}}
    \toprule
    \textbf{Name} & \textbf{Meaning} & \textbf{Notes} \\ \midrule
    \ruby{能力}{のう|りょく} & ability & \\
    \ruby{力}{ちから} & force/strength/power & \\
    \ruby{信用}{しん|よう} & trust/confidence/reputation (past) & \href{https://japanese.stackexchange.com/q/24275}{[SE]} \\
    \ruby{信頼}{しん|らい} & trust/confidence/reliance/faith (future) & \href{https://japanese.stackexchange.com/q/24275}{[SE]} \\
    \ruby{簡易}{かん|い} & simplicity/ease/convenience & \\
    \ruby{人気}{にん|き} & popularity/public favour & \\
    \ruby{本物}{ほん|もの} & genuine article/real deal & \\\
    \ruby{真実}{しん|じつ} & truth/reality & \href{https://ja.hinative.com/questions/21280744}{[HN]} \\
    & & \\
    もちもち & springy texture/elastic & \\
    & & \\
    % & & \\
\bottomrule
\end{tabular}%
}
\captionof{table}{Nouns: positive traits, strengths.}
\label{tbl:appendix-vocab-nouns-positive-traits-strengths}
\end{center}

\subsubsection{Negative traits, weaknesses}
\begin{center}
\centering
\resizebox{\linewidth}{!}{%
% Help: \multicolumn{2}{c}{}, \multirow{2}{*}{}, cmidrule(l){3-5}
\begin{tabular}{@{}lll@{}}
    \toprule
    \textbf{Name} & \textbf{Meaning} & \textbf{Notes} \\ \midrule
    \ruby{失礼}{しつ|れい} & discourtesy/impoliteness & also a な-adjective \\
    \ruby{無礼}{ぶ|れい} & rudeness/discourtesy/insolence (stronger) & also a な-adjective \\
    \ruby{偽物}{にせ|もの} & fake article/forgery/counterfeit/imiation  & \\
    & & \\
    & & \\
    % & & \\
\bottomrule
\end{tabular}%f
}
\captionof{table}{Nouns: negative traits, weaknesses.}
\label{tbl:appendix-vocab-nouns-negative-traits-weaknesses}
\end{center}

\subsubsection{Subjects}
\begin{center}
\resizebox{\linewidth}{!}{%
% Help: \multicolumn{2}{c}{}, \multirow{2}{*}{}, cmidrule(l){3-5}
\begin{tabular}{@{}lll@{}}
    \toprule
    \textbf{Name} & \textbf{Meaning} & \textbf{Notes} \\ \midrule
    \ruby{数学}{すう|がく} & mathematics & \\
    \ruby{科学}{か|がく} & science & \\
    \ruby{工学}{こう|がく} & engineering & \\
    \ruby{経済学}{けい|ざい|がく} & economics & \\
    \ruby{地理学}{ち|り|がく} & geography & \\
    \ruby{歴史学}{れき|し|がく} & history & \\
    \ruby{計算機科学}{けい|さん|き|か|がく} & computer science & \\
    \ruby{情報工学}{じょう|ほう|こう|がく} & information engineering & \\
    & & \\
    % & & \\
    \bottomrule
\end{tabular}%
}
\captionof{table}{Nouns: subjects.}
\label{tbl:appendix-vocab-nouns-subjects}
\end{center}

\subsubsection{Creatures and divinity}
\begin{center}
\centering
\resizebox{\linewidth}{!}{%
% Help: \multicolumn{2}{c}{}, \multirow{2}{*}{}, cmidrule(l){3-5}
\begin{tabular}{@{}lll@{}}
    \toprule
    \textbf{Name} & \textbf{Meaning} & \textbf{Notes} \\ \midrule
    \ruby{神}{かみ} & god/deity/divinity/spirit & \\
    \ruby{天使}{てん|し} & angel & \\
    \ruby{巫女}{み|こ}/\ruby{神子}{み|こ} & shrine maiden & \href{https://detail.chiebukuro.yahoo.co.jp/qa/question_detail/q1424312974}{[YJ]} \\
    \ruby{御朱印}{ご|しゅ|いん} & seal stamp at shrines and temples & \\
    \ruby{南無阿弥陀仏}{な|む|あ|み|だ|ぶつ} & hail Amitabha Buddha & \\
    & & \\
    \ruby{悪魔}{あく|ま} & devil/demon & \\
    & & \\
    & & \\
    & & \\
    % & & \\
\bottomrule
\end{tabular}%
}
\captionof{table}{Nouns: creatures and divinity.}
\label{tbl:appendix-vocab-nouns-creatures-and-divinity}
\end{center}



\subsubsection{Nature}
\begin{center}
\resizebox{\linewidth}{!}{%
% Help: \multicolumn{2}{c}{}, \multirow{2}{*}{}, cmidrule(l){3-5}
\begin{tabular}{@{}lll@{}}
    \toprule
    \textbf{Name} & \textbf{Meaning} & \textbf{Notes} \\ \midrule
    \ruby{光}{ひかり} & light & \\
    \ruby{水}{みず} & water & \\
    \ruby{水素}{すい|そ} & hydrogen & \\
    \ruby{炭素}{たん|そ} & carbon & \\
    \ruby{酸素}{さん|そ} & oxygen & \\
    \ruby{一酸化炭素}{いっ|さん|か|たん|そ} & carbon monoxide & \\
    \ruby{二酸化炭素}{に|さん|か|たん|そ} & carbon dioxide & \\
    \ruby{雨}{あめ} & rain & \\
    \ruby{雪}{ゆき} & snow & \\
    \ruby{虹}{にじ} & rainbow & \\
    \ruby[g]{時雨}{しぐれ} & seasonal rain/rain in late autumn-early winter & \\
    \ruby{花}{はな} & flower/blossom/bloom/petal & \\
    \ruby{桜}{さくら} & cherry tree/cherry blossom & \\
    \ruby{満開}{まん|かい} & full bloom (esp.\ of cherry blossom) & \\
    & & \\
    & & \\
    % & & \\
    \bottomrule
\end{tabular}%
}
\captionof{table}{Nouns: nature.}
\label{tbl:appendix-vocab-nouns-nature}
\end{center}

\subsubsection{Cosmic}
\begin{center}
\resizebox{\linewidth}{!}{%
% Help: \multicolumn{2}{c}{}, \multirow{2}{*}{}, cmidrule(l){3-5}
\begin{tabular}{@{}lll@{}}
    \toprule
    \textbf{Name} & \textbf{Meaning} & \textbf{Notes} \\ \midrule
    \ruby{流星}{りゅう|せい} & meteor/shooting star & \\
    \ruby{月見}{つき|み} & moon viewing (eighth lunar month) & \\
    & & \\
    \bottomrule
\end{tabular}%
}
\captionof{table}{Nouns: cosmic.}
\label{tbl:appendix-vocab-nouns-cosmic}
\end{center}


\subsubsection{Physical units}
\begin{center}
\resizebox{\linewidth}{!}{%
% Help: \multicolumn{2}{c}{}, \multirow{2}{*}{}, cmidrule(l){3-5}
\begin{tabular}{@{}lll@{}}
    \toprule
    \textbf{Name} & \textbf{Meaning} & \textbf{Notes} \\ \midrule
    \ruby{摂氏}{せっ|し} & Celsius/centigrade & also: セし \\
    \ruby{摂氏温度}{せっ|し|おん|ど} & degrees Celsius & also: セし温度 \\
    & & \\
    & & \\
    & & \\
    & & \\
    % & & \\
    \bottomrule
\end{tabular}%
}
\captionof{table}{Nouns: physical units.}
\label{tbl:appendix-vocab-nouns-physical-units}
\end{center}


\subsubsection{Hygiene}
\begin{center}
\resizebox{\linewidth}{!}{%
% Help: \multicolumn{2}{c}{}, \multirow{2}{*}{}, cmidrule(l){3-5}
\begin{tabular}{@{}lll@{}}
    \toprule
    \textbf{Name} & \textbf{Meaning} & \textbf{Notes} \\ \midrule
    うんこ/ウンコ & poop & \\
    ごみ/ゴミ & trash/rubbish/garbage/refuse & \\
    & & \\
    & & \\
    \bottomrule
\end{tabular}%
}
\captionof{table}{Nouns: hygiene.}
\label{tbl:appendix-vocab-nouns-hygiene}
\end{center}

\end{multicols}

\subsection{Adjectives}

\subsubsection{Emotions}
\begin{center}
\centering
\resizebox{\linewidth}{!}{%
% Help: \multicolumn{2}{c}{}, \multirow{2}{*}{}, cmidrule(l){3-5}
\begin{tabular}{@{}lcll@{}}
    \toprule
    \textbf{Descriptor} & \textbf{Cat.} & \textbf{Meaning} & \textbf{Notes} \\ \midrule
    \ruby{嬉}{うれ}しい & い & happy/glad/delighted & \\
    \ruby{寂}{さび}しい & い & lonely & \\
    \ruby{欲}{ほ}しい & い & desired/wanted & \\
    \ruby{悲}{かな}しい & い & sad/miserable & \\
    \ruby{楽}{たの}しい & い & fun/enjoyable/happy & \\
    \ruby{懐}{なつ}かしい & い & nostalgic/fondly-remembered/missed & \\
    \ruby{安心}{あん|しん} & な & relieved & \\
    & & & \\
    \ruby{暖}{あたた}かい & い & pleasantly warm & \\
    \ruby{暑}{あつ}い & い & hot & \\
    \ruby{熱}{あつ}い & い & hot (to the touch) & \\
    \ruby{小寒}{こ|さむ}い & い & chilly/a little cold & \\
    \ruby{寒}{さむ}い & い & cold (weather) & \\
    \ruby{眠}{ねむ}い & い & sleepy/drowsy & \\
    \ruby{恥}{は}ずかしい & い & embarrassed/ashamed/humiliated & \\
    & & & \\
    & & & \\
    & & & \\
    % & & & \\
\bottomrule
\end{tabular}%
}
\captionof{table}{Adjectives: emotions.}
\label{tbl:appendix-vocab-adjectives-emotions}
\end{center}


\subsubsection{Health}
\begin{center}
\centering
\resizebox{\linewidth}{!}{%
% Help: \multicolumn{2}{c}{}, \multirow{2}{*}{}, cmidrule(l){3-5}
\begin{tabular}{@{}lcll@{}}
    \toprule
    \textbf{Descriptor} & \textbf{Cat.} & \textbf{Meaning} & \textbf{Notes} \\ \midrule
    \ruby{大丈夫}{だい|じょう|ぶ} & な & alright/problem-free/without fear & \\
    \ruby{元気}{げん|き} & な & lively/well/in good health & \\
    \ruby{暇}{ひま} & な & free/available & also a noun \\
    & & & \\
    % & & & \\
\bottomrule
\end{tabular}%
}
\captionof{table}{Adjectives: health.}
\label{tbl:appendix-vocab-adjectives-health}
\end{center}


\subsubsection{Positive traits, strengths}
\begin{center}
\centering
\resizebox{\linewidth}{!}{%
% Help: \multicolumn{2}{c}{}, \multirow{2}{*}{}, cmidrule(l){3-5}
\begin{tabular}{@{}lcll@{}}
    \toprule
    \textbf{Descriptor} & \textbf{Cat.} & \textbf{Meaning} & \textbf{Notes} \\ \midrule
    いい/\ruby{良}{よ}い/よい & い & good/nice/agreeable/OK & \href{https://salon.mainichi-kotoba.jp/archives/670}{[MK]}\\
    かわいい & い & cute/adorable/charming/lovely/pretty & \\
    かっこいい & い & cool/attractive/stylish & \\
    \ruby{面白}{おも|しろ}い & い & interesting/fascinating/funny/entertaining & \\
    \ruby{上手}{じょう|ず} & な & skilful/proficient/adept & \\
    \ruby{簡単}{かん|たん} & な & easy/simple & \\
    \ruby{好}{す}き & な & likeable/favourite & \\
    \ruby{大好}{だい|す}き & な & strongly liked/loved & \\
    おいしい & い & good-tasting/delicious/tasty & \\
    & & & \\
    ふわふわ & な & soft/fluffy/spongy & \\
    & & & \\
    \ruby{綺麗}{き|れい} & な & pretty/beautiful/clean/tidy & \\
    \ruby{最高}{さい|こう} & な & best/highest & \\
    やばい & い & terrific/amazing/cool & colloquial, slang \\
    \ruby{信}{しん}じられない & い & unbelievable/incredible & \\
    \ruby{静}{しず}か & な & quiet/silent/calm/peaceful & \\
    \ruby{親切}{しん|せつ} & な & kind/generous/gentle (action) & formal; \href{https://ja.hinative.com/question_summaries/112079}{[HN]} \\
    \ruby{優}{やさ}しい & い & kind/affectionate/gentle (character) & speech; \href{https://ja.hinative.com/question_summaries/112079}{[HN]} \\
    & & & \\

    [お]やすい & い & easy & \\
    \ruby{見}{み}やすい & い & easy to see & \\
    \ruby{気安}{き|や}い & い & Relaxed/familiar/friendly & \\
    \ruby{読}{よ}みやすい & い & easy to read/legible & \\
    \ruby{使}{つか}いやすい & い & easy to use & \\
    \ruby{熱}{ね}しやすい & い & excitable & \\
    \ruby{飲}{の}みやすい & い & easy to drink/swallow & \\
    \ruby{疲}{つか}れやすい & い & easily fatigued & \\
    \ruby{住}{す}みやすい & い & Comfortable/convenient to live in (of a neighbourhood) & \\
    わかりやすい & い & easy to understand & \\
    \ruby{覚}{おぼ}えやすい & い & easy to learn/remember & \\
    \ruby{心安}{こころ|やす}い & い & friendly/familiar/intimate & \\
    \ruby{有名}{ゆう|めい} & な & famous & \\
    & & & \\
    \ruby{真面目}{ま|じ|め} & な & serious/sober/earnest/grave & \\
    まじ/マジ & な & serious/not joking & abbreviation \\
    \ruby{当}{あ}たり\ruby{前}{まえ} & な & natural/obvious/common/ordinary/the norm & \\
    \ruby{本当}{ほう|とう} & な & real/true/genuine/authentic & \href{https://ja.hinative.com/questions/21280744}{[HN]} \\
    & & & \\
    % & & & \\
\bottomrule
\end{tabular}%
}
\captionof{table}{Adjectives: positive traits, strengths.}
\label{tbl:appendix-vocab-adjectives-positive-traits-strengths}
\end{center}


\subsubsection{Negative traits, weaknesses}
\begin{center}
\centering
\resizebox{\linewidth}{!}{%
% Help: \multicolumn{2}{c}{}, \multirow{2}{*}{}, cmidrule(l){3-5}
\begin{tabular}{@{}lcll@{}}
    \toprule
    \textbf{Descriptor} & \textbf{Cat.} & \textbf{Meaning} & \textbf{Notes} \\ \midrule
    ない & い & non-existent/not being there & \\
    おかしい & い & laughable/ridiculous/strange/weird/suspicious & \\
    \ruby{寒}{さむ}い & い & lame/corny (joke) & \\
    ダサい & い & lame/uncool & slang \\
    \ruby{下手}{へ|た} & な & unskilful/poor/awkward & \\
    \ruby{難}{むずか}しい & い & difficult/troublesome/impossible (euphemism) & \\
    \ruby{嫌}{きら}い & \textredbf{な} & disliked/hated & \\
    \ruby{大嫌}{だい|きら}い & \textredbf{な} & strongly disliked/hated & \\
    \ruby{嫌}{いや} & な & reluctant/disagreeable & \\
    & & & \\
    \ruby{悪}{わる}い & い & bad/poor/undesirable/at fault & also an interjection \\
    \ruby{危}{あぶ}ない & い & dangerous/risky & also an interjection \\
    やばい & い & dangerous/risky/awful/crazy/unhinged & colloquial, slang \\
    まずい & い & bad taste/unpleasant/awful/problematic/unfavourable & \\
    \ruby{最悪}{さい|あく} & な & worst & \\
    \ruby{最低}{さい|てい} & な & lowest/worst & \\
    \ruby{邪悪}{じゃ|あく} & な & evil/wicked & \\
    \ruby{失礼}{しつ|れい} & な & discourteous/impolite & also a noun \\
    \ruby{無礼}{ぶ|れい} & な & rude/discourteous/insolent (stronger) & also a noun \\
    \ruby{無理}{む|り} & な & impossible/no way/unreasonable & \\
    & & & \\
    \ruby{飽}{あ}きやすい & い & easily bored/fickle/quick to lose interest & \\
    \ruby{感じ}{かん|じ}やすい & い & sensitive/susceptible & also: センシティブ \\
    \ruby{受}{う}けやすい & い & susceptible/vulnerable/prone to & \\
    \ruby{騙}{だま}されやすい & い & guillible/naive & \\
    & & & \\
    % & & & \\
\bottomrule
\end{tabular}%
}
\captionof{table}{Adjectives: negative traits, weaknesses.}
\label{tbl:appendix-vocab-adjectives-negative-traits-weaknesses}
\end{center}

\subsubsection{Amounts and sizes}
\begin{center}
\centering
\resizebox{\linewidth}{!}{%
% Help: \multicolumn{2}{c}{}, \multirow{2}{*}{}, cmidrule(l){3-5}
\begin{tabular}{@{}lcll@{}}
    \toprule
    \textbf{Descriptor} & \textbf{Cat.} & \textbf{Meaning} & \textbf{Notes} \\ \midrule
    \ruby{大}{おお}きい & い & big/large/great & \\
    \ruby{高}{たか}い & い & high/tall; expensive & \\
    & & & \\
    & & & \\
    & & & \\
    & & & \\
    & & & \\
    \ruby{小}{ちい}さい & い & small/little/tiny & \\
    うまい & い & skilful/good/delicious & \\
    \ruby{低}{ひく}い & い & low/short & \\
    \ruby{安}{やす}い & い & lheap & \\
    & & & \\
    \ruby{久}{ひさ}しい & い & long (time that has passed)/old (story) & \\
    \ruby{久}{ひさ}しぶり & な & long time (since the last time) & \\
    & & & \\
    % & & & \\
\bottomrule
\end{tabular}%
}
\captionof{table}{Adjectives: amounts and sizes.}
\label{tbl:appendix-vocab-adjectives-amounts-and-sizes}
\end{center}

\subsection{Verbs}

\subsubsection{Physical}
\begin{center}
\centering
\resizebox{\linewidth}{!}{%
% Help: \multicolumn{2}{c}{}, \multirow{2}{*}{}, cmidrule(l){3-5}
\begin{tabular}{@{}lclllcll@{}}
    \toprule
    \multicolumn{4}{c}{\textbf{Transitive}} & \multicolumn{4}{c}{\textbf{Intransitive}} \\ \cmidrule(r){1-4} \cmidrule(l){5-8}
    \textbf{Action} & \textbf{Cat.} & \textbf{Meaning} & \textbf{Notes} & \textbf{Action} & \textbf{Cat.} & \textbf{Meaning} & \textbf{Notes} \\ \midrule
    \ruby{当}{あ}てる & る & to hit & & \ruby{当}{あ}たる & う & to be hit & \\
    \ruby{打}{う}つ & う & to hit (strong) & \href{https://ja.hinative.com/questions/3867085}{[HN]} & \ruby{打}{う}たれる & る & to be hit (strong) & \\
    \ruby{打}{ぶ}つ & う & to hit someone & \href{https://ja.hinative.com/questions/4651279\#answer-39822392}{[HN]} & ? & & & \\
    ぶつける & る & to hit someone's head/crash into & \href{https://ja.hinative.com/questions/18725588}{[HN]} & ぶつかる & う & to be hit/crashed (large objects) & \href{https://ja.hinative.com/questions/94519\#answer-237544}{[HN]} \\
    ボッコボコにする & E & to severely beat up & & ? & & & \\
    \ruby{横}{よこ}たえる & る & to lay down & & \ruby{横}{よこ}たわる & う & to be laid down/stretched out & \\
    \ruby{持}{も}つ & う & to hold (in hand)/take/carry/possess; hold meeting & & - & & & \\
    & & & & & & & \\
    & & & & & & & \\
    & & & & & & & \\
    & & & & & & & \\
    % & & & & & & & \\
\bottomrule
\end{tabular}%
}
\captionof{table}{Verbs: physical.}
\label{tbl:appendix-vocab-verbs-physical}
\end{center}



\subsubsection{Directions}
\begin{center}
\centering
\resizebox{\linewidth}{!}{%
% Help: \multicolumn{2}{c}{}, \multirow{2}{*}{}, cmidrule(l){3-5}
\begin{tabular}{@{}lclllcll@{}}
    \toprule
    \multicolumn{4}{c}{\textbf{Transitive}} & \multicolumn{4}{c}{\textbf{Intransitive}} \\ \cmidrule(r){1-4} \cmidrule(l){5-8}
    \textbf{Action} & \textbf{Cat.} & \textbf{Meaning} & \textbf{Notes} & \textbf{Action} & \textbf{Cat.} & \textbf{Meaning} & \textbf{Notes} \\ \midrule
    - & & & & \ruby{行}{い}く & う & to go/move through/proceed/reach (information/phase) & \\
    - & & & & <て-form>いく & う & gradually/progressively/steadily & \aux, い sometimes omitted in casual speech \\
    \ruby{持}{も}っていく & う & to take/bring/carry something along & & - & & & \\
    - & & & & くる & E & to come/approach/arrive & (\ruby{来}{く}る)\\
    - & & & & <て-form>くる & E & to do <て-form> and come back & \aux, e.g.\ 「\ruby{行}{い}ってくる」 \\
    \ruby{持}{も}ってくる & う & to take/bring/carry something over & & - & & & \\
    \ruby{帰}{かえ}す & う & to send back/home (animate) & \href{https://ja.hinative.com/questions/23865042}{[HN]} & \ruby{帰}{かえ}る & \textredbf{う} & to return/go back/go home (animate) & \href{https://ja.hinative.com/questions/23865042}{[HN]} \\
    \ruby{還}{かえ}す & う & to send back to origin (grander scale) & \href{https://ja.hinative.com/questions/23865042}{[HN]}, \href{https://kurashi-memocho.com/113.html}{[KRS]} & \ruby{還}{かえ}る & \textredbf{う} & to return back to origin (grander scale) & \href{https://ja.hinative.com/questions/23865042}{[HN]}, \href{https://kurashi-memocho.com/113.html}{[KRS]} \\
    \ruby{返}{かえ}す & う & to return/put something back (inaminate) & \href{https://ja.hinative.com/questions/23865042}{[HN]} & \ruby{返}{かえ}る & \textredbf{う} & to return/go back (inaminate) & \href{https://ja.hinative.com/questions/23865042}{[HN]} \\
    & & & & & & & \\
    ? & & & & \ruby{上}{のぼ}る & う & to go up/upwards (focus on process) & \href{https://dictionary.goo.ne.jp/word/\%E4\%B8\%8A\%E3\%82\%8B/}{[goo]}\\
    ? & & & & \ruby{登}{のぼ}る & う & to ascend to a higher place & \href{https://dictionary.goo.ne.jp/word/\%E4\%B8\%8A\%E3\%82\%8B/}{[goo]} \\
    ? & & & & \ruby{昇}{のぼ}る & う & to rise (sun); be promoted in rank & \href{https://dictionary.goo.ne.jp/word/\%E4\%B8\%8A\%E3\%82\%8B/}{[goo]} \\
    \ruby{乗}{の}せる & る & to pick up passenger/load goods & & \ruby{乗}{の}る & う & to board/embark & \\
    \ruby{上}{あ}げる & る & to raise/elevate & & \ruby{上}{あ}がる & う & to be raised/elevated (focus on destination) & \href{https://dictionary.goo.ne.jp/thsrs/15966/meaning/m1u/}{[goo]}, \href{https://hugkum.sho.jp/582833}{[HK]}\\
    & & & & & & & \\
    ? & & & & \ruby{下}{くだ}る & う & to go down/downwards (focus on process) & \\
    ? & & & & \ruby{沈}{しず}む & う & to set (sun); be sunken/submerged & \\
    ? & & & & \ruby{下}{お}りる & る & to descend to a lower place & \\
    \ruby{降}{お}ろす & う & to drop off passenger/unload goods & & \ruby{降}{お}りる & る & to alight/disembark & \\
    \ruby{下}{お}ろす & う & to take down/bring down/lower & & \ruby{下}{さ}がる & う & to go downwards\emph{/backwards} (focus on destination) & \href{https://ja.hinative.com/questions/7054838\#answer-36801861}{[HN]} \\
    & & & & & & & \\
    \ruby{入}{い}れる & る & to put in/bring in/let in/insert/install (software) & & \ruby{入}{はい}る & \textredbf{う} & to enter/arrive/join/get into/fit into & \href{https://ja.hinative.com/questions/15301215}{[HN]} \\
    & & & & & & & \\
    & & & & \ruby{昇進}{しょう|しん}する & E & to promote/rise in rank (workplace) & \\
    & & & & & & & \\
    \ruby{減}{へ}らす & う & to decrease & & \ruby{減}{へ}る & \textredbf{う} & to be decreased & \\
    \ruby{増}{ふ}やす & う & to increase & & \ruby{増}{ふ}える & る & to be increased & \\
    & & & & & & & \\
    & & & & & & & \\
    % & & & & & & & \\
\bottomrule
\end{tabular}%
}
\captionof{table}{Verbs: directions.}
\label{tbl:appendix-vocab-verbs-directions}
\end{center}



\subsubsection{Emotions}
\begin{center}
\centering
\resizebox{\linewidth}{!}{%
% Help: \multicolumn{2}{c}{}, \multirow{2}{*}{}, cmidrule(l){3-5}
\begin{tabular}{@{}lclllcll@{}}
    \toprule
    \multicolumn{4}{c}{\textbf{Transitive}} & \multicolumn{4}{c}{\textbf{Intransitive}} \\ \cmidrule(r){1-4} \cmidrule(l){5-8}
    \textbf{Action} & \textbf{Cat.} & \textbf{Meaning} & \textbf{Notes} & \textbf{Action} & \textbf{Cat.} & \textbf{Meaning} & \textbf{Notes} \\ \midrule
    - & & & & \ruby{泣}{な}く & う & to cry & \\
    & & & & & & & \\
    & & & & & & & \\
    & & & & & & & \\
    & & & & & & & \\
    % & & & & & & & \\
\bottomrule
\end{tabular}%
}
\captionof{table}{Verbs: emotions.}
\label{tbl:appendix-vocab-verbs-emotions}
\end{center}


\subsubsection{Production}
\begin{center}
\centering
\resizebox{\linewidth}{!}{%
% Help: \multicolumn{2}{c}{}, \multirow{2}{*}{}, cmidrule(l){3-5}
\begin{tabular}{@{}lclllcll@{}}
    \toprule
    \multicolumn{4}{c}{\textbf{Transitive}} & \multicolumn{4}{c}{\textbf{Intransitive}} \\ \cmidrule(r){1-4} \cmidrule(l){5-8}
    \textbf{Action} & \textbf{Cat.} & \textbf{Meaning} & \textbf{Notes} & \textbf{Action} & \textbf{Cat.} & \textbf{Meaning} & \textbf{Notes} \\ \midrule
    & & & & & & & \\
    \ruby{作}{つく}る & う & to make/prepare (food)/grow (agriculture)/cultivate (people) & \href{https://dictionary.goo.ne.jp/word/\%E4\%BD\%9C\%E3\%82\%8B}{[goo]} & - & & & \\
    \ruby{造}{つく}る & う & to construct (large-scale buildings, manufacturing) & \href{https://dictionary.goo.ne.jp/word/\%E4\%BD\%9C\%E3\%82\%8B}{[goo]} & - & & & \\
    \ruby{創}{つく}る & う & to create/compose (artistic)/start a business & \href{https://dictionary.goo.ne.jp/word/\%E4\%BD\%9C\%E3\%82\%8B}{[goo]} & - & & & \\
    \ruby{書}{か}く & う & to write & & ? & & & \\
    \ruby{描}{か}く & う & to draw/paint & & ? & & & \\
    \ruby{描}{えが}く & う & to imagine; to depict (abstract concept) & & ? & & & \\
    \ruby{歌}{うた}う & う & to sing & & \ruby{歌}{うた}う & う & to sing \\
    - & & & & \ruby{踊}{おど}る & う & to dance (a hopping dance) & \\
    - & & & & \ruby{小躍}{こ|おど}りする & E & to dance for joy & \\
    \ruby{続}{つづ}ける & る & to continue & \aux & \ruby{続}{つづ}く & う & to continue & \\
    \ruby{話}{はな}す & う & to talk/speak & & - & & & \\
    - & & & & \ruby{泳}{およ}ぐ & う & to swim/weave through a crowd & \\
    \ruby{勉強}{べん|きょう}する & E & to study & & \ruby{勉強}{べん|きょう}する & E & to work hard &  \\
    - & & & & \ruby{無理}{む|り}する & E & to work/try too hard & \\
    \multirow{2}{*}{\ruby{切}{き}る} & \multirow{2}{*}{\textredbf{う}} & to cut/open (sealed); turn off (lights/appliance); hang up & & \multirow{2}{*}{\ruby{切}{き}れる} & \multirow{2}{*}{る} & to be cut/broken; stop working; be disconnected; & \\
    & & (conversation); shuffle/discard (cards/tiles); punch (ticket) & & & & be shuffled (cards/tiles); run out (stock); break up & \\
    & & & & キレる & る & to snap/flip/get angry/lose one's temper &  \\
    <stem>\ruby{切}{き}る & \textredbf{う} & completely & \aux & & & & \\
    - & & & & \ruby{歩}{ある}く & う & to walk & \\
    & & & & ぶらぶらする & E & to walk leisurely/aimlessly & also an adverb \\
    - & & & & \ruby{走}{はし}る & \textredbf{う} & to run; drive (vehicle); flash (lightning); wind (road) & \\
    - & & & & \ruby{満開}{まん|かい}する & E & to be in full bloom (esp.\ of cherry blossom) & \\

    \ruby{取}{と}る & う & to take (notes/break/time)/obtain/pass/obtain & & \ruby{取}{と}れる & る & to come off (button/handle/lid) & \\
    とる & う & to have/take/consume (a meal/vitamins) & (\ruby{摂}{と}る) & & & & \\
    \ruby{撮}{と}る & う & to take a photograph & & \ruby{撮}{と}れる & る & to be taken (photograph) & \\
    \ruby{録}{と}る & う & to record an audio or video & & \ruby{録}{と}れる & る & to be recorded/caught on tape (audio or video) & \\
    \ruby{捕}{と}る & う & to catch an object/capture an animal & & \ruby{捕}{と}れる & る & to be caught (object)/captured (animal) & \\
    \ruby{採}{と}る & う & to adopt (method/proposal); to collect/gather (flowers/plants) & & \ruby{採}{と}れる & る & to be collected/gathered (flowers/plants) & \\
    \ruby{摘}{つ}む & う & to pick/pluck (flowers); to nip/snip/cut/trim & & つまむ & う & to pick up (with chopsticks/tweezers)/pinch/hold & \\
    \ruby{集}{あつ}める & る & to collect/assemble/gather (collectibles/people/information) & & \ruby{集}{あつ}まる & う & to be collected/assembled/gathered & \\
    & & & & & & & \\
    & & & & & & & \\
    % & & & & & & & \\
\bottomrule
\end{tabular}%
}
\captionof{table}{Verbs: production.}
\label{tbl:appendix-vocab-verbs-production}
\end{center}
To use \ruby{続}{つづ}ける as an auxiliary verb, suffix it to the stem of the main verb (e.g.\ 「\ruby{歌}{うた}い\ruby{続}{つづ}ける」 means to continue to sing, and \ruby{歌}{うた}い is the stem of \ruby{歌}{うた}う).

\subsubsection{Consumption}
\begin{center}
\centering
\resizebox{\linewidth}{!}{%
% Help: \multicolumn{2}{c}{}, \multirow{2}{*}{}, cmidrule(l){3-5}
\begin{tabular}{@{}lclllcll@{}}
    \toprule
    \multicolumn{4}{c}{\textbf{Transitive}} & \multicolumn{4}{c}{\textbf{Intransitive}} \\ \cmidrule(r){1-4} \cmidrule(l){5-8}
    \textbf{Action} & \textbf{Cat.} & \textbf{Meaning} & \textbf{Notes} & \textbf{Action} & \textbf{Cat.} & \textbf{Meaning} & \textbf{Notes} \\ \midrule
    \ruby{見}{み}る & る & to see/observe & & \ruby{見}{み}える & る & to be seen/visible & \\
    \ruby{見}{み}つける & る & to find/discover/detect & & \ruby{見}{み}つかる & う & to be found/discovered & \\
    ばらす & う & to expose/disclose/leak a secret & colloquial & ばれる & る & to be exposed/found out/leak a secret & \\
    \ruby{聞}{き}く & う & to hear & & \ruby{聞}{き}こえる & る & to be heard/audible & \\
    \ruby{聴}{き}く & う & to listen attentively (music) & & ? & & & \\
    \ruby{食}{た}べる & る & to eat & & - & & & \\
    \ruby{食}{た}べすぎる & る & to overeat & & - & & & \\
    かぶる & う & to put on (head)/be covered with/shoulder responsibility & & - & & & \\
    \ruby{着}{き}る & る & to wear (upper body) & & - & & & \\
    \ruby{履}{は}く & う & to put on (lower body: pants, shoes) & & - & & & \\
    \ruby{遊}{あそ}ぶ & う & to play & & & & & \\
    \ruby{遊}{あそ}ばす & う & to entertain/amuse someone & & - & & & \\
    \ruby{飲}{の}む & う & to drink/swallow/take medicine & & - & & & \\
    \ruby{呑}{の}む & う & to gulp/swallow whole & & - & & & \\
    \ruby{買}{か}う & う & to buy & & - & & & \\
    & & & & & & & \\
    & & & & & & & \\
    % & & & & & & & \\
\bottomrule
\end{tabular}%
}
\captionof{table}{Verbs: consumption.}
\label{tbl:appendix-vocab-verbs-consumption}
\end{center}


\subsubsection{Interaction}
\begin{center}
\centering
\resizebox{\linewidth}{!}{%
% Help: \multicolumn{2}{c}{}, \multirow{2}{*}{}, cmidrule(l){3-5}
\begin{tabular}{@{}lclllcll@{}}
    \toprule
    \multicolumn{4}{c}{\textbf{Transitive}} & \multicolumn{4}{c}{\textbf{Intransitive}} \\ \cmidrule(r){1-4} \cmidrule(l){5-8}
    \textbf{Action} & \textbf{Cat.} & \textbf{Meaning} & \textbf{Notes} & \textbf{Action} & \textbf{Cat.} & \textbf{Meaning} & \textbf{Notes} \\ \midrule
    - & & & & ある & う & to exist/have (inaminate) & \\
    - & & & & いる & る & to exist (animate) & (\ruby{居}{い}る) \\
    - & & & & いる & る & to need/want & (\ruby{要}{い}る)\\
    & & & & なる & う & to become/consist of/be completed & \\
    & & & & お<stem>になる & う & to do <stem> & \aux \\
    & & & & ご<noun>になる & う & to do <noun> & \aux \\
    & & & & & & & \\
    \ruby{思}{おも}う & う & to think/believe/judge/imagine/recall & & - & & & \\
    \ruby{訊}{き}く & う & to ask/enquire & & ? & & & \\
    ? & & & & (ご)\ruby{注意}{ちゅう|い}する & E & to pay attention/remind/caution & \\
    & & & & & & & \\
    \ruby{感謝}{かん|しゃ}する & E & to thank & & \ruby{感謝}{かん|しゃ}する & E & to be thanked & \\
    \ruby{信}{しん}じる & る & to believe/trust/have faith in & & - & & & \\
    \ruby{信用}{しん|よう}する & E & to trust (information/source; past) & \href{https://japanese.stackexchange.com/q/24275}{[SE]} & - & & & \\
    \ruby{信頼}{しん|らい}する & E & to trust (a person/organisation; future) & \href{https://japanese.stackexchange.com/q/24275}{[SE]} & - & & & \\
    \ruby{一緒}{いっ|しょ}にする & E & to do together/unite/mix & & & & & \\
    \ruby{考}{かん}える & る & to consider/think over/reflect on & & - & & & \\
    \ruby{教}{おし}える & る & to teach/inform & & - & & & \\
    \ruby{習}{なら}う & う & to take lessons/learn/be trained (under a teacher) & & - & & & \\
    \ruby{練習}{れん|しゅう}する & E & to practise/train/drill & & - & & & \\
    \ruby{自習}{じ|しゅう}する & E & to self-study & & - & & & \\
    % The following line break is necessary to prevent interpretation of the next [

    [<with list>と] \ruby{一緒}{いっ|しょ}になる & う & to rendezvous/join/meet together/get married with & \htc & & & \\  % Hard To Categorise: neither strictly transitive nor strictly intransitive?
    & & & & なる & う & to become/get/attain/reach/turn into & \\
    & & & & わかる & う & to understand/comprehend & \\
    & & & & \ruby{質問}{しつ|もん}をする & E & to ask a question & \htc \\
    & & & & All WOSURU family & & & \\
    \ruby{待}{ま}つ & う & to wait & & \ruby{待}{ま}つ & う & to wait & \\
    & & & & & & & \\
    \ruby{馬鹿}{ば|か}にする & E & to make fun of/look down on/make light of & & - & & & \\
    \ruby{気}{き}にする & E & (negative nuance) to mind/care/worry about something & & \ruby{気}{き}にする & E & (negative nuance) to mind/care/worry about something & \\
    - & & & & \ruby{答}{こた}える & る & To answer/reply & \\
    \ruby{登録}{とう|ろく}する & E & to be entered into a register/to register & \href{https://dictionary.goo.ne.jp/word/\%e7\%99\%bb\%e9\%8c\%b2/}{[goo]} & & & & \\
    - & & & & \ruby{入会}{にゅう|かい}する & E & to enrol/admit into a club/society/mailing list & \href{https://ja.hinative.com/questions/22502664}{[HN]} \\
    - & & & & \ruby{退会}{たい|かい}する & E & to withdraw/resign from a club/society/mailing list & \\
    - & & & & \ruby{加入}{か|にゅう}する & E & to join a group/project & \href{https://ja.hinative.com/questions/22502664}{[HN]} \\
    & & & & & & & \\
    - & & & & \ruby{会}{あ}う & う & to meet/encounter & \\
    - & & & & \ruby{逢}{あ}う & う & to meet/encounter (close friends/romantic) & \href{https://ja.hinative.com/questions/22148235}{[HN]} \\
    - & & & & \ruby{遭}{あ}う & う & to have an undesired meeting/experience/accident & \\
    ? & & & & \ruby{電話}{でん|わ}する & E & to call (phone call) & \\
    - & & & & \ruby{勝}{か}つ & う & to win (personal) & \\
    - & & & & \ruby{勝利}{しょう|り}する & E & to win (larger scale) & \\
    - & & & & \ruby{負}{ま}ける & る & to lose (personal)/succumb to/give in to & \\
    - & & & & \ruby{敗北}{はい|ぼく}する & E & to lose (larger scale)& \\
    & & & & & & & \\
    % & & & & & & & \\
\bottomrule
\end{tabular}%
}
\captionof{table}{Verbs: interaction.}
\label{tbl:appendix-vocab-verbs-interaction}
\end{center}


\subsubsection{Health}
\begin{center}
\centering
\resizebox{\linewidth}{!}{%
% Help: \multicolumn{2}{c}{}, \multirow{2}{*}{}, cmidrule(l){3-5}
\begin{tabular}{@{}lclllcll@{}}
    \toprule
    \multicolumn{4}{c}{\textbf{Transitive}} & \multicolumn{4}{c}{\textbf{Intransitive}} \\ \cmidrule(r){1-4} \cmidrule(l){5-8}
    \textbf{Action} & \textbf{Cat.} & \textbf{Meaning} & \textbf{Notes} & \textbf{Action} & \textbf{Cat.} & \textbf{Meaning} & \textbf{Notes} \\ \midrule
    & & & & \ruby{寝}{ね}る & る & to lie down/go to bed & \\
    & & & & \ruby{生}{い}きる & る & to live/come to life/make a living & \\
    & & & & \ruby{死}{し}ぬ & う & to die/pass away & \\
    & & & & & & & \\
    % & & & & & & & \\
\bottomrule
\end{tabular}%
}
\captionof{table}{Verbs: health.}
\label{tbl:appendix-vocab-verbs-health}
\end{center}


\subsubsection{Change}
\begin{center}
\centering
\resizebox{\linewidth}{!}{%
% Help: \multicolumn{2}{c}{}, \multirow{2}{*}{}, cmidrule(l){3-5}
\begin{tabular}{@{}lclllcll@{}}
    \toprule
    \multicolumn{4}{c}{\textbf{Transitive}} & \multicolumn{4}{c}{\textbf{Intransitive}} \\ \cmidrule(r){1-4} \cmidrule(l){5-8}
    \textbf{Action} & \textbf{Cat.} & \textbf{Meaning} & \textbf{Notes} & \textbf{Action} & \textbf{Cat.} & \textbf{Meaning} & \textbf{Notes} \\ \midrule
    ? & & & & \ruby{声変}{こえ|が}わりする & E & to break voice & \\
    & & & & & & & \\
    & & & & & & & \\
    & & & & & & & \\
    & & & & & & & \\
    % & & & & & & & \\
\bottomrule
\end{tabular}%
}
\captionof{table}{Verbs: change.}
\label{tbl:appendix-vocab-verbs-change}
\end{center}


\subsection{Adverbs}
Adverbs modify both verbs and adjectives. They may also modify entire noun phrases or sentences.

\subsubsection{Intensity modifiers}
\begin{center}
\centering
\resizebox{\linewidth}{!}{%
% Help: \multicolumn{2}{c}{}, \multirow{2}{*}{}, cmidrule(l){3-5}
\begin{tabular}{@{}lll@{}}
    \toprule
    \textbf{Modifier} & \textbf{Meaning} & \textbf{Notes} \\ \midrule
    あまり<negative> & not very & \\
    % <so>のあまり<verb>& so much <so> that you <verb> & 嬉しさのあまり\ruby{泣}{な}いた。\\
    \ruby{別}{べつ}に<negative> & not particularly & \\
    \ruby{別}{べつ}に & nothing/not really & \\
    つまり & in short/in other words & \\
    \ruby{全然}{ぜん|ぜん}<negative> & not at all & \\
    \ruby{多分}{た|ぶん} & probably/perhaps & \\
    もう & already; not any more/longer; again/another & again/another: used with counting 1 \\
    \ruby{確}{たし}かに & certainly/for sure/indeed/really & \\
    \ruby{少}{すこ}し & somewhat/slightly/a little & \\
    もしや & possibly/perhaps/by some chance & \\
    もしかし & maybe/perhaps/by some chance & \\
    \ruby{全部}{ぜん|ぶ} & entirely/wholly/altogether & also a noun \\
    & & \\
    ぷにぷに & squishy/springy/bouncy (chubby when used on person) & new cat.? \\
    ふわふわ & lightly/buoyantly & also an adjective \\
    \ruby{大}{だい} & large/big/great/severe & \prefix. \htc; technically な-adj/noun \\
    & & \\
    さっさと & immediately/without delay/hurriedly/quickly & \\
    ずっと & the whole time/continuously; much (more); (by) far & \\
    & & \\
    <with>と\ruby{一緒}{いっ|しょ}に<verb> & together with & \\
    \ruby{久}{ひさ}しぶりに & for the first time in a while/after a long time & \\
    \ruby{早}{はや}く & Early/soon/quickly/swiftly/rapidly & \\
    & & \\
    \ruby{例}{たと}えば & for example/for instance & \\
    ぶらぶら & (walking) leisurely/aimlessly & also a verb \\
    & & \\
    % & & \\
\bottomrule
\end{tabular}%
}
\captionof{table}{Adverbs: intensity modifiers.}
\label{tbl:appendix-vocab-adverbs-intensity}
\end{center}

\end{document}
