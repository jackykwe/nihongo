\documentclass[../nihongo-gakushuu-kyouzai.tex]{subfiles}
\begin{document}
\appendix
\setcounter{section}{1}
\section{Vocabulary mega summary}
\tableofcontents


\subsection{Interjections and expressions}

\begin{center}
\resizebox{\linewidth}{!}{%
% Help: \multicolumn{2}{c}{}, \multirow{2}{*}{}, cmidrule(l){3-5}
\begin{tabular}{@{}lll@{}}
    \toprule
    \textbf{Interjection} & \textbf{Meaning} & \textbf{Notes} \\ \midrule
    はい & Yes/that is correct/I'm here/pardon? & \\
    うん & Yes/yeah/mhmm & \\
    \ruby{大丈夫}{だい|じょう|ぶ} & No thanks/I'm good & \\
    いえ/いいえ & No & \\
    ううん/うーん & Um/well/no & \\
    おはよう & Good morning & \\
    おはようございます & Good morning & Polite \\
    お\ruby{休}{やす}み[なさい] & Good night & \\
    ようこそ & Welcome & \\
    いらっしゃいませ & Welcome (in shops and restaurants) & \\
    ごちそうさま & Thank you for the meal & \\
    ごちそうさまでした & Thank you for the meal & Polite \\
    どうも & Thanks (abbreviation) & \\
    どうもありがと & Thank you very much & \\
    \ruby{何}{なに}か & What (are you trying to say/do you mean)? & \\
    もしもし & Hello (on phone)/excuse me (calling out to someone) & \\
    \ruby{助}{たす}けて & Help! & \\
    お\ruby{待}{ま}たせしました & Thank you for waiting/sorry to have kept you waiting & Polite \\
    \ruby{失礼}{しつ|れい}します & Goodbye/excuse me (while I take my leave) & Polite \\
    \ruby{失礼}{しつ|れい}しました & I'm sorry/excuse me/my apologies & Polite \\
    & & \\
    % & & \\
    \bottomrule
\end{tabular}%
}
\captionof{table}{Interections and expressions.}
\label{tbl:appendix-vocab-interjections-and-expressions}
\end{center}

\begin{multicols}{2}
[
\subsection{Nouns}
]


\subsubsection{Counting (generic)}
\begin{center}
\resizebox{\linewidth}{!}{%
% Help: \multicolumn{2}{c}{}, \multirow{2}{*}{}, cmidrule(l){3-5}
\begin{tabular}{@{}lll@{}}
    \toprule
    \textbf{Name} & \textbf{Meaning} & \textbf{Notes} \\ \midrule
    ichi ni san... & & \\
    \ruby{三}{みっ}つ/\ruby{3}{みっ}つ & three & \\
    & & \\
    & & \\
    % & & \\
    \bottomrule
\end{tabular}%
}
\captionof{table}{Nouns: counting (generic).}
\label{tbl:appendix-vocab-nouns-counting-generic}
\end{center}



\subsubsection{Counting people}
\begin{center}
\resizebox{\linewidth}{!}{%
% Help: \multicolumn{2}{c}{}, \multirow{2}{*}{}, cmidrule(l){3-5}
\begin{tabular}{@{}lll@{}}
    \toprule
    \textbf{Name} & \textbf{Meaning} & \textbf{Notes} \\ \midrule
    \ruby{一人}{ひと|り}/\ruby{1人}{ひと|り} & One person; being alone/single & \\
    & & \\
    \ruby{一緒}{いっ|しょ} & Together & \\
    % & & \\
    \bottomrule
\end{tabular}%
}
\captionof{table}{Nouns: counting people.}
\label{tbl:appendix-vocab-nouns-counting-people}
\end{center}


\subsubsection{Counting age}
\begin{center}
\resizebox{\linewidth}{!}{%
% Help: \multicolumn{2}{c}{}, \multirow{2}{*}{}, cmidrule(l){3-5}
\begin{tabular}{@{}lll@{}}
    \toprule
    \textbf{Name} & \textbf{Meaning} & \textbf{Notes} \\ \midrule
    & & \\
    & & \\
    % & & \\
    \bottomrule
\end{tabular}%
}
\captionof{table}{Nouns: counting age.}
\label{tbl:appendix-vocab-nouns-counting-age}
\end{center}


\subsubsection{Counting thin/flat things}
\begin{center}
\resizebox{\linewidth}{!}{%
% Help: \multicolumn{2}{c}{}, \multirow{2}{*}{}, cmidrule(l){3-5}
\begin{tabular}{@{}lll@{}}
    \toprule
    \textbf{Name} & \textbf{Meaning} & \textbf{Notes} \\ \midrule
    & & \\
    & & \\
    % & & \\
    \bottomrule
\end{tabular}%
}
\captionof{table}{Nouns: counting thin/flat things.}
\label{tbl:appendix-vocab-nouns-counting-thin-flat-things}
\end{center}


\subsubsection{Counting thin long things (bottles)}
\begin{center}
\resizebox{\linewidth}{!}{%
% Help: \multicolumn{2}{c}{}, \multirow{2}{*}{}, cmidrule(l){3-5}
\begin{tabular}{@{}lll@{}}
    \toprule
    \textbf{Name} & \textbf{Meaning} & \textbf{Notes} \\ \midrule
    & & \\
    & & \\
    % & & \\
    \bottomrule
\end{tabular}%
}
\captionof{table}{Nouns: counting thin long things (bottles).}
\label{tbl:appendix-vocab-nouns-counting-thin-long-things-bottles}
\end{center}


\subsubsection{Counting drinks (cups/glasses)}
\begin{center}
\resizebox{\linewidth}{!}{%
% Help: \multicolumn{2}{c}{}, \multirow{2}{*}{}, cmidrule(l){3-5}
\begin{tabular}{@{}lll@{}}
    \toprule
    \textbf{Name} & \textbf{Meaning} & \textbf{Notes} \\ \midrule
    & & \\
    & & \\
    % & & \\
    \bottomrule
\end{tabular}%
}
\captionof{table}{Nouns: counting drinks (cups/glasses).}
\label{tbl:appendix-vocab-nouns-counting-drinks-cups-glasses}
\end{center}


\subsubsection{Counting machines/vehicles}
\begin{center}
\resizebox{\linewidth}{!}{%
% Help: \multicolumn{2}{c}{}, \multirow{2}{*}{}, cmidrule(l){3-5}
\begin{tabular}{@{}lll@{}}
    \toprule
    \textbf{Name} & \textbf{Meaning} & \textbf{Notes} \\ \midrule
    & & \\
    & & \\
    % & & \\
    \bottomrule
\end{tabular}%
}
\captionof{table}{Nouns: counting machines/vehicles.}
\label{tbl:appendix-vocab-nouns-counting-machines-vehicles}
\end{center}


\subsubsection{Counting books}
\begin{center}
\resizebox{\linewidth}{!}{%
% Help: \multicolumn{2}{c}{}, \multirow{2}{*}{}, cmidrule(l){3-5}
\begin{tabular}{@{}lll@{}}
    \toprule
    \textbf{Name} & \textbf{Meaning} & \textbf{Notes} \\ \midrule
    & & \\
    & & \\
    % & & \\
    \bottomrule
\end{tabular}%
}
\captionof{table}{Nouns: counting books.}
\label{tbl:appendix-vocab-nouns-counting-books}
\end{center}


\subsubsection{Counting clothes}
\begin{center}
\resizebox{\linewidth}{!}{%
% Help: \multicolumn{2}{c}{}, \multirow{2}{*}{}, cmidrule(l){3-5}
\begin{tabular}{@{}lll@{}}
    \toprule
    \textbf{Name} & \textbf{Meaning} & \textbf{Notes} \\ \midrule
    & & \\
    & & \\
    % & & \\
    \bottomrule
\end{tabular}%
}
\captionof{table}{Nouns: counting clothes.}
\label{tbl:appendix-vocab-nouns-counting-clothes}
\end{center}


\subsubsection{Counting small things}
\begin{center}
\resizebox{\linewidth}{!}{%
% Help: \multicolumn{2}{c}{}, \multirow{2}{*}{}, cmidrule(l){3-5}
\begin{tabular}{@{}lll@{}}
    \toprule
    \textbf{Name} & \textbf{Meaning} & \textbf{Notes} \\ \midrule
    \ruby{個}{こ} & Counter for small things & \\
    \ruby{個々}{こ|こ} & Individual & E.g.\ 「そのクラスの個々のメンバー\dots」, like CN's 个个\\
    & & \\
    & & \\
    % & & \\
    \bottomrule
\end{tabular}%
}
\captionof{table}{Nouns: counting small things.}
\label{tbl:appendix-vocab-nouns-counting-small-things}
\end{center}


\subsubsection{Counting shoes and socks}
\begin{center}
\resizebox{\linewidth}{!}{%
% Help: \multicolumn{2}{c}{}, \multirow{2}{*}{}, cmidrule(l){3-5}
\begin{tabular}{@{}lll@{}}
    \toprule
    \textbf{Name} & \textbf{Meaning} & \textbf{Notes} \\ \midrule
    & & \\
    & & \\
    % & & \\
    \bottomrule
\end{tabular}%
}
\captionof{table}{Nouns: counting shoes and socks.}
\label{tbl:appendix-vocab-nouns-counting-shoes-and-socks}
\end{center}


\subsubsection{Counting houses}
\begin{center}
\resizebox{\linewidth}{!}{%
% Help: \multicolumn{2}{c}{}, \multirow{2}{*}{}, cmidrule(l){3-5}
\begin{tabular}{@{}lll@{}}
    \toprule
    \textbf{Name} & \textbf{Meaning} & \textbf{Notes} \\ \midrule
    & & \\
    & & \\
    % & & \\
    \bottomrule
\end{tabular}%
}
\captionof{table}{Nouns: counting houses.}
\label{tbl:appendix-vocab-nouns-counting-houses}
\end{center}



\subsubsection{Counting floors}
\begin{center}
\resizebox{\linewidth}{!}{%
% Help: \multicolumn{2}{c}{}, \multirow{2}{*}{}, cmidrule(l){3-5}
\begin{tabular}{@{}lll@{}}
    \toprule
    \textbf{Name} & \textbf{Meaning} & \textbf{Notes} \\ \midrule
    & & \\
    & & \\
    % & & \\
    \bottomrule
\end{tabular}%
}
\captionof{table}{Nouns: counting floors.}
\label{tbl:appendix-vocab-nouns-counting-floors}
\end{center}



\subsubsection{Counting small animals}
\begin{center}
\resizebox{\linewidth}{!}{%
% Help: \multicolumn{2}{c}{}, \multirow{2}{*}{}, cmidrule(l){3-5}
\begin{tabular}{@{}lll@{}}
    \toprule
    \textbf{Name} & \textbf{Meaning} & \textbf{Notes} \\ \midrule
    & & \\
    & & \\
    % & & \\
    \bottomrule
\end{tabular}%
}
\captionof{table}{Nouns: counting small animals.}
\label{tbl:appendix-vocab-nouns-counting-small-animals}
\end{center}


\subsubsection{Counting big animals}
\begin{center}
\resizebox{\linewidth}{!}{%
% Help: \multicolumn{2}{c}{}, \multirow{2}{*}{}, cmidrule(l){3-5}
\begin{tabular}{@{}lll@{}}
    \toprule
    \textbf{Name} & \textbf{Meaning} & \textbf{Notes} \\ \midrule
    & & \\
    & & \\
    % & & \\
    \bottomrule
\end{tabular}%
}
\captionof{table}{Nouns: counting big animals.}
\label{tbl:appendix-vocab-nouns-counting-big-animals}
\end{center}



\subsubsection{Counting hours}
\begin{center}
\resizebox{\linewidth}{!}{%
% Help: \multicolumn{2}{c}{}, \multirow{2}{*}{}, cmidrule(l){3-5}
\begin{tabular}{@{}lll@{}}
    \toprule
    \textbf{Name} & \textbf{Meaning} & \textbf{Notes} \\ \midrule
    & & \\
    & & \\
    % & & \\
    \bottomrule
\end{tabular}%
}
\captionof{table}{Nouns: counting hours.}
\label{tbl:appendix-vocab-nouns-counting-hours}
\end{center}


\subsubsection{Counting minutes}
\begin{center}
\resizebox{\linewidth}{!}{%
% Help: \multicolumn{2}{c}{}, \multirow{2}{*}{}, cmidrule(l){3-5}
\begin{tabular}{@{}lll@{}}
    \toprule
    \textbf{Name} & \textbf{Meaning} & \textbf{Notes} \\ \midrule
    & & \\
    & & \\
    % & & \\
    \bottomrule
\end{tabular}%
}
\captionof{table}{Nouns: counting minutes.}
\label{tbl:appendix-vocab-nouns-counting-minutes}
\end{center}


\subsubsection{Counting seconds}
\begin{center}
\resizebox{\linewidth}{!}{%
% Help: \multicolumn{2}{c}{}, \multirow{2}{*}{}, cmidrule(l){3-5}
\begin{tabular}{@{}lll@{}}
    \toprule
    \textbf{Name} & \textbf{Meaning} & \textbf{Notes} \\ \midrule
    & & \\
    & & \\
    % & & \\
    \bottomrule
\end{tabular}%
}
\captionof{table}{Nouns: counting seconds.}
\label{tbl:appendix-vocab-nouns-counting-seconds}
\end{center}


\subsubsection{Counting positions}
\begin{center}
\resizebox{\linewidth}{!}{%
% Help: \multicolumn{2}{c}{}, \multirow{2}{*}{}, cmidrule(l){3-5}
\begin{tabular}{@{}lll@{}}
    \toprule
    \textbf{Name} & \textbf{Meaning} & \textbf{Notes} \\ \midrule
    & & \\
    & & \\
    % & & \\
    \bottomrule
\end{tabular}%
}
\captionof{table}{Nouns: counting positions.}
\label{tbl:appendix-vocab-nouns-counting-positions}
\end{center}


\subsubsection{Counting occurrences}
\begin{center}
\resizebox{\linewidth}{!}{%
% Help: \multicolumn{2}{c}{}, \multirow{2}{*}{}, cmidrule(l){3-5}
\begin{tabular}{@{}lll@{}}
    \toprule
    \textbf{Name} & \textbf{Meaning} & \textbf{Notes} \\ \midrule
    & & \\
    & & \\
    % & & \\
    \bottomrule
\end{tabular}%
}
\captionof{table}{Nouns: counting occurrences.}
\label{tbl:appendix-vocab-nouns-counting-occurrences}
\end{center}


\subsubsection{Calendar months and days of a week}
\begin{center}
\resizebox{\linewidth}{!}{%
% Help: \multicolumn{2}{c}{}, \multirow{2}{*}{}, cmidrule(l){3-5}
\begin{tabular}{@{}lll@{}}
    \toprule
    \textbf{Name} & \textbf{Meaning} & \textbf{Notes} \\ \midrule
    & & \\
    & & \\
    % & & \\
    \bottomrule
\end{tabular}%
}
\captionof{table}{Nouns: Calendar months and days of a week.}
\label{tbl:appendix-vocab-nouns-calendar-months-and-days-of-a-week}
\end{center}


\subsubsection{Calendar days}
\begin{center}
\resizebox{\linewidth}{!}{%
% Help: \multicolumn{2}{c}{}, \multirow{2}{*}{}, cmidrule(l){3-5}
\begin{tabular}{@{}lll@{}}
    \toprule
    \textbf{Name} & \textbf{Meaning} & \textbf{Notes} \\ \midrule
    & & \\
    & & \\
    % & & \\
    \bottomrule
\end{tabular}%
}
\captionof{table}{Nouns: calendar days.}
\label{tbl:appendix-vocab-nouns-calendar-days}
\end{center}


\subsubsection{Directions}
\begin{center}
\resizebox{\linewidth}{!}{%
% Help: \multicolumn{2}{c}{}, \multirow{2}{*}{}, cmidrule(l){3-5}
\begin{tabular}{@{}lll@{}}
    \toprule
    \textbf{Name} & \textbf{Meaning} & \textbf{Notes} \\ \midrule
    \ruby{上}{うえ} & Up & \\
    \ruby{下}{した} & Down & \\
    \ruby{左}{ひだり} & Left & \\
    \ruby{右}{みぎ} & Right & \\
    \ruby{上}{のぼ}り & Upwards/upbound/ascent & \\
    \ruby{下}{くだ}り & Donwards/downbound/descent & \\
    & & \\
    \ruby{遠回}{とお|まわ}り & Detour/roundabout way & \\
    \ruby{昇進}{しょう|しん} & Promotion/rise in rank & E.g.\ Workplace: \ruby{昇進人事}{しょう|しん|じん|じ} \\
    \ruby{格上}{かく|あ}げ & Status upgrade/promotion & E.g.\ Friendship status \\
    \ruby{降格}{こう|かく} & Demotion/drop in rank & Eg.\ Workplace: \ruby{降格人事}{こう|かく|じん|じ} \\
    \ruby{格下}{かく|さ}げ & Status downgrade/demotion & E.g.\ Friendship status  \\
    & & \\
    & & \\
    & & \\
    % & & \\
    \bottomrule
\end{tabular}%
}
\captionof{table}{Nouns: directions.}
\label{tbl:appendix-vocab-nouns-directions}
\end{center}

\subsubsection{Places}
\begin{center}
\resizebox{\linewidth}{!}{%
% Help: \multicolumn{2}{c}{}, \multirow{2}{*}{}, cmidrule(l){3-5}
\begin{tabular}{@{}lll@{}}
    \toprule
    \textbf{Name} & \textbf{Meaning} & \textbf{Notes} \\ \midrule
    \ruby{町} & Street/neighbourhood & \\
    \ruby{町内会}{ちょう|ない|かい} & Neighbourhood association & \\
    \ruby{図書館}{と|しょ|かん} & Library & \\
    \ruby{扉}{とびら} & Door/gate/opening & Also: ドア \\
    \ruby{橋}{はし} & Bridge & \\
    \ruby{山}{やま} & Mountain/hill & \\
    \ruby{山々}{やま|やま} & Mountains/hills & \\
    ビル & (Multi-floor) building & \\
    レストラン & Restaurant (Western) & \\
    & & \\
    & & \\
    % & & \\
    \bottomrule
\end{tabular}%
}
\captionof{table}{Nouns: places.}
\label{tbl:appendix-vocab-nouns-places}
\end{center}


\subsubsection{Small objects: stationery}
\begin{center}
\resizebox{\linewidth}{!}{%
% Help: \multicolumn{2}{c}{}, \multirow{2}{*}{}, cmidrule(l){3-5}
\begin{tabular}{@{}lll@{}}
    \toprule
    \textbf{Name} & \textbf{Meaning} & \textbf{Notes} \\ \midrule
    \ruby{書}{しょ} & Book/document & \\
    \ruby{辞書}{じ|しょ} & Dictionary & \\
    \ruby{本}{ほん} & Book/volume/script & \\
    \ruby{帳}{ちょう} & Book/register & \\
    \ruby{紙}{かみ} & Paper & \\
    & & \\
    & & \\
    % & & \\
    \bottomrule
\end{tabular}%
}
\captionof{table}{Nouns: small objects: stationery.}
\label{tbl:appendix-vocab-nouns-small-objects-stationery}
\end{center}

\subsubsection{Date}
\begin{center}
\resizebox{\linewidth}{!}{%
% Help: \multicolumn{2}{c}{}, \multirow{2}{*}{}, cmidrule(l){3-5}
\begin{tabular}{@{}lll@{}}
    \toprule
    \textbf{Name} & \textbf{Meaning} & \textbf{Notes} \\ \midrule
    \ruby[g]{昨日}{きのう} & Yesterday & Also an adverb. \\
    \ruby[g]{今日}{きょう} & Today & Also an adverb. \\
    \ruby[g]{明日}{あした} & Tomorrow, as in CN's 明天 & Also an adverb. \\
    \ruby{翌日}{よく|じつ} & Next day, as in CN's 隔(一)天 & \\
    \ruby{翌々日}{よく|よく|じつ} & Two days later, as in CN's 隔(两)天 & \\
    \ruby{日}{ひ}にち & (Referring to) the date of an event & 「<event>の日にち」 \\
    \ruby{日々}{ひ|び} & Everyday/daily & \\
    & & \\
    & & \\
    \ruby{誕生日}{たん|じょう|び} & Birthday & \\
    & & \\
    \bottomrule
\end{tabular}%
}
\captionof{table}{Nouns: date.}
\label{tbl:appendix-vocab-nouns-date}
\end{center}

\subsubsection{Time}
\begin{center}
\resizebox{\linewidth}{!}{%
% Help: \multicolumn{2}{c}{}, \multirow{2}{*}{}, cmidrule(l){3-5}
\begin{tabular}{@{}lll@{}}
    \toprule
    \textbf{Name} & \textbf{Meaning} & \textbf{Notes} \\ \midrule
    \ruby{時}{とき} & Time/moment & E.g.\ 「16\ruby{歳}{さい}の\ruby{時}{とき}私は\ruby{声変}{こえ|が}わりした。」\\
    \ruby{刻}{とき} & (Referring to) time of day & \\
    \ruby{秋}{とき} & Important time & \\
    \ruby{時々}{とき|どき} & Sometimes/occasionally & \\
    & & \\
    \bottomrule
\end{tabular}%
}
\captionof{table}{Nouns: time.}
\label{tbl:appendix-vocab-nouns-time}
\end{center}



\subsubsection{Pronouns and question words}
Gramatically, pronouns are used in place of nouns and noun phrases.
\begin{center}
\resizebox{\linewidth}{!}{%
% Help: \multicolumn{2}{c}{}, \multirow{2}{*}{}, cmidrule(l){3-5}
\begin{tabular}{@{}lll@{}}
    \toprule
    \textbf{Name} & \textbf{Meaning} & \textbf{Notes} \\ \midrule
    \ruby{何}{なに} & What & \\
    \ruby{何}{なに}か & Something & Also an interjection. \\
    \ruby{誰}{だれ} & Who & \\
    & & \\
    \ruby{私}{わたくし} & I/me & Formal \\
    \ruby{私}{あたし} & I/me & Feminine, less common \\
    うち & I/me & Feminine, familiar (Kanji is \ruby{内}{うち}) \\
    \ruby{私}{わたし} & I/me & Slightly formal/distant \\
    \ruby{僕}{ぼく} & I/me & Masculine, distant \\
    \ruby{自分}{じ|ぶん} & Myself/oneself/himself/herself/I/me & Distant \\
    \ruby{俺}{おれ} & I/me & Masculine, familiar \\
    & & \\
    & & \\
    こちら & This one/way/direction (here, closer to speaker) & \\
    これ & This one/way/direction (here, closer to speaker) & Slightly informal \\
    そちら & That one/way/direction (there, closer to listener) & \\
    それ & That one/way/direction (there, closer to listener) & Slightly informal \\
    あちら & That one/way/direction (there, distant) & \\
    あれ & That one/way/direction (there, distant) & Slightly informal \\
    & & \\
    あなた & You & Rude/distant \\
    あなたたち & You (plural) & Rude/distant \\
    お\ruby{前}{まえ} & You & Rude \\
    お\ruby{前}{まえ}ら & You (plural) & Rude \\
    % & & \\
    \bottomrule
\end{tabular}%
}
\captionof{table}{Nouns: pronouns and question words.}
\label{tbl:appendix-vocab-nouns-pronouns-and-question-words}
\end{center}

\subsubsection{Roles and occupations}
\begin{center}
\resizebox{\linewidth}{!}{%
% Help: \multicolumn{2}{c}{}, \multirow{2}{*}{}, cmidrule(l){3-5}
\begin{tabular}{@{}lll@{}}
    \toprule
    \textbf{Name} & \textbf{Meaning} & \textbf{Notes} \\ \midrule
    \ruby{女}{おんあ} & Female/woman & \\
    \ruby{女子}{じょ|し} & Woman/girl & \\
    \ruby{女子高生}{じょ|し|こう|せい} & Female high-school student & \\
    \ruby{男}{おとこ} & Make/man & \\
    \ruby{男子}{だん|し} & Man/boy & \\
    \ruby{男子高生}{だん|し|こう|せい} & Male high-school student & \\
    \ruby{学生}{がく|せい} & Student & \\
    \ruby{小学生}{しょう|がく|せい} & Elementary/primary school student& \\
    \ruby{中学生}{ちゅう|がく|せい} & Junior high/middle school student & \\
    \ruby{高校生}{こう|こう|せい} & High school student & \\
    \ruby{大学生}{だい|がく|せい} & Univeristy student & \\
    \ruby{初学者}{しょ|がく|しゃ} & Beginner & \\
    \ruby{将軍}{しょう|ぐん} & General (military, historical) & \\
    \ruby{店長}{\textbf{て}ん|ちょう} & Shop manager & \\
    \ruby{友達}{とも|だち} & Friend & \\
    \ruby{問題児}{もん|だい|じ} & Problem child & \\
    & & \\
    % & & \\
    \bottomrule
\end{tabular}%
}
\captionof{table}{Nouns: roles and occupations.}
\label{tbl:appendix-vocab-nouns-roles-and-occupations}
\end{center}


\subsubsection{Family}
\begin{center}
\resizebox{\linewidth}{!}{%
% Help: \multicolumn{2}{c}{}, \multirow{2}{*}{}, cmidrule(l){3-5}
\begin{tabular}{@{}lll@{}}
    \toprule
    \textbf{Name} & \textbf{Meaning} & \textbf{Notes} \\ \midrule
    \ruby{親子}{おや|こ} & Parent and child & \\
    & & \\
    & & \\
    & & \\
    & & \\
    % & & \\
    \bottomrule
\end{tabular}%
}
\captionof{table}{Nouns: family.}
\label{tbl:appendix-vocab-nouns-family}
\end{center}


\subsubsection{Body parts}
\begin{center}
\resizebox{\linewidth}{!}{%
% Help: \multicolumn{2}{c}{}, \multirow{2}{*}{}, cmidrule(l){3-5}
\begin{tabular}{@{}lll@{}}
    \toprule
    \textbf{Name} & \textbf{Meaning} & \textbf{Notes} \\ \midrule
    \ruby{髪}{かみ} & Hair (on the head) & Also: ヘア \\
    & & \\
    & & \\
    % & & \\
    \bottomrule
\end{tabular}%
}
\captionof{table}{Nouns: body parts.}
\label{tbl:appendix-vocab-nouns-body-parts}
\end{center}


\subsubsection{Clothing}
\begin{center}
\resizebox{\linewidth}{!}{%
% Help: \multicolumn{2}{c}{}, \multirow{2}{*}{}, cmidrule(l){3-5}
\begin{tabular}{@{}lll@{}}
    \toprule
    \textbf{Name} & \textbf{Meaning} & \textbf{Notes} \\ \midrule
    \ruby{制服}{せい|ふく} & Uniform & \\
    & & \\
    & & \\
    & & \\
    % & & \\
    \bottomrule
\end{tabular}%
}
\captionof{table}{Nouns: clothing.}
\label{tbl:appendix-vocab-nouns-clothing}
\end{center}


\subsubsection{Emotions}
\begin{center}
\resizebox{\linewidth}{!}{%
% Help: \multicolumn{2}{c}{}, \multirow{2}{*}{}, cmidrule(l){3-5}
\begin{tabular}{@{}lll@{}}
    \toprule
    \textbf{Name} & \textbf{Meaning} & \textbf{Notes} \\ \midrule
    \ruby{涙}{なみだ} & Tears & \\
    \ruby{一人}{ひと|り}ぼっち & Aloneness/loneliness/solitude & \\
    & & \\
    % & & \\
    \bottomrule
\end{tabular}%
}
\captionof{table}{Nouns: emotions.}
\label{tbl:appendix-vocab-nouns-emotions}
\end{center}


\subsubsection{Production}
\begin{center}
\resizebox{\linewidth}{!}{%
% Help: \multicolumn{2}{c}{}, \multirow{2}{*}{}, cmidrule(l){3-5}
\begin{tabular}{@{}lll@{}}
    \toprule
    \textbf{Name} & \textbf{Meaning} & \textbf{Notes} \\ \midrule
    \ruby{歌}{うた} & Song/singing & \\
    \ruby{踊}{おど}り & Dance & \\
    \ruby{切符}{きっ|ぷ} & Ticket & \\
    \ruby{小躍}{こ|おど}り & Dancing/jumping for joy & \\
    \ruby{結果}{けっ|か} & Result/outcome/consequence & \\
    \ruby{雑誌}{ざっ|し} & Journal/magazine & \\
    \ruby{音楽}{おん|がく} & Music & \\
    \ruby{歌}{うた} & Singing/song & \\
    \ruby{曲}{きょく} & Piece/composition/song/track & \\
    \ruby{作}{つく}り & The making/production/components of & \\
    つもり & Plan/intention; assumption; estimation & \\
    & & \\
    % & & \\
    \bottomrule
\end{tabular}%
}
\captionof{table}{Nouns: production.}
\label{tbl:appendix-vocab-nouns-production]}
\end{center}


\subsubsection{Consumption}
\begin{center}
\resizebox{\linewidth}{!}{%
% Help: \multicolumn{2}{c}{}, \multirow{2}{*}{}, cmidrule(l){3-5}
\begin{tabular}{@{}lll@{}}
    \toprule
    \textbf{Name} & \textbf{Meaning} & \textbf{Notes} \\ \midrule
    かばん & Bag/briefcase/basket & \\
    \ruby{袋}{ふくろ} & Bag/sack/pouch & \\
    \ruby{丼}{どんぶり}/\ruby{丼}{どん} & Porcelain bowl/meat served over rice & \\
    ご\ruby{飯}{はん} & Cooked rice/meal & \\
    \ruby{朝}{あさ}ご\ruby{飯}{はん} & Breakfast & \\
    \ruby{昼}{ひる}ご\ruby{飯}{はん} & Lunch & \\
    \ruby{晩}{ばん}ご\ruby{飯}{はん} & Dinner & \\
    \ruby{卵}{たまご} & Eggs/egg/roe & \\
    そば & Buckwheat & \\
    \ruby{抹茶}{まっ|ちゃ} & Matcha, powdered green tea & \\
    \ruby{試験}{し|けん} & Examination/test & Also: テスト \\
    \ruby{魚}{さかな} & Fish & \\
    \ruby{肉}{にく} & Meat & \\
    \ruby{野菜}{や|さい} & Vegetable & Also: ベジタブル \\
    \ruby{食}{た}べ\ruby{物}{もの} & Food & \\
    \ruby{飲}{の}み & The act of drinking & \\
    \ruby{飲}{の}み\ruby{物}{もの} & Beverage & \\
    ビル & Bill/invoice & \\
    \ruby{値段}{ね|だん} & Price/cost & \\
    金 & Money & \\
    お\ruby{金}{かね} & Money & Polite. \\
    & & \\
    & & \\
    & & \\
    % & & \\
    \bottomrule
\end{tabular}%
}
\captionof{table}{Nouns: consumption.}
\label{tbl:appendix-vocab-nouns-consumption}
\end{center}



\subsubsection{Interaction}
\begin{center}
\resizebox{\linewidth}{!}{%
% Help: \multicolumn{2}{c}{}, \multirow{2}{*}{}, cmidrule(l){3-5}
\begin{tabular}{@{}lll@{}}
    \toprule
    \textbf{Name} & \textbf{Meaning} & \textbf{Notes} \\ \midrule
    \ruby{感謝}{かん|しゃ} & Thanks/gratitude/appreciation & \\
    \ruby{問題}{もん|だい} & Problem/question & \\
    \ruby{質問}{しつ|もん} & Question/enquiry & \\
    \ruby{情報}{じょう|ほう} & Information/news/intelligence & \\
    & & \\
    setsumei & & \\
    & & \\
    & & \\
    % & & \\
    \bottomrule
\end{tabular}%
}
\captionof{table}{Nouns: interaction.}
\label{tbl:appendix-vocab-nouns-interaction}
\end{center}


\subsubsection{Health}
\begin{center}
\resizebox{\linewidth}{!}{%
% Help: \multicolumn{2}{c}{}, \multirow{2}{*}{}, cmidrule(l){3-5}
\begin{tabular}{@{}lll@{}}
    \toprule
    \textbf{Name} & \textbf{Meaning} & \textbf{Notes} \\ \midrule
    \ruby{寝袋}{ね|ぶくろ} & Sleeping bag & \\
    \ruby{休}{やす}み & Rest/vacation & \\
    \ruby{春休}{はる|やす}み & Spring break/vacation & \\
    \ruby{夏休}{なつ|やす}み & Summer vacation & \\
    \ruby{秋休}{あき|やす}み & Autumn break/vacation & \\
    \ruby{冬休}{ふゆ|やす}み & Winter vacation & \\
    \ruby{月見}{つき|み} & Japanese equivalent of CN's mid-autumn festival (\ruby{同}{おな}じ\ruby{日}{ひ}) & \\
    & & \\
    % & & \\
    \bottomrule
\end{tabular}%
}
\captionof{table}{Nouns: health.}
\label{tbl:appendix-vocab-nouns-health}
\end{center}



\subsubsection{Positive traits, strengths}
\begin{center}
\centering
\resizebox{\linewidth}{!}{%
% Help: \multicolumn{2}{c}{}, \multirow{2}{*}{}, cmidrule(l){3-5}
\begin{tabular}{@{}lll@{}}
    \toprule
    \textbf{Name} & \textbf{Meaning} & \textbf{Notes} \\ \midrule
    \ruby{能力}{のう|りょく} & Ability & \\
    \ruby{力}{ちから} & Force/strength/power & \\
    \ruby{信用}{しん|よう} & Trust/confidence/reputation (past) & \href{https://japanese.stackexchange.com/q/24275}{[SE]} \\
    \ruby{信頼}{しん|らい} & Trust/confidence/reliance/faith (future) & \href{https://japanese.stackexchange.com/q/24275}{[SE]} \\
    \ruby{簡易}{かん|い} & Simplicity/ease/convenience & \\
    \ruby{安心感}{あん|しん|かん} & Sense of security & \\
    & & \\
    & & \\
    もちもち & Springy texture/elastic & \\
    & & \\
    % & & \\
\bottomrule
\end{tabular}%
}
\captionof{table}{Nouns: positive traits, strengths.}
\label{tbl:appendix-vocab-nouns-positive-traits-strengths}
\end{center}

\subsubsection{Negative traits, weaknesses}
\begin{center}
\centering
\resizebox{\linewidth}{!}{%
% Help: \multicolumn{2}{c}{}, \multirow{2}{*}{}, cmidrule(l){3-5}
\begin{tabular}{@{}lll@{}}
    \toprule
    \textbf{Name} & \textbf{Meaning} & \textbf{Notes} \\ \midrule
    \ruby{失礼}{しつ|れい} & Discourtesy/impoliteness & Also a な-adjective. \\
    \ruby{無礼}{ぶ|れい} & Rudeness/discourtesy/insolence (stronger) & Also a な-adjective. \\
    & & \\
    & & \\
    & & \\
    % & & \\
\bottomrule
\end{tabular}%
}
\captionof{table}{Nouns: negative traits, weaknesses.}
\label{tbl:appendix-vocab-nouns-negative-traits-weaknesses}
\end{center}

\subsubsection{Subjects}
\begin{center}
\resizebox{\linewidth}{!}{%
% Help: \multicolumn{2}{c}{}, \multirow{2}{*}{}, cmidrule(l){3-5}
\begin{tabular}{@{}lll@{}}
    \toprule
    \textbf{Name} & \textbf{Meaning} & \textbf{Notes} \\ \midrule
    \ruby{数学}{すう|がく} & Mathematics & \\
    & & \\
    \bottomrule
\end{tabular}%
}
\captionof{table}{Nouns: subjects.}
\label{tbl:appendix-vocab-nouns-subjects}
\end{center}

\subsubsection{Creatures and divinity}
\begin{center}
\centering
\resizebox{\linewidth}{!}{%
% Help: \multicolumn{2}{c}{}, \multirow{2}{*}{}, cmidrule(l){3-5}
\begin{tabular}{@{}lll@{}}
    \toprule
    \textbf{Name} & \textbf{Meaning} & \textbf{Notes} \\ \midrule
    \ruby{神}{かみ} & God/deity/divinity/spirit & \\
    \ruby{天使}{てん|し} & Angel & \\
    \ruby{巫女}{み|こ}/\ruby{神子}{み|こ} & Shrine maiden & \href{https://detail.chiebukuro.yahoo.co.jp/qa/question_detail/q1424312974}{[YJ]} \\
    & & \\
    & & \\
    \ruby{悪魔}{あく|ま} & Devil/demon & \\
    & & \\
    & & \\
    & & \\
    % & & \\
\bottomrule
\end{tabular}%
}
\captionof{table}{Nouns: creatures and divinity.}
\label{tbl:appendix-vocab-nouns-creatures-and-divinity}
\end{center}



\subsubsection{Nature}
\begin{center}
\resizebox{\linewidth}{!}{%
% Help: \multicolumn{2}{c}{}, \multirow{2}{*}{}, cmidrule(l){3-5}
\begin{tabular}{@{}lll@{}}
    \toprule
    \textbf{Name} & \textbf{Meaning} & \textbf{Notes} \\ \midrule
    \ruby{光}{ひかり} & Light & \\
    \ruby{水}{みず} & Water & \\
    & & \\
    & & \\
    & & \\
    % & & \\
    \bottomrule
\end{tabular}%
}
\captionof{table}{Nouns: nature.}
\label{tbl:appendix-vocab-nouns-nature}
\end{center}


\subsubsection{Cosmic}
\begin{center}
\resizebox{\linewidth}{!}{%
% Help: \multicolumn{2}{c}{}, \multirow{2}{*}{}, cmidrule(l){3-5}
\begin{tabular}{@{}lll@{}}
    \toprule
    \textbf{Name} & \textbf{Meaning} & \textbf{Notes} \\ \midrule
    \ruby{流星}{りゅう|せい} & Meteor/shooting star & \\
    \ruby{月見}{つき|み} & Moon viewing (eighth lunar month) & \\
    & & \\
    & & \\
    \bottomrule
\end{tabular}%
}
\captionof{table}{Nouns: cosmic.}
\label{tbl:appendix-vocab-nouns-cosmic}
\end{center}



\subsubsection{Hygiene}
\begin{center}
\resizebox{\linewidth}{!}{%
% Help: \multicolumn{2}{c}{}, \multirow{2}{*}{}, cmidrule(l){3-5}
\begin{tabular}{@{}lll@{}}
    \toprule
    \textbf{Name} & \textbf{Meaning} & \textbf{Notes} \\ \midrule
    うんこ/ウンコ & Poop & \\
    ごみ/ゴミ & Trash/rubbish/garbage/refuse & \\
    & & \\
    & & \\
    \bottomrule
\end{tabular}%
}
\captionof{table}{Nouns: hygiene.}
\label{tbl:appendix-vocab-nouns-hygiene}
\end{center}

\end{multicols}

\subsection{Adjectives}

\subsubsection{Emotions}
\begin{center}
\centering
\resizebox{\linewidth}{!}{%
% Help: \multicolumn{2}{c}{}, \multirow{2}{*}{}, cmidrule(l){3-5}
\begin{tabular}{@{}lcll@{}}
    \toprule
    \textbf{Descriptor} & \textbf{Cat.} & \textbf{Meaning} & \textbf{Notes} \\ \midrule
    \ruby{嬉}{うれ}しい & い & Happy/glad/delighted & \\
    \ruby{寂}{さび}しい & い & Lonely & \\
    \ruby{欲}{ほ}しい & い & Wanted/desired & \\
    & & & \\
    \ruby{暖}{あたた}かい & い & Pleasantly warm & \\
    \ruby{暑}{あつ}い & い & Hot & \\
    \ruby{熱}{あつ}い & い & Hot (to the touch) & \\
    \ruby{小寒}{こ|さむ}い & い & Chilly/a little cold & \\
    \ruby{寒}{さむ}い & い & Cold (weather) & \\
    \ruby{眠}{ねむ}い & い & Sleepy/drowsy & \\
    & & & \\
    & & & \\
    & & & \\
    % & & & \\
\bottomrule
\end{tabular}%
}
\captionof{table}{Adjectives: emotions.}
\label{tbl:appendix-vocab-adjectives-emotions}
\end{center}


\subsubsection{Health}
\begin{center}
\centering
\resizebox{\linewidth}{!}{%
% Help: \multicolumn{2}{c}{}, \multirow{2}{*}{}, cmidrule(l){3-5}
\begin{tabular}{@{}lcll@{}}
    \toprule
    \textbf{Descriptor} & \textbf{Cat.} & \textbf{Meaning} & \textbf{Notes} \\ \midrule
    \ruby{大丈夫}{だい|じょう|ぶ} & な & Alright/problem-free/without fear & \\
    \ruby{元気}{げん|き} & な & Lively/well/in good health & \\
    & & & \\
    & & & \\
    % & & & \\
\bottomrule
\end{tabular}%
}
\captionof{table}{Adjectives: health.}
\label{tbl:appendix-vocab-adjectives-health}
\end{center}


\subsubsection{Positive traits, strengths}
\begin{center}
\centering
\resizebox{\linewidth}{!}{%
% Help: \multicolumn{2}{c}{}, \multirow{2}{*}{}, cmidrule(l){3-5}
\begin{tabular}{@{}lcll@{}}
    \toprule
    \textbf{Descriptor} & \textbf{Cat.} & \textbf{Meaning} & \textbf{Notes} \\ \midrule
    かわいい & い & Cute/adorable/charming/lovely/pretty & \\
    \ruby{面白}{おも|しろ}い & い & Interesting/fascinating/funny/entertaining & \\
    \ruby{上手}{じょう|ず} & な & Skilful/proficient/adept & \\
    \ruby{簡単}{かん|たん} & な & Easy/simple & \\
    \ruby{好}{す}き & な & Likeable/favourite & \\
    \ruby{大好}{だい|す}き & な & Strongly liked/loved & \\
    おいしい & い & Good-tasting/delicious/tasty & \\
    & & & \\
    ふわふわ & な & Soft/fluffy/spongy & \\
    & & & \\
    \ruby{綺麗}{き|れい} & な & Pretty/beautiful/clean/tidy & \\
    \ruby{最高}{さい|こう} & な & Best/highest & \\
    やばい & い & Terrific/amazing/cool & Colloquial, slang \\
    \ruby{信}{しん}じられない & い & Unbelievable/incredible & \\
    \ruby{静}{しず}か & な & Quiet/silent/calm/peaceful & \\
    \ruby{親切}{しん|せつ} & Kind/generous/gentle & & \\
    & & & \\
    & & & \\
    & & & \\
    % & & & \\
\bottomrule
\end{tabular}%
}
\captionof{table}{Adjectives: positive traits, strengths.}
\label{tbl:appendix-vocab-adjectives-positive-traits-strengths}
\end{center}


\subsubsection{Negative traits, weaknesses}
\begin{center}
\centering
\resizebox{\linewidth}{!}{%
% Help: \multicolumn{2}{c}{}, \multirow{2}{*}{}, cmidrule(l){3-5}
\begin{tabular}{@{}lcll@{}}
    \toprule
    \textbf{Descriptor} & \textbf{Cat.} & \textbf{Meaning} & \textbf{Notes} \\ \midrule
    おかしい & い & Laughable/ridiculous/strange/weird/suspicious & \\
    \ruby{寒}{さむ}い & い & Lame/corny (joke) & \\
    ダサい & い & Lame/uncool & Slang \\
    \ruby{下手}{へ|た} & な & Unskilful/poor/awkward & \\
    \ruby{難}{むずか}しい & い & Difficult/troublesome/impossible (euphemism) & \\
    \ruby{嫌}{きら}い & \textred{\textbf{な}} & Disliked/hated & \\
    \ruby{大嫌}{だい|きら}い & \textred{\textbf{な}} & Strongly disliked/hated & \\
    & & & \\
    やばい & い & Dangerous/risky/awful/crazy/unhinged & Colloquial, slang \\
    まずい & い & Bad taste/unpleasant/awful/problematic/unfavourable & \\
    \ruby{最悪}{さい|あく} & な & Worst & \\
    \ruby{最低}{さい|てい} & な & Lowest/worst & \\
    \ruby{邪悪}{じゃ|あく} & な & Evil/wicked & \\
    \ruby{失礼}{しつ|れい} & な & Discourteous/impolite & Also a noun. \\
    \ruby{無礼}{ぶ|れい} & な & Rude/discourteous/insolent (stronger) & Also a noun. \\
    \ruby{無理}{む|り} & な & Impossible/no way/unreasonable & \\
    & & & \\
    % & & & \\
\bottomrule
\end{tabular}%
}
\captionof{table}{Adjectives: negative traits, weaknesses.}
\label{tbl:appendix-vocab-adjectives-negative-traits-weaknesses}
\end{center}

\subsubsection{Amounts and sizes}
\begin{center}
\centering
\resizebox{\linewidth}{!}{%
% Help: \multicolumn{2}{c}{}, \multirow{2}{*}{}, cmidrule(l){3-5}
\begin{tabular}{@{}lcll@{}}
    \toprule
    \textbf{Descriptor} & \textbf{Cat.} & \textbf{Meaning} & \textbf{Notes} \\ \midrule
    \ruby{大}{おお}きい & い & Big/large/great & \\
    \ruby{高}{たか}い & い & High/tall & \\
    & & & \\
    & & & \\
    & & & \\
    & & & \\
    & & & \\
    \ruby{小}{ちい}さい & い & Small/little/tiny & \\
    うまい & い & Skilful/good/delicious & \\
    \ruby{低}{ひく}い & い & Low/short & \\
    & & & \\
    % & & & \\
\bottomrule
\end{tabular}%
}
\captionof{table}{Adjectives: amounts and sizes.}
\label{tbl:appendix-vocab-adjectives-amounts-and-sizes}
\end{center}

\subsection{Verbs}

\subsubsection{Physical}
\begin{center}
\centering
\resizebox{\linewidth}{!}{%
% Help: \multicolumn{2}{c}{}, \multirow{2}{*}{}, cmidrule(l){3-5}
\begin{tabular}{@{}lclllcll@{}}
    \toprule
    \multicolumn{4}{c}{\textbf{Transitive}} & \multicolumn{4}{c}{\textbf{Intransitive}} \\ \cmidrule(r){1-4} \cmidrule(l){5-8}
    \textbf{Action} & \textbf{Cat.} & \textbf{Meaning} & \textbf{Notes} & \textbf{Action} & \textbf{Cat.} & \textbf{Meaning} & \textbf{Notes} \\ \midrule
    \ruby{当}{あ}てる & る & To hit & & \ruby{当}{あ}たる & う & To be hit & \\
    \ruby{打}{う}つ & う & To hit (strong) & \href{https://ja.hinative.com/questions/3867085}{[HN]} & \ruby{打}{う}たれる & る & To be hit (strong) & \\
    \ruby{打}{ぶ}つ & う & To hit someone & \href{https://ja.hinative.com/questions/4651279\#answer-39822392}{[HN]} & ? & & & \\
    ぶつける & る & To hit someone's head/crash into & \href{https://ja.hinative.com/questions/18725588}{[HN]} & ぶつかる & う & To be hit/crashed (large objects) & \href{https://ja.hinative.com/questions/94519\#answer-237544}{[HN]} \\
    ボッコボコにする & E & To severely beat up & & ? & & & \\
    & & & & & & & \\
    & & & & & & & \\
    & & & & & & & \\
    & & & & & & & \\
    & & & & & & & \\
    % & & & & & & & \\
\bottomrule
\end{tabular}%
}
\captionof{table}{Verbs: physical.}
\label{tbl:appendix-vocab-verbs-physical}
\end{center}



\subsubsection{Directions}
\begin{center}
\centering
\resizebox{\linewidth}{!}{%
% Help: \multicolumn{2}{c}{}, \multirow{2}{*}{}, cmidrule(l){3-5}
\begin{tabular}{@{}lclllcll@{}}
    \toprule
    \multicolumn{4}{c}{\textbf{Transitive}} & \multicolumn{4}{c}{\textbf{Intransitive}} \\ \cmidrule(r){1-4} \cmidrule(l){5-8}
    \textbf{Action} & \textbf{Cat.} & \textbf{Meaning} & \textbf{Notes} & \textbf{Action} & \textbf{Cat.} & \textbf{Meaning} & \textbf{Notes} \\ \midrule
    ? & & & & \ruby{上}{のぼ}る & う & To go up/upwards (focus on process) & \href{https://dictionary.goo.ne.jp/word/\%E4\%B8\%8A\%E3\%82\%8B/}{[goo]}\\
    ? & & & & \ruby{登}{のぼ}る & う & To ascend to a higher place & \href{https://dictionary.goo.ne.jp/word/\%E4\%B8\%8A\%E3\%82\%8B/}{[goo]} \\
    ? & & & & \ruby{昇}{のぼ}る & う & To rise (sun); be promoted in rank & \href{https://dictionary.goo.ne.jp/word/\%E4\%B8\%8A\%E3\%82\%8B/}{[goo]} \\
    \ruby{乗}{の}せる & る & To pick up passenger/load goods & & \ruby{乗}{の}る & う & To board/embark & \\
    \ruby{上}{あ}げる & る & To raise/elevate & & \ruby{上}{あ}がる & う & To be raised/elevated (focus on destination) & \href{https://dictionary.goo.ne.jp/thsrs/15966/meaning/m1u/}{[goo]}, \href{https://hugkum.sho.jp/582833}{[HK]}\\
    & & & & & & & \\
    ? & & & & \ruby{下}{くだ}る & う & To go down/downwards (focus on process) & \\
    ? & & & & \ruby{沈}{しず}む & う & To set (sun); be sunken/submerged & \\
    ? & & & & \ruby{下}{お}りる & る & To descend to a lower place & \\
    \ruby{降}{お}ろす & う & To drop off passenger/unload goods & & \ruby{降}{お}りる & る & To alight/disembark & \\
    \ruby{下}{お}ろす & う & To take down/bring down/lower & & \ruby{下}{さ}がる & う & To go downwards\emph{/backwards} (focus on destination) & \href{https://ja.hinative.com/questions/7054838\#answer-36801861}{[HN]} \\
    & & & & & & & \\
    & & & & & & & \\
    & & & & & & & \\
    & & & & \ruby{昇進}{しょう|しん}する & E & To promote/rise in rank (workplace) & \\
    & & & & & & & \\
    % & & & & & & & \\
\bottomrule
\end{tabular}%
}
\captionof{table}{Verbs: directions.}
\label{tbl:appendix-vocab-verbs-directions}
\end{center}



\subsubsection{Emotions}
\begin{center}
\centering
\resizebox{\linewidth}{!}{%
% Help: \multicolumn{2}{c}{}, \multirow{2}{*}{}, cmidrule(l){3-5}
\begin{tabular}{@{}lclllcll@{}}
    \toprule
    \multicolumn{4}{c}{\textbf{Transitive}} & \multicolumn{4}{c}{\textbf{Intransitive}} \\ \cmidrule(r){1-4} \cmidrule(l){5-8}
    \textbf{Action} & \textbf{Cat.} & \textbf{Meaning} & \textbf{Notes} & \textbf{Action} & \textbf{Cat.} & \textbf{Meaning} & \textbf{Notes} \\ \midrule
    - & & & & \ruby{泣}{な}く & う & To cry & \\
    & & & & & & & \\
    & & & & & & & \\
    & & & & & & & \\
    & & & & & & & \\
    % & & & & & & & \\
\bottomrule
\end{tabular}%
}
\captionof{table}{Verbs: emotions.}
\label{tbl:appendix-vocab-verbs-emotions}
\end{center}


\subsubsection{Production}
\begin{center}
\centering
\resizebox{\linewidth}{!}{%
% Help: \multicolumn{2}{c}{}, \multirow{2}{*}{}, cmidrule(l){3-5}
\begin{tabular}{@{}lclllcll@{}}
    \toprule
    \multicolumn{4}{c}{\textbf{Transitive}} & \multicolumn{4}{c}{\textbf{Intransitive}} \\ \cmidrule(r){1-4} \cmidrule(l){5-8}
    \textbf{Action} & \textbf{Cat.} & \textbf{Meaning} & \textbf{Notes} & \textbf{Action} & \textbf{Cat.} & \textbf{Meaning} & \textbf{Notes} \\ \midrule
    - & & & & ある & う & To exist/have (inaminate) & \\
    - & & & & いる & る & To exist (animate) & \\
    & & & & & & & \\
    \ruby{作}{つく}る & う & To make/prepare (food)/grow (agriculture)/cultivate (people) & \href{https://dictionary.goo.ne.jp/word/\%E4\%BD\%9C\%E3\%82\%8B}{[goo]} & - & & & \\
    \ruby{造}{つく}る & う & To construct (large-scale buildings, manufacturing) & \href{https://dictionary.goo.ne.jp/word/\%E4\%BD\%9C\%E3\%82\%8B}{[goo]} & - & & & \\
    \ruby{創}{つく}る & う & To create/compose (artistic)/start a business & \href{https://dictionary.goo.ne.jp/word/\%E4\%BD\%9C\%E3\%82\%8B}{[goo]} & - & & & \\
    \ruby{書}{か}く & う & To write & & ? & & & \\
    \ruby{描}{か}く & う & To draw/paint & & ? & & & \\
    \ruby{描}{えが}く & う & To imagine; to depict (abstract concept) & & ? & & & \\
    \ruby{歌}{うた}う & う & To sing & & \ruby{歌}{うた}う & う & To sing \\
    - & & & & \ruby{踊}{おど}る & う & To dance (a hopping dance) & \\
    - & & & & \ruby{小躍}{こ|おど}りする & E & To dance for joy & \\
    \ruby{続}{つづ}ける & る & To continue & \aux. & \ruby{続}{つづ}く & う & To continue & \\
    - & & & & \ruby{無理}{む|り}する & E & To work/try too hard & \\
    & & & & & & & \\
    % & & & & & & & \\
\bottomrule
\end{tabular}%
}
\captionof{table}{Verbs: production.}
\label{tbl:appendix-vocab-verbs-production}
\end{center}
To use \ruby{続}{つづ}ける as an auxiliary verb, suffix it to the stem of the main verb (e.g.\ 「\ruby{歌}{うた}い\ruby{続}{つづ}ける」 means to continue to sing, and \ruby{歌}{うた}い is the stem of \ruby{歌}{うた}う).

\subsubsection{Consumption}
\begin{center}
\centering
\resizebox{\linewidth}{!}{%
% Help: \multicolumn{2}{c}{}, \multirow{2}{*}{}, cmidrule(l){3-5}
\begin{tabular}{@{}lclllcll@{}}
    \toprule
    \multicolumn{4}{c}{\textbf{Transitive}} & \multicolumn{4}{c}{\textbf{Intransitive}} \\ \cmidrule(r){1-4} \cmidrule(l){5-8}
    \textbf{Action} & \textbf{Cat.} & \textbf{Meaning} & \textbf{Notes} & \textbf{Action} & \textbf{Cat.} & \textbf{Meaning} & \textbf{Notes} \\ \midrule
    \ruby{見}{み}る & る & To see/observe & & \ruby{見}{み}える & る & To be seen/visible & \\
    \ruby{見}{み}つける & る & To find/discover/detect & & \ruby{見}{み}つかる & う & To be found/discovered & \\
    ばらす & う & To expose/disclose/leak a secret & Colloquial. & ばれる & る & To be exposed/found out/leak a secret & \\
    \ruby{聞}{き}く & う & To hear & & \ruby{聞}{き}こえる & る & To be heard/audible & \\
    \ruby{聴}{き}く & う & To listen attentively (music) & & ? & & & \\
    \ruby{食}{た}べる & る & To eat & & - & & & \\
    \ruby{食}{た}べすぎる & る & To overeat & & - & & & \\
    & & & & & & & \\
    & & & & & & & \\
    & & & & & & & \\
    % & & & & & & & \\
\bottomrule
\end{tabular}%
}
\captionof{table}{Verbs: consumption.}
\label{tbl:appendix-vocab-verbs-consumption}
\end{center}


\subsubsection{Interaction}
\begin{center}
\centering
\resizebox{\linewidth}{!}{%
% Help: \multicolumn{2}{c}{}, \multirow{2}{*}{}, cmidrule(l){3-5}
\begin{tabular}{@{}lclllcll@{}}
    \toprule
    \multicolumn{4}{c}{\textbf{Transitive}} & \multicolumn{4}{c}{\textbf{Intransitive}} \\ \cmidrule(r){1-4} \cmidrule(l){5-8}
    \textbf{Action} & \textbf{Cat.} & \textbf{Meaning} & \textbf{Notes} & \textbf{Action} & \textbf{Cat.} & \textbf{Meaning} & \textbf{Notes} \\ \midrule
    \ruby{訊}{き}く & う & To ask/enquire & & ? & & & \\
    ? & & & & (ご)\ruby{注意}{ちゅう|い}する & E & To pay attention/remind/caution & \\
    & & & & & & & \\
    \ruby{感謝}{かん|しゃ}する & E & To thank & & \ruby{感謝}{かん|しゃ}する & E & To be thanked & \\
    \ruby{信}{しん}じる & る & To believe/trust/have faith in & & - & & & \\
    \ruby{信用}{しん|よう}する & E & To trust (information/source; past) & \href{https://japanese.stackexchange.com/q/24275}{[SE]} & - & & & \\
    \ruby{信頼}{しん|らい}する & E & To trust (a person/organisation; future) & \href{https://japanese.stackexchange.com/q/24275}{[SE]} & - & & & \\
    \ruby{一緒}{いっ|しょ}にする & E & To do together/unite/mix & & & & & \\
    % This line break is necessary to prevent interpretation of the next [

    [<with list>と] \ruby{一緒}{いっ|しょ}になる & う & To rendezvous/join/meet together/get married with & \htc & & & \\  % Hard To Categorise: neither strictly transitive nor strictly intransitive?
    & & & & なる & う & To become/get/attain/reach/turn into & \\
    & & & & わかる & う & To understand/comprehend & \\
    & & & & \ruby{質問}{しつ|もん}をする & E & To ask a question & \htc. \\
    & & & & All WOSURU family & & & \\
    % & & & & & & & \\
\bottomrule
\end{tabular}%
}
\captionof{table}{Verbs: interaction.}
\label{tbl:appendix-vocab-verbs-interaction}
\end{center}


\subsubsection{Health}
\begin{center}
\centering
\resizebox{\linewidth}{!}{%
% Help: \multicolumn{2}{c}{}, \multirow{2}{*}{}, cmidrule(l){3-5}
\begin{tabular}{@{}lclllcll@{}}
    \toprule
    \multicolumn{4}{c}{\textbf{Transitive}} & \multicolumn{4}{c}{\textbf{Intransitive}} \\ \cmidrule(r){1-4} \cmidrule(l){5-8}
    \textbf{Action} & \textbf{Cat.} & \textbf{Meaning} & \textbf{Notes} & \textbf{Action} & \textbf{Cat.} & \textbf{Meaning} & \textbf{Notes} \\ \midrule
    \ruby{訊}{き}く & う & To ask/enquire & & ? & & & \\
    & & & & & & & \\
    & & & & & & & \\
    & & & & & & & \\
    & & & & & & & \\
    % & & & & & & & \\
\bottomrule
\end{tabular}%
}
\captionof{table}{Verbs: health.}
\label{tbl:appendix-vocab-verbs-health}
\end{center}


\subsubsection{Change}
\begin{center}
\centering
\resizebox{\linewidth}{!}{%
% Help: \multicolumn{2}{c}{}, \multirow{2}{*}{}, cmidrule(l){3-5}
\begin{tabular}{@{}lclllcll@{}}
    \toprule
    \multicolumn{4}{c}{\textbf{Transitive}} & \multicolumn{4}{c}{\textbf{Intransitive}} \\ \cmidrule(r){1-4} \cmidrule(l){5-8}
    \textbf{Action} & \textbf{Cat.} & \textbf{Meaning} & \textbf{Notes} & \textbf{Action} & \textbf{Cat.} & \textbf{Meaning} & \textbf{Notes} \\ \midrule
    ? & & & & \ruby{声変}{こえ|が}わりする & E & To break voice & \\
    & & & & & & & \\
    & & & & & & & \\
    & & & & & & & \\
    & & & & & & & \\
    % & & & & & & & \\
\bottomrule
\end{tabular}%
}
\captionof{table}{Verbs: change.}
\label{tbl:appendix-vocab-verbs-change}
\end{center}


\subsection{Adverbs}
Adverbs modify both verbs and adjectives. They may also modify entire noun phrases or sentences.

\subsubsection{Intensity modifiers}
\begin{center}
\centering
\resizebox{\linewidth}{!}{%
% Help: \multicolumn{2}{c}{}, \multirow{2}{*}{}, cmidrule(l){3-5}
\begin{tabular}{@{}lll@{}}
    \toprule
    \textbf{Modifier} & \textbf{Meaning} & \textbf{Notes} \\ \midrule
    あまり<negative> & not very & \\
    % <so>のあまり<verb>& so much <so> that you <verb> & 嬉しさのあまり\ruby{泣}{な}いた。\\
    \ruby{別}{べつ}に<negative> & not particularly & \\
    \ruby{別}{べつ}に & nothing/not really & \\
    & & \\
    もしや & Perhaps/possibly/by some chance & \\
    & & \\
    ぷにぷに & Squishy/springy/bouncy (chubby when used on person) & New cat.? \\
    ふわふわ & Lightly/buoyantly & Also an adjective. \\
    \ruby{大}{だい} & Large/big/great/severe & \prefix. \htc. Technically な-adj/noun. \\
    & & \\
    さっさと & Immediately/without delay/hurriedly/quickly & \\
    ずっと & The whole time/continuously; much (more); (by) far & \\
    & & \\
    <with>と\ruby{一緒}{いっ|しょ}に<verb> & Together with & \\
    & & \\
    % & & \\
\bottomrule
\end{tabular}%
}
\captionof{table}{Adverbs: intensity modifiers.}
\label{tbl:appendix-vocab-adverbs-intensity}
\end{center}

\end{document}
