\documentclass[../nihongo-gakushuu-kyouzai-supplementary.tex]{subfiles}
\begin{document}
\appendix
\setcounter{section}{1}

\section{Basic nouns} \label{sec:appendix-basic-nouns}

\subsection{Numbers}
% Help: \SetCell[r=2,c=2]{c,m} <content>, \cmidrule[l]{3-4}
% Help: colspec: X[ratio, horizontal alignment] columns grow to fit width=\linewidth
%                  negative ratios: shrink to fit content and may not grow to full ratio
% Help: colspec: l/c/r columns do not grow
\longtabse[0.75]  % scale factor
{Nouns: numbers.}  % caption
{tbl:appendix-vocab-basic-nouns-numbers}  % label
{}  % outer specification options
{
    colspec={X[-1,c]X[-3,l]X[3,l]X[-3,l]},
    rowhead=1,
    % width=\linewidth,  % useful only with X columns
}  % inner specification options
{
    \toprule
    & \textbf{Name} & \textbf{Meaning} & \textbf{Notes} \\
    \midrule
    & \ruby{零}{れい}/\ruby{0}{れい}/\ruby{〇}{れい} & zero & \href{https://www.nhk.or.jp/bunken/summary/kotoba/gimon/062.html}{[NHK]}; also: ゼロ/ (with 漢字 numerals) \\
    & \ruby{一}{いち}/\ruby{1}{いち}/\ruby{壱}{いち} & one & \\
    & \ruby{二}{に}/\ruby{2}{に}/\ruby{弐}{に} & two & \\
    & \ruby{三}{さん}/\ruby{3}{さん}/\ruby{参}{さん} & three & \\
    \textlegacybullet & \ruby{四}{よん}/\ruby{4}{よん}/\ruby{四}{し} & four & \href{https://ja.hinative.com/questions/14367966}{[HN]}; \ruby{肆}{し} is obsolete; \ruby{四}{し} only when counting up/compound 漢字/banks \\
    & \ruby{五}{ご}/\ruby{5}{ご}/\ruby{五}{ご} & five & \ruby{伍}{ご} is obsolete \\
    & \ruby{六}{ろく}/\ruby{6}{ろく}/\ruby{六}{ろく} & six & \ruby{陸}{ろく} is obsolete \\
    \textlegacybullet & \ruby{七}{なな}/\ruby{7}{なな}/\ruby{七}{しち} & seven & \href{https://ja.hinative.com/questions/24548637}{[HN]}; \ruby{漆}{しち} is obsolete \\
    & \ruby{八}{はち}/\ruby{8}{はち}/\ruby{八}{はち} & eight & \ruby{捌}{はち} is obsolete \\
    & \ruby{九}{きゅう}/\ruby{9}{きゅう}/\ruby{九}{きゅう} & nine & \ruby{玖}{きゅう} is obsolete  \\
    & \ruby{十}{じゅう}/\ruby[g]{一〇}{じゅう}/\ruby[g]{10}{じゅう}/\ruby{拾}{じゅう} & ten & \\
    & \ruby{十一}{じゅう|いち}/\ruby[g]{11}{じゅういち} & eleven & \\
    & \ruby{十二}{じゅう|に}/\ruby[g]{12}{じゅうに} & twelve & \\
    & \ruby{十三}{じゅう|さん}/\ruby[g]{13}{じゅうさん} & thirteen & \\
    & \ruby{十四}{じゅう|よん}/\ruby[g]{14}{じゅうよん} & fourteen & \\
    & \ruby{十五}{じゅう|ご}/\ruby[g]{15}{じゅうご} & fifteen & \\
    & \ruby{十六}{じゅう|ろく}/\ruby[g]{16}{じゅうろく} & sixteen  & \\
    & \ruby{十七}{じゅう|なな}/\ruby[g]{17}{じゅうなな} & seventeen & \\
    & \ruby{十八}{じゅう|はち}/\ruby[g]{18}{じゅうはち} & eighteen & \\
    & \ruby{十九}{じゅう|きゅう}/\ruby[g]{19}{じゅうきゅう} & nineteen & \\
    & \ruby{二十}{に|じゅう}/\ruby{二〇}{に|じゅう}/\ruby[g]{20}{にじゅう} & twenty & \\
    & \ruby{三十}{さん|じゅう}/\ruby{三〇}{さん|じゅう}/\ruby[g]{30}{さんじゅう} & thirty & \\
    & \ruby{四十}{よん|じゅう}/\ruby{四〇}{よん|じゅう}/\ruby[g]{40}{よんじゅう} & fourty & \\
    & \ruby{五十}{ご|じゅう}/\ruby{五〇}{ご|じゅう}/\ruby[g]{50}{ごじゅう} & fifty & \\
    & \ruby{六十}{ろく|じゅう}/\ruby{六〇}{ろく|じゅう}/\ruby[g]{60}{ろくじゅう} & sixty & \\
    & \ruby{七十}{なな|じゅう}/\ruby{七〇}{なな|じゅう}/\ruby[g]{70}{ななじゅう} & seventy & \\
    & \ruby{八十}{はち|じゅう}/\ruby{八〇}{はち|じゅう}/\ruby[g]{80}{はちじゅう} & eighty & \\
    & \ruby{九十}{きゅう|じゅう}/\ruby{九〇}{きゅう|じゅう}/\ruby[g]{90}{きゅうじゅう} & ninety & \\
    & \ruby{百}{ひゃく}/\ruby[g]{一〇〇}{ひゃく}/\ruby[g]{100}{ひゃく} & hundred & \\
    & \ruby{二百}{に|ひゃく}/\ruby{2百}{に|ひゃく}/\ruby[g]{二〇〇}{にひゃく}/\ruby[g]{200}{にひゃく} & two hundred & \\
    \textlegacybullet & \ruby{三百}{さん|びゃく}/\ruby{3百}{さん|びゃく}/\ruby[g]{三〇〇}{さんびゃく}/\ruby[g]{300}{さんびゃく} & three hundred& \\
    & \ruby{四百}{よん|ひゃく}/\ruby{4百}{よん|ひゃく}/\ruby[g]{四〇〇}{よんひゃく}/\ruby[g]{400}{よんひゃく} & four hundred & \\
    & \ruby{五百}{ご|ひゃく}/\ruby{5百}{ご|ひゃく}/\ruby[g]{五〇〇}{ごひゃく}/\ruby[g]{500}{ごひゃく} & five hundred & \\
    \textlegacybullet & \ruby{六百}{ろく|ぴゃく}/\ruby{6百}{ろく|ぴゃく}/\ruby[g]{六〇〇}{ろくぴゃく}/\ruby[g]{600}{ろくぴゃく} & six hundred & \\
    & \ruby{七百}{なな|ひゃく}/\ruby{7百}{なな|ひゃく}/\ruby[g]{七〇〇}{ななひゃく}/\ruby[g]{700}{ななひゃく} & seven hundred & \\
    \textlegacybullet & \ruby{八百}{はっ|ぴゃく}/\ruby{8百}{はっ|ぴゃく}/\ruby[g]{八〇〇}{はっぴゃく}/\ruby[g]{800}{はっぴゃく} & eight hundred & \\
    & \ruby{九百}{きゅう|ひゃく}/\ruby{9百}{きゅう|ひゃく}/\ruby[g]{九〇〇}{きゅうひゃく}/\ruby[g]{900}{きゅうひゃく} & nine hundred & \\
    \textlegacybullet & \ruby{千}{せん}/\ruby{一千}{いっ|せん}/\ruby{1千}{いっ|せん}/\ruby[g]{一〇〇〇}{いっせん}/\ruby[g]{1000}{いっせん} & thousand & \\
    & \ruby{二千}{に|せん}/\ruby{2千}{に|せん}/\ruby[g]{二〇〇〇}{にせん}/\ruby[g]{2000}{にせん} & two thousand & \\
    \textlegacybullet & \ruby{三千}{さん|ぜん}/\ruby{3千}{さん|ぜん}/\ruby[g]{三〇〇〇}{さんぜん}/\ruby[g]{3000}{さんぜん} & three thousand & \\
    & \ruby{四千}{よん|せん}/\ruby{4千}{よん|せん}/\ruby[g]{四〇〇〇}{よんせん}/\ruby[g]{4000}{よんせん} & four thousand & \\
    & \ruby{五千}{ご|せん}/\ruby{5千}{ご|せん}/\ruby[g]{五〇〇〇}{ごせん}/\ruby[g]{5000}{ごせん} & five thousand & \\
    & \ruby{六千}{ろく|せん}/\ruby{6千}{ろく|せん}/\ruby[g]{六〇〇〇}{ろくせん}/\ruby[g]{6000}{ろくせん} & six thousand & \\
    & \ruby{七千}{なな|せん}/\ruby{7千}{なな|せん}/\ruby[g]{七〇〇〇}{ななせん}/\ruby[g]{7000}{ななせん} & seven thousand & \\
    \textlegacybullet & \ruby{八千}{はっ|せん}/\ruby{8千}{はっ|せん}/\ruby[g]{八〇〇〇}{はっせん}/\ruby[g]{8000}{はっせん} & eight thousand & \\
    & \ruby{九千}{きゅう|せん}/\ruby{9千}{きゅう|せん}/\ruby[g]{九〇〇〇}{きゅうせん}/\ruby[g]{9000}{きゅうせん} & nine thousand & \\
    & \ruby{一万}{いち|まん}/\ruby{1万}{いち|まん}/\ruby[g]{一〇〇〇〇}{いちまん}/\ruby{1万}{いち|まん}/\ruby{壱萬}{いち|まん} & ten thousand & \\
    & \ruby{十万}{じゅう|まん}/\ruby[g]{10万}{じゅうまん}/\ruby[g]{一〇〇〇〇〇}{じゅうまん}/\ruby[g]{100000}{じゅうまん}/\ruby{拾万}{じゅう|まん}/\ruby{拾萬}{じゅう|まん} & hundred thousand & \\
    & \ruby{百万}{ひゃく|まん}/\ruby[g]{100万}{ひゃくまん}/\ruby[g]{1000000}{ひゃくまん}/\ruby[g]{壱百万}{ひゃくまん}/\ruby[g]{壱百萬}{ひゃくまん} & million & \\
    \textlegacybullet & \ruby{千万}{せん|まん}/\ruby{一千万}{いっ|せん|まん}/\ruby{1千万}{いっ|せん|まん}/\ruby[g]{1000万}{いっせんまん}/\ruby[g]{一〇〇〇万}{いっせんまん}/\ruby[g]{10000000}{いっせんまん} & ten million & \\
    & \ruby{一億}{いち|おく}/\ruby{1億}{いち|おく} & hundred million & \\
    & \ruby{一兆}{いっ|ちょう}/\ruby{1億}{いっ|ちょう} & trillion & \\
    % & & & \\
    \bottomrule
}


\subsection{Counting (generic): つ}
% Help: \SetCell[r=2,c=2]{c,m} <content>, \cmidrule[l]{3-4}
% Help: colspec: X[ratio, horizontal alignment] columns grow to fit width=\linewidth
%                  negative ratios: shrink to fit content and may not grow to full ratio
% Help: colspec: l/c/r columns do not grow
\longtabse[0.75]  % scale factor
{Nouns: counting (generic).}  % caption
{tbl:appendix-vocab-basic-nouns-counting-generic}  % label
{}  % outer specification options
{
    colspec={X[-1,c]X[-3,l]X[3,l]X[-3,l]},
    rowhead=1,
    % width=\linewidth,  % useful only with X columns
}  % inner specification options
{
    \toprule
    & \textbf{Name} & \textbf{Meaning} & \textbf{Notes} \\
    \midrule
    & いくつ & how many? & (\ruby{幾}{いく}つ) \\
    & \ruby{一}{ひと}つ/\ruby{1}{ひと}つ & one item & \\
    & \ruby{二}{ふた}つ/\ruby{2}{ふた}つ & two items & \\
    & \ruby{三}{みっ}つ/\ruby{3}{みっ}つ & three items & \\
    & \ruby{四}{よっ}つ/\ruby{4}{よっ}つ & four items & \\
    & \ruby{五}{いつ}つ/\ruby{5}{いつ}つ & five items & \\
    & \ruby{六}{むっ}つ/\ruby{6}{むっ}つ & six items & \\
    & \ruby{七}{なな}つ/\ruby{7}{なな}つ & seven items & \\
    & \ruby{八}{やっ}つ/\ruby{8}{やっ}つ & eight items & \\
    & \ruby{九}{ここの}つ/\ruby{9}{ここの}つ & nine items & \\
    & \ruby{十}{とお} & ten items & \\
    % & & & \\
    \midrule
    \midrule
    & \ruby{全部}{ぜん|ぶ} & all/entire/whole/altogether & also an adverb \\
    & \ruby{以上}{い|じょう} & and above & \suffix, also an expression \\
    & \ruby{以下}{い|か} & and below/subpar & \suffix \\
    & \ruby{次}{つぎ} & next & \\
    % & & & \\
    \bottomrule
}


\subsection{Counting people: \ruby{人}{にん}}
\href{https://www.tofugu.com/japanese/japanese-counter-nin/}{\hl{Tofugu: TO READ}}

% Help: \SetCell[r=2,c=2]{c,m} <content>, \cmidrule[l]{3-4}
% Help: colspec: X[ratio, horizontal alignment] columns grow to fit width=\linewidth
%                  negative ratios: shrink to fit content and may not grow to full ratio
% Help: colspec: l/c/r columns do not grow
\longtabse[0.75]  % scale factor
{Nouns: counting people.}  % caption
{tbl:appendix-vocab-basic-nouns-counting-people}  % label
{}  % outer specification options
{
    colspec={X[-1,c]X[-3,l]X[3,l]X[-3,l]},
    rowhead=1,
    % width=\linewidth,  % useful only with X columns
}  % inner specification options
{
    \toprule
    & \textbf{Name} & \textbf{Meaning} & \textbf{Notes} \\
    \midrule
    & \ruby{何人}{なん|にん} & how many people? & \\
    \textlegacybullet & \ruby{1人}{ひと|り} & one person; being alone/single & \\
    \textlegacybullet & \ruby{2人}{ふた|り} & two people & \\
    & \ruby{3人}{さん|にん} & three people & \\
    \textlegacybullet & \ruby{4人}{よ|にん} & four people & \\
    & \ruby{5人}{ご|にん} & five people & \\
    & \ruby{6人}{ろく|にん} & six people & \\
    \textlegacybullet & \ruby{7人}{なな|にん}/\ruby{7人}{しち|にん} & seven people; former preferred for clarity \hl{Actually, we don't talk about しち...right? See grammar book when you get there} & \href{https://ja.hinative.com/questions/6001961}{[HN1]}, \href{https://ja.hinative.com/questions/22417892}{[HN2]} \\
    & \ruby{8人}{はち|にん} & eight people & \\
    & \ruby{9人}{じゅう|にん} & nine people & \\
    & \ruby[g]{10人}{きゅうにん} & ten people & \\
    & \ruby[g]{100人}{ひゃくにん} & hundred people & \\
    & \ruby[g]{1000人}{せんにん} & thousand people & \\
    & \ruby{1万人}{いち|まん|にん} & ten thousand people & \\
    % & & & \\
    \midrule
    \midrule
    & \ruby{一緒}{いっ|しょ} & together & \\
    % & & & \\
    \bottomrule
}


\subsection{Counting bus/train stations: \ruby{駅}{えき}}

% Help: \SetCell[r=2,c=2]{c,m} <content>, \cmidrule[l]{3-4}
% Help: colspec: X[ratio, horizontal alignment] columns grow to fit width=\linewidth
%                  negative ratios: shrink to fit content and may not grow to full ratio
% Help: colspec: l/c/r columns do not grow
\longtabse[0.75]  % scale factor
{Nouns: counting bus/train stations.}  % caption
{tbl:appendix-vocab-basic-nouns-counting-bus-train-stations}  % label
{}  % outer specification options
{
    colspec={X[-1,c]X[-3,l]X[3,l]X[-3,l]},
    rowhead=1,
    % width=\linewidth,  % useful only with X columns
}  % inner specification options
{
    \toprule
    & \textbf{Name} & \textbf{Meaning} & \textbf{Notes} \\
    \midrule
    & \ruby{何駅}{なん|えき} & how many stations? & \\
    \textlegacybullet & \ruby{1駅}{ひと|えき} & one station & \\
    \textlegacybullet & \ruby{2駅}{ふた|えき} & two stations & \\
    & \ruby{3駅}{さん|えき} & three stations & \\
    & \ruby{4駅}{よん|えき} & four stations & \\
    & \ruby{5駅}{ご|えき} & five stations & \\
    & \ruby{6駅}{ろく|えき} & six stations & \\
    & \ruby{7駅}{なな|えき} & seven stations & \\
    & \ruby{8駅}{はち|えき} & eight stations & \\
    & \ruby{9駅}{きゅう|えき} & nine stations & \\
    & \ruby[g]{10駅}{じゅうえき} & ten stations & \\
    & \ruby[g]{100駅}{ひゃくえき} & hundred stations & \\
    & \ruby[g]{1000駅}{せんえき} & thousand stations & \\
    & \ruby{1万駅}{いち|まん|えき} & ten thousand stations & \\
    % & & & \\
    \bottomrule
}


\subsection{Counting age: \ruby{歳}{さい}}
\ruby{才}{さい} may be used as a simpler substitute for \ruby{歳}{さい} only in handwriting, but note that \ruby{才}{さい} is technically incorrect as it does not mean age (\href{https://ja.hinative.com/questions/19557790}{[HN]}).

% Help: \SetCell[r=2,c=2]{c,m} <content>, \cmidrule[l]{3-4}
% Help: colspec: X[ratio, horizontal alignment] columns grow to fit width=\linewidth
%                  negative ratios: shrink to fit content and may not grow to full ratio
% Help: colspec: l/c/r columns do not grow
\longtabse[0.75]  % scale factor
{Nouns: counting age.}  % caption
{tbl:appendix-vocab-basic-nouns-counting-age}  % label
{}  % outer specification options
{
    colspec={X[-1,c]X[-3,l]X[3,l]X[-3,l]},
    rowhead=1,
    % width=\linewidth,  % useful only with X columns
}  % inner specification options
{
    \toprule
    & \textbf{Name} & \textbf{Meaning} & \textbf{Notes} \\
    \midrule
    & \ruby{何歳}{なん|さい} & how old? & \\
    \textlegacybullet & \ruby{1歳未満}{いっ|さい|み|まん} & less than one year old & e.g.\ 「〜の\ruby{乳児}{にゅう|じ}」; \href{https://www.nhk.or.jp/bunken/summary/kotoba/gimon/062.html}{[NHK]} \\
    \textlegacybullet & \ruby{1歳}{いっ|さい} & one year old  & \\
    & \ruby{2歳}{に|さい} & two year old  & \\
    & \ruby{3歳}{さん|さい} & three years old  & \\
    & \ruby{4歳}{よん|さい} & four years old  & \\
    & \ruby{5歳}{ご|さい} & five years old  & \\
    & \ruby{6歳}{ろく|さい} & six years old  & \\
    & \ruby{7歳}{なな|さい} & seven years old  & \\
    \textlegacybullet & \ruby{8歳}{はっ|さい} & eight years old  & \\
    & \ruby{9歳}{きゅう|さい} & nine years old  & \\
    \textlegacybullet & \ruby[g]{10歳}{じゅっさい} & ten years old  & じっさい also possible? \\
    & \ruby[g]{二十歳}{はたち}/\ruby[g]{20歳}{はたち} & twenty years old & the only exception, to do with coming-of-age \\
    & \ruby[g]{100歳}{ひゃくさい} & hundred years old & \\
    & \ruby[g]{1000歳}{せんさい} & thousand years old & \\
    & \ruby{1万歳}{いち|まん|さい} & ten thousand years old & \\
    % & & & \\
    \bottomrule
}


\subsection{Counting thin/flat things: \ruby{枚}{まい}}
\href{https://www.tofugu.com/japanese/japanese-counter-mai/}{\hl{Tofugu: TO READ}}

% Help: \SetCell[r=2,c=2]{c,m} <content>, \cmidrule[l]{3-4}
% Help: colspec: X[ratio, horizontal alignment] columns grow to fit width=\linewidth
%                  negative ratios: shrink to fit content and may not grow to full ratio
% Help: colspec: l/c/r columns do not grow
\longtabse[0.75]  % scale factor
{Nouns: counting thin/flat things.}  % caption
{tbl:appendix-vocab-basic-nouns-counting-thin-flat-things}  % label
{}  % outer specification options
{
    colspec={X[-1,c]X[-3,l]X[3,l]X[-3,l]},
    rowhead=1,
    % width=\linewidth,  % useful only with X columns
}  % inner specification options
{
    \toprule
    & \textbf{Name} & \textbf{Meaning} & \textbf{Notes} \\
    \midrule
    & \ruby{何枚}{なん|まい} & how many sheets/thin or flat things? & \\
    & \ruby{1枚}{いち|まい} & one sheet/thin or flat thing & \\
    & \ruby{2枚}{に|まい} & two sheets/thin or flat things & \\
    & \ruby{3枚}{さん|まい} & three sheets/thin or flat things & \\
    & \ruby{4枚}{よん|まい} & four sheets/thin or flat things & \\
    & \ruby{5枚}{ご|まい} & five sheets/thin or flat things & \\
    & \ruby{6枚}{ろく|まい} & six sheets/thin or flat things & \\
    & \ruby{7枚}{なな|まい} & seven sheets/thin or flat things & \\
    & \ruby{8枚}{はち|まい} & eight sheets/thin or flat things & \\
    & \ruby{9枚}{きゅう|まい} & nine sheets/thin or flat things & \\
    & \ruby[g]{10枚}{じゅうまい} & ten sheets/thin or flat things & \\
    & \ruby[g]{100枚}{ひゃくまい} & hundred sheets/thin or flat things & \\
    & \ruby[g]{1000枚}{せんまい} & thousand sheets/thin or flat things & \\
    & \ruby{1万枚}{いち|まん|まい} & ten thousand sheets/thin or flat things & \\
    % & & & \\
    \bottomrule
}


\subsection{Counting thin long things (bottles): \ruby{本}{ほん}}
\href{https://www.tofugu.com/japanese/japanese-counter-hon/}{\hl{Tofugu: TO READ}}

bottles/trains/buses/books/films/phone calls

% Help: \SetCell[r=2,c=2]{c,m} <content>, \cmidrule[l]{3-4}
% Help: colspec: X[ratio, horizontal alignment] columns grow to fit width=\linewidth
%                  negative ratios: shrink to fit content and may not grow to full ratio
% Help: colspec: l/c/r columns do not grow
\longtabse[0.75]  % scale factor
{Nouns: counting thin long things (bottles).}  % caption
{tbl:appendix-vocab-basic-nouns-counting-thin-long-things-bottles}  % label
{}  % outer specification options
{
    colspec={X[-1,c]X[-3,l]X[3,l]X[-3,l]},
    rowhead=1,
    % width=\linewidth,  % useful only with X columns
}  % inner specification options
{
    \toprule
    & \textbf{Name} & \textbf{Meaning} & \textbf{Notes} \\
    \midrule
    \textlegacybullet & \ruby{何本}{なん|ぼん} & how many thin long things? & \\
    \textlegacybullet & \ruby{1本}{いっ|ぽん} & one thin long thing & \\
    & \ruby{2本}{に|ほん} & two thin long things & \\
    \textlegacybullet & \ruby{3本}{さん|ぼん} & three thin long things & \\
    & \ruby{4本}{よん|ほん} & four thin long things & \\
    & \ruby{5本}{ご|ほん} & five thin long things & \\
    \textlegacybullet & \ruby{6本}{ろっ|ぽん} & six thin long things & \\
    & \ruby{7本}{なな|ほん} & seven thin long things & \\
    \textlegacybullet & \ruby{8本}{はっ|ぽん} & eight thin long things & \\
    & \ruby{9本}{きゅう|ほん} & nine thin long things & \\
    \textlegacybullet & \ruby[g]{10本}{じゅっぽん} & ten thin long things & じっぽん also possible? \\
    \textlegacybullet & \ruby[g]{100本}{ひゃっぽん} & one hundred thin long things & \\
    \textlegacybullet & \ruby[g]{1000本}{せんぼん} & one thousand thin long things & \\
    \textlegacybullet & \ruby{1万本}{いち|まん|ぼん} & ten thousand thin long things & \\
    % & & & \\
    \bottomrule
}


\subsection{Counting drinks (cups/glasses): \ruby{杯}{はい}}
% Help: \SetCell[r=2,c=2]{c,m} <content>, \cmidrule[l]{3-4}
% Help: colspec: X[ratio, horizontal alignment] columns grow to fit width=\linewidth
%                  negative ratios: shrink to fit content and may not grow to full ratio
% Help: colspec: l/c/r columns do not grow
\longtabse[0.75]  % scale factor
{Nouns: counting drinks (cups/glasses).}  % caption
{tbl:appendix-vocab-basic-nouns-counting-drinks-cups-glasses}  % label
{}  % outer specification options
{
    colspec={X[-1,c]X[-3,l]X[3,l]X[-3,l]},
    rowhead=1,
    % width=\linewidth,  % useful only with X columns
}  % inner specification options
{
    \toprule
    & \textbf{Name} & \textbf{Meaning} & \textbf{Notes} \\
    \midrule
    \textlegacybullet & \ruby{何杯}{なん|ばい} & how many drinks/cups/glasses? & \\
    \textlegacybullet & \ruby{1杯}{いっ|ぱい} & one drink/cup/glass & \\
    & \ruby{2杯}{に|はい} & two drinks/cups/glasses & \\
    \textlegacybullet & \ruby{3杯}{さん|ばい} & three drinks/cups/glasses & \\
    & \ruby{4杯}{よん|はい} & four drinks/cups/glasses & \\
    & \ruby{5杯}{ご|はい} & five drinks/cups/glasses & \\
    \textlegacybullet & \ruby{6杯}{ろっ|ぱい} & six drinks/cups/glasses & \\
    & \ruby{7杯}{なな|はい} & seven drinks/cups/glasses & \\
    \textlegacybullet & \ruby{8杯}{はっ|ぱい} & eight drinks/cups/glasses & \\
    & \ruby{9杯}{きゅう|はい} & nine drinks/cups/glasses & \\
    \textlegacybullet & \ruby[g]{10杯}{じゅっぱい} & ten drinks/cups/glasses & じっぱい also possible? \\
    \textlegacybullet & \ruby[g]{100杯}{ひゃっぱい} & hundred drinks/cups/glasses & \\
    \textlegacybullet & \ruby[g]{1000杯}{せんばい} & thousand drinks/cups/glasses & \\
    \textlegacybullet & \ruby{1万杯}{いち|まん|ばい} & ten thousand drinks/cups/glasses & \\
    % & & & \\
    \bottomrule
}


\subsection{Counting machines/vehicles: \ruby{台}{だい}}
\href{https://www.tofugu.com/japanese/japanese-counter-dai/}{\hl{Tofugu: TO READ}}

% Help: \SetCell[r=2,c=2]{c,m} <content>, \cmidrule[l]{3-4}
% Help: colspec: X[ratio, horizontal alignment] columns grow to fit width=\linewidth
%                  negative ratios: shrink to fit content and may not grow to full ratio
% Help: colspec: l/c/r columns do not grow
\longtabse[0.75]  % scale factor
{Nouns: counting machines/vehicles.}  % caption
{tbl:appendix-vocab-basic-nouns-counting-machines-vehicles}  % label
{}  % outer specification options
{
    colspec={X[-1,c]X[-3,l]X[3,l]X[-3,l]},
    rowhead=1,
    % width=\linewidth,  % useful only with X columns
}  % inner specification options
{
    \toprule
    & \textbf{Name} & \textbf{Meaning} & \textbf{Notes} \\
    \midrule
    & \ruby{何台}{なん|だい} & how many machines? & \\
    & \ruby{1台}{いち|だい} & one machine & \\
    & \ruby{2台}{に|だい} & two machines & \\
    & \ruby{3台}{さん|だい} & three machines & \\
    & \ruby{4台}{よん|だい} & four machines & \\
    & \ruby{5台}{ご|だい} & five machines & \\
    & \ruby{6台}{ろく|だい} & six machines & \\
    & \ruby{7台}{なな|だい} & seven machines & \\
    & \ruby{8台}{はち|だい} & eight machines & \\
    & \ruby{9台}{きゅう|だい} & nine machines & \\
    & \ruby[g]{10台}{じゅうだい} & ten machines & \\
    & \ruby[g]{100台}{ひゃくだい} & hundred machines & \\
    & \ruby[g]{1000台}{せんだい} & thousand machines & \\
    & \ruby{1万台}{いち|まん|だい} & ten thousand machines & \\
    % & & & \\
    \bottomrule
}


\subsection{Counting books: \ruby{冊}{さつ}}
\href{https://www.tofugu.com/japanese/japanese-counter-satsu/}{\hl{Tofugu: TO READ}}

% Help: \SetCell[r=2,c=2]{c,m} <content>, \cmidrule[l]{3-4}
% Help: colspec: X[ratio, horizontal alignment] columns grow to fit width=\linewidth
%                  negative ratios: shrink to fit content and may not grow to full ratio
% Help: colspec: l/c/r columns do not grow
\longtabse[0.75]  % scale factor
{Nouns: counting books.}  % caption
{tbl:appendix-vocab-basic-nouns-counting-books}  % label
{}  % outer specification options
{
    colspec={X[-1,c]X[-3,l]X[3,l]X[-3,l]},
    rowhead=1,
    % width=\linewidth,  % useful only with X columns
}  % inner specification options
{
    \toprule
    & \textbf{Name} & \textbf{Meaning} & \textbf{Notes} \\
    \midrule
    & \ruby{何冊}{なん|さつ} & how many books? & \\
    \textlegacybullet & \ruby{1冊}{いっ|さつ} & one book & \\
    & \ruby{2冊}{に|さつ} & two books & \\
    & \ruby{3冊}{さん|さつ} & three books & \\
    & \ruby{4冊}{よん|さつ} & four books & \\
    & \ruby{5冊}{ご|さつ} & five books & \\
    & \ruby{6冊}{ろく|さつ} & six books & \\
    & \ruby{7冊}{なな|さつ} & seven books & \\
    \textlegacybullet & \ruby{8冊}{はっ|さつ} & eight books & \\
    & \ruby{9冊}{きゅう|さつ} & nine books & \\
    \textlegacybullet & \ruby[g]{10冊}{じゅっさつ} & ten books & じっさつ also possible? \\
    & \ruby[g]{100冊}{ひゃくさつ} & hundred books & \\
    & \ruby[g]{1000冊}{せんさつ} & thousand books & \\
    & \ruby{1万冊}{いち|まん|さつ} & ten thousand books & \\
    % & & & \\
    \bottomrule
}


\subsection{Counting clothes: \ruby{着}{ちゃく}}
% Help: \SetCell[r=2,c=2]{c,m} <content>, \cmidrule[l]{3-4}
% Help: colspec: X[ratio, horizontal alignment] columns grow to fit width=\linewidth
%                  negative ratios: shrink to fit content and may not grow to full ratio
% Help: colspec: l/c/r columns do not grow
\longtabse[0.75]  % scale factor
{Nouns: counting clothes.}  % caption
{tbl:appendix-vocab-basic-nouns-counting-clothes}  % label
{}  % outer specification options
{
    colspec={X[-1,c]X[-3,l]X[3,l]X[-3,l]},
    rowhead=1,
    % width=\linewidth,  % useful only with X columns
}  % inner specification options
{
    \toprule
    & \textbf{Name} & \textbf{Meaning} & \textbf{Notes} \\
    \midrule
    & \ruby{何着}{なん|ちゃく} & how many dresses? & \\
    \textlegacybullet & \ruby{1着}{いっ|ちゃく} & one dress & \\
    & \ruby{2着}{に|ちゃく} & two dresses & \\
    & \ruby{3着}{さん|ちゃく} & three dresses & \\
    & \ruby{4着}{よん|ちゃく} & four dresses & \\
    & \ruby{5着}{ご|ちゃく} & five dresses & \\
    & \ruby{6着}{ろく|ちゃく} & six dresses & \\
    & \ruby{7着}{なな|ちゃく} & seven dresses & \\
    \textlegacybullet & \ruby{8着}{はっ|ちゃく} & eight dresses & \\
    & \ruby{9着}{きゅう|ちゃく} & nine dresses & \\
    \textlegacybullet & \ruby[g]{10着}{じゅっちゃく} & ten dresses & \\
    & \ruby[g]{100着}{ひゃくちゃく} & hundred dresses & \\
    & \ruby[g]{1000着}{せんちゃく} & thousand dresses & \\
    & \ruby{1万着}{いち|まん|ちゃく} & ten thousand dresses & \\
    % & & & \\
    \bottomrule
}


\subsection{Counting small things: \ruby{個}{こ}}
\href{https://www.tofugu.com/japanese/japanese-counter-ko/}{\hl{Tofugu: TO READ}}

% Help: \SetCell[r=2,c=2]{c,m} <content>, \cmidrule[l]{3-4}
% Help: colspec: X[ratio, horizontal alignment] columns grow to fit width=\linewidth
%                  negative ratios: shrink to fit content and may not grow to full ratio
% Help: colspec: l/c/r columns do not grow
\longtabse[0.75]  % scale factor
{Nouns: counting small things.}  % caption
{tbl:appendix-vocab-basic-nouns-counting-small-things}  % label
{}  % outer specification options
{
    colspec={X[-1,c]X[-3,l]X[3,l]X[-3,l]},
    rowhead=1,
    % width=\linewidth,  % useful only with X columns
}  % inner specification options
{
    \toprule
    & \textbf{Name} & \textbf{Meaning} & \textbf{Notes} \\
    \midrule
    & \ruby{何個}{なん|こ} & how many small things? & \\
    \textlegacybullet & \ruby{1個}{いっ|こ} & one small thing & \\
    & \ruby{2個}{に|こ} & two small things & \\
    & \ruby{3個}{さん|こ} & three small things & \\
    & \ruby{4個}{よん|こ} & four small things & \\
    & \ruby{5個}{ご|こ} & five small things & \\
    \textlegacybullet & \ruby{6個}{ろっ|こ} & six small things & \\
    & \ruby{7個}{なな|こ} & seven small things & \\
    \color{lightgray}\textlegacybullet & \ruby{8個}{はち|こ}\color{lightgray}/\ruby{8個}{はっ|こ} & eight small things & \ruby{8個}{はっ|こ} is casual; \href{https://ja.hinative.com/questions/5127910}{[HN]} \\
    & \ruby{9個}{きゅう|こ} & nine small things & \\
    \textlegacybullet & \ruby[g]{10個}{じゅっこ} & ten small things & \\
    \textlegacybullet & \ruby[g]{100個}{ひゃっこ} & hundred small things & \\
    & \ruby[g]{1000個}{せんこ} & thousand small things & \\
    & \ruby{1万個}{いち|まん|こ} & ten thousand small things & \\
    % & & & \\
    \midrule
    \midrule
    \ruby{個々}{こ|こ} & individual & e.g.\ 「そのクラスの個々のメンバー\dots」, like CN's 个个 \\
    % & & & \\
    \bottomrule
}


\subsection{Counting shoes and socks: \ruby{足}{そく}}
% Help: \SetCell[r=2,c=2]{c,m} <content>, \cmidrule[l]{3-4}
% Help: colspec: X[ratio, horizontal alignment] columns grow to fit width=\linewidth
%                  negative ratios: shrink to fit content and may not grow to full ratio
% Help: colspec: l/c/r columns do not grow
\longtabse[0.75]  % scale factor
{Nouns: counting shoes and socks.}  % caption
{tbl:appendix-vocab-basic-nouns-counting-shoes-and-socks}  % label
{}  % outer specification options
{
    colspec={X[-1,c]X[-3,l]X[3,l]X[-3,l]},
    rowhead=1,
    % width=\linewidth,  % useful only with X columns
}  % inner specification options
{
    \toprule
    & \textbf{Name} & \textbf{Meaning} & \textbf{Notes} \\
    \midrule
    \textlegacybullet & \ruby{何足}{なん|ぞく} & how many pairs of shoes/socks? & \\
    \textlegacybullet & \ruby{1足}{いっ|そく} & one pair of shoes/socks & \\
    & \ruby{2足}{に|そく} & two pairs of shoes/socks & \\
    & \ruby{3足}{さん|そく} & three pairs of shoes/socks & \href{https://ja.hinative.com/questions/22667890}{[HN]} \\
    & \ruby{4足}{よん|そく} & four pairs of shoes/socks & \\
    & \ruby{5足}{ご|そく} & five pairs of shoes/socks & \\
    & \ruby{6足}{ろく|そく} & six pairs of shoes/socks & \\
    & \ruby{7足}{なな|そく} & seven pairs of shoes/socks & \\
    \textlegacybullet & \ruby{8足}{はっ|そく} & eight pairs of shoes/socks & \\
    & \ruby{9足}{きゅう|そく} & nine pairs of shoes/socks & \\
    \textlegacybullet & \ruby[g]{10足}{じゅっそく} & ten pairs of shoes/socks & \\
    & \ruby[g]{100足}{ひゃくそく} & hundred pairs of shoes/socks & also an organism \\
    & \ruby[g]{1000足}{せんそく} & thousand pairs of shoes/socks & \\
    & \ruby{1万足}{いち|まん|そく} & ten thousand pairs of shoes/socks & \\
    % & & & \\
    \bottomrule
}


\subsection{Counting houses: \ruby{軒}{けん}}
\hl{UNSURE TERRITORY, exceptions of exceptions popping out!}

% Help: \SetCell[r=2,c=2]{c,m} <content>, \cmidrule[l]{3-4}
% Help: colspec: X[ratio, horizontal alignment] columns grow to fit width=\linewidth
%                  negative ratios: shrink to fit content and may not grow to full ratio
% Help: colspec: l/c/r columns do not grow
\longtabse[0.75]  % scale factor
{Nouns: counting houses.}  % caption
{tbl:appendix-vocab-basic-nouns-counting-houses}  % label
{}  % outer specification options
{
    colspec={X[-1,c]X[-3,l]X[3,l]X[-3,l]},
    rowhead=1,
    % width=\linewidth,  % useful only with X columns
}  % inner specification options
{
    \toprule
    & \textbf{Name} & \textbf{Meaning} & \textbf{Notes} \\
    \midrule
    \textlegacybullet & \ruby{何軒}{なん|げん} & how many houses? & exception$^2$; \href{https://miyagirh.exblog.jp/21478345/}{[myg]} \\
    \textlegacybullet & \ruby{1軒}{いっ|けん} & one house & \\
    & \ruby{2軒}{に|けん} & two houses & \\
    \textlegacybullet & \ruby{3軒}{さん|げん} & three houses & exception$^2$; PREFERENCE? \href{https://miyagirh.exblog.jp/21478345/}{[myg]} \\
    & \ruby{4軒}{よん|けん} & four houses & \\
    & \ruby{5軒}{ご|けん} & five houses & \\
    \textlegacybullet & \ruby{6軒}{ろっ|けん} & six houses & \\
    & \ruby{7軒}{なな|けん} & seven houses & \\
    \color{lightgray}\textlegacybullet & \ruby{8軒}{はち|けん}\color{lightgray}/\ruby{8軒}{はっ|けん} & eight houses & \textlightgrey{\ruby{8軒}{はっ|けん}} may be casual; \href{https://ja.hinative.com/questions/236852}{[HN]} \\
    & \ruby{9軒}{きゅう|けん} & nine houses & \\
    \textlegacybullet & \ruby[g]{10軒}{じゅっけん} & ten houses & \\
    \textlegacybullet & \ruby[g]{100軒}{ひゃっけん} & hundred houses & \\
    \textlegacybullet & \ruby[g]{1000軒}{せんげん} & thousand houses & exception$^2$? \\
    & \ruby{1万軒}{いち|まん|けん} & ten thousand houses & ? \\
    % & & & \\
    \bottomrule
}


\subsection{Counting floors: \ruby{階}{かい}}
\href{https://www.tofugu.com/japanese/japanese-counter-kai-floors/}{\hl{Tofugu: TO READ}}

% Help: \SetCell[r=2,c=2]{c,m} <content>, \cmidrule[l]{3-4}
% Help: colspec: X[ratio, horizontal alignment] columns grow to fit width=\linewidth
%                  negative ratios: shrink to fit content and may not grow to full ratio
% Help: colspec: l/c/r columns do not grow
\longtabse[0.75]  % scale factor
{Nouns: counting floors.}  % caption
{tbl:appendix-vocab-basic-nouns-counting-floors}  % label
{}  % outer specification options
{
    colspec={X[-1,c]X[-3,l]X[3,l]X[-3,l]},
    rowhead=1,
    % width=\linewidth,  % useful only with X columns
}  % inner specification options
{
    \toprule
    & \textbf{Name} & \textbf{Meaning} & \textbf{Notes} \\
    \midrule
    \textlegacybullet & \ruby{何階}{なん|かい}/\ruby{何階}{なん|がい} & which floor? & \ruby{階}{かい} is special and can choose to rendaku, prefer first for uniformity; \href{https://miyagirh.exblog.jp/21478345/}{[myg]}, \href{https://www.tofugu.com/japanese/japanese-counter-kai-floors/}{[TFG]} \\
    \textlegacybullet & \ruby{1階}{いっ|かい} & first floor (ground floor) & \\
    & \ruby{2階}{に|かい} & second floor & \\
    \textlegacybullet & \ruby{3階}{さん|かい}/\ruby{3階}{さん|がい} & third floor & \ruby{階}{かい} is special and can choose to rendaku, prefer first for uniformity; \href{https://miyagirh.exblog.jp/21478345/}{[myg]}, \href{https://www.tofugu.com/japanese/japanese-counter-kai-floors/}{[TFG]} \\
    & \ruby{4階}{よん|かい} & fourth floor & \\
    & \ruby{5階}{ご|かい} & fifth floor & \\
    \textlegacybullet & \ruby{6階}{ろっ|かい} & sixth floor & \\
    & \ruby{7階}{なな|かい} & seventh floor & \\
    \color{lightgray}\textlegacybullet & \ruby{8階}{はち|かい}\color{lightgray}/\ruby{8階}{はっ|かい} & eighth floor & \textlightgrey{\ruby{8階}{はっ|かい}} is informal; \href{https://ja.hinative.com/questions/236852}{[HN]} \\
    & \ruby{9階}{きゅう|かい} & nine floor & \\
    \textlegacybullet & \ruby[g]{10階}{じゅっかい} & tenth floor & \href{https://ja.hinative.com/questions/236852}{[HN]} \\
    \textlegacybullet & \ruby[g]{100階}{ひゃっかい} & hundredth floor & \\
    & \ruby[g]{1000階}{せんかい} & thousandth floor & \\
    & \ruby{1万階}{いち|まん|かい} & ten thousandth floor & \\
    % & & & \\
    \bottomrule
}



\subsection{Counting locations: \ruby{箇所}{か|しょ}}
\ruby{箇所}{か|しょ} is sometimes written as ヶ所.

% Help: \SetCell[r=2,c=2]{c,m} <content>, \cmidrule[l]{3-4}
% Help: colspec: X[ratio, horizontal alignment] columns grow to fit width=\linewidth
%                  negative ratios: shrink to fit content and may not grow to full ratio
% Help: colspec: l/c/r columns do not grow
\longtabse[0.75]  % scale factor
{Nouns: counting locations.}  % caption
{tbl:appendix-vocab-basic-nouns-counting-locations}  % label
{}  % outer specification options
{
    colspec={X[-1,c]X[-3,l]X[3,l]X[-3,l]},
    rowhead=1,
    % width=\linewidth,  % useful only with X columns
}  % inner specification options
{
    \toprule
    & \textbf{Name} & \textbf{Meaning} & \textbf{Notes} \\
    \midrule
    & \ruby{何箇所}{なん|か|しょ} & how many locations? & \\
    \textlegacybullet & \ruby{1箇所}{いっ|か|しょ} & one location & \\
    & \ruby{2箇所}{に|か|しょ} & two locations & \\
    & \ruby{3箇所}{さん|か|しょ} & three locations &  \\
    & \ruby{4箇所}{よん|か|しょ} & four locations & \\
    & \ruby{5箇所}{ご|か|しょ} & five locations & \\
    \textlegacybullet & \ruby{6箇所}{ろっ|か|しょ} & six locations & \\
    & \ruby{7箇所}{なな|か|しょ} & seven locations & \\
    & \ruby{八箇所}{はち|か|しょ} & eight locations & \\
    & \ruby{9箇所}{きゅう|か|しょ} & nine locations & \\
    \textlegacybullet & \ruby[g]{10箇所}{じゅっかしょ} & ten locations & \\
    \textlegacybullet & \ruby[g]{100箇所}{ひゃっかしょ} & hundred locations & \\
    & \ruby[g]{1000箇所}{せんかしょ} & thousand locations & \\
    & \ruby{1万箇所}{いち|まん|か|しょ} & ten thousand locations & \\
    % & & & \\
    \bottomrule
}


\subsection{Counting small animals: \ruby{匹}{ひき}}
\href{https://www.tofugu.com/japanese/japanese-counter-hiki/}{\hl{Tofugu: TO READ}}

% Help: \SetCell[r=2,c=2]{c,m} <content>, \cmidrule[l]{3-4}
% Help: colspec: X[ratio, horizontal alignment] columns grow to fit width=\linewidth
%                  negative ratios: shrink to fit content and may not grow to full ratio
% Help: colspec: l/c/r columns do not grow
\longtabse[0.75]  % scale factor
{Nouns: counting small animals.}  % caption
{tbl:appendix-vocab-basic-nouns-counting-small-animals}  % label
{}  % outer specification options
{
    colspec={X[-1,c]X[-3,l]X[3,l]X[-3,l]},
    rowhead=1,
    % width=\linewidth,  % useful only with X columns
}  % inner specification options
{
    \toprule
    \textlegacybullet & \textbf{Name} & \textbf{Meaning} & \textbf{Notes} \\
    \midrule
    & \ruby{何匹}{なん|びき} & how many small animals? & \\
    \textlegacybullet & \ruby{1匹}{いっ|ぴき} & one small animal & \\
    & \ruby{2匹}{に|ひき} & two small animals & \\
    \textlegacybullet & \ruby{3匹}{さん|びき} & three small animals & \\
    & \ruby{4匹}{よん|ひき} & four small animals & \\
    & \ruby{5匹}{ご|ひき} & five small animals & \\
    \textlegacybullet & \ruby{6匹}{ろっ|ぴき} & six small animals & \\
    & \ruby{7匹}{なな|ひき} & seven small animals & \\
    \textlegacybullet & \ruby{8匹}{はっ|ぴき} & eight small animals & \\
    & \ruby{9匹}{きゅう|ひき} & nine small animals & \\
    \textlegacybullet & \ruby[g]{10匹}{じゅっぴき} & ten small animals & \\
    \textlegacybullet & \ruby[g]{100匹}{ひゃっぴき} & hundred small animals & \\
    \textlegacybullet & \ruby[g]{1000匹}{せんびき} & thousand small animals & \\
    \textlegacybullet & \ruby{1万匹}{いち|まん|びき} & ten thousand small animals & \\
    % & & & \\
    \bottomrule
}


\subsection{Counting big animals: \ruby{頭}{とう}}
\href{https://www.tofugu.com/japanese/japanese-counter-tou/}{\hl{Tofugu: TO READ}}

% Help: \SetCell[r=2,c=2]{c,m} <content>, \cmidrule[l]{3-4}
% Help: colspec: X[ratio, horizontal alignment] columns grow to fit width=\linewidth
%                  negative ratios: shrink to fit content and may not grow to full ratio
% Help: colspec: l/c/r columns do not grow
\longtabse[0.75]  % scale factor
{Nouns: counting big animals.}  % caption
{tbl:appendix-vocab-basic-nouns-counting-big-animals}  % label
{}  % outer specification options
{
    colspec={X[-1,c]X[-3,l]X[3,l]X[-3,l]},
    rowhead=1,
    % width=\linewidth,  % useful only with X columns
}  % inner specification options
{
    \toprule
    & \textbf{Name} & \textbf{Meaning} & \textbf{Notes} \\
    \midrule
    & \ruby{何頭}{なん|とう} & how many big animals? & \\
    \textlegacybullet & \ruby{1頭}{いっ|とう} & one big animal & \\
    & \ruby{2頭}{に|とう} & two big animals & \\
    & \ruby{3頭}{さん|とう} & three big animals & \\
    & \ruby{4頭}{よん|とう} & four big animals & \\
    & \ruby{5頭}{ご|とう} & five big animals & \\
    & \ruby{6頭}{ろく|とう} & six big animals & \\
    & \ruby{7頭}{なな|とう} & seven big animals & \\
    \textlegacybullet & \ruby{8頭}{はっ|とう} & eight big animals & \\
    & \ruby{9頭}{きゅう|とう} & nine big animals & \\
    \textlegacybullet & \ruby[g]{10頭}{じゅっとう} & ten big animals & \\
    & \ruby[g]{100頭}{ひゃくとう} & hundred big animals & \\
    & \ruby[g]{1000頭}{せんとう} & thousand big animals & \\
    & \ruby{1万頭}{いち|まん|とう} & ten thousand big animals & \\
    % & & & \\
    \bottomrule
}



\subsection{Counting birds and bats: \ruby{羽}{わ}}
\emph{Read the main article on \href{https://www.tofugu.com/japanese/japanese-counter-wa/}{Tofugu}.}

% Be careful: many winged animals don't use the 羽 counter: flying insects, winged monsters, and flying dinosaurs all use \ruby{匹}{ひき} (Table~\ref{tbl:appendix-vocab-basic-nouns-counting-small-animals}) or \ruby{頭}{とう} (Table~\ref{tbl:appendix-vocab-basic-nouns-counting-big-animals}).

All birds are counted with \ruby{羽}{わ} except \emph{very} big birds like ostriches and emus, those use \ruby{頭}{とう} (Table~\ref{tbl:appendix-vocab-basic-nouns-counting-big-animals}).

% Help: \SetCell[r=2,c=2]{c,m} <content>, \cmidrule[l]{3-4}
% Help: colspec: X[ratio, horizontal alignment] columns grow to fit width=\linewidth
%                  negative ratios: shrink to fit content and may not grow to full ratio
% Help: colspec: l/c/r columns do not grow
\longtabse[0.75]  % scale factor
{Nouns: counting birds and bats.}  % caption
{tbl:appendix-vocab-basic-nouns-counting-birds-bats}  % label
{}  % outer specification options
{
    colspec={X[-1,c]X[-3,l]X[3,l]X[-3,l]},
    rowhead=1,
    % width=\linewidth,  % useful only with X columns
}  % inner specification options
{
    \toprule
    & \textbf{Name} & \textbf{Meaning} & \textbf{Notes} \\
    \midrule
    & \ruby{何羽}{なん|わ} & how many birds/bats? & \\
    & \ruby{1羽}{いち|わ} & one bird/bats \\
    & \ruby{2羽}{に|わ} & two birds/bats & \\
    \textlegacybullet & \ruby{3羽}{さん|ば} & three birds/bats & \\
    & \ruby{4羽}{よん|わ} & four birds/bats & \\
    & \ruby{5羽}{ご|わ} & five birds/bats & \\
    & \ruby{6羽}{ろく|わ} & six birds/bats & \\
    & \ruby{7羽}{なな|わ} & seven birds/bats & \\
    & \ruby{8羽}{はち|わ} & eight birds/bats & \\
    & \ruby{9羽}{きゅう|わ} & nine birds/bats & \\
    \textlegacybullet & \ruby[g]{10羽}{じっぱ} & ten birds/bats & \\
    \textlegacybullet & \ruby[g]{100羽}{ひゃくっぱ} & hundred birds/bats & \\
    \textlegacybullet & \ruby[g]{1000羽}{せんば} & thousand birds/bats & \\
    \textlegacybullet & \ruby[g]{1万羽}{いちまんば} & ten thousand birds/bats & \\
    % & & & \\
    \bottomrule
}


\subsection{Counting years: \ruby{年}{ねん}}
\href{https://www.tofugu.com/japanese/japanese-counters-nen/}{\hl{Tofugu: TO READ}}

% Help: \SetCell[r=2,c=2]{c,m} <content>, \cmidrule[l]{3-4}
% Help: colspec: X[ratio, horizontal alignment] columns grow to fit width=\linewidth
%                  negative ratios: shrink to fit content and may not grow to full ratio
% Help: colspec: l/c/r columns do not grow
\longtabse[0.75]  % scale factor
{Nouns: counting years.}  % caption
{tbl:appendix-vocab-basic-nouns-counting-years}  % label
{}  % outer specification options
{
    colspec={X[-1,c]X[-3,l]X[3,l]X[-3,l]},
    rowhead=1,
    % width=\linewidth,  % useful only with X columns
}  % inner specification options
{
    \toprule
    & \textbf{Name} & \textbf{Meaning} & \textbf{Notes} \\
    \midrule
    & \ruby{何年}{なん|ねん} & how many years?/which year? & \\
    & \ruby{1年}{いち|ねん} & one year/first year & \\
    & \ruby{2年}{に|ねん} & two years/second year & \\
    & \ruby{3年}{さん|ねん} & three years/third year & \\
    \textlegacybullet & \ruby{4年}{よ|ねん} & four years/fourth year & \\
    & \ruby{5年}{ご|ねん} & five years/fifth year & \\
    & \ruby{6年}{ろく|ねん} & six years/sixth year & \\
    & \ruby{7年}{なな|ねん} & seven years/seventh year & \\
    & \ruby{8年}{はち|ねん} & eight years/eighth year & \\
    & \ruby{9年}{きゅう|ねん} & nine years/ninth year & \\
    & \ruby[g]{10年}{じゅうねん} & ten years/tenth year & \\
    & \ruby[g]{100年}{ひゃくねん} & hundred years/hundredth year & \\
    & \ruby[g]{1000年}{せんねん} & thousand years/thousandth year & \\
    & \ruby{1万年}{いち|まん|ねん} & ten thousand years/ten thousandth year & \\
    % & & & \\
    \bottomrule
}


\subsection{Calendar months and days of a week: \ruby{月}{がつ} and \ruby{曜日}{よう|び}}
The days of the week are named after the East Asian Seven Luminaries: the sun, the moon, and the five planets visible to the naked eye (Mercury, Venus, Mars, Jupiter, Saturn).

% Help: \SetCell[r=2,c=2]{c,m} <content>, \cmidrule[l]{3-4}
% Help: colspec: X[ratio, horizontal alignment] columns grow to fit width=\linewidth
%                  negative ratios: shrink to fit content and may not grow to full ratio
% Help: colspec: l/c/r columns do not grow
\longtabsea[0.75]  % scale factor
{Nouns: Calendar months and days of a week.}  % caption
{tbl:appendix-vocab-basic-nouns-calendar-months-and-days-of-a-week}  % label
{}  % outer specification options
{
    colspec={X[-1,c]X[-3,l]X[3,l]X[-3,l]},
    rowhead=1,
    % width=\linewidth,  % useful only with X columns
}  % inner specification options
{
    \toprule
    & \textbf{Name} & \textbf{Meaning} & \textbf{Notes} \\
    \midrule
    & \ruby{何月}{なん|がつ} & which month? & \\
    & \ruby{一月}{いち|がつ}/\ruby{1月}{いち|がつ} & January & \\
    & \ruby{二月}{に|がつ}/\ruby{2月}{に|がつ} & February & \\
    & \ruby{三月}{さん|がつ}/\ruby{3月}{さん|がつ} & March & \\
    \textlegacybullet & \ruby{四月}{し|がつ}/\ruby{4月}{し|がつ} & April & \\
    & \ruby{五月}{ご|がつ}/\ruby{5月}{ご|がつ} & May & \\
    & \ruby{六月}{ろく|がつ}/\ruby{6月}{ろく|がつ} & June & \\
    \textlegacybullet & \ruby{七月}{しち|がつ}/\ruby{7月}{しち|がつ} & July & \ruby{七月}{なな|がつ}/\ruby{7月}{なな|がつ} is sometimes used for disambiguation; \href{https://ja.hinative.com/questions/19121670}{[HN]} \\
    & \ruby{八月}{はち|がつ}/\ruby{8月}{はち|がつ} & August & \\
    \textlegacybullet & \ruby{九月}{く|がつ}/\ruby{9月}{く|がつ} & September & \\
    & \ruby{十月}{じゅう|がつ}/\ruby[g]{10月}{じゅうがつ} & October & \\
    & \ruby{十一月}{じゅう|いち|がつ}/\ruby[g]{11月}{じゅういちがつ} & November & \\
    & \ruby{十二月}{じゅう|に|がつ}/\ruby[g]{12月}{じゅうにがつ} & December & \\
    % & & & \\
    \midrule
    \midrule
    & \ruby{何曜日}{なん|よう|び} & which day of the week? & \\
    & \ruby{日}{にち}/\ruby{日曜}{にち|よう}/\ruby{日曜日}{にち|よう|び} & Sunday & Sun \\
    & \ruby{月}{げつ}/\ruby{月曜}{げつ|よう}/\ruby{月曜日}{げつ|よう|び} & Monday & Moon \\
    & \ruby{火}{か}/\ruby{火曜}{か|よう}/\ruby{火曜日}{か|よう|び} & Tuesday & fire/Mars (\ruby{火星}{か|せい}) \\
    & \ruby{水}{すい}/\ruby{水曜}{すい|よう}/\ruby{水曜日}{すい|よう|び} & Wednesday & water/Mercury (\ruby{水星}{すい|せい}) \\
    & \ruby{木}{もく}/\ruby{木曜}{もく|よう}/\ruby{木曜日}{もく|よう|び} & Thursday & wood/Jupiter (\ruby{木星}{もく|せい}) \\
    & \ruby{金}{きん}/\ruby{金曜}{きん|よう}/\ruby{金曜日}{きん|よう|び} & Friday & metal/Venus (\ruby{金星}{きん|せい}) \\
    & \ruby{土}{ど}/\ruby{土曜}{ど|よう}/\ruby{土曜日}{ど|よう|び} & Saturday & earth/Saturn (\ruby{土星}{ど|せい}) \\
    % & & & \\
    \bottomrule
}


\subsection{Counting months: \ruby{ヶ月}{か|げつ}\textlightgrey{/\ruby{月}{つき}}}
\href{https://www.tofugu.com/japanese/japanese-counter-tsuki-gatsu-getsu/}{\hl{Tofugu: TO READ}}

\ruby{月}{つき} is the 和語 reading, so counting uses the 和語 counting system.

% Help: \SetCell[r=2,c=2]{c,m} <content>, \cmidrule[l]{3-4}
% Help: colspec: X[ratio, horizontal alignment] columns grow to fit width=\linewidth
%                  negative ratios: shrink to fit content and may not grow to full ratio
% Help: colspec: l/c/r columns do not grow
\longtabse[0.75]  % scale factor
{Nouns: counting months.}  % caption
{tbl:appendix-vocab-basic-nouns-counting-months}  % label
{}  % outer specification options
{
    colspec={X[-1,c]X[-3,l]X[3,l]X[-3,l]},
    rowhead=1,
    % width=\linewidth,  % useful only with X columns
}  % inner specification options
{
    \toprule
    & \textbf{Name} & \textbf{Meaning} & \textbf{Notes} \\
    \midrule
    & \ruby{何ヶ月}{なん|か|げつ}/\color{lightgray}\ruby{何月}{なん|つき} & how many months? & \\
    \textlegacybullet & \ruby{1ヶ月}{いっ|か|げつ}\color{gray}/\ruby{1月}{ひと|つき} & one month & \textgrey{\ruby{1月}{ひと|つき}} is semi-archaic; \href{https://ja.hinative.com/questions/7822280}{[HN]} \\
    & \ruby{2ヶ月}{に|か|げつ}\color{gray}/\ruby{2月}{ふた|つき} & two months & \textgrey{\ruby{2月}{ふた|つき}} is semi-archaic; \href{https://ja.hinative.com/questions/7822280}{[HN]} \\
    \color{lightgray}\textlegacybullet & \ruby{3ヶ月}{さん|か|げつ}\color{lightgray}/\ruby{3月}{み|つき} & three months & \\
    \textlegacybullet & \ruby{4ヶ月}{よん|か|げつ}\color{lightgray}/\ruby{4月}{よ|つき} & four months & \\
    & \ruby{5ヶ月}{ご|か|げつ}\color{lightgray}/\ruby{5月}{いつ|つき} & five months & \\
    \textlegacybullet & \ruby{6ヶ月}{ろっ|か|げつ}\color{lightgray}/\ruby{6月}{む|つき} & six months & \\
    & \ruby{7ヶ月}{なな|か|げつ}\color{lightgray}/\ruby{7月}{なな|つき} & seven months & \\
    \color{lightgray}\textlegacybullet & \ruby{8ヶ月}{はち|か|げつ}\color{lightgray}/\ruby{8ヶ月}{はっ|か|げつ}\color{lightgray}/\ruby{八月}{や|つき}/\ruby{8月}{や|つき} & eight months & \textlightgrey{\ruby{8ヶ月}{はっ|か|げつ}} may be informal; \href{https://ja.hinative.com/questions/2749483}{[HN]} \\
    & \ruby{9ヶ月}{きゅう|か|げつ}\color{lightgray}/\ruby{9月}{ここの|つき} & nine months & \\
    \textlegacybullet & \ruby[g]{10ヶ月}{じゅっかげつ}\color{lightgray}/\ruby[g]{10月}{とつき} & ten months & \\
    \textlegacybullet & \ruby[g]{100ヶ月}{ひゃっかげつ} & hundred months & \\
    & \ruby[g]{1000ヶ月}{せんかげつ} & thousand months & \\
    & \ruby{1万ヶ月}{いち|まん|か|げつ} & ten thousand months & \\
    % & & & \\
    \bottomrule
}


\subsection{Days of the month: \ruby{日}{たち}/\ruby{日}{か}/\ruby{日}{にち}} \label{sec:appendix-vocab-basic-nouns-calendar-days}
\emph{Read the main article on \href{https://www.tofugu.com/japanese/japanese-counter-ka-nichi/}{Tofugu}.} Also see \href{https://www.instagram.com/reel/C_4iSZdyIPt/}{Instagram reel}.

\color{orange}
For calendar days, the 和語 counting system (\ruby{日}{か}) is used for \{2nd--10th, 14th, 20th, 24th\}. All other numbers use standard 漢語 counting system (\ruby{日}{にち}). Furthermore, \{17th, 27th\} use \ruby{七}{しち}, and \{19th, 29th\} use \ruby{九}{く}.

For ordinal days (\S\ref{sec:appendix-vocab-basic-nouns-ordinal-days}), there are two rules. The 〜\{\ruby{日}{にち}/\ruby{日}{か}\}\ruby{目}{め} schema follows the rules of day intervals (\S\ref{sec:appendix-vocab-basic-nouns-counting-days}), so 和語 counting system (\ruby{日}{か}) is used for \{2nd--10th, 20th\} only. Furthermore, \{1st--31st, 49th\} use the formal readings: \{17th, 27th\} use \ruby{七}{しち} (\ruby{七}{なな} OK for disambiguation) and \{19th, 29th, 49th\} use \ruby{九}{く}. On the other hand, the \ruby{第}{だい}〜\ruby{日}{たち} schema follows these simplified rules: the standard 漢語 counting system is used for all numbers, and \{9th, 19th, 29th, 49th\} use \ruby{九}{く}.

For day intervals (\S\ref{sec:appendix-vocab-basic-nouns-counting-days}), the 和語 counting system (\ruby{日}{か}) is used for \{2nd--10th, 20th\} only. All other numbers use the standard 漢語 counting system (\ruby{日}{にち}). Furthermore, days intervals $\in$\{1--31, 49\} days use the formal readings: \{17, 27\} days use \ruby{七}{しち} (\ruby{七}{なな} OK for disambiguation) and \{19, 29, 49\} days use \ruby{九}{く}.

For o'clocks and hour intervals (\S\ref{sec:appendix-vocab-basic-nouns-counting-oclocks-and-hours}), if the ones place is 4, 7 or 9, then \ruby{四}{よ}、\ruby{七}{しち}、\ruby{九}{く} are used, ad infinitum.
\color{black}

% Help: \SetCell[r=2,c=2]{c,m} <content>, \cmidrule[l]{3-4}
% Help: colspec: X[ratio, horizontal alignment] columns grow to fit width=\linewidth
%                  negative ratios: shrink to fit content and may not grow to full ratio
% Help: colspec: l/c/r columns do not grow
\longtabse[0.75]  % scale factor
{Nouns: calendar days.}  % caption
{tbl:appendix-vocab-basic-nouns-calendar-days}  % label
{}  % outer specification options
{
    colspec={X[-1,c]X[-3,l]X[3,l]X[-3,l]},
    rowhead=1,
    % width=\linewidth,  % useful only with X columns
}  % inner specification options
{
    \toprule
    & \textbf{Name} & \textbf{Meaning} & \textbf{Notes} \\
    \midrule
    & \ruby{何日}{なん|にち} & which day of month/which day?/how many days? & \\
    \textlegacybullet & \ruby[g]{1日}{ついたち}/\ruby{1日}{いっ|ぴ} & first day of month & \ruby{1日}{いっ|ぴ} is sometimes used in business settings; \href{https://www.tofugu.com/japanese/japanese-counter-ka-nichi/}{[TFG]} \\
    \textlegacybullet & \ruby{2日}{ふつ|か} & second day of month & \\
    & \ruby{3日}{みっ|か} & third day of month & \\
    & \ruby{4日}{よっ|か} & fourth day of month & \\
    & \ruby{5日}{いつ|か} & fifth day of month & \\
    \textlegacybullet & \ruby{6日}{むい|か} & sixth day of month & \\
    \textlegacybullet & \ruby{7日}{なの|か} & seventh day of month & \\
    \textlegacybullet & \ruby{8日}{よう|か} & eighth day of month & \\
    & \ruby{9日}{ここの|か} & ninth day of month & \\
    & \ruby[g]{10日}{とおか} & tenth day of month & \\
    & \ruby{11日}{じゅう|いち|にち} & eleventh day of month & \\
    & \ruby{12日}{じゅう|に|にち} & twelfth day of month & \\
    & \ruby{13日}{じゅう|さん|にち} & thirteenth day of month & \\
    \textlegacybullet & \ruby{14日}{じゅう|よっ|か} & fourteenth day of month & \\
    & \ruby{15日}{じゅう|ご|にち} & fifteenth day of month & \\
    & \ruby{16日}{じゅう|ろく|にち} & sixteenth day of month & \\
    \textlegacybullet & \ruby{17日}{じゅう|しち|にち} & seventeenth day of month & \ruby{17日}{じゅう|なな|にち} is sometimes used for disambiguation \\
    & \ruby{18日}{じゅう|はち|にち} & eighteenth day of month & \\
    \textlegacybullet & \ruby{19日}{じゅう|く|にち} & nineteenth day of month & \\
    \textlegacybullet & \ruby[g]{20日}{はつか} & twentieth day of month & \\
    & \ruby{21日}{にじゅう|いち|にち} & twenty-first day of month & \\
    & \ruby{22日}{にじゅう|に|にち} & twenty-second day of month & \\
    & \ruby{23日}{にじゅう|さん|にち} & twenty-third day of month & \\
    \textlegacybullet & \ruby{24日}{にじゅう|よっ|か} & twenty-fourth day of month & \\
    & \ruby{25日}{にじゅう|ご|にち} & twenty-fifth day of month & \\
    & \ruby{26日}{にじゅう|ろく|にち} & twenty-sixth day of month & \\
    \textlegacybullet & \ruby{27日}{にじゅう|しち|にち} & twenty-seventh day of month & \ruby{27日}{にじゅう|なな|にち} is sometimes used for disambiguation \\
    & \ruby{28日}{にじゅう|はち|にち} & twenty-eighth day of month & \\
    \textlegacybullet & \ruby{29日}{にじゅう|く|にち} & twenty-ninth day of month & \\
    & \ruby{30日}{さん|じゅう|にち} & thirtieth of month & \\
    & \ruby{31日}{さんじゅう|いち|にち} & thirty-first day of month & \\
    % & & & \\
    \midrule
    \midrule
    & \ruby{四十九日}{し|じゅう|く|にち} & forty-ninth day after death & \\
    % & & & \\
    \bottomrule
}


\subsection{Ordinal days: 〜\{\ruby{日}{にち}/\ruby{日}{か}\}\ruby{目}{め}/\ruby{第}{だい}〜\ruby{日}{たち}} \label{sec:appendix-vocab-basic-nouns-ordinal-days}
\emph{Read the main article on \href{https://www.tofugu.com/japanese/japanese-counter-ka-nichi/}{Tofugu}.}

\ruby{目}{め} is casual, \ruby{第}{だい} is formal. \ruby{目}{め} is 和語 and follows pronunciations from Section~\ref{sec:appendix-vocab-basic-nouns-counting-days}; \ruby{第}{だい} is 漢語 and forces \ruby{日}{にち} to take its 漢語 reading.

% Help: \SetCell[r=2,c=2]{c,m} <content>, \cmidrule[l]{3-4}
% Help: colspec: X[ratio, horizontal alignment] columns grow to fit width=\linewidth
%                  negative ratios: shrink to fit content and may not grow to full ratio
% Help: colspec: l/c/r columns do not grow
\longtabse[0.75]  % scale factor
{Nouns: ordinal days.}  % caption
{tbl:appendix-vocab-basic-nouns-ordinal-days}  % label
{}  % outer specification options
{
    colspec={X[-1,c]X[-3,l]X[3,l]X[-3,l]},
    rowhead=1,
    % width=\linewidth,  % useful only with X columns
}  % inner specification options
{
    \toprule
    & \textbf{Name} & \textbf{Meaning} & \textbf{Notes} \\
    \midrule
    & \ruby{何日}{なん|にち} & which day of month/which day?/how many days? & \\
    & \ruby{第何日}{だい|なん|にち} & which day? & \\
    & \ruby{1日目}{いち|にち|め}/\ruby{第1日}{だい|いち|にち} & first day & \\
    \textlegacybullet & \ruby{2日目}{ふつ|か|め}/\ruby{第2日}{だい|に|にち} & second day & \\
    & \ruby{3日目}{みっ|か|め}/\ruby{第3日}{だい|さん|にち} & third day & \\
    & \ruby{4日目}{よっ|か|め}/\ruby{第4日}{だい|よん|にち} & fourth day & \\
    & \ruby{5日目}{いつ|か|め}/\ruby{第5日}{だい|ご|にち} & fifth day & \\
    \textlegacybullet & \ruby{6日目}{むい|か|め}/\ruby{第6日}{だい|ろく|にち} & sixth day & \\
    \textlegacybullet & \ruby{7日目}{なの|か|め}/\ruby{第7日}{だい|なな|にち} & seventh day & \\
    \textlegacybullet & \ruby{8日目}{よう|か|め}/\ruby{第8日}{だい|はち|にち} & eighth day & \\
    \textlegacybullet & \ruby{9日目}{ここの|か|め}/\ruby{第9日}{だい|く|にち} & ninth day & \\
    & \ruby[g]{10日目}{とおかめ}/\ruby[g]{第10日}{だいじゅうにち} & tenth day & \\
    & \ruby{11日目}{じゅう|いち|にち|め}/\ruby{第11日}{だい|じゅう|いち|にち} & eleventh day & \\
    & \ruby{12日目}{じゅう|に|にち|め}/\ruby{第12日}{だい|じゅう|に|にち} & twelfth day & \\
    & \ruby{13日目}{じゅう|さん|にち|め}/\ruby{第13日}{だい|じゅう|さん|にち} & thirteenth day & \\
    & \ruby{14日目}{じゅう|よん|にち|め}/\ruby{第14日}{だい|じゅう|よん|にち} & fourteenth day & \\
    & \ruby{15日目}{じゅう|ご|にち|め}/\ruby{第15日}{だい|じゅう|ご|にち} & fifteenth day & \\
    & \ruby{16日目}{じゅう|ろく|にち|め}/\ruby{第16日}{だい|じゅう|ろく|にち} & sixteenth day & \\
    \textlegacybullet & \ruby{17日目}{じゅう|しち|にち|め}/\ruby{第17日}{だい|じゅう|なな|にち} & seventeenth day & \\
    & \ruby{18日目}{じゅう|はち|にち|め}/\ruby{第18日}{だい|じゅう|はち|にち} & eighteenth day & \\
    \textlegacybullet & \ruby{19日目}{じゅう|く|にち|め}/\ruby{第19日}{だい|じゅう|く|にち} & nineteenth day & \\
    \textlegacybullet & \ruby[g]{20日目}{はつかめ}/\ruby{第20日}{だい|に|じゅう|にち} & twentieth day & \\
    & \ruby{21日目}{にじゅう|いち|にち|め}/\ruby{第21日}{だい|にじゅう|いち|にち} & twenty-first day & \\
    & \ruby{22日目}{にじゅう|に|にち|め}/\ruby{第22日}{だい|にじゅう|に|にち} & twenty-second day & \\
    & \ruby{23日目}{にじゅう|さん|にち|め}/\ruby{第23日}{だい|にじゅう|さん|にち} & twenty-third day & \\
    & \ruby{24日目}{にじゅう|よん|にち|め}/\ruby{第24日}{だい|にじゅう|よん|にち} & twenty-fourth day & \\
    & \ruby{25日目}{にじゅう|ご|にち|め}/\ruby{第25日}{だい|にじゅう|ご|にち} & twenty-fifth day & \\
    & \ruby{26日目}{にじゅう|ろく|にち|め}/\ruby{第26日}{だい|にじゅう|ろく|にち} & twenty-sixth day & \\
    \textlegacybullet & \ruby{27日目}{にじゅう|しち|にち|め}/\ruby{第27日}{だい|にじゅう|なな|にち} & twenty-seventh day & \\
    & \ruby{28日目}{にじゅう|はち|にち|め}/\ruby{第28日}{だい|にじゅう|はち|にち} & twenty-eighth day & \\
    \textlegacybullet & \ruby{29日目}{にじゅう|く|にち|め}/\ruby{第29日}{だい|にじゅう|く|にち} & twenty-ninth day & \\
    & \ruby{30日目}{さん|じゅう|にち|め}/\ruby{第30日}{だい|さん|じゅう|にち} & thirtieth day & \\
    & \ruby{37日目}{さんじゅう|なな|にち|め}/\ruby{第37日}{だい|さんじゅう|なな|にち} & thirty-seventh day & \\
    & \ruby{39日目}{さんじゅう|きゅう|にち|め}/\ruby{第39日}{だい|さんじゅう|きゅう|にち} & thirty-ninth day & \\
    \textlegacybullet & \ruby{49日目}{よんじゅう|く|にち|め}/\ruby{第49日}{だい|よんじゅう|く|にち} & forty-ninth day & \\
    & \ruby{59日目}{ごじゅう|きゅう|にち|め}/\ruby{第59日}{だい|ごじゅう|きゅう|にち} & fifty-ninth day & \\
    & \ruby[g]{100日目}{ひゃくにちめ}/\ruby[g]{第100日}{だいひゃくにち} & hundredth day & \\
    & \ruby[g]{1000日目}{せんにちめ}/\ruby[g]{第1000日}{だいせんにち} & thousandth day & \\
    & \ruby{1万日目}{いち|まん|にち|め}/\ruby{第1万日}{だい|いち|まん|にち} & ten thousandth day & \\
    % & & & \\
    \bottomrule
}


\subsection{Counting days: \{\ruby{日}{にち}/\ruby{日}{か}\}\ruby{間}{かん}} \label{sec:appendix-vocab-basic-nouns-counting-days}
\emph{Read the main article on \href{https://www.tofugu.com/japanese/japanese-counter-ka-nichi/}{Tofugu}.}

The 「〜\ruby{間}{かん}」 suffix here means ``interval''. While typically dropped in the absence of ambiguity, it is necessary here to distinguish day intervals from days of the month (\S\ref{sec:appendix-vocab-basic-nouns-calendar-days}; \href{https://en.wikibooks.org/wiki/Japanese/Lessons/Days\#Periods\_of\_days}{[WB]}).

% \textorange{For formal settings and specifically \textbf{days and hours} (with the exception of day of month \ruby{七日}{なの|か}), \ruby{七}{しち} is preferred over \ruby{七}{なな}, though the latter may be used for disambiguation with \ruby{一}{いち}.}

% Help: \SetCell[r=2,c=2]{c,m} <content>, \cmidrule[l]{3-4}
% Help: colspec: X[ratio, horizontal alignment] columns grow to fit width=\linewidth
%                  negative ratios: shrink to fit content and may not grow to full ratio
% Help: colspec: l/c/r columns do not grow
\longtabse[0.75]  % scale factor
{Nouns: counting days.}  % caption
{tbl:appendix-vocab-basic-nouns-counting-days}  % label
{}  % outer specification options
{
    colspec={X[-1,c]X[-3,l]X[3,l]X[-3,l]},
    rowhead=1,
    % width=\linewidth,  % useful only with X columns
}  % inner specification options
{
    \toprule
    & \textbf{Name} & \textbf{Meaning} & \textbf{Notes} \\
    \midrule
    & \ruby{何日間}{なん|にち|かん} & which day of month/which day?/how many days? & \\
    & \ruby{1日間}{いち|にち|かん} & one day & \\
    \textlegacybullet & \ruby{2日間}{ふつ|か|かん} & two days & \\
    & \ruby{3日間}{みっ|か|かん} & three days & \\
    & \ruby{4日間}{よっ|か|かん} & four days & \\
    & \ruby{5日間}{いつ|か|かん} & five days & \\
    \textlegacybullet & \ruby{6日間}{むい|か|かん} & six days & \\
    \textlegacybullet & \ruby{7日間}{なの|か|かん} & seven days & \\
    \textlegacybullet & \ruby{8日間}{よう|か|かん} & eight days & \\
    & \ruby{9日間}{ここの|か|かん} & nine days & \\
    & \ruby[g]{10日間}{とおかかん} & ten days & \\
    & \ruby{11日間}{じゅう|いち|にち|かん} & eleven days & \\
    & \ruby{12日間}{じゅう|に|にち|かん} & twelve days & \\
    & \ruby{13日間}{じゅう|さん|にち|かん} & thirteen days & \\
    & \ruby{14日間}{じゅう|よん|にち|かん} & fourteen days & \\
    & \ruby{15日間}{じゅう|ご|にち|かん} & fifteen days & \\
    & \ruby{16日間}{じゅう|ろく|にち|かん} & sixteen days & \\
    \textlegacybullet & \ruby{17日間}{じゅう|しち|にち|かん} & seventeen days & \ruby{17日間}{じゅう|なな|にち|かん} is sometimes used for disambiguation \\
    & \ruby{18日間}{じゅう|はち|にち|かん} & eighteen days & \\
    \textlegacybullet & \ruby{19日間}{じゅう|く|にち|かん} & nineteen days & \\
    \textlegacybullet & \ruby[g]{20日間}{はつかかん} & twenty days & \\
    & \ruby{21日間}{にじゅう|いち|にち|かん} & twenty-one days & \\
    & \ruby{22日間}{にじゅう|に|にち|かん} & twenty-two days & \\
    & \ruby{23日間}{にじゅう|さん|にち|かん} & twenty-three days & \\
    & \ruby{24日間}{にじゅう|よん|にち|かん} & twenty-four days & \\
    & \ruby{25日間}{にじゅう|ご|にち|かん} & twenty-five days & \\
    & \ruby{26日間}{にじゅう|ろく|にち|かん} & twenty-six days & \\
    \textlegacybullet & \ruby{27日間}{にじゅう|しち|にち|かん} & twenty-seven days & \ruby{27日間}{にじゅう|なな|にち|かん} is sometimes used for disambiguation \\
    & \ruby{28日間}{にじゅう|はち|にち|かん} & twenty-eight days & \\
    \textlegacybullet & \ruby{29日間}{にじゅう|く|にち|かん} & twenty-nine days & \\
    & \ruby{30日間}{さん|じゅう|にち|かん} & thirty days & \\
    & \ruby{37日間}{さんじゅう|なな|にち|かん} & thirty-seven days & \\
    & \ruby{39日間}{さんじゅう|きゅう|にち|かん} & thirty-nine days & \\
    \textlegacybullet & \ruby{49日間}{にじゅう|く|にち|かん} & forty-nine days & \\
    & \ruby{59日間}{にじゅう|きゅう|にち|かん} & fifty-nine days & \\
    & \ruby[g]{100日間}{ひゃくにちかん} & hundred days & \\
    & \ruby[g]{1000日間}{せんにちかん} & thousand days & \\
    & \ruby{1万日間}{いち|まん|にち|かん} & ten thousand days & \\
    % & & & \\
    \bottomrule
}


\subsection{Counting o'clocks and hours: \ruby{時}{じ} and \ruby{時間}{じ|かん}} \label{sec:appendix-vocab-basic-nouns-counting-oclocks-and-hours}
\href{https://www.tofugu.com/japanese/japanese-counter-ji-jikan/}{\hl{Tofugu: TO READ}}

% Help: \SetCell[r=2,c=2]{c,m} <content>, \cmidrule[l]{3-4}
% Help: colspec: X[ratio, horizontal alignment] columns grow to fit width=\linewidth
%                  negative ratios: shrink to fit content and may not grow to full ratio
% Help: colspec: l/c/r columns do not grow
\longtabse[0.75]  % scale factor
{Nouns: counting o'clocks and hours.}  % caption
{tbl:appendix-vocab-basic-nouns-counting-oclocks-and-hours}  % label
{}  % outer specification options
{
    colspec={X[-1,c]X[-3,l]X[3,l]X[-3,l]},
    rowhead=1,
    % width=\linewidth,  % useful only with X columns
}  % inner specification options
{
    \toprule
    & \textbf{Name} & \textbf{Meaning} & \textbf{Notes} \\
    \midrule
    & \ruby{何時}{なん|じ} & which hour (of day)? & \\
    & \ruby{0時}{れい|じ} & zero o'clock (midnight/noon) & \\
    & \ruby{1時}{いち|じ} & one o'clock & \\
    & \ruby{2時}{に|じ} & two o'clock & \\
    & \ruby{3時}{さん|じ} & three o'clock & \\
    \textlegacybullet & \ruby{4時}{よ|じ} & four o'clock & \\
    & \ruby{5時}{ご|じ} & five o'clock & \\
    & \ruby{6時}{ろく|じ} & six o'clock & \\
    \textlegacybullet & \ruby{7時}{しち|じ} & seven o'clock & \ruby{7時}{なな|じ} is sometimes used for disambiguation \\
    & \ruby{8時}{はち|じ} & eight o'clock & \\
    \textlegacybullet & \ruby{9時}{く|じ} & nine o'clock & \\
    & \ruby[g]{10時}{じゅうじ} & ten o'clock & \\
    & \ruby{11時}{じゅう|いち|じ} & eleven o'clock & \\
    & \ruby{12時}{じゅう|に|じ} & twelve o'clock & \\
    & \ruby{13時}{じゅう|さん|じ} & thirteen o'clock & \\
    \textlegacybullet & \ruby{14時}{じゅう|よ|じ} & fourteen o'clock & \\
    & \ruby{15時}{じゅう|ご|じ} & fifteen o'clock & \\
    & \ruby{16時}{じゅう|ろく|じ} & sixteen o'clock & \\
    \textlegacybullet & \ruby{17時}{じゅう|しち|じ} & seventeen o'clock & \ruby{17時}{じゅう|なな|じ} is sometimes used for disambiguation \\
    & \ruby{18時}{じゅう|はち|じ} & eighteen o'clock & \\
    \textlegacybullet & \ruby{19時}{じゅう|く|じ} & nineteen o'clock & \\
    & \ruby{20時}{に|じゅう|じ} & twenty o'clock & \\
    & \ruby{21時}{にじゅう|いち|じ} & twenty-one o'clock & \\
    & \ruby{22時}{にじゅう|に|じ} & twenty-two o'clock & \\
    & \ruby{23時}{にじゅう|さん|じ} & twenty-three o'clock & \\
    \textlegacybullet & \ruby{24時}{にじゅう|よ|じ} & twenty-four o'clock & \\
    % & & & \\
    \midrule
    \midrule
    & \ruby{何時間}{なん|じ|かん} & how many hours? & \\
    & \ruby{1時間}{いち|じ|かん} & one hour & \\
    & \ruby{2時間}{に|じ|かん} & two hours & \\
    & \ruby{3時間}{さん|じ|かん} & three hours & \\
    \textlegacybullet & \ruby{4時間}{よ|じ|かん} & four hours & \\
    & \ruby{5時間}{ご|じ|かん} & five hours & \\
    & \ruby{6時間}{ろく|じ|かん} & six hours & \\
    \textlegacybullet & \ruby{7時間}{しち|じ|かん} & seven hours & \ruby{7時間}{なな|じ|かん} is sometimes used for disambiguation \\
    & \ruby{8時間}{はち|じ|かん} & eight hours & \\
    \textlegacybullet & \ruby{9時間}{く|じ|かん} & nine hours & \\
    & \ruby[g]{10時間}{じゅうじかん} & ten hours & \\
    & \ruby{11時間}{じゅう|いち|じ|かん} & eleven hours & \\
    & \ruby{12時間}{じゅう|に|じ|かん} & twelve hours & \\
    & \ruby{13時間}{じゅう|さん|じ|かん} & thirteen hours & \\
    \textlegacybullet & \ruby{14時間}{じゅう|よ|じ|かん} & fourteen hours & \\
    & \ruby{15時間}{じゅう|ご|じ|かん} & fifteen hours & \\
    & \ruby{16時間}{じゅう|ろく|じ|かん} & sixteen hours & \\
    \textlegacybullet & \ruby{17時間}{じゅう|しち|じ|かん} & seventeen hours & \ruby{17時間}{じゅう|なな|じ|かん} is sometimes used for disambiguation \\
    & \ruby{18時間}{じゅう|はち|じ|かん} & eighteen hours & \\
    \textlegacybullet & \ruby{19時間}{じゅう|く|じ|かん} & nineteen hours & \\
    & \ruby{20時間}{に|じゅう|じ|かん} & twenty hours & \\
    & \ruby{21時間}{にじゅう|いち|じ|かん} & twenty-one hours & \\
    & \ruby{22時間}{にじゅう|に|じ|かん} & twenty-two hours & \\
    & \ruby{23時間}{にじゅう|さん|じ|かん} & twenty-three hours & \\
    \textlegacybullet & \ruby{24時間}{にじゅう|よ|じ|かん} & twenty-four hours & \\
    & \ruby[g]{100時間}{ひゃくじかん} & hundred hours & \\
    & \ruby[g]{1000時間}{せんじかん} & thousand hours & \\
    & \ruby{1万時間}{いち|まん|じ|かん} & ten thousand hours & \\
    % & & & \\
    \bottomrule
}


\subsection{Counting minutes: \ruby{分}{ふん}}
\emph{Read the main article on \href{https://www.tofugu.com/japanese/japanese-counter-fun/}{Tofugu}.}

% Help: \SetCell[r=2,c=2]{c,m} <content>, \cmidrule[l]{3-4}
% Help: colspec: X[ratio, horizontal alignment] columns grow to fit width=\linewidth
%                  negative ratios: shrink to fit content and may not grow to full ratio
% Help: colspec: l/c/r columns do not grow
\longtabse[0.75]  % scale factor
{Nouns: counting minutes.}  % caption
{tbl:appendix-vocab-basic-nouns-counting-minutes}  % label
{}  % outer specification options
{
    colspec={X[-1,c]X[-3,l]X[3,l]X[-3,l]},
    rowhead=1,
    % width=\linewidth,  % useful only with X columns
}  % inner specification options
{
    \toprule
    & \textbf{Name} & \textbf{Meaning} & \textbf{Notes} \\
    \midrule
    \textlegacybullet & \ruby{何分}{なん|ぷん} & how many minutes? & \\
    \textlegacybullet & \ruby{1分}{いっ|ぷん} & one minute & \\
    & \ruby{2分}{に|ふん} & two minutes & \\
    \textlegacybullet & \ruby{3分}{さん|ぷん} & three minutes & \\
    \textlegacybullet & \ruby{4分}{よん|ぷん} & four minutes & \\
    & \ruby{5分}{ご|ふん} & five minutes & \\
    \textlegacybullet & \ruby{6分}{ろっ|ぷん} & six minutes & \\
    & \ruby{7分}{なな|ふん} & seven minutes & \\
    \textlegacybullet & \ruby{8分}{はっ|ぷん} & eight minutes & \\
    & \ruby{9分}{きゅう|ふん} & nine minutes & \\
    \textlegacybullet & \ruby[g]{10分}{じゅっぷん} & ten minutes & \\
    & \ruby{15分}{じゅう|ご|ふん} & fifteen minutes & \\
    \textlegacybullet & \ruby{30分}{さん|じゅっ|ぷん} & thirty minutes & \\
    & \ruby{45分}{よんじゅう|ご|ふん} & fourty-five minutes & \\
    \textlegacybullet & \ruby[g]{100分}{ひゃっぷん} & hundred minutes & \\
    \textlegacybullet & \ruby[g]{1000分}{せんぷん} & thousand minutes & \\
    \textlegacybullet & \ruby{1万分}{いち|まん|ぷん} & ten thousand minutes & \\
    % & & & \\
    \bottomrule
}


\subsection{Counting seconds: \ruby{秒}{びょう}}

% Help: \SetCell[r=2,c=2]{c,m} <content>, \cmidrule[l]{3-4}
% Help: colspec: X[ratio, horizontal alignment] columns grow to fit width=\linewidth
%                  negative ratios: shrink to fit content and may not grow to full ratio
% Help: colspec: l/c/r columns do not grow
\longtabse[0.75]  % scale factor
{Nouns: counting seconds.}  % caption
{tbl:appendix-vocab-basic-nouns-counting-seconds}  % label
{}  % outer specification options
{
    colspec={X[-1,c]X[-3,l]X[3,l]X[-3,l]},
    rowhead=1,
    % width=\linewidth,  % useful only with X columns
}  % inner specification options
{
    \toprule
    & \textbf{Name} & \textbf{Meaning} & \textbf{Notes} \\
    \midrule
    & \ruby{何秒}{なん|びょう} & how many seconds? & \\
    & \ruby{1秒}{いち|びょう} & one second & \\
    & \ruby{2秒}{に|びょう} & two seconds & \\
    & \ruby{3秒}{さん|びょう} & three seconds & \\
    & \ruby{4秒}{よん|びょう} & four seconds & \\
    & \ruby{5秒}{ご|びょう} & five seconds & \\
    & \ruby{6秒}{ろく|びょう} & six seconds & \\
    & \ruby{7秒}{なな|びょう} & seven seconds & \\
    & \ruby{8秒}{はち|びょう} & eight seconds & \\
    & \ruby{9秒}{きゅう|びょう} & nine seconds & \\
    & \ruby[g]{10秒}{じゅうびょう} & ten seconds & \\
    & \ruby[g]{100秒}{ひゃくびょう} & hundred seconds & \\
    & \ruby[g]{1000秒}{せんびょう} & thousand seconds & \\
    & \ruby{1万秒}{いち|まん|びょう} & ten thousand seconds & \\
    % & & & \\
    \bottomrule
}


\subsection{Counting positions: \ruby{番}{ばん}}
% Help: \SetCell[r=2,c=2]{c,m} <content>, \cmidrule[l]{3-4}
% Help: colspec: X[ratio, horizontal alignment] columns grow to fit width=\linewidth
%                  negative ratios: shrink to fit content and may not grow to full ratio
% Help: colspec: l/c/r columns do not grow
\longtabse[0.75]  % scale factor
{Nouns: counting positions.}  % caption
{tbl:appendix-vocab-basic-nouns-counting-positions}  % label
{}  % outer specification options
{
    colspec={X[-1,c]X[-3,l]X[3,l]X[-3,l]},
    rowhead=1,
    % width=\linewidth,  % useful only with X columns
}  % inner specification options
{
    \toprule
    & \textbf{Name} & \textbf{Meaning} & \textbf{Notes} \\
    \midrule
    & \ruby{何番}{なん|ばん} & which position? & \\
    & \ruby{1番}{いち|ばん} & first position & \\
    & \ruby{2番}{に|ばん} & second position & \\
    & \ruby{3番}{さん|ばん} & third position & \\
    & \ruby{4番}{よん|ばん} & fourth position & \\
    & \ruby{5番}{ご|ばん} & fifth position & \\
    & \ruby{6番}{ろく|ばん} & sixth position & \\
    & \ruby{7番}{なな|ばん} & seventh position & \\
    & \ruby{8番}{はち|ばん} & eighth position & \\
    & \ruby{9番}{きゅう|ばん} & ninth position & \\
    & \ruby[g]{10番}{じゅうばん} & tenth position & \\
    & \ruby[g]{100番}{ひゃくばん} & hundredth position & \\
    & \ruby[g]{1000番}{せんばん} & thousandth position & \\
    & \ruby{1万番}{いち|まん|ばん} & ten thousandth position & \\
    % & & & \\
    \bottomrule
}


\subsection{Counting occurrences: \ruby{回}{かい}}
\href{https://www.tofugu.com/japanese/japanese-counter-kai-times/}{\hl{Tofugu: TO READ}}

% Help: \SetCell[r=2,c=2]{c,m} <content>, \cmidrule[l]{3-4}
% Help: colspec: X[ratio, horizontal alignment] columns grow to fit width=\linewidth
%                  negative ratios: shrink to fit content and may not grow to full ratio
% Help: colspec: l/c/r columns do not grow
\longtabse[0.75]  % scale factor
{Nouns: counting occurrences.}  % caption
{tbl:appendix-vocab-basic-nouns-counting-occurrences}  % label
{}  % outer specification options
{
    colspec={X[-1,c]X[-3,l]X[3,l]X[-3,l]},
    rowhead=1,
    % width=\linewidth,  % useful only with X columns
}  % inner specification options
{
    \toprule
    & \textbf{Name} & \textbf{Meaning} & \textbf{Notes} \\
    \midrule
    & \ruby{何回}{なん|かい} & how many times? & \\
    \textlegacybullet & \ruby{1回}{いっ|かい} & one time & \\
    & \ruby{2回}{に|かい} & two times & \\
    & \ruby{3回}{さん|かい} & three times & \\
    & \ruby{4回}{よん|かい} & four times & \\
    & \ruby{5回}{ご|かい} & five times & \\
    \textlegacybullet & \ruby{6回}{ろっ|かい} & six times & \\
    & \ruby{7回}{なな|かい} & seven times & \\
    \color{lightgray}\textlegacybullet & \ruby{8回}{はち|かい}\color{lightgray}/\ruby{8回}{はっ|かい} & eight times & \textlightgrey{\ruby{8回}{はっ|かい}} is probably informal \\
    & \ruby{9回}{きゅう|かい} & nine times & \\
    \textlegacybullet & \ruby[g]{10回}{じゅっかい} & ten times & \\
    \textlegacybullet & \ruby[g]{100回}{ひゃっかい} & hundred times & \\
    & \ruby[g]{1000回}{せんかい} & thousand times & \\
    & \ruby{1万回}{いち|まん|かい} & ten thousand times & \\
    % & & & \\
    \bottomrule
}


\subsection{Counting methods/ways/kinds: \ruby{通}{とお}り}
\href{https://www.tofugu.com/japanese/japanese-counter-kai-times/}{\hl{Tofugu: TO READ}}

% Help: \SetCell[r=2,c=2]{c,m} <content>, \cmidrule[l]{3-4}
% Help: colspec: X[ratio, horizontal alignment] columns grow to fit width=\linewidth
%                  negative ratios: shrink to fit content and may not grow to full ratio
% Help: colspec: l/c/r columns do not grow
\longtabse[0.75]  % scale factor
{Nouns: counting occurrences.}  % caption
{tbl:appendix-vocab-basic-nouns-counting-occurrences}  % label
{}  % outer specification options
{
    colspec={X[-1,c]X[-3,l]X[3,l]X[-3,l]},
    rowhead=1,
    % width=\linewidth,  % useful only with X columns
}  % inner specification options
{
    \toprule
    & \textbf{Name} & \textbf{Meaning} & \textbf{Notes} \\
    \midrule
    & \ruby{何通}{なん|とお}り & how many methods/ways/kinds? & \\
    \textlegacybullet & \ruby{1通}{いっ|とお}り & one method/way/kind & \\
    & \ruby{2通}{に|とお}り & two methods/ways/kinds & \\
    & \ruby{3通}{さん|とお}り & three methods/ways/kinds & \\
    & \ruby{4通}{よん|とお}り & four methods/ways/kinds & \\
    & \ruby{5通}{ご|とお}り & five methods/ways/kinds & \\
    & \ruby{6通}{ろく|とお}り & six methods/ways/kinds & \\
    & \ruby{7通}{なな|とお}り & seven methods/ways/kinds & \\
    \color{lightgray}\textlegacybullet & \ruby{8通}{はち|とお}り\color{lightgray}/\ruby{8通}{はっ|とお}り & eight methods/ways/kinds & \textlightgrey{\ruby{8通}{はっ|とお}り} is probably informal \\
    & \ruby{9通}{きゅう|とお}り & nine methods/ways/kinds & \\
    \textlegacybullet & \ruby[g]{10通}{じゅっとお}り & ten methods/ways/kinds & \\
    & \ruby[g]{100通}{ひゃくとお}り & hundred methods/ways/kinds & \\
    & \ruby[g]{1000通}{せんとお}り & thousand methods/ways/kinds & \\
    & \ruby{1万通}{いち|まん|とお}り & ten thousand methods/ways/kinds & \\
    % & & & \\
    \bottomrule
}


\end{document}
