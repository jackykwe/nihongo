\documentclass[../nihongo-gakushuu-kyouzai.tex]{subfiles}
\begin{document}
\appendix
\setcounter{section}{0}
\section{Grammar mega summary}

\subsection{Conjugation rules summary}
\subsubsection{Nouns}
\begin{table}[h]
\centering
\resizebox{\linewidth}{!}{%
% Help: \multicolumn{2}{c}{}, \multirow{2}{*}{}, cmidrule(l){3-5}
\begin{tabular}{@{}ccll@{}}
    \toprule
    \textbf{Purpose}                & \textbf{Tense}   & \textbf{Casual schema} & \textbf{Polite schema}                          \\ \midrule
    \multirow{4}{*}{State-of-being} & Present-positive & <noun>[だ]             & <noun>です。                                    \\
                                    & Past-positive    & <noun>だった           & <noun>でした。                                  \\
                                    & Present-negative & <noun>じゃない         & <noun>じゃないです。/じゃありません。           \\
                                    & Past-negative    & <noun>じゃなかった     & <noun>じゃなかったです。/じゃありませんでした。 \\
\bottomrule
\end{tabular}%
}
\caption{Noun conjugation rules.}
\label{tbl:appendix-noun-conjugations}
\end{table}

\textorange{In \textgreen{敬語}, 「〜した」indicates the past-tense.}

Also, 「〜ないです。」can be replaced with「〜ありまえん。」, and 「〜なかったです。」 can be replaced with 「〜ありませんでした。」 as alternative polite forms of the present-negative and past-negative respectively.
\subsubsection{な-adjectives}
\begin{table}[h]
\centering
\resizebox{\linewidth}{!}{%
% Help: \multicolumn{2}{c}{}, \multirow{2}{*}{}, cmidrule(l){3-5}
\begin{tabular}{@{}ccll@{}}
    \toprule
    \textbf{Purpose}               & \textbf{Tense}   & \textbf{Casual schema}     & \textbf{Polite schema}                            \\ \midrule
                                   & Present-positive & <na-adj>[だ]               & <na-adj>です。                                    \\
    State-of-being                 & Past-positive    & <na-adj>だった             & <na-adj>でした。                                  \\
    (same as nouns)                & Present-negative & <na-adj>じゃない           & <na-adj>じゃないです。/じゃありません。           \\
                                   & Past-negative    & <na-adj>じゃなかった       & <na-adj>じゃなかったです。/じゃありませんでした。 \\ \midrule
    \multirow{4}{*}{Noun modifier} & Present-positive & <na-adj>\textbf{な}<noun>           &                                          \\
                                   & Past-positive    & <na-adj>だった<noun>       &                                                   \\
                                   & Present-negative & <na-adj>じゃない<noun>     &                                                   \\
                                   & Past-negative    & <na-adj>じゃなかった<noun> &                                                   \\
    \bottomrule
\end{tabular}%
}
\caption{な-adjective conjugation rules.}
\label{tbl:appendix-な-adjective-conjugations}
\end{table}

\subsubsection{い-adjectives}
All い-adjectives end with 〜い that is \ul{not} part of the 漢字's pronunciation.
\begin{table}[h]
\centering
\resizebox{\linewidth}{!}{%
% Help: \multicolumn{2}{c}{}, \multirow{2}{*}{}, cmidrule(l){3-5}
\begin{tabular}{@{}ccll@{}}
    \toprule
    \textbf{Purpose}                & \textbf{Tense}   & \textbf{Casual schema}       & \textbf{Polite schema}                                                \\ \midrule
    \multirow{4}{*}{State-of-being} & Present-positive & <i-adj>い                    & <i-adj>いです。                                                     \\
                                    & Past-positive    & <i-adj>\textbf{か}った       & <i-adj>\textbf{か}ったです。                                          \\
                                    & Present-negative & <i-adj>\textbf{く}ない       & <i-adj>\textbf{く}ないです。/\textbf{く}ありません。           \\
                                    & Past-negative    & <i-adj>\textbf{く}なかった   & <i-adj>\textbf{く}なかったです。/\textbf{く}ありませんでした。 \\ \midrule
    \multirow{4}{*}{Noun modifier}  & Present-positive & <i-adj>い<noun>              &                                                                       \\
                                    & Past-positive    & <i-adj>\textbf{か}った<noun> &                                                                       \\
                                    & Present-negative & <i-adj>\textbf{く}ない<noun> &                                                                       \\
                                    & Past-negative    & <i-adj>\textbf{く}なかった<noun> &                                                                       \\
    \bottomrule
\end{tabular}%
}
\caption{い-adjective conjugation rules.}
\label{tbl:appendix-い-adjective-conjugations}
\end{table}

\color{red}
Exceptions:
\begin{description}
    \item[Adjectives ending with 「〜いい」] When in any form other than present-positive, the root becomes 「〜よい」.
\end{description}
\color{black}

\subsubsection{Verbs}
See Table~\ref{tbl:verb-classification} for a summary of the three categories. In a nutshell, る-verbs is the class of \emph{almost all} \ul{-iru/-eru} verbs; all other verbs are う-verbs. Exception verbs are する and くる. \textorange{Mnemonic: Group I is the most superior; 五段 is superior to 一段; う comes before る in the 平仮名 alphabet chart.}

\begin{table}[h]
\centering
\resizebox{\linewidth}{!}{%
% Help: \multicolumn{2}{c}{}, \multirow{2}{*}{}, cmidrule(l){3-5}
\begin{tabular}{@{}ccll@{}}
    \toprule
    \textbf{Purpose}                & \textbf{Tense}   & \textbf{Casual schema}       & \textbf{Polite schema}                                                \\ \midrule
    \multirow{4}{*}{State-of-being} & Present-positive & <i-adj>い                    & <i-adj>いです。                                                     \\
                                    & Past-positive    & <i-adj>\textbf{か}った       & <i-adj>\textbf{か}ったです。                                          \\
                                    & Present-negative & <i-adj>\textbf{く}ない       & <i-adj>\textbf{く}ないです。/\textbf{く}ありません。           \\
                                    & Past-negative    & <i-adj>\textbf{く}なかった   & <i-adj>\textbf{く}なかったです。/\textbf{く}ありませんでした。 \\ \midrule
    \multirow{4}{*}{Noun modifier}  & Present-positive & <i-adj>い<noun>              &                                                                       \\
                                    & Past-positive    & <i-adj>\textbf{か}った<noun> &                                                                       \\
                                    & Present-negative & <i-adj>\textbf{く}ない<noun> &                                                                       \\
                                    & Past-negative    & <i-adj>\textbf{く}なかった<noun> &                                                                       \\
    \bottomrule
\end{tabular}%
}
\caption{Verb conjugation rules.}
\label{tbl:appendix-い-adjective-conjugations}
\end{table}


\subsection{Particle summary}
% Help: \multicolumn{2}{c}{}, \multirow{2}{*}{}, cmidrule(l){3-5}
\scriptsize
\begin{longtable}[c]{@{}llll@{}}
    \toprule
    \textbf{Particle} & \textbf{Particle name/purpose} & \textbf{Schemae} & \textbf{First seen} \\* \midrule
    は & Introductory topic marker & <main/new topic>は & \S\ref{sec:topic-marker}, \S\ref{sec:particles} \\
    も & Inclusive topic marker & <inclusive topic>も & \S\ref{sec:particles} \\
    が & Subject marker & <subject>が & \S\ref{sec:particles} \\
    を & Direct object marker & <direct object>を<transitive verb> & \S\ref{sec:verb-particles} \\
    & & <location>を<transitive/intransitive motion verb> & \\
    に & Target marker & <target/location>に\textlightgrey{[は/も]}<transitive/intransitive verb> & \S\ref{sec:verb-particles} \\
    &  & <time>[に\textlightgrey{[は/も]}]<transitive/intransitive verb> & \\
    から & From-marker & <from>から & \S\ref{sec:verb-particles} \\
    まで & To-marker & <to>まで & \S\ref{sec:verb-particles} \\
    へ & Direction marker & <direction>へ\textlightgrey{[は/も]}<transitive/intransitive verb> & \S\ref{sec:verb-particles} \\
    で & Context marker & <by-way-of context>で\textlightgrey{[は/も]} & \S\ref{sec:verb-particles} \\
    と & Inclusive noun connector & <noun>と<noun> & \S\ref{sec:noun-related-particles} \\
    や & Vague listing connector & <noun>や<noun> & \S\ref{sec:noun-related-particles} \\
    どか & \textbf{ここで続きます。} &  & \S\ref{sec:noun-related-particles} \\
    の &  &  & \S\ref{sec:noun-related-particles} \\
    の &  &  & \S\ref{sec:noun-related-particles} \\
    % &  &  &  \\*
    \bottomrule
    \caption{All particles seen so far.}
    \label{tbl:} \\
\end{longtable}%
\normalsize



% Full A4 dimensions: width=210, height=297
% Width/height and LR/TB here are before turning into landscape.
\newgeometry{top=2cm,bottom=2cm,left=1cm,right=1cm}
\fancyhfoffset{0pt} % recalculate fancyhdr width, courtesy of https://tex.stackexchange.com/a/440307
\begin{landscape}

\begin{table}[h]
\centering
\resizebox{0.8\linewidth}{!}{%
% Help: \multicolumn{2}{c}{}, \multirow{2}{*}{}, cmidrule(l){3-5}
\begin{tabular}{@{}clllllllll@{}}
    \toprule
    \multirow{2}{*}{\textbf{Category}}   & \multirow{2}{*}{\textbf{Dictionary form}} & \multirow{2}{*}{\textbf{Verb stem}} & \multicolumn{3}{c}{\textbf{Positive}}                                                                                                                  & \multicolumn{3}{c}{\textbf{Negative}}                                                                                                                                                                          & \multirow{2}{*}{\textbf{Examples}} \\ \cmidrule(lr){4-6} \cmidrule(lr){7-9}
                                         &                                           &                                     & \textbf{Present}                         & \textbf{Past}                                         & \textbf{て-form}                     & \textbf{Present}                                        & \textbf{Past}                                                 & \textbf{て-form}                         & \\ \midrule
    \multirow{3}{*}{Noun/な-adjective}   & 「」                                      &                                     & 「」\textgreen{[だ]}                     & 「」\textgreen{だった}                                & 「」\textgreen{で}                   & 「」\textgreen{じゃない}                                & 「」\textgreen{じゃなかった}                                  & 「」\textgreen{じゃなくて}               & \multirow{3}{*}{学生、友達} \\
                                         &                                           &                                     & 「」\textgreen{です。}                   & 「」\textgreen{でした$\!\!^\dagger$。}                &                                      & 「」\textgreen{じゃないです。}                          & 「」\textgreen{じゃなかったです。}                            &                                          & \\
                                         &                                           &                                     &                                          &                                                       &                                      & 「」\textgreen{じゃありません。}                        & 「」\textgreen{じゃありませんでした。}                        &                                          & \\ \midrule
    な-adjective                         & 「」                                      &                                     & 「」\textgreen{な}                       & 「」\textgreen{だった}                                & 「」\textgreen{で}                   & 「」\textgreen{じゃない}                                & 「」\textgreen{じゃなかった}                                  & 「」\textgreen{じゃなくて}               & 元気、綺麗、好き、\textred{嫌い}  \\ \midrule
    \multirow{9}{*}{い-adjective}        & 「」い                                    &                                     & 「」い                                   & 「」\textblue{かった}                                 & 「」\textblue{くて}                  & 「」\textblue{くない}                                   & 「」\textblue{くなかった}                                     & 「」\textblue{くなくて}                  & \multirow{3}{*}{嬉しい、気持ち悪い、難しい} \\
                                         &                                           &                                     & 「」い\textgreen{です。}                 & 「」\textblue{かった}\textgreen{です。}               &                                      & 「」\textblue{くない}\textgreen{です。}                 & 「」\textblue{くなかった}\textgreen{です。}                   &                                          & \\
                                         &                                           &                                     &                                          &                                                       &                                      & 「」\textblue{く}\textgreen{ありません。}               & 「」\textblue{く}\textgreen{ありませんでした。}               &                                          & \\ \cmidrule(l){2-10}
                                         & いい                                      &                                     & いい                                     & \textred{よ}\textblue{かった}                         & \textred{よ}\textblue{くて}          & \textred{よ}\textblue{くない}                           & \textred{よ}\textblue{くなかった}                             & \textred{よ}\textblue{くなかくて}        & \\
                                         &                                           &                                     & いい\textgreen{です。}                   & \textred{よ}\textblue{かった}\textgreen{です。}       &                                      & \textred{よ}\textblue{くない}\textgreen{です。}         & \textred{よ}\textblue{くなかった}\textgreen{です。}           &                                          & \\
                                         &                                           &                                     &                                          &                                                       &                                      & \textred{よ}\textblue{く}\textgreen{ありません。}       & \textred{よ}\textblue{く}\textgreen{ありませんでした。}       &                                          & \\ \cmidrule(l){2-10}
                                         & かっこいい                                &                                     & かっこいい                               & かっこ\textred{よ}\textblue{かった}                   & かっこ\textred{よ}\textblue{くて}    & かっこ\textred{よ}\textblue{くない}                     & かっこ\textred{よ}\textblue{くなかった}                       & かっこ\textred{よ}\textblue{くなかくて}  & \\
                                         &                                           &                                     & かっこいい\textgreen{です。}             & かっこ\textred{よ}\textblue{かった}\textgreen{です。} &                                      & かっこ\textred{よ}\textblue{くない}\textgreen{です。}   & かっこ\textred{よ}\textblue{くなかった}\textgreen{です。}     &                                          & \\
                                         &                                           &                                     &                                          &                                                       &                                      & かっこ\textred{よ}\textblue{く}\textgreen{ありません。} & かっこ\textred{よ}\textblue{く}\textgreen{ありませんでした。} &                                          & \\ \midrule
    \multirow{20}{*}{う-verb}            & 「」す                                    &                                     & 「」す                                   & 「」\textblue{した}                                   & 「」\textblue{して}                  & 「」\textblue{さない}                                   & 「」\textblue{さなかった}                                     & 「」\textblue{さなくて}                  & \multirow{2}{*}{話す、出す} \\
                                         &                                           & 「」\textblue{し}                   & 「」\textblue{します。}                  & 「」\textblue{しました。}                             &                                      & 「」\textblue{しません。}                               & 「」\textblue{しませんでした。}                               &                                          & \\ \cmidrule(l){2-10}
                                         & 「」く                                    &                                     & 「」く                                   & 「」\textblue{いた}                                   & 「」\textblue{いて}                  & 「」\textblue{かない}                                   & 「」\textblue{かなかった}                                     & 「」\textblue{かなくて}                  & \multirow{2}{*}{聞く、書く、描く} \\
                                         &                                           & 「」\textblue{き}                   & 「」\textblue{きます。}                  & 「」\textblue{きました。}                             &                                      & 「」\textblue{きません。}                               & 「」\textblue{きませんでした。}                               &                                          & \\ \cmidrule(l){2-10}
                                         & 行く                                      &                                     & 行く                                     & 行\textblue{った}                                     & 行\textblue{って}                    & 行\textblue{かない}                                     & 行\textblue{かなかった}                                       & 行\textblue{かなくて}                    & \\
                                         &                                           & 行\textblue{き}                     & 行\textblue{きます。}                    & 行\textblue{きました。}                               &                                      & 行\textblue{きません。}                                   & 行\textblue{きませんでした。}                                 &                                          & \\ \cmidrule(l){2-10}
                                         & 「」ぐ                                    &                                     & 「」ぐ                                   & 「」\textblue{いだ}                                   & 「」\textblue{いで}                  & 「」\textblue{がない}                                   & 「」\textblue{がなかった}                                     & 「」\textblue{がなくて}                  & \multirow{2}{*}{泳ぐ} \\
                                         &                                           & 「」\textblue{ぎ}                   & 「」\textblue{ぎます。}                  & 「」\textblue{ぎました。}                             &                                      & 「」\textblue{ぎません。}                               & 「」\textblue{ぎませんでした。}                               &                                          & \\ \cmidrule(l){2-10}
                                         & 「」む                                    &                                     & 「」む                                   & 「」\textblue{んだ}                                   & 「」\textblue{んで}                  & 「」\textblue{まない}                                   & 「」\textblue{まなかった}                                     & 「」\textblue{まなくて}                  & \multirow{2}{*}{飲む、読む、\ruby{盗}{ぬす}む} \\
                                         &                                           & 「」\textblue{み}                   & 「」\textblue{みます。}                  & 「」\textblue{みました。}                             &                                      & 「」\textblue{みません。}                               & 「」\textblue{みませんでした。}                               &                                          & \\ \cmidrule(l){2-10}
                                         & 「」ね                                    &                                     & 「」ね                                   & 「」\textblue{んだ}                                   & 「」\textblue{んで}                  & 「」\textblue{なない}                                   & 「」\textblue{ななかった}                                     & 「」\textblue{ななくて}                  & \multirow{2}{*}{死ね} \\
                                         &                                           & 「」\textblue{に}                   & 「」\textblue{にます。}                  & 「」\textblue{にました。}                             &                                      & 「」\textblue{にません。}                               & 「」\textblue{にませんでした。}                               &                                          & \\ \cmidrule(l){2-10}
                                         & 「」ぶ                                    &                                     & 「」ぶ                                   & 「」\textblue{んだ}                                   & 「」\textblue{んで}                  & 「」\textblue{ばない}                                   & 「」\textblue{ばなかった}                                     & 「」\textblue{ばなくて}                  & \multirow{2}{*}{遊ぶ} \\
                                         &                                           & 「」\textblue{び}                   & 「」\textblue{びます。}                  & 「」\textblue{びました。}                             &                                      & 「」\textblue{びません。}                               & 「」\textblue{びませんでした。}                               &                                          & \\ \cmidrule(l){2-10}
                                         & 「」る                                    &                                     & 「」る                                   & 「」\textblue{った}                                   & 「」\textblue{って}                  & 「」\textblue{らない}                                   & 「」\textblue{らなかった}                                     & 「」\textblue{らなくて}                  & \multirow{2}{*}{\textred{知る}、\textred{切る}、\textred{\ruby{帰}{かえ}る}、\textred{\ruby{走}{はし}る}、\ruby{降}{ふ}る} \\
                                         &                                           & 「」\textblue{り}                   & 「」\textblue{ります。}                  & 「」\textblue{りました。}                             &                                      & 「」\textblue{りません。}                               & 「」\textblue{りませんでした。}                               &                                          & \\ \cmidrule(l){2-10}
                                         & ある                                      &                                     & ある                                     & あ\textblue{った}                                     & あ\textblue{って}                    & \textred{ない}                                          & \textred{なかった}                                            & \textred{なくて}                         & \\
                                         &                                           & あ\textblue{り}                     & あ\textblue{ります。}                    & あ\textblue{りました。}                               &                                      & あ\textred{りません。}                                  & あ\textred{りませんでした。}                                  &                                          & \\ \cmidrule(l){2-10}
                                         & 「」つ                                    &                                     & 「」つ                                   & 「」\textblue{った}                                   & 「」\textblue{って}                  & 「」\textblue{たない}                                   & 「」\textblue{たなかった}                                     & 「」\textblue{たなくて}                  & \multirow{2}{*}{待つ} \\
                                         &                                           & 「」\textblue{ち}                   & 「」\textblue{ちます。}                  & 「」\textblue{ちました。}                             &                                      & 「」\textblue{ちません。}                               & 「」\textblue{ちませんでした。}                               &                                          & \\ \cmidrule(l){2-10}
                                         & 「」う                                    &                                     & 「」う                                   & 「」\textblue{った}                                   & 「」\textblue{って}                  & 「」\textblue{わない}                                   & 「」\textblue{わなかった}                                     & 「」\textblue{わなくて}                  & \multirow{2}{*}{買う、\ruby{会}{あ}う} \\
                                         &                                           & 「」\textblue{い}                   & 「」\textblue{います。}                  & 「」\textblue{いました。}                             &                                      & 「」\textblue{いません。}                               & 「」\textblue{いませんでした。}                               &                                          & \\ \midrule
    \multirow{2}{*}{る-verb}             & 「」る                                    &                                     & 「」る                                   & 「」\textblue{た}                                     & 「」\textblue{て}                    & 「」\textblue{ない}                                     & 「」\textblue{なかった}                                       & 「」\textblue{なくて}                    & \multirow{2}{*}{いる、食べる、\ruby{出}{で}る、見る*} \\
                                         &                                           & 「」                                & 「」\textblue{ます。}                    & 「」\textblue{ました。}                               &                                      & 「」\textblue{ません。}                                 & 「」\textblue{ませんでした。}                                 &                                          & \\ \midrule
    \multirow{7}{*}{Exception verb}    & 「」する                                    &                                     & 「」する                                 & 「」\textblue{した}                                   & 「」\textblue{して}                  & 「」\textblue{しない}                                   & 「」\textblue{しなかった}                                     & 「」\textblue{しなくて}                  & \multirow{2}{*}{勉強する、楽しみにする、質問をする} \\
                                         &                                           & \textblue{し}                       & 「」\textblue{します。}                  & 「」\textblue{しました。}                             &                                      & 「」\textblue{しません。}                               & 「」\textblue{しませんでした。}                               &                                          & \\ \cmidrule(l){2-10}
                                         & くる                                      &                                     & くる                                     & \textred{き}\textblue{た}                             & \textred{き}\textblue{て}            & \textred{こ}\textblue{ない}                             & \textred{こ}\textblue{なかった}                               & \textred{こ}\textblue{なくて}            & \\[0.5em]
                                         &                                           & \textred{き}                        & \textred{き}\textblue{ます。}            & \textred{き}\textblue{ました。}                       &                                      & \textred{き}\textblue{ません。}                         & \textred{き}\textblue{ませんでした。}                         &                                          & \\ \cmidrule(l){2-10} \\[-0.75em]
                                         & \ruby{来}{く}る                           &                                     & \ruby{来}{く}る                          & \textred{\ruby{来}{き}}\textblue{た}                  & \textred{\ruby{来}{き}}\textblue{て} & \textred{\ruby{来}{こ}}\textblue{ない}                  & \textred{\ruby{来}{こ}}\textblue{なかった}                    & \textred{\ruby{来}{こ}}\textblue{なくて} & \\[0.5em]
                                         &                                           & \textred{\ruby{来}{き}}             & \textred{\ruby{来}{き}}\textblue{ます。} & \textred{\ruby{来}{き}}\textblue{ました。}            &                                      & \textred{\ruby{来}{き}}\textblue{ません。}              & \textred{\ruby{来}{き}}\textblue{ませんでした。}              &                                          & \\ \bottomrule
\end{tabular}%
}
\caption{All conjugation rules, for nouns, adjectives and verbs. [] means optional; 「」 is a dictionary-form placeholer; \textgreen{green means additive} (without modifying the dictionary-form); \textblue{blue means substitutive} (modifies the dictionary-form); \textred{red means exception}.}
\label{tbl:appendix-conjugation}
\end{table}

\end{landscape}
\restoregeometry
\clearpage


\end{document}
