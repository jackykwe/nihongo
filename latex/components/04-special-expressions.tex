\documentclass[../nihongo-gakushuu-kyouzai.tex]{subfiles}
\begin{document}

\setcounter{section}{3}
\section{Special expressions} \label{sec:special-expressions}

This section is named ``special expressions'' because except the first few subsections, most of the grammar here applies to more specific use-cases. However, this ``special expressions'' section as as whole is necessary for everyday conversations.

\subsection{Causative, passive and causative-passive forms} \label{sec:causative-passive-causative-passive-forms}
We finally learn the last three major verb conjugations.

\subsubsection{Causative form 〜\ruby{◯}{〜あ}\{せる/す\}} \label{sec:causative-form}
The causative form of verbs indicate that someone \emph{was made to} perform the verb. It has two senses: making somebody do something, and letting someone to do something. \textred{Disambiguation between the two senses is based on context!}

When the causative form is used with あげる/くれる, it almost always means to ``let someone do''. Otherwise, it usually means ``make someone do''.

For detailed conjugation rules, refer to Appendix~\ref{appendix:conjugation-rules-summary}. All causative form verbs are る-verbs. Further conjugations follow る-verb rules.

\textorange{When listening, 〜\ruby{◯}{〜あ}せ is the signal for passive voice, though for る-verbs there is potential conflict with the potential form.}

E.g.\ 全部\textbf{食べさせた}。 (Made/let someone eat it all.)

E.g.\ 全部\textbf{食べさせてくれた}。 (Let someone eat it all.)

E.g.\ 先生が学生に宿題をたくさん\textbf{させた}。 (Teacher made students do lots of homework.)

E.g.\ 先生が質問をたくさん\textbf{聞かせてくれた}。 (Teacher let someone ask lots of questions.; 聞く $\to$ 聞かせる (causative) $\to$ 聞かせてくれる (to give the favour))

E.g.\ 今日は仕事を\textbf{休ませてください}。 (Please let me rest from work today.; 休む $\to$ 休ませる (causative) $\to$ 休ませてください (desiderative of causative, ``let-do'' sense))

E.g.\ その部長は、よく\ruby{長}{ちょう}時間\textbf{\ruby{働}{はたら}かせる}。 (That manager often makes people work long hours.; \ruby{働}{かたら}く (to work) $\to$ 働かせる (causative))

When \emph{asking} for permission to let someone do something (including letting yourself do something), it's more common to use the 「〜てもいい」 grammar (\S\ref{sec:permission}).

E.g.\ お手洗いに\textbf{行かせてくれません}か。 (Can you let me go to the toilet? (sounds like a prisoner, even in English))

E.g.\ お手洗いに\textbf{行ってもいい}ですか。 (Is it ok to go to the toilet?)

In very rough/casual slang, the causative form may be expressed with the う-verb ending 〜す. For detailed conjugation rules, refer to Appendix~\ref{appendix:conjugation-rules-summary}. All causative form verbs are う-verbs. Further conjugations follow う-verb rules (with す ending).

E.g.\ 同じことを何回も\textbf{言わす}\textsuperscript{言わせる}な! (Don't make me say the same thing again and again!)

E.g.\ お腹空いているんだから、なんか\textbf{食べさしてくれ}\textsuperscript{食べさせてくれ}よ。 (I'm hungry, so let me eat something.; 食べる $\to$ 食べさす $\to$ 食べさしてくれる $\to$ 食べさしてくれ (imperative))

\subsubsection{Interlude: passivisation, direct and indirect/adversative passives} \label{sec:direct-and-indirect-adversative-passive}
\emph{Read the supplementary materials on \href{https://www.tofugu.com/japanese-grammar/verb-passive-form-rareru\#direct-vs-indirect-passive}{[TFG1]} and \href{https://www.tofugu.com/japanese-grammar/particle-ni/\#in-passive-sentences}{[TFG2]}.}

\emph{There is no such thing as a ``suffering passive''. I'll thus avoid mentioning the term ``adversative passive''. (\href{https://www.guidetojapanese.org/blog/2005/09/09/no-suffering-passive/}{[TK]})}

Note that parts-of-speech (e.g.\ object, subject) are purely grammatical (syntactic) roles. An entity being the performer of the action or the receiver of the action's effects are semantic roles. Semantic roles do not change under passivisation (otherwise the sentence's meaning changes), whereas grammatical roles may change.

(SL) In English, passivisation can only be applied to a sentence containing a transitive verb. When it occurs, \ul{the object is promoted to the subject position}, and the subject is demoted to an oblique (a non-required argument).


We now distinguish between the direct and indirect/adversative passives \emph{in Japanese}. The necessary grammatical elements are underlined.
\begin{itemize}
    \item \textbf{Direct passive \textred{\ul{(transitive only)}}}: used to express that the \ul{transitive verb} was done to \ul{someone/something (subject experiencer; marked by が)}, by someone (origin oblique/optional performer; marked by に). The effect that the action has on the experiencer is very obvious, since the verb is done directly to them (promoted to subject grammatical role). The subject (demoted to an oblique) is often omitted as it's not important. The Japanese direct passive is equivalent to the English passive.

    In Japanese, the standard SOV sentence schema is <subject>が<object>を<transitive v>, or <performer>が<experiencer>を<transitive v> in semantic terms. After passivisation into the \textbf{direct passive}, it becomes <performer>に<experiencer>が<transitive v passive form>. Notice that the semantic performer is now the syntactic origin (marked by に), and the semantic experiencer is now the syntactic subject (marked by が/は). This is very similar to もらう's usage of に as a origin particle (\S\ref{sec:receiving}, \href{https://www.tofugu.com/japanese-grammar/particle-ni/\#in-social-interactions-and-transactions}{[TFG]}), since \textorange{the passive subject is the receiver (experiencer) in the transaction (action)}.

    E.g.\ ピカソに\textbf{キスされた}。 (I was kissed by Picasso.; remember that in passive voice に marks the origin of the action)

    E.g.\ タバコが\textbf{吸われた}。 (A cigarette was smoked.)

    \item \textbf{Indirect passive (both transitive and intransitive)}: used to express that a \ul{someone (origin performer; marked by に)} did a \ul{transitive/intransitive verb}, involving an optional direct object (only for transitive verbs; marked by を), and it had an effect on someone (experiencer; marked by は). ??????Notice that compared to the direct passive, there are now three necessary grammatical elements.????? It is often used to complain about something, thus it's called the adversative passive. There is no equivalent in English.

    E.g.\ (私は)(あのおじさんに)タバコを吸われた。 (I got smoked on (by that man).; \href{https://www.tofugu.com/japanese-grammar/verb-passive-form-rareru\#direct-vs-indirect-passive}{[TFG1]})

    In the above example, the passive verb is 吸われた, the experiencer is me, and the performer is あのおじさん (marked by origin particle に). The indirect passive emphasises that this act of smoking was done to me, and that I did not have control over it, creating the nuance that the act was a nuisance.

    E.g.\ ブラット・ピットに目の前でタバコを\textbf{吸われて}、\ruby{気絶}{き|ぜつ}するかと思った。 (I got smoked on by Brad Pit right in front of me, and I thought I was going to faint.; \ruby{気絶}{き|ぜつ}: loss of consciousness; more natural translation would be ``Brad Pitt smoked right in front of me, and I thought I was going to faint.''; \href{https://www.tofugu.com/japanese-grammar/verb-passive-form-rareru\#direct-vs-indirect-passive}{[TFG1]})

    In the above example, the indirect passive is used, but the effect may not be adversative depending on the context: I could be fainting from the smoke, or from the fact that it's Brad Pitt.

    Except for a few cases, the indirect passive is always formed with verbs that were intentionally performed by some\emph{one}. The performer of the verb (marked by に) is therefore usually a person. Only (some) weather-related verbs are acceptable exceptions to this rule, where the performer is a weather-related object.

    E.g.\ \st{私は\textred{本に}頭に落ちられた。 (My head was fallen on by a book.; \href{https://www.tofugu.com/japanese-grammar/verb-passive-form-rareru\#direct-vs-indirect-passive}{[TFG1]}) \textred{Unnatural sentence: the performer should not be an object.}}

    E.g.\ \ruby{突然}{とつ|ぜん}雨に\textbf{降られた}。 (All of a sudden, I was rained on.; 降る is intransitive; \href{https://www.tofugu.com/japanese-grammar/verb-passive-form-rareru\#direct-vs-indirect-passive}{[TFG1]})

    E.g.\ \ruby{花子}{かな|こ}が\ruby{隣}{となり}の学生にピアノを朝まで\ruby{弾}{ひ}かれた。(Hanako had the student next to her play the piano on her until morning.; a more natural translation is ``Hanako was adversely affected by the student next to her playing the piano until morning.''; \href{https://en.wikipedia.org/wiki/Passive\_voice\#Adversative\_passive}{[Wiki]})
\end{itemize}


\subsubsection{Passive form  〜\ruby{◯}{〜あ}れる} \label{sec:causative-form}

The passive voice in Japanese is often used in written essays and articles. Read more about passivisation in Section~\ref{sec:direct-and-indirect-adversative-passive}.

For detailed conjugation rules, refer to Appendix~\ref{appendix:conjugation-rules-summary}. All passive form verbs are る-verbs. Further conjugations follow る-verb rules. \textred{Note that for る-verbs and 来る, the passive form is identical to the potential form (〜られる); disambiguation requires context and is otherwise impossible.}

\textorange{When listening, 〜\ruby{◯}{〜あ}れ is the signal for passive voice, though for る-verbs there is potential conflict with the potential form.}

In Japanese, the standard SOV sentence schema is <subject>が<object>を<transitive v>, or <performer>が<experiencer>を<transitive v> in semantic terms. After passivisation into the \textbf{direct passive}, it becomes <performer>に<experiencer>が<transitive v passive form>. Notice that the semantic performer is now the syntactic origin (marked by に), and the semantic experiencer is now the syntactic subject (marked by が/は). This is very similar to もらう's usage of に as a origin particle (\S\ref{sec:receiving}, \href{https://www.tofugu.com/japanese-grammar/particle-ni/\#in-social-interactions-and-transactions}{[TFG]}), since \textorange{the passive subject is the receiver (experiencer) in the transaction (action)}.

E.g.\ ポリッジが誰かに\textbf{食べられた}! (The porridge was eaten by somebody!)

E.g.\ みんなに変だと\textbf{言われます}。 (I am told by everybody that (I'm/it's) strange.)

E.g.\ 光の\ruby{速}{はや}さを\ruby{超}{こ}えるのは、不可能だと\textbf{思われる}。 (Exceeding the speed of light is thought to be impossible.)

E.g.\ この教科書は多くの人に\textbf{読まれている}。 (This textbook is being read by a large number of people.)

E.g.\ \ruby{外国人}{がい|こく|じん}に質問を\textbf{聞かれた}が、答えられなかった。 (I was asked a question by a foreigner, but I couldn't answer.)

E.g.\ このパッケージには、あらゆる物が\textbf{\ruby{含}{ふく}まれている}。 (Everything is included in this package.; あらゆる: all, \ruby{含}{ふく}む: to contain)

% Because an indirect sentence is more polite in Japanese, the passive voice is used to show a level of politeness above the normal ます form.
In a similar sense to how it's more polite to address someone indirectly, and how it's more polite to ask negative questions than positive ones (\S\ref{sec:positive-negative-questions}), the passive form makes the sentence less direct because the subject (now the experiencer) does not directly perform the action. In increasing levels of politeness of expressing ``what will you do?'':
\begin{enumerate}[label=\arabic*.]
    \item (active voice) どうする?
    \item (active voice, polite) どうしますか。
    \item (passive voice, polite) どうされますか。
    \item (\hl{???} voice, honorific, \S\hl{???}) どうなさいますか。
    \item (\hl{???} voice, honorific, less certainty, \S\hl{???}) どうなさいますでしょうか。
\end{enumerate}
With increasing indirection and politeness, the sentence grows longer and longer.

E.g.\ \ruby{領収証}{りょう|しゅう|しょう}はどう\textbf{されます}か。 (What about your receipt?)

E.g.\ 明日の\ruby{会議}{かい|ぎ}に\textbf{行かれる}んですか。 (Are you going to tomorrow's meeting?; passive voice deliberately used for politeness)

\subsubsection{Caustive-passive form 〜\ruby{◯}{〜あ}せられる} \label{sec:causative-passive-form}

The causative-passive form is used to express the idea that the action of ``making someone do something'' was performed to that person, or in short, the idea that someone was made to do something. The verb is first conjugated to the causative then the passive, never the other way round.

For detailed conjugation rules, refer to Appendix~\ref{appendix:conjugation-rules-summary}. All passive form verbs are る-verbs. Further conjugations follow る-verb rules.

The causative-passive form is a variant of the passive voice, so the に particle marks the \emph{origin}, i.e.\ the performer of the action.

\textorange{When listening, 〜\ruby{◯}{〜あ}せられ is the signal for causative-passive voice.}

E.g.\ 朝ごはんは食べたくなかったのに、\textbf{食べさせられた}。 (Despite not wanting to eat breakfast, I was made to eat it.)

E.g.\ 日本では、お酒を\textbf{飲ませられる}ことが多い。 (In Japan, the event of being made to drink is numerous.)

E.g.\ あいつに二時間も\textbf{待たせられた}。 (I was made to wait 2 hours by that guy.)

E.g.\ \ruby{親}{おや}に毎日宿題を\textbf{させられる}。 (I was made to do homework everyday by my parents.; \ruby{親}{おや}: parents)

\textred{This shortened causative-passive form only exists for う-verbs with $*\setminus$\{す\}-ending.}

In very rough/casual slang, the shortened causative-passive form (derived from the shortened causaive form, \S\ref{sec:causative-form}) may be used. This form only exists for う-verbs with the exception of those with a す-ending in dictionary form, because wherever the shortened causative form ends with 〜さす, the shortened causative-passive form would have 「$\cdots$\textred{ささ}れる」 in it, which is not allowed.

E.g.\ 学生が\ruby{廊下}{ろう|か}に\textbf{\ruby{立}{た}たされた}\textsuperscript{立たせられた}。 (The stuednt was made to stand in the hall.; \ruby{立}{た}つ: to stand)

E.g.\ 日本では、お酒を\textbf{飲まされる}\textsuperscript{飲ませられる}ことが多い。 (In Japan, the event of being made to drink is numerous.)

E.g.\ あいつに二時間も\textbf{待たされた}\textsuperscript{待たせられた}。 (I was made to wait 2 hours by that guy.)

\end{document}
