\documentclass[../nihongo-gakushuu-kyouzai.tex]{subfiles}
\begin{document}

\setcounter{section}{0}
\section{The writing system}
\subsection{平仮名(ひらがな)}
Some general notes:
\begin{itemize}
    \item The ん character is rarely used by itself, but suffixed to another character to add the ``n'' sound.
\end{itemize}
\subsubsection{Mnemonics}

% Help: \SetCell[r=2,c=2]{c,m} <content>, \cmidrule[l]{3-4}
% Help: colspec: X[ratio, horizontal alignment] columns grow to fit width=\linewidth
%                  negative ratios: shrink to fit content and may not grow to full ratio
% Help: colspec: l/c/r columns do not grow
\longtabse[1]  % scale factor
{平仮名 mnemonic table. $^\dagger$Particle romanisations.}  % caption
{tbl:hiragana-mnemonics}  % label
{
    colspec={ccccl},
    rowhead=1,
    % width=\linewidth,  % useful only with X columns
}  % inner specification options
{
    \toprule
    \SetCell[c=2]{c,m} \textbf{平仮名} & & {\textbf{Hepburn}\\\textbf{romanisation}} & {\textbf{Keyboard}\\\textbf{command}} & \textbf{Mnemonic} \\
    \midrule
    あ & \emph{あ} & a & \texttt{a} & ``A'' shape \\
    い & \emph{い} & i & \texttt{i} & \ul{ee}l \\
    う & \emph{う} & u & \texttt{u} & ``u'' shape \\
    え & \emph{え} & e & \texttt{e} & \ul{e}xotic swan \\
    お & \emph{お} & o & \texttt{o} & double ``o'' shape \\
    か & \emph{か} & ka & \texttt{ka} & 咖啡 \\
    が & \emph{が} & ga & \texttt{ga} &  \\
    き & \emph{き} & ki & \texttt{ki} & \ul{ke}y \\
    きゃ & \emph{きゃ} & kya & \texttt{kya} &  \\
    きゅ & \emph{きゅ} & kyu & \texttt{kyu} &  \\
    きょ & \emph{きょ} & kyo & \texttt{kyo} &  \\
    ぎ & \emph{ぎ} & gi & \texttt{gi} &  \\
    ぎゃ & \emph{ぎゃ} & gya & \texttt{gya} &  \\
    ぎゅ & \emph{ぎゅ} & gyu & \texttt{gyu} &  \\
    ぎょ & \emph{ぎょ} & gyo & \texttt{gyo} &  \\
    く & \emph{く} & ku & \texttt{ku} & bird \ul{ku}-ku \\
    ぐ & \emph{ぐ} & gu & \texttt{gu} &  \\
    け & \emph{け} & ke & \texttt{ke} & \ul{ke}lp (loose kelp) \\
    げ & \emph{げ} & ge & \texttt{ge} &  \\
    こ & \emph{こ} & ko & \texttt{ko} & \ul{co}-habiting worms \\
    ご & \emph{ご} & go & \texttt{go} &  \\
    さ & \emph{さ} & sa & \texttt{sa} & \ul{sa}lsa (two hand stir) / NOT ``5'' \\
    ざ & \emph{ざ} & za & \texttt{za} &  \\
    し & \emph{し} & shi & \textlightgrey{\texttt{si}/}\texttt{shi} & \ul{shee}p; shepherd's crook \\
    しゃ & \emph{しゃ} & sha & \texttt{sha} &  \\
    しゅ & \emph{しゅ} & shu & \texttt{shu} &  \\
    しょ & \emph{しょ} & sho & \texttt{sho} &  \\
    じ & \emph{じ} & ji & \textlightgrey{\texttt{zi}/}\texttt{ji} &  \\
    じゃ & \emph{じゃ} & ja & \textlightgrey{\texttt{jya}/}\texttt{ja} &  \\
    じゅ & \emph{じゅ} & ju & \textlightgrey{\texttt{jyu}/}\texttt{ju} &  \\
    じょ & \emph{じょ} & jo & \textlightgrey{\texttt{jyo}/}\texttt{jo} &  \\
    す & \emph{す} & su & \texttt{su} & \ul{sw}ing \\
    ず & \emph{ず} & zu & \texttt{zu} & \\
    せ & \emph{せ} & se & \texttt{se} & \ruby{世界}{せ|かい} \\
    ぜ & \emph{ぜ} & ze & \texttt{ze} &  \\
    そ & \emph{そ} & so & \texttt{so} & \ul{so}da / ``sword'' shape \\
    ぞ & \emph{ぞ} & zo & \texttt{zo} &  \\
    た & \emph{た} & ta & \texttt{ta} & ``ta'' shape \\
    だ & \emph{だ} & da & \texttt{da} &  \\
    ち & \emph{ち} & chi & \textlightgrey{\texttt{ti}/}\texttt{chi} & the ``5'' \\
    ちゃ & \emph{ちゃ} & cha & \texttt{cha} &  \\
    ちゅ & \emph{ちゅ} & chu & \texttt{chu} &  \\
    ちょ & \emph{ちょ} & cho & \texttt{cho} &  \\
    ぢ & \emph{ぢ} & ji & \textred{\texttt{di}} &  \\
    ぢゃ & \emph{ぢゃ} & ja & \textred{\texttt{dya}} &  \\
    ぢゅ & \emph{ぢゅ} & ju & \textred{\texttt{dyu}} &  \\
    ぢょ & \emph{ぢょ} & jo & \textred{\texttt{dyo}} &  \\
    つ & \emph{つ} & tsu & \textlightgrey{\texttt{tu}/}\texttt{tsu} & \ul{tsu}nami \\
    づ & \emph{づ} & zu & \color{red} \texttt{du} &  \\
    て & \emph{て} & te & \texttt{te} & \ul{te}lescope \\
    で & \emph{で} & de & \texttt{de} &  \\
    と & \emph{と} & to & \texttt{to} & \ul{to}e with splinter \\
    ど & \emph{ど} & do & \texttt{do} &  \\
    な & \emph{な} & na & \texttt{na} & \ul{nu}n praying to cross \\
    に & \emph{に} & ni & \texttt{ni} & \ul{nee}dle \\
    にゃ & \emph{にゃ} & nya & \texttt{nya} &  \\
    にゅ & \emph{にゅ} & nyu & \texttt{nyu} &  \\
    にょ & \emph{にょ} & nyo & \texttt{nyo} &  \\
    ぬ & \emph{ぬ} & nu & \texttt{nu} & \ul{noo}dles \textbf{with tail} \\
    ね & \emph{ね} & ne & \texttt{ne} & ねこ (\ul{ne}ko) \textbf{with tail} \\
    の & \emph{の} & no & \texttt{no} & pig \ul{no}se \\
    は & \emph{は} & ha/wa$^\dagger$ & \textred{\texttt{ha}} & ``Ha'' shape \\
    ば & \emph{ば} & ba & \texttt{ba} &  \\
    ぱ & \emph{ぱ} & pa & \texttt{pa} &  \\
    ひ & \emph{ひ} & hi & \texttt{hi} & \ul{hee}l / \ul{he} has a big nose \\
    ひゃ & \emph{ひゃ} & hya & \texttt{hya} &  \\
    ひゅ & \emph{ひゅ} & hyu & \texttt{hyu} &  \\
    ひょ & \emph{ひょ} & hyo & \texttt{hyo} &  \\
    び & \emph{び} & bi & \texttt{bi} &  \\
    びゃ & \emph{びゃ} & bya & \texttt{bya} &  \\
    びゅ & \emph{びゅ} & byu & \texttt{byu} &  \\
    びょ & \emph{びょ} & byo & \texttt{byo} &  \\
    ぴ & \emph{ぴ} & pi & \texttt{pi} &  \\
    ぴゃ & \emph{ぴゃ} & pya & \texttt{pya} &  \\
    ぴゅ & \emph{ぴゅ} & pyu & \texttt{pyu} &  \\
    ぴょ & \emph{ぴょ} & pyo & \texttt{pyo} &  \\
    ふ & \emph{ふ} & fu & \textlightgrey{\texttt{hu}/}\texttt{fu} & Mount \ul{Fu}ji \\
    ぶ & \emph{ぶ} & bu & \texttt{bu} &  \\
    ぷ & \emph{ぷ} & pu & \texttt{pu} &  \\
    へ & \emph{へ} & he/e$^\dagger$ & \textred{\texttt{he}} & \ul{he}adband / Mount St. \ul{He}lens \\
    べ & \emph{べ} & be & \texttt{be} &  \\
    ぺ & \emph{ぺ} & pe & \texttt{pe} &  \\
    ほ & \emph{ほ} & ho & \texttt{ho} & mutated santa says \ul{ho} ho ho\\
    ぼ & \emph{ぼ} & bo & \texttt{bo} &  \\
    ぽ & \emph{ぽ} & po & \texttt{po} &  \\
    ま & \emph{ま} & ma & \texttt{ma} & mutated mom with snake tail \\
    み & \emph{み} & mi & \texttt{mi} & \ul{me} just turned 21 \\
    みゃ & \emph{みゃ} & mya & \texttt{mya} &  \\
    みゅ & \emph{みゅ} & myu & \texttt{myu} &  \\
    みょ & \emph{みょ} & myo & \texttt{myo} &  \\
    む & \emph{む} & mu & \texttt{mu} & cow says \ul{moo} \\
    め & \emph{め} & me & \texttt{me} & eye shape \textbf{without tail} \\
    も & \emph{も} & mo & \texttt{mo} & \ul{mo}re worms to catch \ul{mo}re fish \\
    や & \emph{や} & ya & \texttt{ya} & \ul{ya}cht with anchor down \\
    ゆ & \emph{ゆ} & yu & \texttt{yu} & \ul{u}-tensils \\
    よ & \emph{よ} & yo & \texttt{yo} & ``yo'' shape \\
    ら & \emph{ら} & ra & \texttt{ra} & \ul{ra}bbit \\
    り & \emph{り} & ri & \texttt{ri} & reeds \\
    りゃ & \emph{りゃ} & rya & \texttt{rya} &  \\
    りゅ & \emph{りゅ} & ryu & \texttt{ryu} &  \\
    りょ & \emph{りょ} & ryo & \texttt{ryo} &  \\
    る & \emph{る} & ru & \texttt{ru} & weird \ul{rou}te \textbf{with tail} \\
    れ & \emph{れ} & re & \texttt{re} & \ul{re}tching guy kneeled down \\
    ろ & \emph{ろ} & ro & \texttt{ro} & normal \ul{ro}ad \textbf{without tail} \\
    わ & \emph{わ} & wa & \texttt{wa} & \ul{wa}llaby / \ul{wa}sp \\
    を & \emph{を} & wo/o$^\dagger$ & \texttt{wo} & \ul{wo}ah the water is cold \\
    ん & \emph{ん} & nn & \textred{\texttt{nn}} & ``n'' shape \\
    ぁ & \emph{ぁ} & ? & \textlightgrey{\texttt{la}/}\textred{\texttt{xa}} &  \\
    ぃ & \emph{ぃ} & ? & \textlightgrey{\texttt{li}/}\textred{\texttt{xi}} &  \\
    ぅ & \emph{ぅ} & ? & \textlightgrey{\texttt{lu}/}\textred{\texttt{xu}} &  \\
    ぇ & \emph{ぇ} & ? & \textlightgrey{\texttt{le}/}\textred{\texttt{xe}} &  \\
    ぉ & \emph{ぉ} & ? & \textlightgrey{\texttt{lo}/}\textred{\texttt{xo}} &  \\
    ゃ & \emph{ゃ} & ? & \textlightgrey{\texttt{lya}/}\textred{\texttt{xya}} &  \\
    ゅ & \emph{ゅ} & ? & \textlightgrey{\texttt{lyu}/}\textred{\texttt{xyu}} &  \\
    ょ & \emph{ょ} & ? & \textlightgrey{\texttt{lyo}/}\textred{\texttt{xyo}} &  \\
    っ & \emph{っ} & $^{\texttt{+1}}$\texttt{>} & \textlightgrey{\texttt{ltu}/\texttt{ltsu}/\textred{\texttt{xtsu}}/}repeat \texttt{>} & \\
    \bottomrule
}


\subsection{片仮名(カタカナ)}
Some general notes:
\begin{itemize}
    \item Usage of the ・ symbol to denote word boundaries is completely optional.
\end{itemize}
\subsubsection{Mnemonics}

% Help: \SetCell[r=2,c=2]{c,m} <content>, \cmidrule[l]{3-4}
% Help: colspec: X[ratio, horizontal alignment] columns grow to fit width=\linewidth
%                  negative ratios: shrink to fit content and may not grow to full ratio
% Help: colspec: l/c/r columns do not grow
\longtabse[1]  % scale factor
{片仮名 mnemonic table. Some entries were taken from \href{https://en.wikipedia.org/wiki/Hepburn_romanization\#Extended_katakana}{Wikipedia (Hepburn Romanisation)} but only the orange and blue ones are taken, since the beige and purple ones are regarded as unofficial (by me).}  % caption
{tbl:katakana-mnemonics}  % label
{
    colspec={ccccl},
    rowhead=1,
    % width=\linewidth,  % useful only with X columns
}  % inner specification options
{
    \toprule
    \SetCell[c=2]{c,m} \textbf{片仮名} & & {\textbf{Hepburn}\\\textbf{romanisation}} & {\textbf{Keyboard}\\\textbf{command}} & \textbf{Mnemonic} \\
    \midrule
    ア & \emph{ア} & a & \texttt{a} & ``A'' shape \\
    イ & \emph{イ} & i & \texttt{i} & \ul{e}agle perched \\
    \color{blue} イェ & \color{blue} \emph{イェ} & \color{blue} ye & \color{blue} \texttt{ye} & \\
    ウ & \emph{ウ} & u & \texttt{u} & same shape as う \\
    \color{blue} ウィ & \color{blue} \emph{ウィ} & \color{blue} wi & \color{blue} \texttt{wi} & \\
    \color{blue} ウェ & \color{blue} \emph{ウェ} & \color{blue} we & \color{blue} \texttt{we} & \\
    \color{blue} ウォ & \color{blue} \emph{ウォ} & \color{blue} wo & \color{red} \texttt{uxo} & \\
    % U R COOKED... https://en.wikipedia.org/wiki/Hepburn_romanization#Extended_katakana
    \color{blue} ヴ & \color{blue} \emph{ヴ} & \color{blue} vu & \color{blue} \texttt{vu} & \\
    \color{blue} ヴァ & \color{blue} \emph{ヴァ} & \color{blue} va & \color{blue} \texttt{va} & \\
    \color{blue} ヴィ & \color{blue} \emph{ヴィ} & \color{blue} vi & \color{blue} \texttt{vi} & \\
    \color{blue} ヴュ & \color{blue} \emph{ヴュ} & \color{blue} vyu & \color{blue} \texttt{vyu} & \\
    \color{blue} ヴェ & \color{blue} \emph{ヴェ} & \color{blue} ve & \color{blue} \texttt{ve} & \\
    \color{blue} ヴォ & \color{blue} \emph{ヴォ} & \color{blue} vo & \color{blue} \texttt{vo} & \\
    エ & \emph{エ} & e & \texttt{e} & \ul{e}ngineer bar \\
    オ & \emph{オ} & o & \texttt{o} & \ul{o}pera talent (才) singing \\
    カ & \emph{カ} & ka & \texttt{ka} & same shape as か \\
    ガ & \emph{ガ} & ga & \texttt{ga} &  \\
    キ & \emph{キ} & ki & \texttt{ki} & same shape as き \\
    キャ & \emph{キャ} & kya & \texttt{kya} &  \\
    キュ & \emph{キュ} & kyu & \texttt{kyu} &  \\
    キョ & \emph{キョ} & kyo & \texttt{kyo} &  \\
    ギ & \emph{ギ} & gi & \texttt{gi} &  \\
    ギャ & \emph{ギャ} & gya & \texttt{gya} &  \\
    ギュ & \emph{ギュ} & gyu & \texttt{gyu} &  \\
    ギョ & \emph{ギョ} & gyo & \texttt{gyo} &  \\
    ク & \emph{ク} & ku & \texttt{ku} & \ul{coo}k's hat \\
    \color{blue} クァ & \color{blue} \emph{クァ} & \color{blue} kwa & \color{blue} \texttt{kwa} & \\
    \color{blue} クィ & \color{blue} \emph{クィ} & \color{blue} kwi & \color{blue} \texttt{kwi} & \\
    \color{blue} クェ & \color{blue} \emph{クェ} & \color{blue} kwe & \color{blue} \texttt{kwe} & \\
    \color{blue} クォ & \color{blue} \emph{クォ} & \color{blue} kwo & \color{blue} \texttt{kwo} & \\
    グ & \emph{グ} & gu & \texttt{gu} &  \\
    \color{blue} グァ & \color{blue} \emph{グァ} & \color{blue} gwa & \color{blue} \texttt{gwa} & \\
    ケ & \emph{ケ} & ke & \texttt{ke} & ``k'' shape \\
    ゲ & \emph{ゲ} & ge & \texttt{ge} &  \\
    コ & \emph{コ} & ko & \texttt{ko} & broken 口 (CN) / two \ul{co}rners \\
    ゴ & \emph{ゴ} & go & \texttt{go} &  \\
    サ & \emph{サ} & sa & \texttt{sa} & \ul{sa}rdines and \ul{sa}lmon (bigger) \\
    ザ & \emph{ザ} & za & \texttt{za} &  \\
    シ & \emph{シ} & shi & \textlightgrey{\texttt{si}/}\texttt{shi} & same direction as し \\
    シャ & \emph{シャ} & sha & \texttt{sha} &  \\
    シュ & \emph{シュ} & shu & \texttt{shu} &  \\
    \color{blue} シェ & \color{blue} \emph{シュ} & \color{blue} she & \color{blue} \texttt{she} &  \\
    ショ & \emph{ショ} & sho & \texttt{sho} &  \\
    ジ & \emph{ジ} & ji & \textlightgrey{\texttt{zi}/}\texttt{ji} &  \\
    ジャ & \emph{ジャ} & ja & \textlightgrey{\texttt{jya}/}\texttt{ja} &  \\
    ジュ & \emph{ジュ} & ju & \textlightgrey{\texttt{jyu}/}\texttt{ju} &  \\
    \color{blue} ジェ & \color{blue} \emph{ジェ} & \color{blue} je & \color{blue} \textlightgrey{\texttt{jye}/}\texttt{je} &  \\
    ジョ & \emph{ジョ} & jo & \textlightgrey{\texttt{jyo}/}\texttt{jo} &  \\
    ス & \emph{ス} & su & \texttt{su} & \ul{su}perman \\
    % \color{blue} スィ & \color{blue} \emph{スィ} & \color{blue} si & \color{red} \texttt{suxi} & \\
    ズ & \emph{ズ} & zu & \texttt{zu} &  \\
    % \color{blue} ズィ & \color{blue} \emph{ズィ} & \color{blue} zi & \color{red} \texttt{zuxi} & \\
    セ & \emph{セ} & se & \texttt{se} & same shape as せ \\
    ゼ & \emph{ゼ} & ze & \texttt{ze} &  \\
    ソ & \emph{ソ} & so & \texttt{so} & \ul{se}wing needles \\
    ゾ & \emph{ゾ} & zo & \texttt{zo} &  \\
    タ & \emph{タ} & ta & \texttt{ta} & \ul{ti}dal wave \\
    ダ & \emph{ダ} & da & \texttt{da} &  \\
    チ & \emph{チ} & chi & \textlightgrey{\texttt{ti}/}\texttt{chi} & \ul{chee}r / \ruby{千}{ち} \\
    チャ & \emph{チャ} & cha & \texttt{cha} &  \\
    チュ & \emph{チュ} & chu & \texttt{chu} &  \\
    \color{blue} チェ & \color{blue} \emph{チェ} & \color{blue} che & \color{blue} \texttt{che} &  \\
    チョ & \emph{チョ} & cho & \texttt{cho} &  \\
    ヂ & \emph{ヂ} & ji & \textred{\texttt{di}} &  \\
    ヂャ & \emph{ヂャ} & ja & \textred{\texttt{dya}} &  \\
    ヂュ & \emph{ヂュ} & ju & \textred{\texttt{dyu}} &  \\
    ヂョ & \emph{ヂョ} & jo & \textred{\texttt{dyo}} &  \\
    ツ & \emph{ツ} & tsu & \textlightgrey{\texttt{tu}/}\texttt{tsu} & same direction as つ \\
    \color{blue} ツァ & \color{blue} \emph{ツァ} & \color{blue} tsa & \color{blue} \texttt{tsa} & \emph{Italian ``z''}\\
    \color{blue} ツィ & \color{blue} \emph{ツィ} & \color{blue} tsi & \color{blue} \texttt{tsi} & \emph{Italian ``z''}\\
    \color{blue} ツェ & \color{blue} \emph{ツェ} & \color{blue} tse & \color{blue} \texttt{tse} & \emph{Italian ``z''}\\
    \color{blue} ツォ & \color{blue} \emph{ツォ} & \color{blue} tso & \color{blue} \texttt{tso} & \emph{Italian ``z''}\\
    ヅ & \emph{ヅ} & zu & \color{red} \texttt{du} &  \\
    テ & \emph{テ} & te & \texttt{te} & \ul{te}lephone pole \\
    \color{blue} ティ & \color{blue}\emph{ティ} & \color{blue} ti & \color{red} \texttt{texi} & \emph{``par\ul{ty}''}\\
    \color{blue} テュ & \color{blue}\emph{ティ} & \color{blue} tyu & \color{red} \texttt{texyu} & \\
    デ & \emph{デ} & de & \texttt{de} &  \\
    \color{blue} ディ & \color{blue}\emph{ディ} & \color{blue} di & \color{red} \texttt{dexi} & \emph{``can\ul{dy}''}\\
    \color{blue} デュ & \color{blue}\emph{デュ} & \color{blue} dyu & \color{red} \texttt{dexyu} & \\
    ト & \emph{ト} & to & \texttt{to} & \ul{to}tem pole \\
    \color{blue} トゥ & \color{blue}\emph{トゥ} & \color{blue} tu & \color{red} \texttt{toxu} & \emph{``two''}\\
    ド & \emph{ド} & do & \texttt{do} &  \\
    \color{blue} ドゥ & \color{blue}\emph{ドゥ} & \color{blue} du & \color{red} \texttt{dowu} & \emph{``dew''}\\
    ナ & \emph{ナ} & na & \texttt{na} & \ul{na}rwhal \\
    ニ & \emph{ニ} & ni & \texttt{ni} & same shape as に \\
    ニャ & \emph{ニャ} & nya & \texttt{nya} &  \\
    ニュ & \emph{ニュ} & nyu & \texttt{nyu} &  \\
    ニョ & \emph{ニョ} & nyo & \texttt{nyo} &  \\
    ヌ & \emph{ヌ} & nu & \texttt{nu} & \ul{noo}dles with chopsticks \\
    ネ & \emph{ネ} & ne & \texttt{ne} & \ul{ne}ckerchief \\
    ノ & \emph{ノ} & no & \texttt{no} & long \ul{no}se \\
    ハ & \emph{ハ} & ha & \textred{\texttt{ha}} & \ruby{八}{ハチ} / 八 (CN) \\
    バ & \emph{バ} & ba & \texttt{ba} &  \\
    パ & \emph{パ} & pa & \texttt{pa} &  \\
    ヒ & \emph{ヒ} & hi & \texttt{hi} & smile \ul{he}he \\
    ヒャ & \emph{ヒャ} & hya & \texttt{hya} &  \\
    ヒュ & \emph{ヒュ} & hyu & \texttt{hyu} &  \\
    ヒョ & \emph{ヒョ} & hyo & \texttt{hyo} &  \\
    ビ & \emph{ビ} & bi & \texttt{bi} &  \\
    ビャ & \emph{ビャ} & bya & \texttt{bya} &  \\
    ビュ & \emph{ビュ} & byu & \texttt{byu} &  \\
    ビョ & \emph{ビョ} & byo & \texttt{byo} &  \\
    ピ & \emph{ピ} & pi & \texttt{pi} &  \\
    ピャ & \emph{ピャ} & pya & \texttt{pya} &  \\
    ピュ & \emph{ピュ} & pyu & \texttt{pyu} &  \\
    ピョ & \emph{ピョ} & pyo & \texttt{pyo} &  \\
    フ & \emph{フ} & fu & \textlightgrey{\texttt{hu}/}\texttt{fu} & \ul{fl}ag \\
    \color{blue} ファ & \color{blue} \emph{ファ} & \color{blue} fa & \color{blue} \texttt{fa} & \\
    \color{blue} フィ & \color{blue} \emph{フィ} & \color{blue} fi & \color{blue} \texttt{fi} & \\
    \color{blue} フュ & \color{blue} \emph{フュ} & \color{blue} fyu & \color{blue} \texttt{fyu} & \\
    \color{blue} フェ & \color{blue} \emph{フェ} & \color{blue} fe & \color{blue} \texttt{fe} & \\
    \color{blue} フォ & \color{blue} \emph{フォ} & \color{blue} fo & \color{blue} \texttt{fo} & \\
    ブ & \emph{ブ} & bu & \texttt{bu} &  \\
    プ & \emph{プ} & pu & \texttt{pu} &  \\
    ヘ & \emph{ヘ} & he & \textred{\texttt{he}} & same shape as へ \\
    ベ & \emph{ベ} & be & \texttt{be} &  \\
    ペ & \emph{ペ} & pe & \texttt{pe} &  \\
    ホ & \emph{ホ} & ho & \texttt{ho} & \ul{ho}ly cross \\
    % \color{blue} ホゥ & \color{blue} \emph{ホゥ} & \color{blue} hu & \color{red} \texttt{hoxu} & \\
    ボ & \emph{ボ} & bo & \texttt{bo} &  \\
    ポ & \emph{ポ} & po & \texttt{po} &  \\
    マ & \emph{マ} & ma & \texttt{ma} & \ul{ma}th angles \\
    ミ & \emph{ミ} & mi & \texttt{mi} & \ul{mi}ssiles \\
    ミャ & \emph{ミャ} & mya & \texttt{mya} &  \\
    ミュ & \emph{ミュ} & myu & \texttt{myu} &  \\
    ミョ & \emph{ミョ} & myo & \texttt{myo} &  \\
    ム & \emph{ム} & mu & \texttt{mu} & cow face, says \ul{moo} \\
    メ & \emph{メ} & me & \texttt{me} & Arlecchino's eyes (め) \\
    モ & \emph{モ} & mo & \texttt{mo} & same shape as も \\
    ヤ & \emph{ヤ} & ya & \texttt{ya} & same shape as や \\
    ユ & \emph{ユ} & yu & \texttt{yu} & \ul{u}-turn \\
    ヨ & \emph{ヨ} & yo & \texttt{yo} & \ul{yo}ga pose \\
    ラ & \emph{ラ} & ra & \texttt{ra} & \ul{ra}ptor \\
    リ & \emph{リ} & ri & \texttt{ri} & reeds \\
    リャ & \emph{リャ} & rya & \texttt{rya} &  \\
    リュ & \emph{リュ} & ryu & \texttt{ryu} &  \\
    リョ & \emph{リョ} & ryo & \texttt{ryo} &  \\
    ル & \emph{ル} & ru & \texttt{ru} & tree \ul{roo}ts \\
    レ & \emph{レ} & re & \texttt{re} & \ul{re}d hair / right side of ル \\
    ロ & \emph{ロ} & ro & \texttt{ro} & cyclic \ul{ro}ad \\
    ワ & \emph{ワ} & wa & \texttt{wa} & \ul{wa}termelon slice \\
    ヲ & \emph{ヲ} & wo & \texttt{wo} & \ul{o}atmeal bowl \\
    ン & \emph{ン} & nn & \textred{\texttt{nn}} & N/A \\
    ァ & \emph{ァ} & ? & \textlightgrey{\texttt{la}/}\textred{\texttt{xa}} &  \\
    ィ & \emph{ィ} & ? & \textlightgrey{\texttt{li}/}\textred{\texttt{xi}} &  \\
    ゥ & \emph{ゥ} & ? & \textlightgrey{\texttt{lu}/}\textred{\texttt{xu}} &  \\
    ェ & \emph{ェ} & ? & \textlightgrey{\texttt{le}/}\textred{\texttt{xe}} &  \\
    ォ & \emph{ォ} & ? & \textlightgrey{\texttt{lo}/}\textred{\texttt{xo}} &  \\
    ャ & \emph{ャ} & ? & \textlightgrey{\texttt{lya}/}\textred{\texttt{xya}} &  \\
    ュ & \emph{ュ} & ? & \textlightgrey{\texttt{lyu}/}\textred{\texttt{xyu}} &  \\
    ョ & \emph{ョ} & ? & \textlightgrey{\texttt{lyo}/}\textred{\texttt{xyo}} &  \\
    ー & \emph{ー} & \texttt{<}$^{\texttt{+1}}$ & \textred{\texttt{$-$} key} &  \\
    ッ & \emph{ッ} & $^{\texttt{+1}}$\texttt{>} & \textlightgrey{\texttt{ltu}/\texttt{ltsu}/\textred{\texttt{xtsu}}/}repeat \texttt{>} &  \\ \bottomrule
}


\subsection{仮名 Summary}

% Help: \SetCell[r=2,c=2]{c,m} <content>, \cmidrule[l]{3-4}
% Help: colspec: X[ratio, horizontal alignment] columns grow to fit width=\linewidth
%                  negative ratios: shrink to fit content and may not grow to full ratio
% Help: colspec: l/c/r columns do not grow
\longtabse[1]  % scale factor
{仮名 summary table. $^\dagger$Particle romanisation applies only for 平仮名.}  % caption
{tbl:kana-summary}  % label
{
    colspec={cccccc},
    rowhead=1,
    % width=\linewidth,  % useful only with X columns
}  % inner specification options
{
    \toprule
    \SetCell[c=2]{c,m} \textbf{平仮名} & & \SetCell[c=2]{c,m} \textbf{片仮名} & &  {\textbf{Hepburn}\\\textbf{romanisation}} & {\textbf{Keyboard}\\\textbf{command}} & \textbf{Mnemonic} \\
    \midrule
    あ & \emph{あ} & ア & \emph{ア} & a & \texttt{a} \\
    い & \emph{い} & イ & \emph{イ} & i & \texttt{i}\\
    & & イェ & \emph{イェ} & ye & \texttt{ye} \\
    う & \emph{う} & ウ & \emph{ウ} & u & \texttt{u} \\
    & & ウィ & \emph{ウィ} & wi & \texttt{wi} \\
    & & ウェ & \emph{ウェ} & we & \texttt{we} \\
    & & ウォ & \emph{ウォ} & wo & \color{red} \texttt{uxo} \\
    & & ヴ & \emph{ヴ} & vu & \texttt{vu} \\
    & & ヴァ & \emph{ヴァ} & va & \texttt{va} \\
    & & ヴィ & \emph{ヴィ} & vi & \texttt{vi} \\
    & & ヴュ & \emph{ヴュ} & vyu & \texttt{vyu} \\
    & & ヴェ & \emph{ヴェ} & ve & \texttt{ve} \\
    & & ヴォ & \emph{ヴォ} & vo & \texttt{vo} \\
    え & \emph{え} & エ & \emph{エ} & e & \texttt{e} \\
    お & \emph{お} & オ & \emph{オ} & o & \texttt{o} \\
    か & \emph{か} & カ & \emph{カ} & ka & \texttt{ka} \\
    が & \emph{が} & ガ & \emph{ガ} & ga & \texttt{ga} \\
    き & \emph{き} & キ & \emph{キ} & ki & \texttt{ki} \\
    きゃ & \emph{きゃ} & キャ & \emph{キャ} & kya & \texttt{kya} \\
    きゅ & \emph{きゅ} & キュ & \emph{キュ} & kyu & \texttt{kyu} \\
    きょ & \emph{きょ} & キョ & \emph{キョ} & kyo & \texttt{kyo} \\
    ぎ & \emph{ぎ} & ギ & \emph{ギ} & gi & \texttt{gi} \\
    ぎゃ & \emph{ぎゃ} & ギャ & \emph{ギャ} & gya & \texttt{gya} \\
    ぎゅ & \emph{ぎゅ} & ギュ & \emph{ギュ} & gyu & \texttt{gyu} \\
    ぎょ & \emph{ぎょ} & ギョ & \emph{ギョ} & gyo & \texttt{gyo} \\
    く & \emph{く} & ク & \emph{ク} & ku & \texttt{ku} \\
    & & クァ & \emph{クァ} & kwa & \texttt{kwa} \\
    & & クィ & \emph{クィ} & kwi & \texttt{kwi} \\
    & & クェ & \emph{クェ} & kwe & \texttt{kwe} \\
    & & クォ & \emph{クォ} & kwo & \texttt{kwo} \\
    ぐ & \emph{ぐ} & グ & \emph{グ} & gu & \texttt{gu} \\
    & & グァ & \emph{グァ} & gwa & \texttt{gwa} \\
    け & \emph{け} & ケ & \emph{ケ} & ke & \texttt{ke} \\
    げ & \emph{げ} & ゲ & \emph{ゲ} & ge & \texttt{ge} \\
    こ & \emph{こ} & コ & \emph{コ} & ko & \texttt{ko} \\
    ご & \emph{ご} & ゴ & \emph{ゴ} & go & \texttt{go} \\
    さ & \emph{さ} & サ & \emph{サ} & sa & \texttt{sa} \\
    ざ & \emph{ざ} & ザ & \emph{ザ} & za & \texttt{za} \\
    し & \emph{し} & シ & \emph{シ} & shi & \textlightgrey{\texttt{si}/}\texttt{shi} \\
    しゃ & \emph{しゃ} & シャ & \emph{シャ} & sha & \texttt{sha} \\
    しゅ & \emph{しゅ} & シュ & \emph{シュ} & shu & \texttt{shu} \\
    & & シェ & \emph{シュ} & she & \texttt{she} \\
    しょ & \emph{しょ} & ショ & \emph{ショ} & sho & \texttt{sho} \\
    じ & \emph{じ} & ジ & \emph{ジ} & ji & \textlightgrey{\texttt{zi}/}\texttt{ji} \\
    じゃ & \emph{じゃ} & ジャ & \emph{ジャ} & ja & \textlightgrey{\texttt{jya}/}\texttt{ja} \\
    じゅ & \emph{じゅ} & ジュ & \emph{ジュ} & ju & \textlightgrey{\texttt{jyu}/}\texttt{ju} \\
    & & ジェ & \emph{ジェ} & je & \textlightgrey{\texttt{jye}/}\texttt{je} \\
    じょ & \emph{じょ} & ジョ & \emph{ジョ} & jo & \textlightgrey{\texttt{jyo}/}\texttt{jo} \\
    す & \emph{す} & ス & \emph{ス} & su & \texttt{su} \\
    % & & スィ & \emph{スィ} & si & \color{red} \texttt{suxi} \\
    ず & \emph{ず} & ズ & \emph{ズ} & zu & \texttt{zu} \\
    % & & ズィ & \emph{ズィ} & zi & \color{red} \texttt{zuxi} \\
    せ & \emph{せ} & セ & \emph{セ} & se & \texttt{se} \\
    ぜ & \emph{ぜ} & ゼ & \emph{ゼ} & ze & \texttt{ze} \\
    そ & \emph{そ} & ソ & \emph{ソ} & so & \texttt{so} \\
    ぞ & \emph{ぞ} & ゾ & \emph{ゾ} & zo & \texttt{zo} \\
    た & \emph{た} & タ & \emph{タ} & ta & \texttt{ta} \\
    だ & \emph{だ} & ダ & \emph{ダ} & da & \texttt{da} \\
    ち & \emph{ち} & チ & \emph{チ} & chi & \textlightgrey{\texttt{ti}/}\texttt{chi} \\
    ちゃ & \emph{ちゃ} & チャ & \emph{チャ} & cha & \texttt{cha} \\
    ちゅ & \emph{ちゅ} & チュ & \emph{チュ} & chu & \texttt{chu} \\
    & & チェ & \emph{チェ} & che & \texttt{che} \\
    ちょ & \emph{ちょ} & チョ & \emph{チョ} & cho & \texttt{cho} \\
    ぢ & \emph{ぢ} & ヂ & \emph{ヂ} & ji & \textred{\texttt{di}} \\
    ぢゃ & \emph{ぢゃ} & ヂャ & \emph{ヂャ} & ja & \textred{\texttt{dya}} \\
    ぢゅ & \emph{ぢゅ} & ヂュ & \emph{ヂュ} & ju & \textred{\texttt{dyu}} \\
    ぢょ & \emph{ぢょ} & ヂョ & \emph{ヂョ} & jo & \textred{\texttt{dyo}} \\
    つ & \emph{つ} & ツ & \emph{ツ} & tsu & \textlightgrey{\texttt{tu}/}\texttt{tsu} \\
    & & ツァ & \emph{ツァ} & tsa & \texttt{tsa} \\
    & & ツィ & \emph{ツィ} & tsi & \texttt{tsi} \\
    & & ツェ & \emph{ツェ} & tse & \texttt{tse} \\
    & & ツォ & \emph{ツォ} & tso & \texttt{tso} \\
    づ & \emph{づ} & ヅ & \emph{ヅ} & zu & \color{red} \texttt{du} \\
    て & \emph{て} & テ & \emph{テ} & te & \texttt{te} \\
    & & ティ &\emph{ティ} & ti & \color{red} \texttt{texi} \\
    & & テュ &\emph{ティ} & tyu & \color{red} \texttt{texyu} \\
    で & \emph{で} & デ & \emph{デ} & de & \texttt{de} \\
    & & ディ &\emph{ディ} & di & \color{red} \texttt{dexi} \\
    & & デュ &\emph{デュ} & dyu & \color{red} \texttt{dexyu} \\
    と & \emph{と} & ト & \emph{ト} & to & \texttt{to} \\
    & & トゥ &\emph{トゥ} & tu & \color{red} \texttt{toxu} \\
    ど & \emph{ど} & ド & \emph{ド} & do & \texttt{do} \\
    & & ドゥ &\emph{ドゥ} & du & \color{red} \texttt{dowu} \\
    な & \emph{な} & ナ & \emph{ナ} & na & \texttt{na} \\
    に & \emph{に} & ニ & \emph{ニ} & ni & \texttt{ni} \\
    にゃ & \emph{にゃ} & ニャ & \emph{ニャ} & nya & \texttt{nya} \\
    にゅ & \emph{にゅ} & ニュ & \emph{ニュ} & nyu & \texttt{nyu} \\
    にょ & \emph{にょ} & ニョ & \emph{ニョ} & nyo & \texttt{nyo} \\
    ぬ & \emph{ぬ} & ヌ & \emph{ヌ} & nu & \texttt{nu} \\
    ね & \emph{ね} & ネ & \emph{ネ} & ne & \texttt{ne} \\
    の & \emph{の} & ノ & \emph{ノ} & no & \texttt{no} \\
    は & \emph{は} & ハ & \emph{ハ} & ha & \textred{\texttt{ha}} \\
    ば & \emph{ば} & バ & \emph{バ} & ba & \texttt{ba} \\
    ぱ & \emph{ぱ} & パ & \emph{パ} & pa & \texttt{pa} \\
    ひ & \emph{ひ} & ヒ & \emph{ヒ} & hi & \texttt{hi} \\
    ひゃ & \emph{ひゃ} & ヒャ & \emph{ヒャ} & hya & \texttt{hya} \\
    ひゅ & \emph{ひゅ} & ヒュ & \emph{ヒュ} & hyu & \texttt{hyu} \\
    ひょ & \emph{ひょ} & ヒョ & \emph{ヒョ} & hyo & \texttt{hyo} \\
    び & \emph{び} & ビ & \emph{ビ} & bi & \texttt{bi} \\
    びゃ & \emph{びゃ} & ビャ & \emph{ビャ} & bya & \texttt{bya} \\
    びゅ & \emph{びゅ} & ビュ & \emph{ビュ} & byu & \texttt{byu} \\
    びょ & \emph{びょ} & ビョ & \emph{ビョ} & byo & \texttt{byo} \\
    ぴ & \emph{ぴ} & ピ & \emph{ピ} & pi & \texttt{pi} \\
    ぴゃ & \emph{ぴゃ} & ピャ & \emph{ピャ} & pya & \texttt{pya} \\
    ぴゅ & \emph{ぴゅ} & ピュ & \emph{ピュ} & pyu & \texttt{pyu} \\
    ぴょ & \emph{ぴょ} & ピョ & \emph{ピョ} & pyo & \texttt{pyo} \\
    ふ & \emph{ふ} & フ & \emph{フ} & fu & \textlightgrey{\texttt{hu}/}\texttt{fu} \\
    & & ファ & \emph{ファ} & fa & \texttt{fa} \\
    & & フィ & \emph{フィ} & fi & \texttt{fi} \\
    & & フュ & \emph{フュ} & fyu & \texttt{fyu} \\
    & & フェ & \emph{フェ} & fe & \texttt{fe} \\
    & & フォ & \emph{フォ} & fo & \texttt{fo} \\
    ぶ & \emph{ぶ} & ブ & \emph{ブ} & bu & \texttt{bu} \\
    ぷ & \emph{ぷ} & プ & \emph{プ} & pu & \texttt{pu} \\
    へ & \emph{へ} & ヘ & \emph{ヘ} & he & \textred{\texttt{he}} \\
    べ & \emph{べ} & ベ & \emph{ベ} & be & \texttt{be} \\
    ぺ & \emph{ぺ} & ペ & \emph{ペ} & pe & \texttt{pe} \\
    ほ & \emph{ほ} & ホ & \emph{ホ} & ho & \texttt{ho} \\
    % & & ホゥ & \emph{ホゥ} & hu & \color{red} \texttt{hoxu} \\
    ぼ & \emph{ぼ} & ボ & \emph{ボ} & bo & \texttt{bo} \\
    ぽ & \emph{ぽ} & ポ & \emph{ポ} & po & \texttt{po} \\
    ま & \emph{ま} & マ & \emph{マ} & ma & \texttt{ma} \\
    み & \emph{み} & ミ & \emph{ミ} & mi & \texttt{mi} \\
    みゃ & \emph{みゃ} & ミャ & \emph{ミャ} & mya & \texttt{mya} \\
    みゅ & \emph{みゅ} & ミュ & \emph{ミュ} & myu & \texttt{myu} \\
    みょ & \emph{みょ} & ミョ & \emph{ミョ} & myo & \texttt{myo} \\
    む & \emph{む} & ム & \emph{ム} & mu & \texttt{mu} \\
    め & \emph{め} & メ & \emph{メ} & me & \texttt{me} \\
    も & \emph{も} & モ & \emph{モ} & mo & \texttt{mo} \\
    や & \emph{や} & ヤ & \emph{ヤ} & ya & \texttt{ya} \\
    ゆ & \emph{ゆ} & ユ & \emph{ユ} & yu & \texttt{yu} \\
    よ & \emph{よ} & ヨ & \emph{ヨ} & yo & \texttt{yo} \\
    ら & \emph{ら} & ラ & \emph{ラ} & ra & \texttt{ra} \\
    り & \emph{り} & リ & \emph{リ} & ri & \texttt{ri} \\
    りゃ & \emph{りゃ} & リャ & \emph{リャ} & rya & \texttt{rya} \\
    りゅ & \emph{りゅ} & リュ & \emph{リュ} & ryu & \texttt{ryu} \\
    りょ & \emph{りょ} & リョ & \emph{リョ} & ryo & \texttt{ryo} \\
    る & \emph{る} & ル & \emph{ル} & ru & \texttt{ru} \\
    れ & \emph{れ} & レ & \emph{レ} & re & \texttt{re} \\
    ろ & \emph{ろ} & ロ & \emph{ロ} & ro & \texttt{ro} \\
    わ & \emph{わ} & ワ & \emph{ワ} & wa & \texttt{wa} \\
    を & \emph{を} & ヲ & \emph{ヲ} & wo & \texttt{wo} \\
    ん & \emph{ん} & ン & \emph{ン} & nn & \textred{\texttt{nn}} \\
    ぁ & \emph{ぁ} & ァ & \emph{ァ} & ? & \textlightgrey{\texttt{la}/}\textred{\texttt{xa}} \\
    ぃ & \emph{ぃ} & ィ & \emph{ィ} & ? & \textlightgrey{\texttt{li}/}\textred{\texttt{xi}} \\
    ぅ & \emph{ぅ} & ゥ & \emph{ゥ} & ? & \textlightgrey{\texttt{lu}/}\textred{\texttt{xu}} \\
    ぇ & \emph{ぇ} & ェ & \emph{ェ} & ? & \textlightgrey{\texttt{le}/}\textred{\texttt{xe}} \\
    ぉ & \emph{ぉ} & ォ & \emph{ォ} & ? & \textlightgrey{\texttt{lo}/}\textred{\texttt{xo}} \\
    ゃ & \emph{ゃ} & ャ & \emph{ャ} & ? & \textlightgrey{\texttt{lya}/}\textred{\texttt{xya}} \\
    ゅ & \emph{ゅ} & ュ & \emph{ュ} & ? & \textlightgrey{\texttt{lyu}/}\textred{\texttt{xyu}} \\
    ょ & \emph{ょ} & ョ & \emph{ョ} & ? & \textlightgrey{\texttt{lyo}/}\textred{\texttt{xyo}} \\
    & & ー & \emph{ー} & \texttt{<}$^{\texttt{+1}}$ & \textred{\texttt{$-$} key} \\
    っ & \emph{っ} & ッ & \emph{ッ} & $^{\texttt{+1}}$\texttt{>} & \textlightgrey{\texttt{ltu}/\texttt{ltsu}/\textred{\texttt{xtsu}}/}repeat \texttt{>} \\
    \bottomrule
}


\subsection{[Interlude] Morphemes, phonemes, phones}
\emph{This entire section is courtesy of SL.}

\textbf{Phonemes} are the smallest unit of mental representation of sound. They do not carry meaning by themselves, but they can alter the meaning pictured by the listener.

\textbf{Morphemes} are the smallest unit of meaning, and comprise two levels: a phonological level and a semantic level. The phonological level states how it is pronounced (a string of phonemes), and the semantic level states what meaning is attached to the phonology.

For instance, \ruby{日々}{ひ|び} contains a repetition of the same phoneme because the sound and meaning of the two 漢字 are identical. In contrast, \ruby{日日}{ひ|にち} contains two different morphemes, because the sound (and meaning) of the two 漢字 are different!

When determining whether a morpheme is repeated or not, consider the sound and meaning first before looking at the orthography. ``Orthography is truly an afterthought[, in the design of languages].''


\subsection{Iteration marks}
\emph{Read main article on \href{https://en.wikipedia.org/wiki/Iteration_mark\#Japanese}{Wikipedia}.}

Only the (horizontal text) 漢字 iteration mark 々 is commonly used today. It is used to represent a \ul{duplicated character representing the same morpheme}. For example, \ruby{日々}{ひ|び} means ``daily, day after day''.

Writing 々 instead of repeating the 漢字 is preferred, provided that:
\begin{enumerate}[label=\arabic*.]
    \item (tl;dr: morpheme is repeated) the reading of the repeated 漢字 must be the same, though certain changes are permitted such as \emph{rendaku} (unvoiced consonant becomes voiced, i.e.\ the dakuten, e.g.\ in \ruby{人々}{ひと|びと}, ひ $\to$ び) and \emph{gemination} (consonant lengthening, i.e.\ the っ, e.g.\ in \ruby{刻々}{こっ|こく}), and
    \item the repetition must be within a single word/phrase.
\end{enumerate}
If the above aren't satisfied:
\begin{itemize}
    \item If repetition isn't repetition of the same morpheme, for disambiguation the second 漢字 is spelt out in 平仮名 (e.g.\ 日にち).
    \item If repetition crosses word boundaries, then the 漢字 is repeated (e.g.\ \ruby{民主主義}{みん|しゅ|しゅ|ぎ}, democracy).

    There are exceptions to this! 民主々義 is rarely used but exists. A notable exception is in the signages for neighbourhood associations 「〜\ruby{町内会}{ちょう|ない|かい}」. Because the name of neighbourhoods often end in 〜\ruby{町}{ちょう}, suffixing with 〜町内会 yields 〜\ruby{町町内会}{ちょう|ちょう|ない|かい}, which is then informally abbreviated to 〜町々内会, despite the repetition crossing a word boundary.

\end{itemize}

Intrepretations when 々 is used:
\begin{itemize}
    \item Reduplication (linguistics terminology) to indicate plurality

    \ruby{人々}{ひと|びと} (people)、\ruby{日々}{ひ|び} (daily/day after day)、\ruby{山々}{やま|やま} (mountains)
    \item Various alterations in meaning
    \begin{itemize}
        \item \ruby{個}{こ} (piece) $\to$ \ruby{個々}{こ|こ} (individually)
        \item \ruby{時}{とき} (time) $\to$ \ruby{時々}{とき|どき} (sometimes)
        \item \ruby{翌日}{よく|じつ} (next day, as in 隔天/隔一天 (CN)) $\to$ \ruby{翌々日}{よく|よく|じつ} (next next day, as in 隔两天 (CN))

        \emph{Note that 翌日 is not the same as \ruby[g]{明日}{あした}, just like how 隔天 and 明天 are used in different contexts in CN!}
    \end{itemize}
\end{itemize}

Repetition marks can be typed using commands in Table~\ref{tbl:miscellaneous-keyboard-commands}.

% Help: \SetCell[r=2,c=2]{c,m} <content>, \cmidrule[l]{3-4}
% Help: colspec: X[ratio, horizontal alignment] columns grow to fit width=\linewidth
%                  negative ratios: shrink to fit content and may not grow to full ratio
% Help: colspec: l/c/r columns do not grow
\longtabse[1]  % scale factor
{Miscellaneous keyboard commands. Today, ゝ, ゞ, ヽ and ヾ only appear in proper names. As examples, じゝ $=$ じし and じゞ $=$ じじ.}  % caption
{tbl:miscellaneous-keyboard-commands}  % label
{
    colspec={cccl},
    rowhead=1,
    % width=\linewidth,  % useful only with X columns
}  % inner specification options
{
    \toprule
    & {Keyboard\\command} & Purpose \\
    \midrule
    ゝ & \emph{ゝ} & \texttt{onaji} $\to$ space$^\texttt{*}$ & 平仮名 previous character repeater (enforce without dakuten) \\
    ゞ & \emph{ゞ} & \texttt{onaji} $\to$ space$^\texttt{*}$ & 平仮名 previous character repeater (enforce with dakuten)\\
    ヽ & \emph{ヽ} & \texttt{onaji} $\to$ space$^\texttt{*}$ & 片仮名 previous character repeater (enforce without dakuten)\\
    ヾ & \emph{ヾ} & \texttt{onaji} $\to$ space$^\texttt{*}$ & 片仮名 previous character repeater (enforce with dakuten)\\
    々 & \emph{々} & \texttt{noma} $\to$ space$^\texttt{*}$ & 漢字 previous character repeater (ノ$+$マ) \\
    % & & &  \\
    \bottomrule
}


\subsection{漢字}
Some preliminary notes:
\begin{itemize}
    \item There exists over $\SI{40000}{}$ 漢字 but only about $\SI{2000}{}$ account for $>95\%$ of characters actually used in written text.
    \item There are no spaces in Japanese, so 漢字 is necessary for distinguishing separate words within a sentence, and discriminating between homophones.
    \item Words that mean practically the same thing can have different 漢字 to distinguish nuances.

    Here's an example:
    \begin{itemize}
        \item \ruby{訊}{き}く means to ask.
        \item \ruby{聞}{き}く means to listen, or to ask.
        \item \ruby{聴}{き}く means to listen attentively. Preferred when talking about listening to music.
    \end{itemize}

    Another example:
    \begin{itemize}
        \item \ruby{見}{み}る means to see.
        \item \ruby{観}{み}る means to watch a movie.
    \end{itemize}

    Another example:
    \begin{itemize}
        \item \ruby{書}{か}く means to write.
        \item \ruby{描}{か}く means to draw.

        When depicting/imagining an \ul{abstract} image (e.g.\ a scene in a book ), we use \ruby{描}{えが}く.
    \end{itemize}

    Another example:
    \begin{itemize}
        \item The different pronuncations \ruby[g]{今日}{きょう}, \ruby{今日}{こん|にち} and \ruby{今日}{こん|じつ} are each preferred in different contexts.
    \end{itemize}
\end{itemize}


\subsection{Pronunciation} \label{sec:pronunciation}
It is not practical to memorise or attempt to logically create rules for pitches, especially since it can change depending on the context or the dialect. Even the intonations provided in dictionaries are there for guidance; they morph when used in different contexts.

The only practical approach is to get the general sense of pitches is by mimicking native Japanese speakers with careful listening and practice.

Some special notes:
\begin{itemize}
    \item Voiced consonants vibrate the vocal cords, while unvoiced consonants don't (see \href{https://www.tofugu.com/japanese/japanese-pronunciation/\#vocal-cords}{Tofugu} article).
    \item In the modern 東京 dialect, ず and づ are pronounced exactly the same way: ``zu'', as expressed in their identical Hepburn romanisation (\S\ref{sec:hepburn-romanisation}).
    \item When in the middle of words, がぎぐげご may be pronounced with a ``ng-'' start instead of a ``g-'' start. This is a regional variation (that's not too uncommon); both ways are acceptable (see \href{https://www.tofugu.com/japanese/japanese-pronunciation/\#nasal-}{Tofugu} article).
    \item The native Japanese speaker will pronounce the ``v'' family (ヴ、ヴァ、ヴィ、ヴェ、ヴォ、ヴュ) as /b/.
    \item Vowel extensions (\S\ref{sec:vowel-extension}) are pronounced as vowel extensions; do not pronounce the extender if it's a different vowel! For example, \ruby{先生}{せん|せい} is pronounced \emph{sen-se} with an elongated trailing ``e'' vowel. There is no ``i'' vowel sound!
    \item Almost every 漢字 character has two different readings (see \S\ref{sec:on-yomi-and-kun-yomi}):
    \begin{itemize}
        \item \ruby{音読}{おん|よ}み: Chinese-derived. Used in compound 漢字 and idioms (both known as \ruby{熟語}{じゅく|ご}).
        \item \ruby{訓読}{くん|よ}み: native Japanese. Used in solo 漢字, solo 漢字 appended with \ruby{送}{おく}り仮名, adjectives and verbs.

        The purpose of trailing 送り仮名 is to preserve the pronunciation of the 漢字, even as the word is conjugated\footnote{\textbf{Conjugation}: change of word form to fit a given context.} to different forms. It is also used to differentiate transitive and intransitive verbs (\S\ref{sec:transitive-intransitive-verbs}).
    \end{itemize}

    Note that although sometimes you may see 音読み pronunciations written in 片仮名 and 訓読み pronunciations written in 平仮名, this is only used in dictionaries for differentiation. In standard 振り仮名, only 平仮名 is used.

    \item The actual readings of 漢字 can change slightly in compound words to make them easier to say (e.g.\ 一本 is いっぽん instead of いっほん).

    When repeating 漢字 using 々, \emph{rendaku} (see \S\ref{sec:rendaku}; unvoiced consonant becomes voiced, i.e.\ the dakuten, e.g.\ in 人々, ひ $\to$ び) and \emph{gemination} (consonant lengthening, i.e.\ the っ, e.g.\ \ruby{刻々}{こっ|こく}) \emph{may} occur.
\end{itemize}


\subsubsection{Vowel extension} \label{sec:vowel-extension}
Vowel extensions follow the rules in Table~\ref{tbl:vowel-extension}. For notes on pronunciation, see Section~\ref{sec:pronunciation}.

% Help: \SetCell[r=2,c=2]{c,m} <content>, \cmidrule[l]{3-4}
% Help: colspec: X[ratio, horizontal alignment] columns grow to fit width=\linewidth
%                  negative ratios: shrink to fit content and may not grow to full ratio
% Help: colspec: l/c/r columns do not grow
\longtabse[1]  % scale factor
{Vowel extension rules. Exceptions are bracketed in \textblue{blue}. /a/ is the phoneme representation.}  % caption
{tbl:vowel-extension}  % label
{
    colspec={cccl},
    rowhead=1,
    % width=\linewidth,  % useful only with X columns
}  % inner specification options
{
    \toprule
    \SetCell[r=2]{c,m} \textbf{Vowel to extend} & \SetCell[c=2]{c,m} \textbf{Extend by appending} & & \SetCell[r=2]{c,m} \textbf{Example} \\* \cmidrule[lr]{2-3}
    & \textbf{平仮名} & \textbf{片仮名} & \\
    \midrule
    /a/ & あ & \SetCell[r=5]{c,m} ー & お\underline{ばあ}さん、お\underline{かあ}さん \\*
    /i/ & い & & \\*
    /u/ & う & & \\*
    /e/ & \textred{い} \textblue{(え)} & & \ruby{先生}{せん|せい}、\ruby{学生}{がく|せい}、\textblue{(お\underline{ねえ}さん)} \\*
    /o/ & \textred{う} \textblue{(お)} & & き\underline{ょう}、おは\underline{よう}、\textblue{(\underline{おお}きい)}、\textblue{(\ruby{遠回}{\ul{とお}|まわ}り)} \\
    \bottomrule
}


\subsection{Hepburn romanisation} \label{sec:hepburn-romanisation}
\emph{Read main article on \href{https://en.wikipedia.org/wiki/Hepburn_romanization}{Wikipedia}.}

The official (as of Jan 2024) romanisation system of Japan. There are only a few rules.
\begin{description}
    \item[Vowel extension (\S\ref{sec:vowel-extension})] When vowels ``a'', ``e'', ``o'', ``u'' are extended \ul{as part of the same morpheme}, it is expressed with a macron (overbar), and the extender vowel is dropped. \textred{Extension of ``i'' and the ``e+i'' combination are exceptions: they remain repeated.}

    \begin{itemize}
        \item お\ruby{婆}{ばあ}さん obaasan $\to$ ob\=asan
        \item \textred{新潟 (city name) niigata}
        \item \ruby{数学}{すう|がく} suugaku $\to$ s\=ugaku
        \item お\ruby{姉}{ねえ}さん oneesan $\to$ on\=esan\\
        \textred{先生 sensei}
        \item \ruby{遠回}{とお|まわ}り toomawari $\to$ t\=omawari\\
        勉強 benkyou $\to$ benky\=o
    \end{itemize}

    This does not apply when the repetition crosses word boundaries or morpheme boundaries.

    \begin{itemize}
        \item \ruby{邪悪}{じゃ|あく} jaaku
        \item \ruby{灰色}{はい|いろ} haiiro

        Also for terminal adjectives (\hl{???}): いい ii
        \item \ruby{湖}{みずうみ} mizuumi

        Also for terminal verbs (\hl{???}): \ruby{食}{く}う kuu (eat)
        \item \ruby{濡}{ぬ}れ\ruby{縁}{えん} nureen (``open veranda (roofed hallway)'')
        \item \ruby{小躍}{こ|おど}り koodori (dance of joy)

        \ruby{仔牛}{こ|うし} koushi (calf)

        Also for terminal verbs (\hl{???}): \ruby{迷う}{まよ|う} mayou (to get lost)
    \end{itemize}
    \item[片仮名 loanwords] The macron is used iff 「ー」 is used to extend a vowel.
    \item[Japanese words adopted into English] Common place names like Tokyo, Kyoto and Osaka, while properly romanised as t\=oky\=o, ky\=oto and \=osaka, are simply romanised as Tokyo, Kyoto and Osaka.
    \item[Particles] When は、へ、を are used as particles, they are romanised as wa, e and o respectively.
    \item[Syllabic ん] ん is romanised as n' (with the apostrophe) if appearing immediately before any lone vowels or ``y''. This is to disambiguate んあ、んい、んう、んえ、んお、んや、んゆ、んよ (n'a, n'i, n'u, n'e, n'o, n'ya, n'yu, n'yo) from な、に、ぬ、ね、の、にゃ、にゅ、にょ (na, ni, nu, ne, no, nya, nyu, nyo) respectively.

    Examples: \ruby{簡易}{かん|に} kan'i (simple), \ruby{信用}{しん|よう} shin'y\=o (trust).
    \item[Geminated consonants (っ、ッ)] Double the next consonant, except if ``ch'' is repeated: in that case we use ``tch'' instead of ``cch''.

    Examples: \ruby{抹茶}{まっ|ちゃ} maccha $\to$ matcha, こっち kocchi $\to$ kotchi
\end{description}


\subsection{Rendaku} \label{sec:rendaku}
\emph{Read the main articles on \href{https://www.tofugu.com/japanese/rendaku/}{Tofugu}.}

Rendaku occurs when multiple words join together to form one \textbf{compound word}, and the initially unvoiced consonant of a second word becomes a voiced consonant. In written form, the second word's first syllable gains a dakuten/handakuten mark. All dakuten/handakuten-marked characters have voiced consonant beginnings (the ``p'' sound for handakuten-marked characters are considered to be ``semi-voiced'').

Compound words comprise words that can independently exist as words on their own.

Here are some general rendaku rules, though note that exceptions exist:
\begin{itemize}
    \item Basic conditions:
    \begin{enumerate}
        \item Two words come together to form a compound word.
        \item The leading consonant of the second word is unvoiced.
        \item The leading consonant of the second word is one of the four sets of characters that can change into a voiced consonant with dakuten or handakuten (``k'', ``s'', ``t'', or ``h'').
        \item Surrounding the leading consonant of the second word are voiced vowels (or sometimes nasals like ん: those do not stop rendaku).
    \end{enumerate}
    \item If the first word ends in つ or ん, the ``h'' leading consonant of the second word usually rendakus to ``p'', ``b'' otherwise by default.

    E.g.\ \ruby[(-)]{出}{しゅつ} $+$ \ruby[(-)]{発}{はつ} $=$ \ruby[(-)]{出発}{しゅっ|ぱつ}.

    E.g.\ \ruby[(-)]{鉛}{えん} $+$ \ruby[(-)]{筆}{ひつ} $=$ \ruby[(-)]{鉛筆}{えん|ぴつ}.
    \item When the second word is of Japanese origin (\ruby{和語}{わ|ご}), and the basic conditions above are met, then it undergoes rendaku. When the second word is of Chinese origin or a foreign loanword (\ruby{漢語}{かん|ご}、\ruby{外来語}{がい|らい|ご}), rendaku is \emph{usually} prevented (unless the 漢語 is \emph{vulgarised}, meaning it's become so common that it's treated as a 和語 word, which doesn't prevent rendaku).

    This is for avoidance of ambiguity in spoken language: 和語 words mostly start with unvoiced consonants, so rendaku makes it clear that a compound word is used instead of two independent words. 漢語 and 外来語 words can and often start with voiced consonants, so rendaku would cause confusion by morphing rendaku-ed words into another different word.

    A vulgarised exception is \ruby[(-)]{夫婦}{ふう|ふ} $+$ \ruby[(-)]{喧嘩}{けん|か} $=$ \ruby{夫婦喧嘩}{ふう|ふ|げん|か} (\ruby{喧嘩}{けん|か} is vulgarised). Other vulgarised words: \ruby{会社}{かい|しゃ}、\ruby{写真}{しゃ|しん}.

    \item When the first word is a 漢語 (in 漢語 $+$ 和語 compound words, where the first element uses the 音読み reading and the second element uses the 訓読み reading), rendaku can be blocked. A notable exmaple is 「\ruby{大}{だい}〜」.

    In \ruby{熟語}{じゅく|ご} compound words where both constituent words use 音読み, rendaku does not occur in the grand majority of cases.

    \item Lyman's Law: If the second word has a voiced consonant or handakuten anywhere in it, rendaku does not occur.

    This may be explained by the observation that two voiced consonants don't appear together side by side in single 和語 words or phrases.

    ``Lyman's Law in reverse''` says that sometimes, when the second word has a second voiced consonant, that can become unvoiced and the first consonant becomes voiced (rendaku). This is a rarity and won't be discussed further here.
    \item If voicing in the first word is too close to the second word, rendaku may (50/50) not occur. ``Japanese doesn't really like having a bunch of dakuten and handakuten very close to each other.''
    \item In words that come together to mean "X and Y," rendaku does not occur. In English, such words are rare but examples include ``bittersweet'', ``stir-fry'' and ``sleepwalk''.

    E.g.\ 山川 can mean either ``mountains and rivers'' or ``a mountain river''. For the former, when both sides are ``equal status'', rendaku does not happen and it is read \ruby{山川}{やま|かわ}. For the latter, when the first word is a noun modifier for the second, rendaku happens and it is read \ruby{山川}{やま|がわ}.

    E.g.\ \ruby{白黒}{しろ|くろ} does not rendaku (white and black are equals), but \ruby{色白}{いろ|じろ} and \ruby{色黒}{いろ|ぐろ} do (colour is a descriptor of white/black).
    \item Repeating onomatopoeia do not rendaku.

    E.g.\ キラキラ as the sound of something sparkling/glittering, does not rendaku.
    \item Certain prefixes block rendaku and certain suffixes resist rendaku.

    Blocking prefixes:
    \begin{itemize}
        \item \ruby{半}{はん}〜 (half)
        \item \ruby{御}{お}〜/\ruby{御}{み}〜 (honorific)

        E.g.\ お\ruby{手洗い}{て|あ|らい}、\ruby{御心}{み|こころ}
        \item \ruby{毎}{まい}〜 (every)
        \item \ruby{一}{ひと}〜 (one)
        \item \ruby{二}{ふた}〜 (two)
        \item \ruby{片}{かた}〜 (one-sided)

        E.g.\ \ruby{片仮名}{かた|か|な} (rendaku blocked) vs.\ \ruby{平仮名}{ひら|が|な} (rendaku happens).
        \item \ruby{唐}{から}〜 (Chinese)
        \item \ruby{白}{しろ}〜 (white)
        \item \ruby{黒}{くろ}〜 (black)
    \end{itemize}

    Resisting suffixes:
    \begin{itemize}
        \item 〜\ruby{先}{さき} (previous/tip)
        \item 〜\ruby{紐}{ひも} (string/cord)
        \item 〜\ruby{浜}{はま} (beach)
        \item 〜\ruby{姫}{ひめ} (princess)
        \item 〜\ruby{煙}{けむり} (smoke)
        \item 〜\ruby{土}{つち} (dirt)
        \item 〜\ruby{潮}{しお} (tide)
        \item 〜\ruby{血}{けつ} (blood)
        \item 〜\ruby{下}{した} (below)
    \end{itemize}
\end{itemize}


\subsection{音読み and 訓読み and mixing the two} \label{sec:on-yomi-and-kun-yomi}
\emph{Read the main article on \href{https://www.tofugu.com/japanese/onyomi-kunyomi/}{Tofugu1} and \href{https://www.tofugu.com/japanese/weird-kanji-readings/}{Tofugu2}.}

\subsubsection{History and why 漢字 is the way it is}
According to Tofugu, 漢字 was imported from China via religious texts (which the Japanese appreciated a lot). They merged the Chinese writing system into olden Japanese, but olden Japanese was already established and had its own set of pronunciations. The Japanese decided to adopt Chinese orthography, while co-opting both the Chinese-derived pronunciations (音読み) and the native Japanese pronunciations (訓読み).

While 漢字 words adopted in the above manner have both 音読み and 訓読み readings, some 漢字 words used today only have one:
\begin{itemize}
    \item Those that only have 音読み readings were imported from China wholesale, either because the concept didn't yet exist in Japanese vocabulary, or because there were multiple incompatible native terms for the concept which cannot be unified properly (back then, Japan wasn't one unified country, but comprised unrelated groups with unique systems of government, and presumably sub-languages).

    E.g.\ \ruby{肉}{にく} (meat)、\ruby{材}{ざい} (lumber)、\ruby{感}{かん} (feeling)、\ruby{点}{てん} (point)、\ruby{医}{い} (doctor)、\ruby{茶}{ちゃ} (tea)、\ruby{胃}{い} (stomach)、\ruby{職}{しょく} (work)、\ruby{象}{ぞう} (elephant)、\ruby{秒}{びょう} (time second).
    \item Those that only have 訓読み readings were invented in Japan for a concept that was native to Japan.

    E.g.\ \ruby{畑}{はたけ} (field)、\ruby{姫}{ひめ} (princess)、\ruby{匂}{にお}い (fragrant)、\ruby{峠}{とうげ} (mountain pass)、\ruby{枠}{わく} (frame)、\ruby{籾}{もみ} (unhulled rice)、\ruby{鰯}{いわし} (sardine)、\ruby{栃}{とち} (horses' chestnut)、\ruby{込}{こ}む (to be crowded)、\ruby{咲}{さ}く (to bloom).
\end{itemize}

Furthermore, olden China was always in a state of infighting and changing of powers. As power in China changed, so did the ``official'' language. The introduction of 漢字 from China to Japan happaned over a long period of time, across many Chinese powers and thus many ``then-official'' versions of the olden Chinese language. This explains why some 漢字 have multiple 音読み readings: Japanese scholars decided to co-opt them as new readings came, without deprecating the ``older'' readings. \emph{(``There were three major reading adoption periods in the history of the Japanese language: \ruby{呉音}{ご|おん} (4--6th century; the Wu Dynasty's pronunciation), \ruby{漢音}{かん|おん} (7--9th century; the Han Dynasty's pronunciation), and \ruby{唐音}{とう|おん} (1185--1573 the ``modern day Mandarin Chinese'' pronunciation).)}

The presence of multiple 訓読み has a different historical explanation. Spoken Japanese existed before written Japanese. Multiple similar ``senses'' of a concept (e.g.\ to raise, to rise, to climb) have different pronunciations in spoken language, but were gathered under the same orthography when the written language was developed (e.g.\ 「上〜」).

\subsection{When to use 音読み and 訓読み}

\begin{itemize}
    \item (lone 訓読み 漢字: majority) These comprise majority of beginner words found in textbooks. Mostly nouns. 訓読み is used.

    \item (lone 音読み 漢字: minority) These are characters with significant meaning, and includes things like counters and single 漢字 numbers. 音読み is used.

    \item (音読み 漢字 compounds: majority; \ruby{熟語}{じゅく|ご}) These are compound 漢字 words without any trailing 送り仮名. All the constituent 漢字 have Chinese origins and thus 音読み is used.

    \item (訓読み 漢字 compounds: minority) These are a special class of compound 漢字 and comprise nature concepts (especially the very Japanese ones) and cardinal directions. 訓読み is used for all the constituent 漢字.

    \item (訓読み 漢字 with trailing 送り仮名) Mostly adjectives and verbs, with the occasional nouns. When trailing 送り仮名 are present, 訓読み is used most of the time.

    \item (\ruby{重箱}{ジュウ|ばこ}読み 漢字 compounds) These compound 漢字 words take words with mixed origins: the first Chinese-derived and the second Japanese-derived. The first word takes 音読み, the second word takes 訓読み.

    E.g.\ \ruby{金色}{キン|いろ} (gold colour).

    \item (\ruby{湯桶}{ゆ|トウ}読み 漢字 compounds) These compound 漢字 words take words with mixed origins: the first Japanese-derived and the second Chinese-derived. The first word takes 訓読み, the second word takes 音読み.

    E.g.\ \ruby{場所}{ば|ショ} (place), \ruby{合気道}{あい|キ|ドウ} (martial arts Aikido).

    \item (\ruby{当}{あ}て\ruby{字}{じ} I: borrow Chinese reading, \ul{invent} meaning) Part of the precursor to modern-day 平仮名 and 片仮名: \ruby{万葉仮名}{ま|にょう|が|な}. This archaic orthographic system used (multiple) Chinese characters to represent each Japanese sound.

    E.g.\ \ruby{亜米利加}{ア|マイ|リ|カ} (America; today we use \ruby{米国}{べい|こく} or アメリカ), \ruby{仏蘭西}{フ|ラン|ス} (France; today we use \ruby{仏国}{ふっ|こく} or フランス), \ruby{寿司}{ス|シ} (sushi), \ruby{亜細亜}{ア|ジ|ア} (Asia; today we use アジア), \ruby{珈琲}{コー|ヒー} (coffee; today we use コーヒー), \ruby[g]{流石}{サスガ} (as expected; today we use さすが), \ruby{沢山}{タク|サン} (many; today we use たくさん).

    \item (\ruby{当}{あ}て\ruby{字}{じ} II: borrow Chinese meaning, \ul{invent} reading) These were adopted when a concept could not yet be expressed directly in the adopted Chinese orthography at the time, but when broken down into simpler concepts, can be expressed using Chinese orthography at the time. The pronunciation follows neither 音読み nor 訓読み of the borrowed Chinese orthography, but instead how the concept would be pronounced in spoken Japanese at the time (that does not later fall into 訓読み).

    E.g.\ \ruby[g]{煙草}{たばこ} (Tobacco (smoke $+$ grass); today we use タバコ), \ruby[g]{台詞}{せりふ} (speech; today we use セリフ), \ruby[g]{南瓜}{かぼちゃ} (Japanese squash/pumpkin; today we use かぼちゃ), \ruby[g]{海老}{えび} (shrimp; today we use エビ), \ruby{海苔}{の|り} (Japanese seaweed; today we use のり).

    \item (\ruby{当}{あ}て\ruby{字}{じ} III: borrow both Chinese meaning and the associated reading) Sometimes Japanese scholars were able to find Chinese orthography whose meaning and reading were both desired when representing a concept. These are happy coincidences, perhaps.

    E.g.\ \ruby{合羽}{カッ|ぱ} (raincoat; today we use カッパ), \ruby{倶楽部}{ク|ラ|ブ} (club; today we use クラブ), \ruby{算盤}{そろ|バン} (abacus; today we use そろばん), \ruby{剃刀}{かみ|そり} (razor; today we use カミソリ), \ruby[g]{田舎}{いなか} (countryside).

    \textblue{Honestly, any time you see pronunciations that don't fall under 音読み or 訓読み, it's safe to assume they're under 当て字, where things are borrowed and crafted from somewhere else; it's a bit unclear and the concept is probably even more complicated than what Tofugu presents (which is an incomplete overview): some of the pronunciations in 当て字 III come from nowhere.}

    \item (lone 外来語 漢字) These are foreign loanwords that attained their own 漢字. These are units of measure (e.g.\ metric system) and common words. These readings have been adopted into 訓読み.

    E.g.\ \ruby{米}{メートル} (metres; today we use メートル or ㍍), \ruby{頁}{ページ} (page; today we use ページ or ㌻), \ruby{零}{ゼロ} (zilch; today we use ゼロ).
\end{itemize}

Finally, there's the bizarre class of Japanese names (\ruby{名乗}{な|の}り) which we best leave untouched here.

\end{document}
