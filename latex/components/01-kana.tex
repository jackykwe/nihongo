\documentclass[../nihongo-gakushuu-kyouzai.tex]{subfiles}
\graphicspath{ {../imgs/} }
\begin{document}
\onehalfspacing  % for 振り仮名

\section{The writing system}
\subsection{平仮名(ひらがな)}
Some general notes:
\begin{itemize}
    \item The ん character is rarely used by itself, but suffixed to another character to add the ``n'' sound.
\end{itemize}
\subsubsection{Mnemonics}

% Help: \multirow{2}{*}{} & \multirow{2}{*}{} & \multicolumn{3}{c}{} \\* \cmidrule(l){3-5}
\begin{longtable}[c]{@{}ccccl@{}}
    \toprule
    \multicolumn{2}{c}{\multirow{2}{*}{平仮名}} & Hepburn & Keyboard & \multirow{2}{*}{Mnemonic} \\
    \multicolumn{2}{c}{} & romanisation & command & \\* \midrule
    あ & {\sffamily あ} & a & \texttt{a} & ``A'' shape \\
    い & {\sffamily い} & i & \texttt{i} & \ul{ee}l \\
    う & {\sffamily う} & u & \texttt{u} & ``u'' shape \\
    え & {\sffamily え} & e & \texttt{e} & \ul{e}xotic swan \\
    お & {\sffamily お} & o & \texttt{o} & double ``o'' shape \\
    か & {\sffamily か} & ka & \texttt{ka} & 咖啡 \\
    が & {\sffamily が} & ga & \texttt{ga} &  \\
    き & {\sffamily き} & ki & \texttt{ki} & \ul{ke}y \\
    きゃ & {\sffamily きゃ} & kya & \texttt{kya} &  \\
    きゅ & {\sffamily きゅ} & kyu & \texttt{kyu} &  \\
    きょ & {\sffamily きょ} & kyo & \texttt{kyo} &  \\
    ぎ & {\sffamily ぎ} & gi & \texttt{gi} &  \\
    ぎゃ & {\sffamily ぎゃ} & gya & \texttt{gya} &  \\
    ぎゅ & {\sffamily ぎゅ} & gyu & \texttt{gyu} &  \\
    ぎょ & {\sffamily ぎょ} & gyo & \texttt{gyo} &  \\
    く & {\sffamily く} & ku & \texttt{ku} & bird \ul{ku}-ku \\
    ぐ & {\sffamily ぐ} & gu & \texttt{gu} &  \\
    け & {\sffamily け} & ke & \texttt{ke} & \ul{ke}lp (loose kelp) \\
    げ & {\sffamily げ} & ge & \texttt{ge} &  \\
    こ & {\sffamily こ} & ko & \texttt{ko} & \ul{co}-habiting worms \\
    ご & {\sffamily ご} & go & \texttt{go} &  \\
    さ & {\sffamily さ} & sa & \texttt{sa} & \ul{sa}lsa (two hand stir) / NOT ``5'' \\
    ざ & {\sffamily ざ} & za & \texttt{za} &  \\
    し & {\sffamily し} & shi & \textlightgrey{\texttt{si}/}\texttt{shi} & \ul{shee}p; shepherd's crook \\
    しゃ & {\sffamily しゃ} & sha & \texttt{sha} &  \\
    しゅ & {\sffamily しゅ} & shu & \texttt{shu} &  \\
    しょ & {\sffamily しょ} & sho & \texttt{sho} &  \\
    じ & {\sffamily じ} & ji & \textlightgrey{\texttt{zi}/}\texttt{ji} &  \\
    じゃ & {\sffamily じゃ} & ja & \textlightgrey{\texttt{jya}/}\texttt{ja} &  \\
    じゅ & {\sffamily じゅ} & ju & \textlightgrey{\texttt{jyu}/}\texttt{ju} &  \\
    じょ & {\sffamily じょ} & jo & \textlightgrey{\texttt{jyo}/}\texttt{jo} &  \\
    す & {\sffamily す} & su & \texttt{su} & \ul{sw}ing \\
    ず & {\sffamily ず} & zu & \texttt{zu} & \\
    せ & {\sffamily せ} & se & \texttt{se} & \ruby{世界}{せ|かい} \\
    ぜ & {\sffamily ぜ} & ze & \texttt{ze} &  \\
    そ & {\sffamily そ} & so & \texttt{so} & \ul{so}da / ``sword'' shape \\
    ぞ & {\sffamily ぞ} & zo & \texttt{zo} &  \\
    た & {\sffamily た} & ta & \texttt{ta} & ``ta'' shape \\
    だ & {\sffamily だ} & da & \texttt{da} &  \\
    ち & {\sffamily ち} & chi & \textlightgrey{\texttt{ti}/}\texttt{chi} & the ``5'' \\
    ちゃ & {\sffamily ちゃ} & cha & \texttt{cha} &  \\
    ちゅ & {\sffamily ちゅ} & chu & \texttt{chu} &  \\
    ちょ & {\sffamily ちょ} & cho & \texttt{cho} &  \\
    ぢ & {\sffamily ぢ} & ji & \textred{\texttt{di}} &  \\
    ぢゃ & {\sffamily ぢゃ} & ja & \textred{\texttt{dya}} &  \\
    ぢゅ & {\sffamily ぢゅ} & ju & \textred{\texttt{dyu}} &  \\
    ぢょ & {\sffamily ぢょ} & jo & \textred{\texttt{dyo}} &  \\
    つ & {\sffamily つ} & tsu & \textlightgrey{\texttt{tu}/}\texttt{tsu} & \ul{tsu}nami \\
    づ & {\sffamily づ} & zu & \color{red} \texttt{du} &  \\
    て & {\sffamily て} & te & \texttt{te} & \ul{te}lescope \\
    で & {\sffamily で} & de & \texttt{de} &  \\
    と & {\sffamily と} & to & \texttt{to} & \ul{to}e with splinter \\
    ど & {\sffamily ど} & do & \texttt{do} &  \\
    な & {\sffamily な} & na & \texttt{na} & \ul{nu}n praying to cross \\
    に & {\sffamily に} & ni & \texttt{ni} & \ul{nee}dle \\
    にゃ & {\sffamily にゃ} & nya & \texttt{nya} &  \\
    にゅ & {\sffamily にゅ} & nyu & \texttt{nyu} &  \\
    にょ & {\sffamily にょ} & nyo & \texttt{nyo} &  \\
    ぬ & {\sffamily ぬ} & nu & \texttt{nu} & \ul{noo}dles \textbf{with tail} \\
    ね & {\sffamily ね} & ne & \texttt{ne} & ねこ (\ul{ne}ko) \textbf{with tail} \\
    の & {\sffamily の} & no & \texttt{no} & pig \ul{no}se \\
    は & {\sffamily は} & ha/wa$^\dagger$ & \textred{\texttt{ha}} & ``Ha'' shape \\
    ば & {\sffamily ば} & ba & \texttt{ba} &  \\
    ぱ & {\sffamily ぱ} & pa & \texttt{pa} &  \\
    ひ & {\sffamily ひ} & hi & \texttt{hi} & \ul{hee}l / \ul{he} has a big nose \\
    ひゃ & {\sffamily ひゃ} & hya & \texttt{hya} &  \\
    ひゅ & {\sffamily ひゅ} & hyu & \texttt{hyu} &  \\
    ひょ & {\sffamily ひょ} & hyo & \texttt{hyo} &  \\
    び & {\sffamily び} & bi & \texttt{bi} &  \\
    びゃ & {\sffamily びゃ} & bya & \texttt{bya} &  \\
    びゅ & {\sffamily びゅ} & byu & \texttt{byu} &  \\
    びょ & {\sffamily びょ} & byo & \texttt{byo} &  \\
    ぴ & {\sffamily ぴ} & pi & \texttt{pi} &  \\
    ぴゃ & {\sffamily ぴゃ} & pya & \texttt{pya} &  \\
    ぴゅ & {\sffamily ぴゅ} & pyu & \texttt{pyu} &  \\
    ぴょ & {\sffamily ぴょ} & pyo & \texttt{pyo} &  \\
    ふ & {\sffamily ふ} & fu & \textlightgrey{\texttt{hu}/}\texttt{fu} & Mount \ul{Fu}ji \\
    ぶ & {\sffamily ぶ} & bu & \texttt{bu} &  \\
    ぷ & {\sffamily ぷ} & pu & \texttt{pu} &  \\
    へ & {\sffamily へ} & he/e$^\dagger$ & \textred{\texttt{he}} & \ul{he}adband / Mount St. \ul{He}lens \\
    べ & {\sffamily べ} & be & \texttt{be} &  \\
    ぺ & {\sffamily ぺ} & pe & \texttt{pe} &  \\
    ほ & {\sffamily ほ} & ho & \texttt{ho} & mutated santa says \ul{ho} ho ho\\
    ぼ & {\sffamily ぼ} & bo & \texttt{bo} &  \\
    ぽ & {\sffamily ぽ} & po & \texttt{po} &  \\
    ま & {\sffamily ま} & ma & \texttt{ma} & mutated mom with snake tail \\
    み & {\sffamily み} & mi & \texttt{mi} & \ul{me} just turned 21 \\
    みゃ & {\sffamily みゃ} & mya & \texttt{mya} &  \\
    みゅ & {\sffamily みゅ} & myu & \texttt{myu} &  \\
    みょ & {\sffamily みょ} & myo & \texttt{myo} &  \\
    む & {\sffamily む} & mu & \texttt{mu} & cow says \ul{moo} \\
    め & {\sffamily め} & me & \texttt{me} & eye shape \textbf{without tail} \\
    も & {\sffamily も} & mo & \texttt{mo} & \ul{mo}re worms to catch \ul{mo}re fish \\
    や & {\sffamily や} & ya & \texttt{ya} & \ul{ya}cht with anchor down \\
    ゆ & {\sffamily ゆ} & yu & \texttt{yu} & \ul{u}-tensils \\
    よ & {\sffamily よ} & yo & \texttt{yo} & ``yo'' shape \\
    ら & {\sffamily ら} & ra & \texttt{ra} & \ul{ra}bbit \\
    り & {\sffamily り} & ri & \texttt{ri} & reeds \\
    りゃ & {\sffamily りゃ} & rya & \texttt{rya} &  \\
    りゅ & {\sffamily りゅ} & ryu & \texttt{ryu} &  \\
    りょ & {\sffamily りょ} & ryo & \texttt{ryo} &  \\
    る & {\sffamily る} & ru & \texttt{ru} & weird \ul{rou}te \textbf{with tail} \\
    れ & {\sffamily れ} & re & \texttt{re} & \ul{re}tching guy kneeled down \\
    ろ & {\sffamily ろ} & ro & \texttt{ro} & normal \ul{ro}ad \textbf{without tail} \\
    わ & {\sffamily わ} & wa & \texttt{wa} & \ul{wa}llaby / \ul{wa}sp \\
    を & {\sffamily を} & wo/o$^\dagger$ & \texttt{wo} & \ul{wo}ah the water is cold \\
    ん & {\sffamily ん} & nn & \textred{\texttt{nn}} & ``n'' shape \\
    ぁ & {\sffamily ぁ} & ? & \textlightgrey{\texttt{la}/}\textred{\texttt{xa}} &  \\
    ぃ & {\sffamily ぃ} & ? & \textlightgrey{\texttt{li}/}\textred{\texttt{xi}} &  \\
    ぅ & {\sffamily ぅ} & ? & \textlightgrey{\texttt{lu}/}\textred{\texttt{xu}} &  \\
    ぇ & {\sffamily ぇ} & ? & \textlightgrey{\texttt{le}/}\textred{\texttt{xe}} &  \\
    ぉ & {\sffamily ぉ} & ? & \textlightgrey{\texttt{lo}/}\textred{\texttt{xo}} &  \\
    ゃ & {\sffamily ゃ} & ? & \textlightgrey{\texttt{lya}/}\textred{\texttt{xya}} &  \\
    ゅ & {\sffamily ゅ} & ? & \textlightgrey{\texttt{lyu}/}\textred{\texttt{xyu}} &  \\
    ょ & {\sffamily ょ} & ? & \textlightgrey{\texttt{lyo}/}\textred{\texttt{xyo}} &  \\
    っ & {\sffamily っ} & $^{\texttt{+1}}$\texttt{>} & \textlightgrey{\texttt{ltu}/\texttt{ltsu}/\textred{\texttt{xtsu}}/}repeat \texttt{>} &  \\* \bottomrule
    \caption{平仮名 mnemonic table. $^\dagger$Particle romanisations.}
    \label{tbl:hiragana-mnemonics} \\
\end{longtable}


\subsection{片仮名(カタカナ)}
Some general notes:
\begin{itemize}
    \item Usage of the ・ symbol to denote word boundaries is completely optional.
\end{itemize}
\subsubsection{Mnemonics}
\begin{longtable}[c]{@{}ccccl@{}}
    \toprule
    \multicolumn{2}{c}{\multirow{2}{*}{片仮名}} & Hepburn & Keyboard & \multirow{2}{*}{Mnemonic} \\
    \multicolumn{2}{c}{} & romanisation & command & \\* \midrule
    ア & {\sffamily ア} & a & \texttt{a} & ``A'' shape \\
    イ & {\sffamily イ} & i & \texttt{i} & \ul{e}agle perched \\
    \color{blue} イェ & \color{blue} {\sffamily イェ} & \color{blue} ye & \color{blue} \texttt{ye} & \\
    ウ & {\sffamily ウ} & u & \texttt{u} & same shape as う \\
    \color{blue} ウィ & \color{blue} {\sffamily ウィ} & \color{blue} wi & \color{blue} \texttt{wi} & \\
    \color{blue} ウェ & \color{blue} {\sffamily ウェ} & \color{blue} we & \color{blue} \texttt{we} & \\
    \color{blue} ウォ & \color{blue} {\sffamily ウォ} & \color{blue} wo & \color{red} \texttt{uxo} & \\
    % U R COOKED... https://en.wikipedia.org/wiki/Hepburn_romanization#Extended_katakana
    \color{blue} ヴ & \color{blue} {\sffamily ヴ} & \color{blue} vu & \color{blue} \texttt{vu} & \\
    \color{blue} ヴァ & \color{blue} {\sffamily ヴァ} & \color{blue} va & \color{blue} \texttt{va} & \\
    \color{blue} ヴィ & \color{blue} {\sffamily ヴィ} & \color{blue} vi & \color{blue} \texttt{vi} & \\
    \color{blue} ヴュ & \color{blue} {\sffamily ヴュ} & \color{blue} vyu & \color{blue} \texttt{vyu} & \\
    \color{blue} ヴェ & \color{blue} {\sffamily ヴェ} & \color{blue} ve & \color{blue} \texttt{ve} & \\
    \color{blue} ヴォ & \color{blue} {\sffamily ヴォ} & \color{blue} vo & \color{blue} \texttt{vo} & \\
    エ & {\sffamily エ} & e & \texttt{e} & \ul{e}ngineer bar \\
    オ & {\sffamily オ} & o & \texttt{o} & \ul{o}pera talent (才) singing \\
    カ & {\sffamily カ} & ka & \texttt{ka} & same shape as か \\
    ガ & {\sffamily ガ} & ga & \texttt{ga} &  \\
    キ & {\sffamily キ} & ki & \texttt{ki} & same shape as き \\
    キャ & {\sffamily キャ} & kya & \texttt{kya} &  \\
    キュ & {\sffamily キュ} & kyu & \texttt{kyu} &  \\
    キョ & {\sffamily キョ} & kyo & \texttt{kyo} &  \\
    ギ & {\sffamily ギ} & gi & \texttt{gi} &  \\
    ギャ & {\sffamily ギャ} & gya & \texttt{gya} &  \\
    ギュ & {\sffamily ギュ} & gyu & \texttt{gyu} &  \\
    ギョ & {\sffamily ギョ} & gyo & \texttt{gyo} &  \\
    ク & {\sffamily ク} & ku & \texttt{ku} & \ul{coo}k's hat \\
    \color{blue} クァ & \color{blue} {\sffamily クァ} & \color{blue} kwa & \color{blue} \texttt{kwa} & \\
    \color{blue} クィ & \color{blue} {\sffamily クィ} & \color{blue} kwi & \color{blue} \texttt{kwi} & \\
    \color{blue} クェ & \color{blue} {\sffamily クェ} & \color{blue} kwe & \color{blue} \texttt{kwe} & \\
    \color{blue} クォ & \color{blue} {\sffamily クォ} & \color{blue} kwo & \color{blue} \texttt{kwo} & \\
    グ & {\sffamily グ} & gu & \texttt{gu} &  \\
    \color{blue} グァ & \color{blue} {\sffamily グァ} & \color{blue} gwa & \color{blue} \texttt{gwa} & \\
    ケ & {\sffamily ケ} & ke & \texttt{ke} & ``k'' shape \\
    ゲ & {\sffamily ゲ} & ge & \texttt{ge} &  \\
    コ & {\sffamily コ} & ko & \texttt{ko} & broken 口 (CN) / two \ul{co}rners \\
    ゴ & {\sffamily ゴ} & go & \texttt{go} &  \\
    サ & {\sffamily サ} & sa & \texttt{sa} & \ul{sa}rdines and \ul{sa}lmon (bigger) \\
    ザ & {\sffamily ザ} & za & \texttt{za} &  \\
    シ & {\sffamily シ} & shi & \textlightgrey{\texttt{si}/}\texttt{shi} & same direction as し \\
    シャ & {\sffamily シャ} & sha & \texttt{sha} &  \\
    シュ & {\sffamily シュ} & shu & \texttt{shu} &  \\
    \color{blue} シェ & \color{blue} {\sffamily シュ} & \color{blue} she & \color{blue} \texttt{she} &  \\
    ショ & {\sffamily ショ} & sho & \texttt{sho} &  \\
    ジ & {\sffamily ジ} & ji & \textlightgrey{\texttt{zi}/}\texttt{ji} &  \\
    ジャ & {\sffamily ジャ} & ja & \textlightgrey{\texttt{jya}/}\texttt{ja} &  \\
    ジュ & {\sffamily ジュ} & ju & \textlightgrey{\texttt{jyu}/}\texttt{ju} &  \\
    \color{blue} ジェ & \color{blue} {\sffamily ジェ} & \color{blue} je & \color{blue} \textlightgrey{\texttt{jye}/}\texttt{je} &  \\
    ジョ & {\sffamily ジョ} & jo & \textlightgrey{\texttt{jyo}/}\texttt{jo} &  \\
    ス & {\sffamily ス} & su & \texttt{su} & \ul{su}perman \\
    % \color{blue} スィ & \color{blue} {\sffamily スィ} & \color{blue} si & \color{red} \texttt{suxi} & \\
    ズ & {\sffamily ズ} & zu & \texttt{zu} &  \\
    % \color{blue} ズィ & \color{blue} {\sffamily ズィ} & \color{blue} zi & \color{red} \texttt{zuxi} & \\
    セ & {\sffamily セ} & se & \texttt{se} & same shape as せ \\
    ゼ & {\sffamily ゼ} & ze & \texttt{ze} &  \\
    ソ & {\sffamily ソ} & so & \texttt{so} & \ul{se}wing needles \\
    ゾ & {\sffamily ゾ} & zo & \texttt{zo} &  \\
    タ & {\sffamily タ} & ta & \texttt{ta} & \ul{ti}dal wave \\
    ダ & {\sffamily ダ} & da & \texttt{da} &  \\
    チ & {\sffamily チ} & chi & \textlightgrey{\texttt{ti}/}\texttt{chi} & \ul{chee}r / \ruby{千}{ち} \\
    チャ & {\sffamily チャ} & cha & \texttt{cha} &  \\
    チュ & {\sffamily チュ} & chu & \texttt{chu} &  \\
    \color{blue} チェ & \color{blue} {\sffamily チェ} & \color{blue} che & \color{blue} \texttt{che} &  \\
    チョ & {\sffamily チョ} & cho & \texttt{cho} &  \\
    ヂ & {\sffamily ヂ} & ji & \textred{\texttt{di}} &  \\
    ヂャ & {\sffamily ヂャ} & ja & \textred{\texttt{dya}} &  \\
    ヂュ & {\sffamily ヂュ} & ju & \textred{\texttt{dyu}} &  \\
    ヂョ & {\sffamily ヂョ} & jo & \textred{\texttt{dyo}} &  \\
    ツ & {\sffamily ツ} & tsu & \textlightgrey{\texttt{tu}/}\texttt{tsu} & same direction as つ \\
    \color{blue} ツァ & \color{blue} {\sffamily ツァ} & \color{blue} tsa & \color{blue} \texttt{tsa} & \emph{Italian ``z''}\\
    \color{blue} ツィ & \color{blue} {\sffamily ツィ} & \color{blue} tsi & \color{blue} \texttt{tsi} & \emph{Italian ``z''}\\
    \color{blue} ツェ & \color{blue} {\sffamily ツェ} & \color{blue} tse & \color{blue} \texttt{tse} & \emph{Italian ``z''}\\
    \color{blue} ツォ & \color{blue} {\sffamily ツォ} & \color{blue} tso & \color{blue} \texttt{tso} & \emph{Italian ``z''}\\
    ヅ & {\sffamily ヅ} & zu & \color{red} \texttt{du} &  \\
    テ & {\sffamily テ} & te & \texttt{te} & \ul{te}lephone pole \\
    \color{blue} ティ & \color{blue}{\sffamily ティ} & \color{blue} ti & \color{red} \texttt{texi} & \emph{``par\ul{ty}''}\\
    \color{blue} テュ & \color{blue}{\sffamily ティ} & \color{blue} tyu & \color{red} \texttt{texyu} & \\
    デ & {\sffamily デ} & de & \texttt{de} &  \\
    \color{blue} ディ & \color{blue}{\sffamily ディ} & \color{blue} di & \color{red} \texttt{dexi} & \emph{``can\ul{dy}''}\\
    \color{blue} デュ & \color{blue}{\sffamily デュ} & \color{blue} dyu & \color{red} \texttt{dexyu} & \\
    ト & {\sffamily ト} & to & \texttt{to} & \ul{to}tem pole \\
    \color{blue} トゥ & \color{blue}{\sffamily トゥ} & \color{blue} tu & \color{red} \texttt{toxu} & \emph{``two''}\\
    ド & {\sffamily ド} & do & \texttt{do} &  \\
    \color{blue} ドゥ & \color{blue}{\sffamily ドゥ} & \color{blue} du & \color{red} \texttt{dowu} & \emph{``dew''}\\
    ナ & {\sffamily ナ} & na & \texttt{na} & \ul{na}rwhal \\
    ニ & {\sffamily ニ} & ni & \texttt{ni} & same shape as に \\
    ニャ & {\sffamily ニャ} & nya & \texttt{nya} &  \\
    ニュ & {\sffamily ニュ} & nyu & \texttt{nyu} &  \\
    ニョ & {\sffamily ニョ} & nyo & \texttt{nyo} &  \\
    ヌ & {\sffamily ヌ} & nu & \texttt{nu} & \ul{noo}dles with chopsticks \\
    ネ & {\sffamily ネ} & ne & \texttt{ne} & \ul{ne}ckerchief \\
    ノ & {\sffamily ノ} & no & \texttt{no} & long \ul{no}se \\
    ハ & {\sffamily ハ} & ha & \textred{\texttt{ha}} & \ruby{八}{ハチ} / 八 (CN) \\
    バ & {\sffamily バ} & ba & \texttt{ba} &  \\
    パ & {\sffamily パ} & pa & \texttt{pa} &  \\
    ヒ & {\sffamily ヒ} & hi & \texttt{hi} & smile \ul{he}he \\
    ヒャ & {\sffamily ヒャ} & hya & \texttt{hya} &  \\
    ヒュ & {\sffamily ヒュ} & hyu & \texttt{hyu} &  \\
    ヒョ & {\sffamily ヒョ} & hyo & \texttt{hyo} &  \\
    ビ & {\sffamily ビ} & bi & \texttt{bi} &  \\
    ビャ & {\sffamily ビャ} & bya & \texttt{bya} &  \\
    ビュ & {\sffamily ビュ} & byu & \texttt{byu} &  \\
    ビョ & {\sffamily ビョ} & byo & \texttt{byo} &  \\
    ピ & {\sffamily ピ} & pi & \texttt{pi} &  \\
    ピャ & {\sffamily ピャ} & pya & \texttt{pya} &  \\
    ピュ & {\sffamily ピュ} & pyu & \texttt{pyu} &  \\
    ピョ & {\sffamily ピョ} & pyo & \texttt{pyo} &  \\
    フ & {\sffamily フ} & fu & \textlightgrey{\texttt{hu}/}\texttt{fu} & \ul{fl}ag \\
    \color{blue} ファ & \color{blue} {\sffamily ファ} & \color{blue} fa & \color{blue} \texttt{fa} & \\
    \color{blue} フィ & \color{blue} {\sffamily フィ} & \color{blue} fi & \color{blue} \texttt{fi} & \\
    \color{blue} フュ & \color{blue} {\sffamily フュ} & \color{blue} fyu & \color{blue} \texttt{fyu} & \\
    \color{blue} フェ & \color{blue} {\sffamily フェ} & \color{blue} fe & \color{blue} \texttt{fe} & \\
    \color{blue} フォ & \color{blue} {\sffamily フォ} & \color{blue} fo & \color{blue} \texttt{fo} & \\
    ブ & {\sffamily ブ} & bu & \texttt{bu} &  \\
    プ & {\sffamily プ} & pu & \texttt{pu} &  \\
    ヘ & {\sffamily ヘ} & he & \textred{\texttt{he}} & same shape as へ \\
    ベ & {\sffamily ベ} & be & \texttt{be} &  \\
    ペ & {\sffamily ペ} & pe & \texttt{pe} &  \\
    ホ & {\sffamily ホ} & ho & \texttt{ho} & \ul{ho}ly cross \\
    % \color{blue} ホゥ & \color{blue} {\sffamily ホゥ} & \color{blue} hu & \color{red} \texttt{hoxu} & \\
    ボ & {\sffamily ボ} & bo & \texttt{bo} &  \\
    ポ & {\sffamily ポ} & po & \texttt{po} &  \\
    マ & {\sffamily マ} & ma & \texttt{ma} & \ul{ma}th angles \\
    ミ & {\sffamily ミ} & mi & \texttt{mi} & \ul{mi}ssiles \\
    ミャ & {\sffamily ミャ} & mya & \texttt{mya} &  \\
    ミュ & {\sffamily ミュ} & myu & \texttt{myu} &  \\
    ミョ & {\sffamily ミョ} & myo & \texttt{myo} &  \\
    ム & {\sffamily ム} & mu & \texttt{mu} & cow face, says \ul{moo} \\
    メ & {\sffamily メ} & me & \texttt{me} & Arlecchino's eyes (め) \\
    モ & {\sffamily モ} & mo & \texttt{mo} & same shape as も \\
    ヤ & {\sffamily ヤ} & ya & \texttt{ya} & same shape as や \\
    ユ & {\sffamily ユ} & yu & \texttt{yu} & \ul{u}-turn \\
    ヨ & {\sffamily ヨ} & yo & \texttt{yo} & \ul{yo}ga pose \\
    ラ & {\sffamily ラ} & ra & \texttt{ra} & \ul{ra}ptor \\
    リ & {\sffamily リ} & ri & \texttt{ri} & reeds \\
    リャ & {\sffamily リャ} & rya & \texttt{rya} &  \\
    リュ & {\sffamily リュ} & ryu & \texttt{ryu} &  \\
    リョ & {\sffamily リョ} & ryo & \texttt{ryo} &  \\
    ル & {\sffamily ル} & ru & \texttt{ru} & tree \ul{roo}ts \\
    レ & {\sffamily レ} & re & \texttt{re} & \ul{re}d hair / right side of ル \\
    ロ & {\sffamily ロ} & ro & \texttt{ro} & cyclic \ul{ro}ad \\
    ワ & {\sffamily ワ} & wa & \texttt{wa} & \ul{wa}termelon slice \\
    ヲ & {\sffamily ヲ} & wo & \texttt{wo} & \ul{o}atmeal bowl \\
    ン & {\sffamily ン} & nn & \textred{\texttt{nn}} & N/A \\
    ァ & {\sffamily ァ} & ? & \textlightgrey{\texttt{la}/}\textred{\texttt{xa}} &  \\
    ィ & {\sffamily ィ} & ? & \textlightgrey{\texttt{li}/}\textred{\texttt{xi}} &  \\
    ゥ & {\sffamily ゥ} & ? & \textlightgrey{\texttt{lu}/}\textred{\texttt{xu}} &  \\
    ェ & {\sffamily ェ} & ? & \textlightgrey{\texttt{le}/}\textred{\texttt{xe}} &  \\
    ォ & {\sffamily ォ} & ? & \textlightgrey{\texttt{lo}/}\textred{\texttt{xo}} &  \\
    ャ & {\sffamily ャ} & ? & \textlightgrey{\texttt{lya}/}\textred{\texttt{xya}} &  \\
    ュ & {\sffamily ュ} & ? & \textlightgrey{\texttt{lyu}/}\textred{\texttt{xyu}} &  \\
    ョ & {\sffamily ョ} & ? & \textlightgrey{\texttt{lyo}/}\textred{\texttt{xyo}} &  \\
    ー & {\sffamily ー} & \texttt{<}$^{\texttt{+1}}$ & \textred{\texttt{$-$} key} &  \\
    ッ & {\sffamily ッ} & $^{\texttt{+1}}$\texttt{>} & \textlightgrey{\texttt{ltu}/\texttt{ltsu}/\textred{\texttt{xtsu}}/}repeat \texttt{>} &  \\* \bottomrule
    \caption{片仮名 mnemonic table. Some entries were taken from \href{https://en.wikipedia.org/wiki/Hepburn_romanization\#Extended_katakana}{Wikipedia (Hepburn Romanisation)} but only the orange and blue ones are taken, since the beige and purple ones are regarded as unofficial (by me).}
    \label{tbl:katakana-mnemonics} \\
\end{longtable}


\subsection{仮名 Summary}
\begin{longtable}[c]{@{}cccccc@{}}
    \toprule
    \multicolumn{2}{c}{\multirow{2}{*}{平仮名}} & \multicolumn{2}{c}{\multirow{2}{*}{片仮名}} & Hepburn & Keyboard \\
    \multicolumn{2}{c}{} & \multicolumn{2}{c}{} & romanisation & command \\* \midrule
    あ & {\sffamily あ} & ア & {\sffamily ア} & a & \texttt{a} \\
    い & {\sffamily い} & イ & {\sffamily イ} & i & \texttt{i}\\
 & {\sffamily } & イェ & {\sffamily イェ} & ye & \texttt{ye} \\
    う & {\sffamily う} & ウ & {\sffamily ウ} & u & \texttt{u} \\
 & {\sffamily } & ウィ & {\sffamily ウィ} & wi & \texttt{wi} \\
 & {\sffamily } & ウェ & {\sffamily ウェ} & we & \texttt{we} \\
 & {\sffamily } & ウォ & {\sffamily ウォ} & wo & \color{red} \texttt{uxo} \\
 & {\sffamily } & ヴ & {\sffamily ヴ} & vu & \texttt{vu} \\
 & {\sffamily } & ヴァ & {\sffamily ヴァ} & va & \texttt{va} \\
 & {\sffamily } & ヴィ & {\sffamily ヴィ} & vi & \texttt{vi} \\
 & {\sffamily } & ヴュ & {\sffamily ヴュ} & vyu & \texttt{vyu} \\
 & {\sffamily } & ヴェ & {\sffamily ヴェ} & ve & \texttt{ve} \\
 & {\sffamily } & ヴォ & {\sffamily ヴォ} & vo & \texttt{vo} \\
    え & {\sffamily え} & エ & {\sffamily エ} & e & \texttt{e} \\
    お & {\sffamily お} & オ & {\sffamily オ} & o & \texttt{o} \\
    か & {\sffamily か} & カ & {\sffamily カ} & ka & \texttt{ka} \\
    が & {\sffamily が} & ガ & {\sffamily ガ} & ga & \texttt{ga} \\
    き & {\sffamily き} & キ & {\sffamily キ} & ki & \texttt{ki} \\
    きゃ & {\sffamily きゃ} & キャ & {\sffamily キャ} & kya & \texttt{kya} \\
    きゅ & {\sffamily きゅ} & キュ & {\sffamily キュ} & kyu & \texttt{kyu} \\
    きょ & {\sffamily きょ} & キョ & {\sffamily キョ} & kyo & \texttt{kyo} \\
    ぎ & {\sffamily ぎ} & ギ & {\sffamily ギ} & gi & \texttt{gi} \\
    ぎゃ & {\sffamily ぎゃ} & ギャ & {\sffamily ギャ} & gya & \texttt{gya} \\
    ぎゅ & {\sffamily ぎゅ} & ギュ & {\sffamily ギュ} & gyu & \texttt{gyu} \\
    ぎょ & {\sffamily ぎょ} & ギョ & {\sffamily ギョ} & gyo & \texttt{gyo} \\
    く & {\sffamily く} & ク & {\sffamily ク} & ku & \texttt{ku} \\
 & {\sffamily } & クァ & {\sffamily クァ} & kwa & \texttt{kwa} \\
 & {\sffamily } & クィ & {\sffamily クィ} & kwi & \texttt{kwi} \\
 & {\sffamily } & クェ & {\sffamily クェ} & kwe & \texttt{kwe} \\
 & {\sffamily } & クォ & {\sffamily クォ} & kwo & \texttt{kwo} \\
    ぐ & {\sffamily ぐ} & グ & {\sffamily グ} & gu & \texttt{gu} \\
 & {\sffamily } & グァ & {\sffamily グァ} & gwa & \texttt{gwa} \\
    け & {\sffamily け} & ケ & {\sffamily ケ} & ke & \texttt{ke} \\
    げ & {\sffamily げ} & ゲ & {\sffamily ゲ} & ge & \texttt{ge} \\
    こ & {\sffamily こ} & コ & {\sffamily コ} & ko & \texttt{ko} \\
    ご & {\sffamily ご} & ゴ & {\sffamily ゴ} & go & \texttt{go} \\
    さ & {\sffamily さ} & サ & {\sffamily サ} & sa & \texttt{sa} \\
    ざ & {\sffamily ざ} & ザ & {\sffamily ザ} & za & \texttt{za} \\
    し & {\sffamily し} & シ & {\sffamily シ} & shi & \textlightgrey{\texttt{si}/}\texttt{shi} \\
    しゃ & {\sffamily しゃ} & シャ & {\sffamily シャ} & sha & \texttt{sha} \\
    しゅ & {\sffamily しゅ} & シュ & {\sffamily シュ} & shu & \texttt{shu} \\
     & {\sffamily } & シェ & {\sffamily シュ} & she & \texttt{she} \\
    しょ & {\sffamily しょ} & ショ & {\sffamily ショ} & sho & \texttt{sho} \\
    じ & {\sffamily じ} & ジ & {\sffamily ジ} & ji & \textlightgrey{\texttt{zi}/}\texttt{ji} \\
    じゃ & {\sffamily じゃ} & ジャ & {\sffamily ジャ} & ja & \textlightgrey{\texttt{jya}/}\texttt{ja} \\
    じゅ & {\sffamily じゅ} & ジュ & {\sffamily ジュ} & ju & \textlightgrey{\texttt{jyu}/}\texttt{ju} \\
     & {\sffamily } & ジェ & {\sffamily ジェ} & je & \textlightgrey{\texttt{jye}/}\texttt{je} \\
    じょ & {\sffamily じょ} & ジョ & {\sffamily ジョ} & jo & \textlightgrey{\texttt{jyo}/}\texttt{jo} \\
    す & {\sffamily す} & ス & {\sffamily ス} & su & \texttt{su} \\
     % & {\sffamily } & スィ & {\sffamily スィ} & si & \color{red} \texttt{suxi} \\
    ず & {\sffamily ず} & ズ & {\sffamily ズ} & zu & \texttt{zu} \\
     % & {\sffamily } & ズィ & {\sffamily ズィ} & zi & \color{red} \texttt{zuxi} \\
    せ & {\sffamily せ} & セ & {\sffamily セ} & se & \texttt{se} \\
    ぜ & {\sffamily ぜ} & ゼ & {\sffamily ゼ} & ze & \texttt{ze} \\
    そ & {\sffamily そ} & ソ & {\sffamily ソ} & so & \texttt{so} \\
    ぞ & {\sffamily ぞ} & ゾ & {\sffamily ゾ} & zo & \texttt{zo} \\
    た & {\sffamily た} & タ & {\sffamily タ} & ta & \texttt{ta} \\
    だ & {\sffamily だ} & ダ & {\sffamily ダ} & da & \texttt{da} \\
    ち & {\sffamily ち} & チ & {\sffamily チ} & chi & \textlightgrey{\texttt{ti}/}\texttt{chi} \\
    ちゃ & {\sffamily ちゃ} & チャ & {\sffamily チャ} & cha & \texttt{cha} \\
    ちゅ & {\sffamily ちゅ} & チュ & {\sffamily チュ} & chu & \texttt{chu} \\
     & {\sffamily } & チェ & {\sffamily チェ} & che & \texttt{che} \\
    ちょ & {\sffamily ちょ} & チョ & {\sffamily チョ} & cho & \texttt{cho} \\
    ぢ & {\sffamily ぢ} & ヂ & {\sffamily ヂ} & ji & \textred{\texttt{di}} \\
    ぢゃ & {\sffamily ぢゃ} & ヂャ & {\sffamily ヂャ} & ja & \textred{\texttt{dya}} \\
    ぢゅ & {\sffamily ぢゅ} & ヂュ & {\sffamily ヂュ} & ju & \textred{\texttt{dyu}} \\
    ぢょ & {\sffamily ぢょ} & ヂョ & {\sffamily ヂョ} & jo & \textred{\texttt{dyo}} \\
    つ & {\sffamily つ} & ツ & {\sffamily ツ} & tsu & \textlightgrey{\texttt{tu}/}\texttt{tsu} \\
     & {\sffamily } & ツァ & {\sffamily ツァ} & tsa & \texttt{tsa} \\
     & {\sffamily } & ツィ & {\sffamily ツィ} & tsi & \texttt{tsi} \\
     & {\sffamily } & ツェ & {\sffamily ツェ} & tse & \texttt{tse} \\
     & {\sffamily } & ツォ & {\sffamily ツォ} & tso & \texttt{tso} \\
    づ & {\sffamily づ} & ヅ & {\sffamily ヅ} & zu & \color{red} \texttt{du} \\
    て & {\sffamily て} & テ & {\sffamily テ} & te & \texttt{te} \\
     & {\sffamily } & ティ &{\sffamily ティ} & ti & \color{red} \texttt{texi} \\
     & {\sffamily } & テュ &{\sffamily ティ} & tyu & \color{red} \texttt{texyu} \\
    で & {\sffamily で} & デ & {\sffamily デ} & de & \texttt{de} \\
     & {\sffamily } & ディ &{\sffamily ディ} & di & \color{red} \texttt{dexi} \\
     & {\sffamily } & デュ &{\sffamily デュ} & dyu & \color{red} \texttt{dexyu} \\
    と & {\sffamily と} & ト & {\sffamily ト} & to & \texttt{to} \\
     & {\sffamily } & トゥ &{\sffamily トゥ} & tu & \color{red} \texttt{toxu} \\
    ど & {\sffamily ど} & ド & {\sffamily ド} & do & \texttt{do} \\
     & {\sffamily } & ドゥ &{\sffamily ドゥ} & du & \color{red} \texttt{dowu} \\
    な & {\sffamily な} & ナ & {\sffamily ナ} & na & \texttt{na} \\
    に & {\sffamily に} & ニ & {\sffamily ニ} & ni & \texttt{ni} \\
    にゃ & {\sffamily にゃ} & ニャ & {\sffamily ニャ} & nya & \texttt{nya} \\
    にゅ & {\sffamily にゅ} & ニュ & {\sffamily ニュ} & nyu & \texttt{nyu} \\
    にょ & {\sffamily にょ} & ニョ & {\sffamily ニョ} & nyo & \texttt{nyo} \\
    ぬ & {\sffamily ぬ} & ヌ & {\sffamily ヌ} & nu & \texttt{nu} \\
    ね & {\sffamily ね} & ネ & {\sffamily ネ} & ne & \texttt{ne} \\
    の & {\sffamily の} & ノ & {\sffamily ノ} & no & \texttt{no} \\
    は & {\sffamily は} & ハ & {\sffamily ハ} & ha & \textred{\texttt{ha}} \\
    ば & {\sffamily ば} & バ & {\sffamily バ} & ba & \texttt{ba} \\
    ぱ & {\sffamily ぱ} & パ & {\sffamily パ} & pa & \texttt{pa} \\
    ひ & {\sffamily ひ} & ヒ & {\sffamily ヒ} & hi & \texttt{hi} \\
    ひゃ & {\sffamily ひゃ} & ヒャ & {\sffamily ヒャ} & hya & \texttt{hya} \\
    ひゅ & {\sffamily ひゅ} & ヒュ & {\sffamily ヒュ} & hyu & \texttt{hyu} \\
    ひょ & {\sffamily ひょ} & ヒョ & {\sffamily ヒョ} & hyo & \texttt{hyo} \\
    び & {\sffamily び} & ビ & {\sffamily ビ} & bi & \texttt{bi} \\
    びゃ & {\sffamily びゃ} & ビャ & {\sffamily ビャ} & bya & \texttt{bya} \\
    びゅ & {\sffamily びゅ} & ビュ & {\sffamily ビュ} & byu & \texttt{byu} \\
    びょ & {\sffamily びょ} & ビョ & {\sffamily ビョ} & byo & \texttt{byo} \\
    ぴ & {\sffamily ぴ} & ピ & {\sffamily ピ} & pi & \texttt{pi} \\
    ぴゃ & {\sffamily ぴゃ} & ピャ & {\sffamily ピャ} & pya & \texttt{pya} \\
    ぴゅ & {\sffamily ぴゅ} & ピュ & {\sffamily ピュ} & pyu & \texttt{pyu} \\
    ぴょ & {\sffamily ぴょ} & ピョ & {\sffamily ピョ} & pyo & \texttt{pyo} \\
    ふ & {\sffamily ふ} & フ & {\sffamily フ} & fu & \textlightgrey{\texttt{hu}/}\texttt{fu} \\
     & {\sffamily } & ファ & {\sffamily ファ} & fa & \texttt{fa} \\
     & {\sffamily } & フィ & {\sffamily フィ} & fi & \texttt{fi} \\
     & {\sffamily } & フュ & {\sffamily フュ} & fyu & \texttt{fyu} \\
     & {\sffamily } & フェ & {\sffamily フェ} & fe & \texttt{fe} \\
     & {\sffamily } & フォ & {\sffamily フォ} & fo & \texttt{fo} \\
    ぶ & {\sffamily ぶ} & ブ & {\sffamily ブ} & bu & \texttt{bu} \\
    ぷ & {\sffamily ぷ} & プ & {\sffamily プ} & pu & \texttt{pu} \\
    へ & {\sffamily へ} & ヘ & {\sffamily ヘ} & he & \textred{\texttt{he}} \\
    べ & {\sffamily べ} & ベ & {\sffamily ベ} & be & \texttt{be} \\
    ぺ & {\sffamily ぺ} & ペ & {\sffamily ペ} & pe & \texttt{pe} \\
    ほ & {\sffamily ほ} & ホ & {\sffamily ホ} & ho & \texttt{ho} \\
     % & {\sffamily } & ホゥ & {\sffamily ホゥ} & hu & \color{red} \texttt{hoxu} \\
    ぼ & {\sffamily ぼ} & ボ & {\sffamily ボ} & bo & \texttt{bo} \\
    ぽ & {\sffamily ぽ} & ポ & {\sffamily ポ} & po & \texttt{po} \\
    ま & {\sffamily ま} & マ & {\sffamily マ} & ma & \texttt{ma} \\
    み & {\sffamily み} & ミ & {\sffamily ミ} & mi & \texttt{mi} \\
    みゃ & {\sffamily みゃ} & ミャ & {\sffamily ミャ} & mya & \texttt{mya} \\
    みゅ & {\sffamily みゅ} & ミュ & {\sffamily ミュ} & myu & \texttt{myu} \\
    みょ & {\sffamily みょ} & ミョ & {\sffamily ミョ} & myo & \texttt{myo} \\
    む & {\sffamily む} & ム & {\sffamily ム} & mu & \texttt{mu} \\
    め & {\sffamily め} & メ & {\sffamily メ} & me & \texttt{me} \\
    も & {\sffamily も} & モ & {\sffamily モ} & mo & \texttt{mo} \\
    や & {\sffamily や} & ヤ & {\sffamily ヤ} & ya & \texttt{ya} \\
    ゆ & {\sffamily ゆ} & ユ & {\sffamily ユ} & yu & \texttt{yu} \\
    よ & {\sffamily よ} & ヨ & {\sffamily ヨ} & yo & \texttt{yo} \\
    ら & {\sffamily ら} & ラ & {\sffamily ラ} & ra & \texttt{ra} \\
    り & {\sffamily り} & リ & {\sffamily リ} & ri & \texttt{ri} \\
    りゃ & {\sffamily りゃ} & リャ & {\sffamily リャ} & rya & \texttt{rya} \\
    りゅ & {\sffamily りゅ} & リュ & {\sffamily リュ} & ryu & \texttt{ryu} \\
    りょ & {\sffamily りょ} & リョ & {\sffamily リョ} & ryo & \texttt{ryo} \\
    る & {\sffamily る} & ル & {\sffamily ル} & ru & \texttt{ru} \\
    れ & {\sffamily れ} & レ & {\sffamily レ} & re & \texttt{re} \\
    ろ & {\sffamily ろ} & ロ & {\sffamily ロ} & ro & \texttt{ro} \\
    わ & {\sffamily わ} & ワ & {\sffamily ワ} & wa & \texttt{wa} \\
    を & {\sffamily を} & ヲ & {\sffamily ヲ} & wo & \texttt{wo} \\
    ん & {\sffamily ん} & ン & {\sffamily ン} & nn & \textred{\texttt{nn}} \\
    ぁ & {\sffamily ぁ} & ァ & {\sffamily ァ} & ? & \textlightgrey{\texttt{la}/}\textred{\texttt{xa}} \\
    ぃ & {\sffamily ぃ} & ィ & {\sffamily ィ} & ? & \textlightgrey{\texttt{li}/}\textred{\texttt{xi}} \\
    ぅ & {\sffamily ぅ} & ゥ & {\sffamily ゥ} & ? & \textlightgrey{\texttt{lu}/}\textred{\texttt{xu}} \\
    ぇ & {\sffamily ぇ} & ェ & {\sffamily ェ} & ? & \textlightgrey{\texttt{le}/}\textred{\texttt{xe}} \\
    ぉ & {\sffamily ぉ} & ォ & {\sffamily ォ} & ? & \textlightgrey{\texttt{lo}/}\textred{\texttt{xo}} \\
    ゃ & {\sffamily ゃ} & ャ & {\sffamily ャ} & ? & \textlightgrey{\texttt{lya}/}\textred{\texttt{xya}} \\
    ゅ & {\sffamily ゅ} & ュ & {\sffamily ュ} & ? & \textlightgrey{\texttt{lyu}/}\textred{\texttt{xyu}} \\
    ょ & {\sffamily ょ} & ョ & {\sffamily ョ} & ? & \textlightgrey{\texttt{lyo}/}\textred{\texttt{xyo}} \\
     & {\sffamily } & ー & {\sffamily ー} & \texttt{<}$^{\texttt{+1}}$ & \textred{\texttt{$-$} key} \\
    っ & {\sffamily っ} & ッ & {\sffamily ッ} & $^{\texttt{+1}}$\texttt{>} & \textlightgrey{\texttt{ltu}/\texttt{ltsu}/\textred{\texttt{xtsu}}/}repeat \texttt{>} \\* \bottomrule
    \caption{仮名 summary table. $^\dagger$Particle romanisation applies only for 平仮名.}
    \label{tbl:kana-summary} \\
\end{longtable}

\subsection{[Interlude] Morphemes, phonemes, phones}
\emph{This entire section is courtesy of SL.}

\textbf{Phonemes} are the smallest unit of mental representation of sound. They do not carry meaning by themselves, but they can alter the meaning pictured by the listener.

\textbf{Morphemes} are the smallest unit of meaning, and comprise two levels: a phonological level and a semantic level. The phonological level states how it is pronounced (a string of phonemes), and the semantic level states what meaning is attached to the phonology.

For instance, \ruby{日々}{ひ|び} contains a repetition of the same phoneme because the sound and meaning of the two 漢字 are identical. In contrast, \ruby{日日}{ひ|にち} contains two different morphemes, because the sound (and meaning) of the two 漢字 are different!

When determining whether a morpheme is repeated or not, consider the sound and meaning first before looking at the orthography. ``Orthography is truly an afterthought[, in the design of languages].''


\subsection{Iteration marks}
\emph{Read main article on \href{https://en.wikipedia.org/wiki/Iteration_mark\#Japanese}{Wikipedia}.}

Only the (horizontal) 漢字 iteration mark 々 is commonly used today. It is used to represent a \ul{duplicated character representing the same morpheme}. For example, \ruby{日々}{ひ|び} means ``daily, day after day''.

Writing 々 instead of repeating the 漢字 is preferred, provided that:
\begin{enumerate}[label=\arabic*.]
    \item (tl;dr: morpheme is repeated) the reading of the repeated 漢字 must be the same, though certain changes are permitted such as \emph{rendaku} (unvoiced consonant becomes voiced, i.e.\ the hakuten, e.g.\ in \ruby{人々}{ひと|びと}, ひ $\to$ び) and \emph{gemination} (consonant lengthening, i.e.\ the っ, e.g.\ in \ruby{刻々}{こっ|こく}), and
    \item the repetition must be within a single word/phrase.
\end{enumerate}
If the above aren't satisfied:
\begin{itemize}
    \item If repetition isn't repetition of the same morpheme, for disambiguation the second 漢字 is spelt out in 平仮名 (e.g.\ 日にち).
    \item If repetition crosses word boundaries, then the 漢字 is repeated (e.g.\ \ruby{民主主義}{みん|しゅ|しゅ|ぎ}, democracy).

    There are exceptions to this! 民主々義 is rarely used but exists. A notable exception is in the signages for neighbourhood associations 「〜\ruby{町内会}{ちょう|ない|かい}」. Because the name of neighbourhoods often end in 〜\ruby{町}{ちょう}, suffixing with 〜町内会 yields 〜\ruby{町町内会}{ちょう|ちょう|ない|かい}, which is then informally abbreviated to 〜町々内会, despite the repetition crossing a word boundary.

\end{itemize}

Intrepretations when 々 is used:
\begin{itemize}
    \item Reduplication (linguistics terminology) to indicate plurality

    \ruby{人々}{ひと|びと} (people)、\ruby{日々}{ひ|び} (daily/day after day)、\ruby{山々}{やま|やま} (mountains)
    \item Various alterations in meaning
    \begin{itemize}
        \item \ruby{個}{こ} (piece) $\to$ \ruby{個々}{こ|こ} (individually)
        \item \ruby{時}{とき} (time) $\to$ \ruby{時々}{とき|どき} (sometimes)
        \item \ruby{翌日}{よく|じつ} (next day, as in 隔天/隔一天 (CN)) $\to$ \ruby{翌々日}{よく|よく|じつ} (next next day, as in 隔两天 (CN))

        \emph{Note that 翌日 is not the same as \ruby[g]{明日}{あした}, just like how 隔天 and 明天 are used in different contexts in CN!}
    \end{itemize}
\end{itemize}

Repetition marks can be typed using commands in Table~\ref{tbl:miscellaneous-keyboard-commands}.
\begin{table}[h]
\centering
% \resizebox{\textwidth}{!}{%
% Help: \multicolumn{2}{c}{}, \multirow{2}{*}{}, cmidrule(l){3-5}
\begin{tabular}{@{}cccl@{}}
    \toprule
    \multicolumn{2}{c}{\multirow{2}{*}{}} & Keyboard & \multirow{2}{*}{Purpose} \\
    \multicolumn{2}{c}{}                  & command  & \\ \midrule
    ゝ & {\sffamily ゝ} & \texttt{onaji} $\to$ space$^\texttt{*}$ & 平仮名 previous character repeater (enforce without dakuten) \\
    ゞ & {\sffamily ゞ} & \texttt{onaji} $\to$ space$^\texttt{*}$ & 平仮名 previous character repeater (enforce with dakuten)\\
    ヽ & {\sffamily ヽ} & \texttt{onaji} $\to$ space$^\texttt{*}$ & 片仮名 previous character repeater (enforce without dakuten)\\
    ヾ & {\sffamily ヾ} & \texttt{onaji} $\to$ space$^\texttt{*}$ & 片仮名 previous character repeater (enforce with dakuten)\\
    々 & {\sffamily 々} & \texttt{noma} $\to$ space$^\texttt{*}$ & 漢字 previous character repeater (ノ$+$マ) \\
    % & {\sffamily } &  &  \\
    % & {\sffamily } &  &  \\
    % & {\sffamily } &  &  \\
    % & {\sffamily } &  &  \\
    % & {\sffamily } &  &  \\
    % & {\sffamily } &  &  \\
    % & {\sffamily } &  &  \\
    \bottomrule
\end{tabular}%
% }
\caption{Miscellaneous keyboard commands. Today, ゝ, ゞ, ヽ and ヾ only appear in proper names. As examples, じゝ $=$ じし and じゞ $=$ じじ.}
\label{tbl:miscellaneous-keyboard-commands}
\end{table}

\subsection{漢字}
Some preliminary notes:
\begin{itemize}
    \item There exists over $\SI{40000}{}$ 漢字 but only about $\SI{2000}{}$ account for $>95\%$ of characters actually used in written text.
    \item There are no spaces in Japanese, so 漢字 is necessary for distinguishing separate words within a sentence, and discriminating between homophones.
    \item Words that mean practically the same thing can have different 漢字 to distinguish nuances.

    Here's an example:
    \begin{itemize}
        \item \ruby{訊}{き}く means to ask.
        \item \ruby{聞}{き}く means to listen, or to ask.
        \item \ruby{聴}{き}く means to listen attentively. Preferred when talking about listening to music.
    \end{itemize}

    Another example:
    \begin{itemize}
        \item \ruby{見}{み}る means to see.
        \item \ruby{観}{み}る means to watch a movie.
    \end{itemize}

    Another example:
    \begin{itemize}
        \item \ruby{書}{か}く means to write.
        \item \ruby{描}{か}く means to draw.

        When depicting an abstract image (e.g.\ a scene in a book), we use \ruby{描}{えが}く.
    \end{itemize}

    Another example:
    \begin{itemize}
        \item The different pronuncations \ruby[g]{今日}{きょう}, \ruby{今日}{こん|にち} and \ruby{今日}{こん|じつ} are each preferred in different contexts.
    \end{itemize}
\end{itemize}

\subsection{Pronunciation} \label{sec:pronunciation}
It is not practical to memorise or attempt to logically create rules for pitches, especially since it can change depending on the context or the dialect. Even the intonations provided in dictionaries are there for guidance; they morph when used in different contexts.

The only practical approach is to get the general sense of pitches is by mimicking native Japanese speakers with careful listening and practice.

Some special notes:
\begin{itemize}
    \item In the modern 東京 dialect, ず and づ are pronounced exactly the same way: ``zu'', as expressed in their identical Hepburn romanisation (\S\ref{sec:hepburn-romanisation}).
    \item The native Japanese speaker will pronounce the ``v'' family (ヴ、ヴァ、ヴィ、ヴェ、ヴォ、ヴュ) as /b/.
    \item Vowel extensions (\S\ref{sec:vowel-extension}) are pronounced as vowel extensions; do not pronounce the extender if it's a different vowel! For example, \ruby{先生}{せん|せい} is pronounced \emph{sen-se} with an elongated trailing ``e'' vowel. There is no ``i'' vowel sound!
    \item Almost every 漢字 character has two different readings:
    \begin{itemize}
        \item \ruby{音読}{おん|よ}み: Chinese-derived. Used in compound 漢字 and idioms (both known as \ruby{熟語}{じゅく|ご}).
        \item \ruby{訓読}{くん|よ}み: native Japanese. Used in solo 漢字, solo 漢字 appended with \ruby{送}{おく}り仮名, adjectives and verbs.

        The purpose of trailing 送り仮名 is to preserve the 音読み pronunciation of the 漢字, even as the word is conjugated\footnote{\textbf{Conjugation}: change of word form to fit a given context.} to different forms. It is also used to differentiate transitive and intransitive verbs (\S\hl{???}).  % TODO
    \end{itemize}
    \item The actual readings of 漢字 can change slightly in compound words to make them easier to say (e.g.\ 一本 is いっぽん instead of いっほん).

    When repeating 漢字 using 々, \emph{rendaku} (unvoiced consonant becomes voiced, i.e.\ the hakuten, e.g.\ in 人々, ひ $\to$ び) and \emph{gemination} (consonant lengthening, i.e.\ the っ, e.g.\ \ruby{刻々}{こっ|こく}) \emph{may} occur.
\end{itemize}

\subsubsection{Vowel extension} \label{sec:vowel-extension}
Vowel extensions follow the rules in Table~\ref{tbl:vowel-extension}. For notes on pronunciation, see Section~\ref{sec:pronunciation}.

\begin{table}[h]
\centering
% \resizebox{\textwidth}{!}{%
% Help: \multirow{2}{*}{} & \multirow{2}{*}{} & \multicolumn{3}{c}{} \\ \cmidrule(l){3-5}
\begin{tabular}{@{}cccl@{}}
    \toprule
    \multirow{2.5}{*}{Vowel to extend} & \multicolumn{2}{c}{Extend by appending} & \multirow{2.5}{*}{Example} \\ \cmidrule(l){2-3}
    & 平仮名 & 片仮名 & \\ \midrule
    /a/ & あ & \multirow{5}{*}{ー} & お\underline{ばあ}さん、お\underline{かあ}さん\\
    /i/ & い & & \\
    /u/ & う & & \\
    /e/ & \textred{い} \textblue{(え)} & & \ruby{先生}{せん|せい}、\ruby{学生}{がく|せい}、\textblue{(お\underline{ねえ}さん)}\\
    /o/ & \textred{う} \textblue{(お)} & & き\underline{ょう}、おは\underline{よう}、\textblue{(\underline{おお}きい)}\\ \bottomrule
\end{tabular}%
% }
\caption{Vowel extension rules. Exceptions are bracketed in \textblue{blue}. /a/ is the phoneme representation.}
\label{tbl:vowel-extension}
\end{table}


\subsection{Hepburn romanisation} \label{sec:hepburn-romanisation}
\emph{Read main article on \href{https://en.wikipedia.org/wiki/Hepburn_romanization}{Wikipedia}.}

The official (as of Jan 2024) romanisation system of Japan. There are only a few rules.
\begin{description}
    \item[Vowel extension (\S\ref{sec:vowel-extension})] When vowels ``a'', ``e'', ``o'', ``u'' are extended \ul{as part of the same morpheme}, it is expressed with a macron (overbar), and the extender vowel is dropped. \textred{Extension of ``i'' and the ``e+i'' combination are exceptions: they remain repeated.}

    \begin{itemize}
        \item お\ruby{婆}{ばあ}さん obaasan $\to$ ob\=asan
        \item \textred{新潟 (city name) niigata}
        \item \ruby{数学}{すう|がく} suugaku $\to$ s\=ugaku
        \item お\ruby{姉}{ねえ}さん oneesan $\to$ on\=esan\\
        \textred{先生 sensei}
        \item \ruby{遠回}{とお|まわ}り toomawari $\to$ t\=omawari\\
        勉強 benkyou $\to$ benky\=o
    \end{itemize}

    This does not apply when the repetition crosses word boundaries or morpheme boundaries.

    \begin{itemize}
        \item \ruby{邪悪}{じゃ|あく} jaaku
        \item \ruby{灰色}{はい|いろ} haiiro

        Also for terminal adjectives (\hl{???}): いい ii
        \item \ruby{湖}{みずうみ} mizuumi

        Also for terminal verbs (\hl{???}): \ruby{食}{く}う kuu (eat)
        \item \ruby{濡}{ぬ}れ\ruby{縁}{えん} nureen (``open veranda (roofed hallway)'')
        \item \ruby{小躍}{こ|おど}り koodori (dance of joy)

        \ruby{仔牛}{こ|うし} koushi (calf)

        Also for terminal verbs (\hl{???}): \ruby{迷う}{まよ|う} mayou (to get lost)
    \end{itemize}
    \item[片仮名 loanwords] The macron is used iff ー is used to extend a vowel.
    \item[Japanese words adopted into English] Common place names like Tokyo, Kyoto and Osaka, while properly romanised as t\=oky\=o, ky\=oto and \=osaka, are simply romanised as Tokyo, Kyoto and Osaka.
    \item[Particles] When は、へ、を are used as particles, they are romanised as wa, e and o respectively.
    \item[Syllabic ん] ん is romanised as n' (with the apostrophe) if appearing immediately before any lone vowels or ``y''. This is to disambiguate んあ、んい、んう、んえ、んお、んや、んゆ、んよ (n'a, n'i, n'u, n'e, n'o, n'ya, n'yu, n'yo) from な、に、ぬ、ね、の、にゃ、にゅ、にょ (na, ni, nu, ne, no, nya, nyu, nyo) respectively.

    Examples: \ruby{簡易}{かん|に} kan'i (simple), \ruby{信用}{しん|よう} shin'y\=o (trust).
    \item[Geminated consonants (っ、ッ)] Double the next consonant, except if ``ch'' is repeated: in that case we use ``tch'' instead of ``cch''.

    Examples: \ruby{抹茶}{まっ|ちゃ} maccha $\to$ matcha, こっち kocchi $\to$ kotchi
\end{description}

\end{document}
