\documentclass[../nihongo-gakushuu-kyouzai.tex]{subfiles}
\begin{document}
\appendix
\setcounter{section}{2}
\setcounter{subsection}{1}

\subsection{\ruby{普通名詞}{ふ|つう|めい|し} (nouns)}

\subsubsection{Meta: Japanese}
% Help: \SetCell[r=2,c=2]{c,m} <content>, \cmidrule[l]{3-4}
% Help: colspec: X[ratio, horizontal alignment] columns grow to fit width=\linewidth
%                  negative ratios: shrink to fit content and may not grow to full ratio
% Help: colspec: l/c/r columns do not grow
\longtabse[0.75]  % scale factor
{Nouns: meta: Japanese.}  % caption
{tbl:appendix-vocab-nouns-meta-language}  % label
{}  % outer specification options
{
    colspec={X[-3,l]X[3,l]X[-3,l]},
    rowhead=1,
    % width=\linewidth,  % useful only with X columns
}  % inner specification options
{
    \toprule
    \textbf{Name} & \textbf{Meaning} & \textbf{Notes} \\
    \midrule
    \ruby{普通名詞}{ふ|つう|めい|し} & common noun & \\
    \ruby{同義語}{どう|ぎ|ご} & synonym & \\
    \ruby{同意語}{どう|い|ご} & synonym & \\
    \ruby{類語}{るい|ご} & quasi-synonym (similar meaning but not interchangeable) & \\
    \ruby{対義語}{たい|ぎ|ご} & antonym & \\
    \ruby{反義語}{はん|ぎ|ご} & antonym & \\
    \ruby{反意語}{はん|い|ご} & antonym & \\
    \ruby{反対語}{はん|たい|ご} & antonym & \\
    % & & \\
    \midrule
    \ruby{主語}{しゅ|ご} & subject & \\
    % & & \\
    \midrule
    \ruby{熟語}{じゅく|ご} & compound kanji/idiom & \\
    \ruby{四字熟語}{よ|じ|じゅく|ご} & four-character compound word (esp.\ idiomatic) & \\
    % & & \\
    \midrule
    \ruby{形容詞}{けい|よう|し} & adjective/い-adjective & \\
    \ruby{形容動詞}{けい|よう|どう|し} & adjectival noun/nominal adjective/quasi-adjective/な-adjective & \\
    \ruby{連体詞}{れん|たい|し} & pre-noun adjectival/adnominal adjective & \\
    % & & \\
    \midrule
    \ruby{動詞}{どう|し} & verb & \\
    \ruby{五段動詞}{ご|だん|どう|し} & う-verb & \\
    \ruby{一段動詞}{いち|だん|どう|し} & る-verb & \\
    \ruby{他動詞}{た|ど|うし} & transitive verb & \\
    \ruby{自動詞}{じ|どう|し} & intransitive verb & \\
    % & & \\
    \midrule
    \ruby{動作主}{どう|さ|しゅ} & agent/performer of an action & \\
    % & & \\
    \midrule
    \ruby{能動態}{のう|どう|たい} & active voice & \\
    \ruby{受動態}{じゅ|どう|たい} & passive voice & \\
    \ruby{受身形}{うけ|み|けい} & passive form & \\
    \ruby{直接受身}{ちょく|せつ|うけ|み} & direct passive & \\
    \ruby{間接受身}{かん|せつ|うけ|み} & indirect passive & \\
    % & & \\
    \midrule
    \ruby{過去形}{か|こ|けい} & past tense & \\
    % & & \\
    \midrule
    \ruby{副詞}{ふく|し} & adverb & \\
    % & & \\
    \midrule
    \ruby{変化}{へん|か} & inflection/conjugation & also a verb, also in Table~\ref{tbl:appendix-vocab-nouns-production} \\
    % & & \\
    \midrule
    \midrule
    \ruby{助詞}{じょ|し} & particle & \\
    % & & \\
    \midrule
    \ruby{接続詞}{せつ|ぞく|し} & conjunction & \\
    % & & \\
    \midrule
    \ruby{感動詞}{かん|どう|し} & interjection & \\
    % & & \\
    \midrule
    \midrule
    \ruby{丁寧語}{てい|ねい|ご} & polite language (e.g.\ ます、です) & \\
    \ruby{尊敬語}{そん|けい|ご} & honorific language & \\
    \ruby{謙譲語}{けん|じょう|ご} & humble language (e.g.\ itadaku) & \\
    \ruby{言}{い}い\ruby{方}{かた} & phrasing/language/wording/way of saying something & \\
    % & & \\
    \midrule
    \ruby{例文}{れい|ぶん} & example sentence & \\
    % & & \\
    \midrule
    \midrule
    \ruby{和語}{わ|ご} & Japanese language & \\
    \ruby{漢語}{かん|ご} & Japanese word of Chinese origin/Sino-Japanese word & \\
    \ruby{外来語}{がい|らい|ご} & loanword in Japanese (esp.\ those of Western origin) & \\
    % & & \\
    \midrule
    \midrule
    \ruby{○}{まる}/\ruby{丸}{まる} & ``correct''/``good''/circle & \\
    \ruby{二重丸}{に|じゅう|まる} & ``very good''/double circle & \\
    \ruby{△}{さんかく}/\ruby{三角}{さん|かく} & not entirely wrong but not entirely right/so-so/average/triangle & \\
    \ruby{×}{ばつ}/バツ/\ruby{罰点}{ばっ|てん} & cross mark/``incorrect'' & \\
    % & & \\
    \bottomrule
}


\subsubsection{Grammatical}
% Help: \SetCell[r=2,c=2]{c,m} <content>, \cmidrule[l]{3-4}
% Help: colspec: X[ratio, horizontal alignment] columns grow to fit width=\linewidth
%                  negative ratios: shrink to fit content and may not grow to full ratio
% Help: colspec: l/c/r columns do not grow
\longtabse[0.75]  % scale factor
{Nouns: grammatical.}  % caption
{tblappendix-vocab-nouns-grammatical}  % label
{}  % outer specification options
{
    colspec={X[-3,l]X[3,l]X[-3,l]},
    rowhead=1,
    % width=\linewidth,  % useful only with X columns
}  % inner specification options
{
    \toprule
    \textbf{Name} & \textbf{Meaning} & \textbf{Notes} \\
    \midrule
    \ruby{物}{もの} & thing/object & \\
    こと & thing/matter & (\ruby{事}{こと}) \\
    <to nominalise>こと & nominalising suffix & \suffix \\
    <noun>こと<alias> & <noun>, also known as <alias>, ... & \suffix \\
    <advice>ことだ & you should/it's important to <advice> & \suffix \\
    % & & \\
    \bottomrule
}


\subsubsection{Physical}
% Help: \SetCell[r=2,c=2]{c,m} <content>, \cmidrule[l]{3-4}
% Help: colspec: X[ratio, horizontal alignment] columns grow to fit width=\linewidth
%                  negative ratios: shrink to fit content and may not grow to full ratio
% Help: colspec: l/c/r columns do not grow
\longtabse[0.75]  % scale factor
{Nouns: physical.}  % caption
{tblappendix-vocab-nouns-physical}  % label
{}  % outer specification options
{
    colspec={X[-3,l]X[3,l]X[-3,l]},
    rowhead=1,
    % width=\linewidth,  % useful only with X columns
}  % inner specification options
{
    \toprule
    \textbf{Name} & \textbf{Meaning} & \textbf{Notes} \\
    \midrule
    ため\ruby{息}{いき} & sigh & \\
    ビンタ & slap (in the face) & also a verb \\
    \ruby{爆発}{ばく|はつ} & explosion/detonation/blast/blowing up/eruption (also of emotion) & also a verb \\
    % & & \\
    \bottomrule
}


\subsubsection{Directions}
% Help: \SetCell[r=2,c=2]{c,m} <content>, \cmidrule[l]{3-4}
% Help: colspec: X[ratio, horizontal alignment] columns grow to fit width=\linewidth
%                  negative ratios: shrink to fit content and may not grow to full ratio
% Help: colspec: l/c/r columns do not grow
\longtabse[0.75]  % scale factor
{Nouns: directions.}  % caption
{tbl:appendix-vocab-nouns-directions}  % label
{}  % outer specification options
{
    colspec={X[-3,l]X[3,l]X[-3,l]},
    rowhead=1,
    % width=\linewidth,  % useful only with X columns
}  % inner specification options
{
    \toprule
    \textbf{Name} & \textbf{Meaning} & \textbf{Notes} \\
    \midrule
    \ruby{上}{うえ} & up & \\
    \ruby{下}{した} & down & \\
    \ruby{左}{ひだり} & left & \\
    \ruby{右}{みぎ} & right & \\
    \ruby{左折}{さ|せつ} & left turn & also a verb \\
    \ruby{右折}{う|せつ} & right turn & also a verb \\
    \ruby{後}{うし}ろ & back/behind/rear (physical) & \href{https://ja.hinative.com/questions/4072942}{[HN]} \\
    \ruby{後}{あと} & behind (esp.\ abstract)/after/later & \href{https://ja.hinative.com/questions/4072942}{[HN]} \\
    \ruby{中}{なか} & inside; middle/centre & \\
    〜\ruby{中}{ちゅう} & during/in the middle of/while (something happening) & \suffix \\
    % & & \\
    \midrule
    \midrule
    \ruby{上}{のぼ}り & upwards/upbound/ascent & \\
    \ruby{下}{くだ}り & downwards/downbound/descent & \\
    \ruby{戻}{もど}り & return/backwards; return (computing) & \\
    % & & \\
    \midrule
    \midrule
    \ruby{遠回}{とお|まわ}り & detour/roundabout way & \\
    % & & \\
    \midrule
    \midrule
    \ruby{昇進}{しょう|しん} & promotion/rise in rank & e.g.\ workplace: \ruby{昇進人事}{しょう|しん|じん|じ}; also a verb \\
    \ruby{格上}{かく|あ}げ & status upgrade/promotion & e.g.\ friendship status \\
    \ruby{降格}{こう|かく} & demotion/drop in rank & e.g.\ workplace: \ruby{降格人事}{こう|かく|じん|じ} \\
    \ruby{格下}{かく|さ}げ & status downgrade/demotion & e.g.\ friendship status  \\
    % & & \\
    \bottomrule
}


\subsubsection{Navigation}
% Help: \SetCell[r=2,c=2]{c,m} <content>, \cmidrule[l]{3-4}
% Help: colspec: X[ratio, horizontal alignment] columns grow to fit width=\linewidth
%                  negative ratios: shrink to fit content and may not grow to full ratio
% Help: colspec: l/c/r columns do not grow
\longtabse[0.75]  % scale factor
{Nouns: navigation.}  % caption
{tbl:appendix-vocab-nouns-navigation}  % label
{}  % outer specification options
{
    colspec={X[-3,l]X[3,l]X[-3,l]},
    rowhead=1,
    % width=\linewidth,  % useful only with X columns
}  % inner specification options
{
    \toprule
    \textbf{Name} & \textbf{Meaning} & \textbf{Notes} \\
    \midrule
    \ruby{立}{た}ち\ruby{入}{い}り & the act of entering & \\
    \ruby{禁止}{きん|し} & prohibition/ban & also: バン/BAN \\
    \ruby{立}{た}ち\ruby{入}{い}り\ruby{検査}{けん|さ} & on-the-spot/on-site inspection & \\
    \ruby{立}{た}ち\ruby{入}{い}り\ruby{禁止}{きん|し} & no entry/no trespassing/keep out/off-limits & \\
    \ruby{止}{と}まれ & stop (road signage) & \\
    % & & \\
    \midrule
    \midrule
    \ruby{始発}{し|はつ} & first departure/train/bus (of the day) & \\
    \ruby{終発}{しゅう|はつ} & last departure/train/bus (of the day) & \href{https://ja.wikipedia.org/wiki/\%E7\%B5\%82\%E9\%9B\%BB}{[Wiki]} \\
    \ruby{最後列車}{さい|ご|れっ|しゃ} & last train (of the day) & \href{https://ja.wikipedia.org/wiki/\%E7\%B5\%82\%E9\%9B\%BB}{[Wiki]} \\
    \ruby{最終電車}{さい|しゅう|でん|しゃ} & last train (of the day) & \href{https://ja.wikipedia.org/wiki/\%E7\%B5\%82\%E9\%9B\%BB}{[Wiki]} \\
    \ruby{終車}{しゅう|しゃ} & last train/bus (of the day) & \href{https://ja.wikipedia.org/wiki/\%E7\%B5\%82\%E9\%9B\%BB}{[Wiki]} \\
    \ruby{終電}{しゅう|でん} & last train (of the day) & \href{https://ja.wikipedia.org/wiki/\%E7\%B5\%82\%E9\%9B\%BB}{[Wiki]} \\
    % & & \\
    \midrule
    \midrule
    \ruby{急行}{きゅう|こう} & the act of hurrying/rushing to somewhere & also a verb \\
    \ruby{緩行}{かん|こう} & the act of going slowly to somewhere & also a verb \\
    % & & \\
    \midrule
    \ruby{普通列車}{ふ|つう|れっ|しゃ} & local train (stops at every station) & \\
    % & & \\
    \midrule
    \ruby{快速列車}{かい|そく|れっ|しゃ} & rapid train & \\
    \ruby{快速}{かい|そく} & rapid train (abbreviated) & \\
    % & & \\
    \midrule
    \ruby{急行列車}{きゅう|こう|れっ|しゃ} & express train & \\
    \ruby{急行}{きゅう|こう} & express train (abbreviated) & \\
    % & & \\
    \midrule
    \ruby{特別急行列車}{とく|べつ|きゅう|こう|れっ|しゃ} & limited express train & \\
    \ruby{特急列車}{とっ|きゅう|れっ|しゃ} & limited express train (abbreviated) & \\
    \ruby{特急}{とっ|きゅう} & limited express train (abbreviated) & \\
    % & & \\
    \bottomrule
}


\subsubsection{Places}
% Help: \SetCell[r=2,c=2]{c,m} <content>, \cmidrule[l]{3-4}
% Help: colspec: X[ratio, horizontal alignment] columns grow to fit width=\linewidth
%                  negative ratios: shrink to fit content and may not grow to full ratio
% Help: colspec: l/c/r columns do not grow
\longtabse[0.75]  % scale factor
{Nouns: places.}  % caption
{tbl:appendix-vocab-nouns-places}  % label
{}  % outer specification options
{
    colspec={X[-3,l]X[3,l]X[-3,l]},
    rowhead=1,
    % width=\linewidth,  % useful only with X columns
}  % inner specification options
{
    \toprule
    \textbf{Name} & \textbf{Meaning} & \textbf{Notes} \\
    \midrule
    \ruby{町}{ちょう}/\ruby{町}{まち} & street/neighbourhood & \href{https://ja.hinative.com/questions/17979737}{[HN1]}, \href{https://hinative.com/questions/20251204}{[HN2]} \\
    \ruby{丁目}{ちょう|め} & district of a town/city block & \\
    \ruby{町内会}{ちょう|ない|かい} & neighbourhood association & \\
    ビル & (multi-floor) building & \\
    レストラン & restaurant (esp.\ Western) & \\
    \ruby{図書館}{と|しょ|かん} & library & \\
    \ruby{銀行}{ぎん|こう} & bank & \\
    \ruby{公園}{こう|えん} & public park & \\
    \ruby{高速道路}{こう|そく|どう|ろ} & highway/expressway & \\
    \ruby{警察署}{けい|さつ|しょ} & police station & \\
    \ruby{食堂}{しょく|どう} & canteen/cafeteria/dining room/restaurant/eatery & \\
    % & & \\
    \midrule
    \midrule
    \ruby{扉}{とびら} & door/gate/opening & also: ドア \\
    \ruby{橋}{はし} & bridge & \\
    % & & \\
    \midrule
    \midrule
    \ruby{家}{いえ} & house (physical entity) & neutral, \href{https://japanese.stackexchange.com/questions/3726/what-is-the-difference-between-\%E3\%81\%84\%E3\%81\%88-and-\%E3\%81\%86\%E3\%81\%A1}{[SE]} \\

    [お]うち & one's home/house/family/household (of speaker, by default) & \href{https://japanese.stackexchange.com/questions/3726/what-is-the-difference-between-\%E3\%81\%84\%E3\%81\%88-and-\%E3\%81\%86\%E3\%81\%A1}{[SE]}, also a pronoun, [honorific] \\
    \ruby{部屋}{へ|や} & room & \\
    \ruby{台所}{だい|どころ} & kitchen & also: キッチン \\
    % & & \\
    \midrule
    \midrule
    \ruby{学校}{がっ|こう} & school & \\
    \ruby{小学}{しょう|がく} & elementary/primary school & \\
    \ruby{中学}{ちゅう|がく} & junior high/middle/lower secondary school & \\
    \ruby{高校}{こう|こう} & senior high school & \\
    \ruby{大学}{だい|がく} & university/college & \\
    % & & \\
    \midrule
    \ruby{学園}{がく|えん} & campus & \\
    \ruby{教室}{きょう|しつ} & classroom/lecture room; university department; class/course; school (for specific discipline) & \\
    % & & \\
    \bottomrule
}


\subsubsection{Vehicles}
% Help: \SetCell[r=2,c=2]{c,m} <content>, \cmidrule[l]{3-4}
% Help: colspec: X[ratio, horizontal alignment] columns grow to fit width=\linewidth
%                  negative ratios: shrink to fit content and may not grow to full ratio
% Help: colspec: l/c/r columns do not grow
\longtabse[0.75]  % scale factor
{Nouns: vehicles.}  % caption
{tbl:appendix-vocab-nouns-vehicles}  % label
{}  % outer specification options
{
    colspec={X[-3,l]X[3,l]X[-3,l]},
    rowhead=1,
    % width=\linewidth,  % useful only with X columns
}  % inner specification options
{
    \toprule
    \textbf{Name} & \textbf{Meaning} & \textbf{Notes} \\
    \midrule
    \ruby{車}{くるま} & car/vehicle & \\
    \ruby{救急車}{きゅう|きゅう|しゃ} & ambulance & \\
    \ruby{列車}{れっ|しゃ} & train/railway train & \\
    \ruby{電車}{でん|しゃ} & electric train & \\
    % & & \\
    \bottomrule
}


\subsubsection{Furniture}
% Help: \SetCell[r=2,c=2]{c,m} <content>, \cmidrule[l]{3-4}
% Help: colspec: X[ratio, horizontal alignment] columns grow to fit width=\linewidth
%                  negative ratios: shrink to fit content and may not grow to full ratio
% Help: colspec: l/c/r columns do not grow
\longtabse[0.75]  % scale factor
{Nouns: furniture.}  % caption
{tbl:appendix-vocab-nouns-furniture}  % label
{}  % outer specification options
{
    colspec={X[-3,l]X[3,l]X[-3,l]},
    rowhead=1,
    % width=\linewidth,  % useful only with X columns
}  % inner specification options
{
    \toprule
    \textbf{Name} & \textbf{Meaning} & \textbf{Notes} \\
    \midrule
    \ruby{椅子}{い|す} & chair/stool & \\
    % & & \\
    \bottomrule
}


\subsubsection{Organisms}
% Help: \SetCell[r=2,c=2]{c,m} <content>, \cmidrule[l]{3-4}
% Help: colspec: X[ratio, horizontal alignment] columns grow to fit width=\linewidth
%                  negative ratios: shrink to fit content and may not grow to full ratio
% Help: colspec: l/c/r columns do not grow
\longtabse[0.75]  % scale factor
{Nouns: organisms.}  % caption
{tbl:appendix-vocab-nouns-organisms}  % label
{}  % outer specification options
{
    colspec={X[-3,l]X[3,l]X[-3,l]},
    rowhead=1,
    % width=\linewidth,  % useful only with X columns
}  % inner specification options
{
    \toprule
    \textbf{Name} & \textbf{Meaning} & \textbf{Notes} \\
    \midrule
    \ruby{猫}{ねこ} & cat & \\
    \ruby{子猫}{こ|ねこ} & kitten & \\
    \ruby{犬}{いぬ} & dog; spy/loser & \\
    \ruby{子犬}{こ|いぬ} & puppy & \\
    \ruby{魚}{さかな} & fish & \\
    \ruby{人間}{にん|げん} & human being/humankind & \\
    ムカデ & centipede & (\ruby[g]{百足}{むかで}); also a counter for shoes/socks \\
    & & \\
    % & & \\
    \bottomrule
}


\subsubsection{Food}
% Help: \SetCell[r=2,c=2]{c,m} <content>, \cmidrule[l]{3-4}
% Help: colspec: X[ratio, horizontal alignment] columns grow to fit width=\linewidth
%                  negative ratios: shrink to fit content and may not grow to full ratio
% Help: colspec: l/c/r columns do not grow
\longtabse[0.75]  % scale factor
{Nouns: food.}  % caption
{tbl:appendix-vocab-nouns-food}  % label
{}  % outer specification options
{
    colspec={X[-3,l]X[3,l]X[-3,l]},
    rowhead=1,
    % width=\linewidth,  % useful only with X columns
}  % inner specification options
{
    \toprule
    \textbf{Name} & \textbf{Meaning} & \textbf{Notes} \\
    \midrule
    ご\ruby{飯}{はん} & cooked rice/meal & \\
    \ruby{朝}{あさ}ご\ruby{飯}{はん} & breakfast & \\
    \ruby{昼}{ひる}ご\ruby{飯}{はん} & lunch & \\
    \ruby{晩}{ばん}ご\ruby{飯}{はん} & dinner & \\
    % & & \\
    \midrule
    \midrule
    \ruby{食}{た}べ\ruby{物}{もの} & food & \\
    \ruby{料理}{りょう|り} & cooking/cuisine/dish & also a verb \\
    \ruby{肉}{にく} & meat & \\
    \ruby{果物}{くだ|もの} & fruit & \\
    \ruby{野菜}{や|さい} & vegetable & also: ベジタブル \\
    \ruby{卵}{たまご} & eggs/egg/roe & \\
    % & & \\
    \midrule
    \midrule
    \ruby{飲}{の}み & the act of drinking & \\
    \ruby{飲}{の}み\ruby{物}{もの} & beverage & \\
    \ruby{抹茶}{まっ|ちゃ} & matcha, powdered green tea & \\
    ジュース & soft drink (usually fruit-based)/sweet drink/juice & \\
    % & & \\
    \midrule
    \midrule
    \ruby{丼}{どんぶり}/\ruby{丼}{どん} & porcelain bowl/meat served over rice & \\
    そば & buckwheat/buckwheat noodles & (\ruby{蕎麦}{そ|ば}) \\

    [お]\ruby{弁当}{べん|とう} & Japanese box lunch & \\

    [お]だし & dashi (Japanese soup stock made from fish and kelp) & ([お]\ruby{出汁}{だ|し}); [polite] \\
    \ruby{味噌汁}{み|そ|しる} & miso soup & \\
    \ruby{飴}{あめ} & (hard) candy & \\
    % & & \\
    \bottomrule
}


\subsubsection{Small objects: stationery}
% Help: \SetCell[r=2,c=2]{c,m} <content>, \cmidrule[l]{3-4}
% Help: colspec: X[ratio, horizontal alignment] columns grow to fit width=\linewidth
%                  negative ratios: shrink to fit content and may not grow to full ratio
% Help: colspec: l/c/r columns do not grow
\longtabse[0.75]  % scale factor
{Nouns: small objects: stationery.}  % caption
{tbl:appendix-vocab-nouns-small-objects-stationery}  % label
{}  % outer specification options
{
    colspec={X[-3,l]X[3,l]X[-3,l]},
    rowhead=1,
    % width=\linewidth,  % useful only with X columns
}  % inner specification options
{
    \toprule
    \textbf{Name} & \textbf{Meaning} & \textbf{Notes} \\
    \midrule
    \ruby{書}{しょ} & book/document & \\
    \ruby{本}{ほん} & book/volume/script & \\
    \ruby{帳}{ちょう} & book/register & \\
    \ruby{紙}{かみ} & paper & \\
    \ruby{手紙}{て|がみ} & letter/mail & \\
    \ruby{葉書}{は|がき} & postcard & \\
    メモ & memo/note & \\
    ノート & note/notebook/exercise book/laptop computer & also a verb \\
    % & & \\
    \midrule
    \midrule
    \ruby{辞書}{じ|しょ} & dictionary & \\
    \midrule
    \midrule
    \ruby{計算機}{けい|さん|き} & calculator & \\
    % & & \\
    \bottomrule
}


\subsubsection{Date}
% Help: \SetCell[r=2,c=2]{c,m} <content>, \cmidrule[l]{3-4}
% Help: colspec: X[ratio, horizontal alignment] columns grow to fit width=\linewidth
%                  negative ratios: shrink to fit content and may not grow to full ratio
% Help: colspec: l/c/r columns do not grow
\longtabse[0.75]  % scale factor
{Nouns: date.}  % caption
{tbl:appendix-vocab-nouns-date}  % label
{}  % outer specification options
{
    colspec={X[-3,l]X[3,l]X[-3,l]},
    rowhead=1,
    % width=\linewidth,  % useful only with X columns
}  % inner specification options
{
    \toprule
    \textbf{Name} & \textbf{Meaning} & \textbf{Notes} \\
    \midrule
    \ruby{日}{ひ}にち & (referring to) the date of an event & 「<event>の日にち」 \\
    \ruby{毎日}{まい|にち} & every day & also an adverb, \href{https://ja.hinative.com/questions/24476486}{[HN]} \\
    \ruby{日々}{ひ|び} & day after day & also an adverb, \href{https://ja.hinative.com/questions/24476486}{[HN]} \\
    % & & \\
    \midrule
    \midrule
    \ruby{週末}{しゅう|まつ} & weekend & \\
    % & & \\
    \midrule
    \midrule
    \ruby{先日}{せん|じつ} & the other day/a few days ago & also an adverb \\
    \ruby[g]{一昨日}{おととい} & day before yesterday (from now) & also an adverb \\
    \ruby[g]{昨日}{きのう} & yesterday (from now) & also an adverb \\
    \ruby[g]{今日}{きょう} & today (now) & also an adverb; \href{https://dictionary.goo.ne.jp/thsrs/12857/meaning/m0u/}{[goo]} \\
    \ruby{本日}{ほん|じつ} & today (now) & also an adverb; formal, \href{https://dictionary.goo.ne.jp/thsrs/12857/meaning/m0u/}{[goo]} \\
    \ruby[g]{明日}{あした} & tomorrow (from now) & also an adverb \\
    \ruby[g]{明後日}{あさって} & day after tomorrow (from now) & also an adverb \\
    % & & \\
    \midrule
    \ruby{前々日}{ぜん|ぜん|じつ} & two days before (an event) & also an adverb \\
    \ruby{前日}{ぜん|じつ} & the day before (an event) & also an adverb \\
    \ruby{当日}{とう|じつ} & the day (of an event) & also an adverb \\
    \ruby{翌日}{よく|じつ} & the day after (an event) & also an adverb \\
    \ruby{翌々日}{よく|よく|じつ} & two days later (an event) & also an adverb \\
    % & & \\
    \midrule
    \midrule
    \ruby{先々週}{せん|せん|しゅう} & two weeks ago (from now) & also an adverb, \href{https://ja.hinative.com/questions/15897169}{[HN]} \\
    \ruby{先週}{せん|しゅう} & last week (from now) & also an adverb, \href{https://ja.hinative.com/questions/15897169}{[HN]} \\
    \ruby{今週}{こん|しゅう} & this week (now) & also an adverb \\
    \ruby{今週末}{こん|しゅう|まつ} & this weekend & \\
    \ruby{来週}{らい|しゅう} & next week (from now) & also an adverb, \href{https://www.goodcross.com/words/22234-2020}{[GC]} \\
    \ruby{來々週}{らい|らい|しゅう} & two weeks later (from now) & \\
    \ruby{次週}{じ|しゅう} & next week (recurring event, e.g.\ TV) & also an adverb, \href{https://www.goodcross.com/words/22234-2020}{[GC]} \\
    % & & \\
    \midrule
    \ruby{前週}{ぜん|しゅう} & the week before (an event) & also an adverb, \href{https://ja.hinative.com/questions/15897169}{[HN]} \\
    \ruby{当週}{とう|しゅう} & the week (of an event) & also an adverb \\
    \ruby{翌週}{よく|しゅう} & the week after (an event) & also an adverb, \href{https://www.goodcross.com/words/22234-2020}{[GC]} \\
    \ruby{翌々週}{よく|よく|しゅう} & two weeks after (an event) & also an adverb, \href{https://www.goodcross.com/words/22234-2020}{[GC]} \\
    % & & \\
    \midrule
    \midrule
    \ruby{先々月}{せん|せん|げつ} & two months ago (from now) & also an adverb \\
    \ruby{先月}{せん|げつ} & last month (from now) & also an adverb \\
    \ruby{今月}{こん|げつ} & this month (now) & also an adverb \\
    \ruby{来月}{らい|げつ} & next month (from now) & also an adverb \\
    % & & \\
    \midrule
    \ruby{前月}{ぜん|げつ} & the month before (an event) & also an adverb \\
    \ruby{当月}{とう|げつ} & the month (of an event) & also an adverb \\
    \ruby{翌月}{よく|げつ} & the month after (an event) & also an adverb \\
    \ruby{翌々月}{よく|よく|げつ} & two months after (an event) & also an adverb \\
    % & & \\
    \midrule
    \midrule
    \ruby[g]{一昨年}{おととし} & two years ago (from now) & casual, also an adverb, \href{https://dictionary.goo.ne.jp/thsrs/12818/meaning/m1u/}{[goo]} \\
    \ruby{去年}{きょ|ねん} & last year (from now) & casual, also an adverb, \href{https://dictionary.goo.ne.jp/thsrs/12818/meaning/m1u/}{[goo]} \\
    \ruby{一昨年}{いっ|さく|ねん} & two years ago (from now) & formal, also an adverb, \href{https://dictionary.goo.ne.jp/thsrs/12818/meaning/m1u/}{[goo]}, \href{https://ja.hinative.com/questions/666233}{[HN]} \\
    \ruby{昨年}{さく|ねん} & last year (from now) & formal, also an adverb, \href{https://dictionary.goo.ne.jp/thsrs/12818/meaning/m1u/}{[goo]} \\
    \ruby[g]{今年}{ことし} & this year (from now) & also an adverb \\
    \ruby{来年}{らい|ねん} & next year (from now) & also an adverb \\
    \ruby{明年}{みょう|ねん} & next year (from now) & formal, also an adverb \\
    \ruby{再来年}{さ|らい|ねん} & two years later (from now) & also an adverb \\
    % & & \\
    \midrule
    \ruby{前々年}{ぜん|ぜん|ねん} & two years before (an event) & \\
    \ruby{前年}{ぜん|ねん} & the year before (an event) & also an adverb \\
    \ruby{当年}{とう|ねん} & the year (of an event) & also an adverb \\
    \ruby{翌年}{よく|ねん} & the year after (an event) & also an adverb \\
    \ruby{翌々年}{よく|よく|ねん} & two years after (an event) & also an adverb \\
    % & & \\
    \midrule
    \midrule
    \ruby{春}{はる} & spring & \\
    \ruby{夏}{なつ} & summer & \\
    \ruby{秋}{あき} & autumn & \\
    \ruby{冬}{ふゆ} & winter & \\
    % & & \\
    \midrule
    \midrule
    \ruby{誕生日}{たん|じょう|び} & birthday & \\
    % & & \\
    \bottomrule
}


\subsubsection{Time}
% Help: \SetCell[r=2,c=2]{c,m} <content>, \cmidrule[l]{3-4}
% Help: colspec: X[ratio, horizontal alignment] columns grow to fit width=\linewidth
%                  negative ratios: shrink to fit content and may not grow to full ratio
% Help: colspec: l/c/r columns do not grow
\longtabse[0.75]  % scale factor
{Nouns: time.}  % caption
{tbl:appendix-vocab-nouns-time}  % label
{}  % outer specification options
{
    colspec={X[-3,l]X[3,l]X[-3,l]},
    rowhead=1,
    % width=\linewidth,  % useful only with X columns
}  % inner specification options
{
    \toprule
    \textbf{Name} & \textbf{Meaning} & \textbf{Notes} \\
    \midrule
    \ruby{時間}{じ|かん} & time (concept)/class period & \\
    % & & \\
    \midrule
    \midrule
    \ruby{時}{とき} & time/hour of day/moment (points to specific instant) & e.g.\ 「16\ruby{歳}{さい}の\ruby{時}{とき}私は\dots」\\
    \ruby{刻}{とき} & (referring to) time of day & \\
    \ruby{秋}{とき} & important time & also: \ruby{秋}{あき} \\
    % & & \\
    \midrule
    \ruby{時々}{とき|どき} & sometimes/occasionally & \\
    % & & \\
    \midrule
    \midrule
    \ruby{今}{いま} & now/immediately & also an adverb \\
    \ruby{今後}{こん|ご} & from now on (ongoing event) & also an adverb, \href{https://dictionary.goo.ne.jp/thsrs/15272/meaning/m1u/}{[goo]} \\
    % & & \\
    \midrule
    \ruby{朝}{あさ} & morning & also an adverb, \href{https://ja.hinative.com/questions/18618956}{[HN]} \\
    \ruby[g]{今朝}{けさ} & this morning & \\
    % & & \\
    \midrule
    \ruby{午前}{ご|ぜん} & before noon/ante meridian (a.m.) & also an adverb, \href{https://ja.hinative.com/questions/18618956}{[HN]} \\
    \ruby{昼}{ひる} & noon & also an adverb\\
    \ruby{午後}{ご|ご} & afternoon/after noon/post meridian (p.m.) & also an adverb \\
    \ruby[g]{今日}{きょう}の\ruby{午後}{ご|ご} & this afternoon & \\
    % & & \\
    \midrule
    \ruby{夕方}{ゆう|がた} & evening/dusk & also an adverb \\
    \ruby{晩}{ばん} & evening/night & also an adverb, \href{https://ja.hinative.com/questions/13398881}{[HN]}, \href{https://dictionary.goo.ne.jp/thsrs/12925/meaning/m1u/}{[goo]} \\
    \ruby{今晩}{こん|ばん} & this evening/tonight & \\
    \ruby{夜}{よる} & evening/night (slightly formal) & also an adverb, \href{https://ja.hinative.com/questions/13398881}{[HN]}, \href{https://dictionary.goo.ne.jp/thsrs/12925/meaning/m1u/}{[goo]} \\
    \ruby{今夜}{こん|や} & this evening/tonight (slightly formal) & \\
    % & & \\
    \midrule
    \midrule
    \ruby{朝日}{あさ|ひ} & morning sun/rising sun (the event) & \href{https://ja.hinative.com/questions/20406767}{[HN]} \\
    \ruby{日}{ひ}の\ruby{出}{で} & sunrise (the moment it rises) & \href{https://ja.hinative.com/questions/20406767}{[HN]} \\
    \ruby{夕日}{ゆう|ひ} & evening sun/setting sun (the event) & \href{https://ja.hinative.com/questions/20210983\#answer-47252259}{[HN]} \\
    \ruby{日}{ひ}の\ruby{入}{い}り & sunset (the moment it sets) & \href{https://ja.hinative.com/questions/20210983\#answer-47252259}{[HN]} \\
    % & & \\
    \midrule
    \midrule
    \ruby{昔}{むかし} & olden days & \\
    \ruby{過去}{か|こ} & the past & also an adverb \\
    それから & and then/after that (from a point in time) & \\
    あれから & since then/after that (a familiar past to both speaker and listener) & \\
    これまで & up to now/so far & also an expression \\
    \midrule
    \ruby{現在}{げん|ざい} & the present & also an adverb \\
    \ruby{最近}{さい|きん} & recently/lately/these days/nowadays & also an adverb \\
    \midrule
    これから & from now on/in the future; from here & also an adverb \\
    それから & and then (from a point in time) & \\
    それまで & until then; to that extent; the end of it/all there is to it & (\ruby{其}{そ}れまで) \\
    \ruby{未来}{み|らい} & the future & \href{https://dictionary.goo.ne.jp/thsrs/15272/meaning/m1u/}{[goo]} \\
    \ruby{将来}{しょう|らい} & future prospects (people/organisations/countries) & also an adverb, \href{https://dictionary.goo.ne.jp/thsrs/15272/meaning/m1u/}{[goo]} \\
    % & & \\
    \midrule
    \ruby{最初}{さい|しょ} & first/beginning & \\
    \ruby{最終}{さい|しゅう} & last/final & \\
    % & & \\
    \midrule
    \midrule
    \ruby{長}{なが}い\ruby{間}{あいだ} & long time/interval & also an adverb \\
    \ruby{永遠}{えい|えん} & eternity & \\
    % & & \\
    \bottomrule
}


\subsubsection{Pronouns and question words}
Gramatically, pronouns are used in place of nouns and noun phrases. There are question words associated with each counter, see the supplementary PDF.

Regarding the こそあど\ruby{言葉}{こと|ば}:
\begin{itemize}
    \item \{こ, ど, あ, ど\} \times \{れ, いつ, なた, こ, ちら, っち\} are pronouns
    \item \{こ, ど, あ, ど\} \times \{の, んな\} are pre-noun adjectivals
    \item \{こ, ど, あ, ど\} \times \{う\} are adverbs
\end{itemize}

\hl{to read all sub articles \href{https://www.tofugu.com/japanese-grammar/kosoado/}{here}}

% Help: \SetCell[r=2,c=2]{c,m} <content>, \cmidrule[l]{3-4}
% Help: colspec: X[ratio, horizontal alignment] columns grow to fit width=\linewidth
%                  negative ratios: shrink to fit content and may not grow to full ratio
% Help: colspec: l/c/r columns do not grow
\longtabse[0.75]  % scale factor
{Nouns: pronouns and question words.}  % caption
{tbl:appendix-vocab-nouns-pronouns-and-question-words}  % label
{}  % outer specification options
{
    colspec={X[-3,l]X[3,l]X[-3,l]},
    rowhead=1,
    % width=\linewidth,  % useful only with X columns
}  % inner specification options
{
    \toprule
    \textbf{Name} & \textbf{Meaning} & \textbf{Notes} \\
    \midrule
    \ruby{誰}{だれ} & who & \\
    どなた & who; what is your name (どなた[\ruby{様}{さま}]\{ですか/でしょうか\}。; [extra politeness and distance]) & honorific, feminine; \href{https://www.tofugu.com/japanese-grammar/konata-sonata-anata-donata/}{[TFG]} \\
    どちら[\ruby{様}{さま}] & who & [honorific] \\
    どいつ & who & rude \\
    \ruby{誰}{だれ}か & somebody & \\
    みんあ[さん] & everybody & \ruby{皆}{みんあ} \\
    \ruby{誰}{も} & nobody & \\
    \ruby{誰}{だれ}でも & anybody & \\
    % & & \\
    \midrule
    \ruby{彼}{かれ} & he/him & also a noun \\
    \ruby{彼女}{かの|じょ} & she/her & also a noun \\
    % & & \\
    \midrule
    こっち/[こちら] & this person (closer to speaker) & equal or higher status; [formal] \\
    そっち/[そちら] & that person (closer to listener) & [formal] \\
    あっち/[あちら] & that person/foreign country (distant) & polite; [formal] \\
    こいつ & this person & familiar, derogatory; also an interjection \\
    こいつめ & this rascal (jokingly)!/this bastard (expletive)! & (こいつ\ruby{奴}{め}); derogatory \\
    そいつ & that person & familiar, derogatory \\
    あいつ & that person & familiar, derogatory \\
    あいつめ & that rascal (jokingly)!/this bastard (expletive)! & (あいつ\ruby{奴}{め}); derogatory \\
    やつ & fellow/guy/chap; he/she/him/her & (\ruby{奴}{やつ}); familiar, derogatory \\
    こいつら & these people & (こいつ\ruby{等}{ら}); familiar, derogatory \\
    そいつら & those people & (そいつ\ruby{等}{ら}); familiar, derogatory \\
    あいつら & those people & (あいつ\ruby{等}{ら}); familiar, derogatory \\
    % & & \\
    \midrule
    \ruby{私}{わたくし} & I/me & formal, \href{https://ja.hinative.com/questions/21654599\#answer-50366344}{[HN]}, \href{https://japanese.stackexchange.com/a/2703}{[SE]} \\
    \ruby{私}{あたし} & I/me & feminine, less common \\
    うち & I/me & (\ruby{内}{うち}); feminine, familiar, also a place \\
    \ruby{私}{わたし} & I/me & slightly formal/distant \\
    \ruby{僕}{ぼく} & I/me & masculine, distant \\
    \ruby{自分}{じ|ぶん} & myself/oneself/yourself/himself/herself/I/me & distant \\
    \ruby{俺}{おれ} & I/me & masculine, familiar \\
    わし & I/me & (\ruby{儂}{わし}); masculine, elderly \\
    \ruby{私}{わたし}たち & we/us & (\ruby{私達}{わたし|たち}) \\
    \ruby{俺}{おれ}ら & we/us & (\ruby{俺等}{おれ|ら}) \\
    こちら & I/me/we/us & \\
    こっちこそ/[こちらこそ] & I (emphasis, used in opposing reply); it is I who should say so & actually an expression; [formal] \\
    こっちの\ruby{方}{かた}こそ/こちらの\ruby{方}{かた}こそ & I (emphasis, used in opposing reply); it is I who should say so & formal \\
    % & & \\
    \midrule
    そちら & you/your family/your company & polite \\
    そちらの\ruby{方}{かた} & you & (more) polite; \href{https://www.tofugu.com/japanese-grammar/koitsu-soitsu-aitsu-doitsu/}{[TFG]} \\
    そっちこそ/[そちらこそ] & you (emphasis, used in opposing reply); it is you who should say so & [formal] \\
    そっちの\ruby{方}{かた}こそ/そちらの\ruby{方}{かた}こそ & you (emphasis, used in opposing reply); it is you who should say so & formal \\
    \ruby{君}{きみ} & you & familiar (equal or lower status) \\
    あなた & you & (\ruby[g]{貴方}{あなた}); rude (if spoken)/distant; archaic: こなた、そなた \\
    \ruby{君}{きみ}たち/\ruby{君}{きみ}ら & you (plural) & (\ruby{君達}{きみ|たち})/(\ruby{君等}{きみ|ら}); familiar (equal or lower status) \\
    あなたたち & you (plural) & (\ruby[g]{貴方}{あなた}\ruby{達}{たち}); rude/distant \\
    \SetCell[c=3]{c,m} \color{red} \textbf{THOU SHALT NOT CROSS THIS LINE} \\
    あんた & you & (\ruby[g]{貴方}{あんた}); rude, expresses annoyance \\
    お\ruby{前}{まえ}/おめー & you & rude \\
    てめえ & you & (\ruby[g]{手前}{てめえ}); derogatory, inviting fight \\
    \ruby{貴様}{き|さま} & you & extremely derogatory, sensitive, inviting fight \\
    お\ruby{前}{まえ}たち/お\ruby{前}{まえ}ら & you (plural) & (お\ruby{前達}{まえ|たち})/(お\ruby{前等}{まえ|ら}); rude \\
    % & & \\
    \midrule
    \midrule
    \ruby{何}{なに} & what & \\
    どういう & what kind/sort of, referring to what was said & (どう\ruby{言}{い}う); actually a pre-noun adjectival \\
    どんあ & what kind/sort of & semi-casual; actually a pre-noun adjectival  \\
    \ruby{何}{なに}か & something & also an interjection \\
    \ruby{全部}{ぜん|ぶ} & all/everything & actually a noun/adverb \\
    \ruby{何}{なに}も & nothing & \\
    \ruby{何}{なん}でも & anything & \\
    どれ & which (three or more) & also an interjection \\
    どの & which/what (way) & actually a pre-noun adjectival \\
    どれか & one of many/some single one from many & \\
    どれも & all/none & \\
    どれでも & any/whichever & \\
    どっち/[どちら] & which (two) & [formal] \\
    どっちか/[どちらか] & one of the two & [formal] \\
    どっちも/[どちらも] & both/neither & [formal] \\
    どっちでも/どちらでも & any of the two & [formal] \\
    これ/こっち/[こちら] & this one (here, closer to speaker) & [formal]\\
    それ/そっち/[そちら] & that one (there, closer to listener) & [formal] \\
    あれ/あっち/[あちら] & that one (there, distant) & [formal] \\
    これら & these ones (here, closer to speaker) & formal and explanatory \\
    それら & those ones (there, closer to listener) & formal and explanatory \\
    あれら & those ones (there, distant) & formal and explanatory \\
    % & & \\
    \midrule
    これ & passionate reference/something (subjective) speaker feels close to & \\
    それ & dispassionate reference/something (objective) speaker maintains a little distance from & \\
    あれ & (mutual) memory reference/something in speaker's (and listener's) distant memory & \\
    アレ & hesitant reference/leave it up to listener to interpret & used in gossip \\
    % & & \\
    \midrule
    \midrule
    いつ & when & (\ruby[g]{何時}{いつ}; $\neq$ \ruby{何時}{なん|じ}) \\
    いつか & sometime & (\ruby[g]{何時}{いつ}か) \\
    いつも & always/never & (\ruby[g]{何時}{いつ}も) \\
    いつでも & anytime & (\ruby[g]{何時}{いつ}でも) \\
    ここ & now (passionate/subjective) & \\
    ここ<duration> & <duration> includes present moment (<duration> either past or future) & \href{https://www.tofugu.com/japanese-grammar/koko-soko-asoko-doko/}{[TFG]} \\
    そこ & then (dispassionate/objective) & \\
    あそこ & then (distant memory) & \\
    % & & \\
    \midrule
    \midrule
    どこ & where & (\ruby[g]{何処}{どこ}) \\
    どっち/[どちら] & where/which way/which direction & [formal of どこ] \\
    どこら\ruby{辺}{へん}/[どの\ruby{辺}{へん}/どの\ruby{辺}{あた}り] & where (approximate) & casual, [semi-formal] \\
    どこか/どっか & somewhere & (\ruby[g]{何処}{どこ}か) \\
    どこ[に]も & everywhere/nowhere & (\ruby[g]{何処}{どこ}も) \\
    どこでも & anywhere & (\ruby[g]{何処}{どこ}でも) \\
    \midrule
    ここ/こっち/こちら & here (closer to speaker, no comparison nuance) & \\
    そこ/そっら/そちら & there (close to listener, no comparison nuance) & \\
    あそこ & there (distant, no comparison nuance) & \\
    % & & \\
    \midrule
    これ/\{こっち/[こちら]\} & this way/direction (here, closer to speaker) \{comparison nuance\} & [formal] \\
    それ/\{そっち/[そちら]\} & that way/direction (there, closer to listener) \{comparison nuance\} & [formal] \\
    あれ/\{あっち/[あちら]\} & that way/direction (there, distant) \{comparison nuance\} & [formal] \\
    \midrule
    ここら\ruby{辺}{へん} & around here/this approximate area & casual \\
    そこら\ruby{辺}{へん} & around there/that approximate area & casual \\
    あそこら\ruby{辺}{へん} & around there/that approximate area (distant) & casual \\
    この\ruby{辺}{へん}/この\ruby{辺}{あた}り & this approximate area/around here & semi-formal \\
    その\ruby{辺}{へん}/その\ruby{辺}{あた}り & that approximate area/around there & semi-formal \\
    % & & \\
    \midrule
    \midrule
    なぜ & why & (\ruby[g]{何故}{なぜ}); direct/formal/rude, \href{https://ja.hinative.com/questions/21654599\#answer-50366344}{[HN]}, \href{https://japanese.stackexchange.com/a/2703}{[SE]} \\
    どうして & why/how/by what means & informal, \href{https://ja.hinative.com/questions/21654599\#answer-50366344}{[HN]}, \href{https://japanese.stackexchange.com/a/2703}{[SE]} \\
    どうしてですか & why/how/by what means & semi-formal, \href{https://ja.hinative.com/questions/21654599\#answer-50366344}{[HN]} \\
    どうやって & how/what way/method, referring to an achievement of something & actually a pre-noun adjectival \\
    なんで & why & (\ruby{何}{なん}で); informal, speech, \href{https://ja.hinative.com/questions/21654599\#answer-50366344}{[HN]}, \href{https://japanese.stackexchange.com/a/2703}{[SE]} \\
    なぜか & for some reason & (\ruby[g]{何故}{なぜ}か) \\
    % & & \\
    \midrule
    \midrule
    どう & how/in what way/how about & \\
    どの & which/what (way) & actually a pre-noun adjectival \\
    どうか/どうも & somehow & also an adverb \\
    どういうわけか & somehow & (どういう\ruby{訳}{わけ}か) \\
    どうでも & anyhow & \\
    % & & \\
    \bottomrule
}



\subsubsection{Pre-noun adjectivals}
\textbf{These are adjectives} that occur directly before nouns. There are $>100$ of them. From Section~\ref{sec:noun-related-particles}, these function as pre-noun noun modifiers. These function similarly to determiners in English.

% Help: \SetCell[r=2,c=2]{c,m} <content>, \cmidrule[l]{3-4}
% Help: colspec: X[ratio, horizontal alignment] columns grow to fit width=\linewidth
%                  negative ratios: shrink to fit content and may not grow to full ratio
% Help: colspec: l/c/r columns do not grow
\longtabse[0.75]  % scale factor
{Nouns: pre-noun adjectivals.}  % caption
{tbl:appendix-vocab-nouns-pre-noun-adjectivals}  % label
{}  % outer specification options
{
    colspec={X[-3,l]X[3,l]X[-3,l]},
    rowhead=1,
    % width=\linewidth,  % useful only with X columns
}  % inner specification options
{
    \toprule
    \textbf{Name} & \textbf{Meaning} & \textbf{Notes} \\
    \midrule
    \ruby{我}{わ}が & my/our & \\
    % & & \\
    \midrule
    \midrule
    どの & which/what (way) & \\
    この & this/these (closer to speaker) & \\
    この<number> & part/number <number> & e.g.\ 「その\ruby{1}{いち}」\\
    その & that/those/the  (closer to listener) & \\
    あの & that/those/the (distant/mutual memory) & \\
    これらの & these (closer to speaker) & (これ\ruby{等}{ら}の); formal and explanatory \\
    それらの & those (closer to listener) & (それ\ruby{等}{ら}の); formal and explanatory \\
    あれらの & those (distant) & (あれ\ruby{等}{ら}の); formal and explanatory \\
    % & & \\
    \midrule
    \midrule
    どんあ & what kind/sort of & semi-casual \\
    こんな & this kind/sort of (closer to speaker/passionate reference) & semi-casual \\
    そんな & that kind/sort of (closer to listener/dispassionate reference) & semi-casual  \\
    あんな & that kind/sort of (distant memory/sentimental) & semi-casual \\
    % & & \\
    \midrule
    どういう & what kind/sort of, referring to what was said & (どう\ruby{言}{い}う) \\
    こういう & this kind/sort of (closer to speaker), referring to what was said & (こう\ruby{言}{い}う) \\
    そういう & that kind/sort of (closer to listener), referring to what was said & (そう\ruby{言}{い}う) \\
    ああいう & that kind/sort of (distant), referring to what was said & (ああ\ruby{言}{い}う) \\
    % & & \\
    \midrule
    どうやって & how/what way/method, referring to an achievement of something & \\
    こうやって & this way/method (closer to speaker), referring to an achievement of something & \\
    そうやって & that way/method (closer to listener), referring to an achievement of something & \\
    ああやって & that way/method (distant), referring to an achievement of something & \\
    % & & \\
    \midrule
    いろんな & various & (\ruby{色}{いろ}んな) \\
    いわゆる & the so-called/so to speak & (\ruby{所謂}{いわ|ゆる})\\
    あらゆる & all/every & \\
    いかなる & every/any kind of/whatsoever/whatever & (\ruby{如何}{い|か}なる) \\
    % & & \\
    \midrule
    \midrule
    \ruby{小}{ち}さな & small/little/tiny & \\
    \ruby{大}{おお}きな & big/large/great & \\
    % & & \\
    \midrule
    \ruby{単}{たん}なる & mere/simple (joke/coincidence) & \\
    ほんの & only (e.g.\ once); just/slight (distance/time); mere (e.g.\ child) & (\ruby{本}{ほん}の) \\
    \ruby{大}{たい}した & considerable/great/important/significant/a big deal & \\
    % & & \\
    \midrule
    \midrule
    \ruby{実}{じつ}の & true/real & \\
    \ruby{主}{おも}なる & main/principal/important & \\
    \ruby{明}{あ}くる & next/following (〜\ruby{日}{ひ}/〜\ruby{朝}{あさ}/\ruby{年}{とし} etc.) & \\
    % & & \\
    \bottomrule
}


\subsubsection{Roles and occupations}
% Help: \SetCell[r=2,c=2]{c,m} <content>, \cmidrule[l]{3-4}
% Help: colspec: X[ratio, horizontal alignment] columns grow to fit width=\linewidth
%                  negative ratios: shrink to fit content and may not grow to full ratio
% Help: colspec: l/c/r columns do not grow
\longtabse[0.75]  % scale factor
{Nouns: roles and occupations.}  % caption
{tbl:appendix-vocab-nouns-roles-and-occupations}  % label
{}  % outer specification options
{
    colspec={X[-3,l]X[3,l]X[-3,l]},
    rowhead=1,
    % width=\linewidth,  % useful only with X columns
}  % inner specification options
{
    \toprule
    \textbf{Name} & \textbf{Meaning} & \textbf{Notes} \\
    \midrule
    \ruby{女}{おんあ} & female/woman & \\
    \ruby{女子}{じょ|し} & woman/girl & \\
    \ruby{男}{おとこ} & make/man & \\
    \ruby{男子}{だん|し} & man/boy & \\
    \ruby{子}{こ}ども & child & \ruby{子供}{こ|ども} may be offensive; \href{https://www.reddit.com/r/LearnJapanese/comments/hkwop3/\%E5\%AD\%90\%E4\%BE\%9B\_vs\_\%E5\%AD\%90\%E3\%81\%A9\%E3\%82\%82_advice/}{[r]} \\
    \ruby[g]{大人}{おとな} & adult/grown-up & \\
    % & & \\
    \midrule
    \midrule
    \ruby{初学者}{しょ|がく|しゃ} & beginner & \\
    \ruby{学生}{がく|せい} & student & \\
    \ruby{後輩}{こう|はい} & junior/younger person & \\
    \ruby{先輩}{せん|ぱい} & senior/superior/elder & \\
    \ruby{小学生}{しょう|がく|せい} & elementary/primary school student& \\
    \ruby{中学生}{ちゅう|がく|せい} & junior high/middle school student & \\
    \ruby{高校生}{こう|こう|せい} & high school student & \\
    \ruby{女子高生}{じょ|し|こう|せい} & female high-school student & \\
    \ruby{男子高生}{だん|し|こう|せい} & male high-school student & \\
    \ruby{大学生}{だい|がく|せい} & university student & \\
    \ruby[g]{博士}{はかせ} & expert/learned person/PhD Dr. & \\
    \ruby{教授}{きょう|じゅ} & professor & \\
    \ruby{校長}{こう|ちょう} & principal/head teacher/headmaster/headmistress & \\
    \ruby{学長}{がく|ちょう} & university president/chancellor/provost & \\
    % & & \\
    \midrule
    \midrule
    \ruby{初代}{しょ|だい} & first generation/founder & \\
    % & & \\
    \midrule
    \ruby{社長}{しゃ|ちょう} & company president/manager/director & \\
    \ruby{課長}{か|ちょう} & section manager/chief & \\
    \ruby{店長}{\textbf{て}ん|ちょう} & shop manager & \\
    % & & \\
    \midrule
    \midrule
    \ruby{友達}{とも|だち} & friend & \\
    \ruby{友人}{ゆう|じん} & friend & formal \\
    \ruby{仲間}{なか|ま} & companion/fellow/friend/mate/comrade/partner/colleague/coworker; group/company/circle & \\
    \ruby{相手}{あい|て} & companion/partner/company; other party/addressee; opponent (sports) & \\
    % & & \\
    \midrule
    \ruby{彼}{かれ} & boyfriend & also a pronoun \\
    \ruby{彼女}{かの|じょ} & girlfriend & also a pronoun \\
    % & & \\
    \midrule
    \midrule
    <noun>の\ruby{卵}{たまご} & aspiring <noun>/expert in the making & \\
    \ruby{将軍}{しょう|ぐん} & general (military, historical) & \\
    \ruby{有名人}{ゆう|めい|じん} & famous person/celebrity/public figure & \\
    \ruby{医者}{い|しゃ} & doctor/physician & \\
    \ruby{警察}{けい|さつ} & police/police officer/police station & \\
    \ruby{警察官}{けい|さつ|かん} & police officer & \\

    [お]\ruby{金持}{かね|も}ち & rich person & \\
    \ruby{冒険者}{ぼう|けん|しゃ} & adventurer & \\
    \ruby{戦士}{せん|し} & soldier/combatant/warrior & \\
    お\ruby{客}{きゃく}さん & guest/visitor/customer/client/shopper/audience/tourist/sightseer/passenger & honorific \\
    お\ruby{客様}{きゃく|さま} & guest/visitor/customer/client/shopper/audience/tourist/sightseer/passenger & honorific \\
    \ruby{天才}{てん|さい} & genius/prodify/natural gift & \\
    \ruby{凡才}{ぼん|さい} & mediocrity/ordinary ability & \\
    \ruby{美少女}{び|しょう|じょ} & beautiful girl & \\
    ネイティブ & native speaker & also an adjective \\
    \ruby{推}{お}し & being a fan/supporter of; one's favourite (member of idol group/anime/team) & slang \\
    % & & \\
    \midrule
    \midrule
    \ruby{変態}{へん|たい} & abnormality; pervert & also in Table~\ref{tbl:appendix-vocab-nouns-production} \\
    \ruby{犯人}{はん|にん} & offender/criminal/culprit & \\
    \ruby{犯罪者}{はん|ざい|しゃ} & criminal/culprit & \\
    % & & \\
    \midrule
    \midrule
    \ruby{問題児}{もん|だい|じ} & problem child & \\
    バカ & idiot/moron/fool & (\ruby{馬鹿}{ば|か}); also an adjective \\
    アホ & fool/idiot/simpleton & (\ruby{阿呆}{あ|ほ}); also an adjective \\
    ボケ & fool/idiot & (\ruby{惚}{ぼ}け) \\
    \ruby{野郎}{や|ろう} & bastard/asshole/son of a bitch & slang, derogatory \\
    ガキ & brat/kid/little devil & (\ruby{餓鬼}{が|き}); slang \\
    ばかやろう & goddamn idiot/moron/nitwit & (\ruby{馬鹿野郎}{ば|か|やろ|う}); slang, derogatory \\
    クソ\ruby{野郎}{や|ろう} & piece of shit/son of a bitch & (\ruby{糞野郎}{くそ|や|ろう}); derogatory \\
    クソガキ & stupid brat/son of a bitch & (\ruby{糞餓鬼}{くそ|が|き}); derogatory \\
    \ruby{畜生}{ちく|しょう} & brute/bastard & \\
    % & & \\
    \bottomrule
}


\subsubsection{Family}
Table~\ref{tbl:appendix-vocab-nouns-family} lists the names in casual/formal manner. We only use casual when referencing our own family members to other people. In all other situations (talking about other people's family, or talking directly to our own family), we use the formal one.

% Help: \SetCell[r=2,c=2]{c,m} <content>, \cmidrule[l]{3-4}
% Help: colspec: X[ratio, horizontal alignment] columns grow to fit width=\linewidth
%                  negative ratios: shrink to fit content and may not grow to full ratio
% Help: colspec: l/c/r columns do not grow
\longtabse[0.75]  % scale factor
{Nouns: family.}  % caption
{tbl:appendix-vocab-nouns-family}  % label
{}  % outer specification options
{
    colspec={X[-3,l]X[3,l]X[-3,l]},
    rowhead=1,
    % width=\linewidth,  % useful only with X columns
}  % inner specification options
{
    \toprule
    \textbf{Name} & \textbf{Meaning} & \textbf{Notes} \\
    \midrule

    [ご]\ruby{両親}{りょう|しん} & parents & [honorific] \\
    \ruby{母}{はは}/[お\ruby{母}{かあ}さん] & mother & humble/[honorific] \\
    \ruby{父}{ちち}/[お\ruby{父}{とう}さん] & father & humble/[honorific] \\
    \ruby{妻}{つま}/[\ruby{奥}{おく}さん] & wife & [honorific] \\
    \ruby{夫}{おっと}/[ご\ruby{主人}{しゅ|じん}] & husband & [honorific] \\
    \ruby{姉}{あね}/[お\ruby{姉}{ねえ}さん] & older sister; young lady/miss/ma'am/older girl & [honorific]\\
    \ruby{兄}{あに}/[お\ruby{兄}{にい}さん] & older brother; young man/buddy/fella/lad & [honorific] \\
    \ruby{妹}{いもうと}[さん] & younger sister & [honorific] \\
    \ruby{弟}{おとうと}[さん] & younger brother & [honorific] \\
    \ruby{娘}{むすめ}[さん] & daughter & [honorific] \\
    \ruby[g]{息子}{むすこ}[さん] & son & [honorific] \\
    % & & \\
    \midrule
    \midrule
    \ruby{親子}{おや|こ} & parent and child & \\
    \ruby{双子}{ふた|ご} & twins & \\
    % & & \\
    \bottomrule
}


\subsubsection{Body parts}
% Help: \SetCell[r=2,c=2]{c,m} <content>, \cmidrule[l]{3-4}
% Help: colspec: X[ratio, horizontal alignment] columns grow to fit width=\linewidth
%                  negative ratios: shrink to fit content and may not grow to full ratio
% Help: colspec: l/c/r columns do not grow
\longtabse[0.75]  % scale factor
{Nouns: body parts.}  % caption
{tbl:appendix-vocab-nouns-body-parts}  % label
{}  % outer specification options
{
    colspec={X[-3,l]X[3,l]X[-3,l]},
    rowhead=1,
    % width=\linewidth,  % useful only with X columns
}  % inner specification options
{
    \toprule
    \textbf{Name} & \textbf{Meaning} & \textbf{Notes} \\
    \midrule
    \ruby{髪}{かみ} & hair (on the head) & also: ヘア \\
    % & & \\
    \midrule
    おっぱい & boobs/breasts & slang/children's language \\
    % & & \\
    \midrule
    \ruby{鉄拳}{てっ|けん} & tightly clenched fist & \\
    % & & \\
    \midrule
    \midrule
    \ruby{体重}{たい|じゅう} & body weight & \\
    % & & \\
    \bottomrule
}


\subsubsection{Clothing}
% Help: \SetCell[r=2,c=2]{c,m} <content>, \cmidrule[l]{3-4}
% Help: colspec: X[ratio, horizontal alignment] columns grow to fit width=\linewidth
%                  negative ratios: shrink to fit content and may not grow to full ratio
% Help: colspec: l/c/r columns do not grow
\longtabse[0.75]  % scale factor
{Nouns: clothing.}  % caption
{tbl:appendix-vocab-nouns-clothing}  % label
{}  % outer specification options
{
    colspec={X[-3,l]X[3,l]X[-3,l]},
    rowhead=1,
    % width=\linewidth,  % useful only with X columns
}  % inner specification options
{
    \toprule
    \textbf{Name} & \textbf{Meaning} & \textbf{Notes} \\
    \midrule
    \ruby{制服}{せい|ふく} & uniform & \\
    シャツ & singlet/inner shirt/buttoned shirt & \\
    Tシャツ & T-shirt (outer shirt) & \\
    % & & \\
    \midrule
    \midrule
    ショーツ & shorts & \\
    パンツ & underpants/panties/swimming trunks/women's trousers & \\
    % & & \\
    \midrule
    \midrule
    \ruby{靴}{くつ} & shoe/shoes/boots/footwear & \\
    % & & \\
    \bottomrule
}


\subsubsection{Emotions}
% Help: \SetCell[r=2,c=2]{c,m} <content>, \cmidrule[l]{3-4}
% Help: colspec: X[ratio, horizontal alignment] columns grow to fit width=\linewidth
%                  negative ratios: shrink to fit content and may not grow to full ratio
% Help: colspec: l/c/r columns do not grow
\longtabse[0.75]  % scale factor
{Nouns: emotions.}  % caption
{tbl:appendix-vocab-nouns-emotions}  % label
{}  % outer specification options
{
    colspec={X[-3,l]X[3,l]X[-3,l]},
    rowhead=1,
    % width=\linewidth,  % useful only with X columns
}  % inner specification options
{
    \toprule
    \textbf{Name} & \textbf{Meaning} & \textbf{Notes} \\
    \midrule
    \ruby{気持}{き|も}ち & feeling/sensation/mood/state of mind & \href{https://dictionary.goo.ne.jp/thsrs/3397/meaning/m0u/}{[goo]} \\
    \ruby{気分}{き|ぶん} & mood/feeling & \href{https://dictionary.goo.ne.jp/thsrs/3397/meaning/m0u/}{[goo]} \\
    \ruby{感}{かん}じ & feeling/sense/impression & \\
    \ruby{空気}{くう|き} & situation/mood/room (esp.\ \ruby{空気}{くう|き}を\ruby{読}{よ}む) & also in Table~\ref{tbl:appendix-vocab-nouns-nature} \\
    \ruby{感情}{かん|じょう} & emotion/feeling/feelings/sentiment & \\
    % & & \\
    \midrule
    \ruby{心}{こころ}の\ruby{声}{こえ} & one's inner voice/what one really thinks & \\
    % & & \\
    \midrule
    \midrule
    \ruby{緊張}{きん|ちょう} & nervousness/stress/tension/strain; tension (between countries/groups) & also a verb \\
    \ruby{心配}{しん|ぱい} & worry/anxiety/uneasiness/fear & also a verb \\
    \ruby{後悔}{こう|かい} & regret/repentance/remorse & also a verb \\
    \midrule
    % & & \\
    \ruby{遠慮}{えん|りょ} & reserve/constraint/hesitation/tact/thoughtfulness; refraining/declining & also a verb \\
    % & & \\
    \midrule
    \ruby{感動}{かん|どう} & being emotionally deeply moved/excited/inspired & also a verb \\
    \ruby{涙}{なみだ} & tears & \\
    \ruby{笑}{わら}/w & LOL/haha & slang \\
    % & & \\
    \midrule
    \midrule
    \ruby{一人}{ひと|り}ぼっち & aloneness/loneliness/solitude & \\
    % & & \\
    \midrule
    \ruby{幸}{しあわ}せ & happiness & also an adjective \\
    \ruby{不幸}{ふ|しあわ}せ & unhappiness & also an adjective \\
    % & & \\
    \midrule
    \ruby{安心感}{あん|しん|かん} & sense of security & \\
    \ruby{不安}{ふ|あん} & anxiety/uneasiness/insecurity & also an adjective\\
    % & & \\
    \bottomrule
}


\subsubsection{Production}
% Help: \SetCell[r=2,c=2]{c,m} <content>, \cmidrule[l]{3-4}
% Help: colspec: X[ratio, horizontal alignment] columns grow to fit width=\linewidth
%                  negative ratios: shrink to fit content and may not grow to full ratio
% Help: colspec: l/c/r columns do not grow
\longtabse[0.75]  % scale factor
{Nouns: production.}  % caption
{tbl:appendix-vocab-nouns-production}  % label
{}  % outer specification options
{
    colspec={X[-3,l]X[3,l]X[-3,l]},
    rowhead=1,
    % width=\linewidth,  % useful only with X columns
}  % inner specification options
{
    \toprule
    \textbf{Name} & \textbf{Meaning} & \textbf{Notes} \\
    \midrule
    \ruby{仕方}{し|かた} & way/method/means towards a goal & \href{https://dictionary.goo.ne.jp/thsrs/14732/meaning/m0u/\%E4\%BB\%95\%E6\%96\%B9/}{[goo]} \\
    \ruby{方法}{ほう|ほう} & (a well-reasoned) way/method/process/procedure & \href{https://dictionary.goo.ne.jp/thsrs/14732/meaning/m0u/\%E4\%BB\%95\%E6\%96\%B9/}{[goo]} \\
    \ruby{手段}{しゅ|だん} & tool needed for going towards a goal & \href{https://dictionary.goo.ne.jp/thsrs/14732/meaning/m0u/\%E4\%BB\%95\%E6\%96\%B9/}{[goo]} \\
    % & & \\
    \midrule
    \midrule
    \ruby{切符}{きっ|ぷ} & ticket & \\
    % & & \\
    \midrule
    \midrule
    \ruby{準備}{じゅん|び} & preparation/arrangements/setup & also a verb \\
    つもり & plan/intention; assumption/belief/thought; estimation & (\ruby{積}{つ}もり) \\
    \ruby{始}{はじ}まり & origin/beginning & \\
    \ruby{開始}{かい|し} & start/commencement/beginning/initiation & slightly formal, also a verb, \href{https://ja.hinative.com/questions/4515521}{[HN]} \\
    スタート & start/beginning & also a verb \\
    \ruby{出発}{しゅっ|ぱつ} & departure/setting off & also a verb \\
    \ruby{発動}{はつ|どう} & kick-start/put into effect (activity/machine/policy) & also a verb \\
    % & & \\
    \midrule
    \ruby{作}{つく}り & the making/production/components of & \\
    % & & \\
    \midrule
    \ruby{変化}{へん|か} & change/variation/alteration/mutation/transfiguration & also a verb, in Table~\ref{tbl:appendix-vocab-nouns-meta-language} \\
    \ruby{変態}{へん|たい} & state of transformation & also in Table~\ref{tbl:appendix-vocab-nouns-roles-and-occupations} \\
    \ruby{進化}{しん|か} & evolution/progress/development/improvement & also a verb \\
    \ruby{展開}{てん|かい} & development/evolution/progression/unfolding/plot twist; expansion (physical/mathematics) & also a verb \\
    % & & \\
    \midrule
    \ruby{終}{お}わり & end/ending/conclusion; it's over & \\
    \ruby{終了}{しゅう|りょう} & end/close/termination & slightly formal, also a verb, \href{https://ja.hinative.com/questions/2620397}{[HN]} \\
    \ruby{完成}{かん|せい} & completion/perfection/accomplishment & also a verb \\
    \ruby{結果}{けっ|か} & result/outcome/consequence & \\
    \ruby{突破}{とっ|ぱ} & breakthrough/overcoming a difficulty; exceeding/rising above & also a verb \\
    % & & \\
    \midrule
    \midrule
    \ruby{話}{はなし} & speech/conversation/topic/subject & \\
    タメ\ruby{口}{ぐち} & casual/informal speech/language & slang \\
    しゃべり & chat/chatter/talk & (\ruby{喋}{しゃべ}り) \\
    \ruby{配信}{はい|しん} & distribution/broadcast/streaming (news/information/media) & \\
    \ruby{音楽}{おん|がく} & music & \\
    \ruby{声}{こえ} & voice (literal and abstract); singing/chirping (of bird/insect) & \\
    \ruby{歌}{うた} & song/singing & \\
    \ruby{曲}{きょく} & piece/composition/song/track & \\
    \ruby{散歩}{さん|ぽ} & stroll & \\
    \ruby{踊}{おど}り & dance & \\
    \ruby{小躍}{こ|おど}り & dancing/jumping for joy & also a verb \\
    \ruby{映画}{えい|が} & movie/film/motion picture & \\
    \ruby{写真}{しゃ|しん} & photograph/photo/picture/snapshot & \\
    % & & \\
    \midrule
    \midrule
    \ruby{雑誌}{ざっ|し} & journal/magazine & \\
    \ruby{物語}{もの|がたり} & story/tale/fable & also: ストーリー \\
    \ruby{伝説}{でん|せつ} & legend/folklore & also: レジェンド \\
    \ruby{伝統}{でん|とう} & tradition & also: トラディション \\
    \ruby{神話}{しん|わ} & myth & also: ミス \\
    % & & \\
    \midrule
    \midrule
    \ruby{計算}{けい|さん} & calculation/computation & also a verb \\
    % & & \\
    \midrule
    \midrule
    \ruby{仕事}{し|ごと} & work/occupation/employment & \\
    % & & \\
    \midrule
    \midrule
    \ruby{盗品}{とう|ひん} & stolen goods & \\
    % & & \\
    \midrule
    \midrule
    \ruby{自動}{じ|どう} & automatic operation & \\
    \ruby{手動}{しゅ|どう} & manual operation & \\
    % & & \\
    \midrule
    \midrule
    \ruby{結婚}{けっ|こん} & marriage & also a verb \\
    % & & \\
    \midrule
    \midrule
    \ruby{冒険}{ぼう|けん} & adventure/venture; risky venture/attempt & \\
    % & & \\
    \bottomrule
}


\subsubsection{Sports}
% Help: \SetCell[r=2,c=2]{c,m} <content>, \cmidrule[l]{3-4}
% Help: colspec: X[ratio, horizontal alignment] columns grow to fit width=\linewidth
%                  negative ratios: shrink to fit content and may not grow to full ratio
% Help: colspec: l/c/r columns do not grow
\longtabse[0.75]  % scale factor
{Nouns: sports.}  % caption
{tbl:appendix-vocab-nouns-sports}  % label
{}  % outer specification options
{
    colspec={X[-3,l]X[3,l]X[-3,l]},
    rowhead=1,
    % width=\linewidth,  % useful only with X columns
}  % inner specification options
{
    \toprule
    \textbf{Name} & \textbf{Meaning} & \textbf{Notes} \\
    \midrule
    \ruby{卓球}{たっ|きゅう} & table tennis/ping-pong & \\
    ピンポン & table tennis/ping-pong & \\
    バドミントン & badminton & \\
    % & & \\
    \bottomrule
}


\subsubsection{Consumption}
% Help: \SetCell[r=2,c=2]{c,m} <content>, \cmidrule[l]{3-4}
% Help: colspec: X[ratio, horizontal alignment] columns grow to fit width=\linewidth
%                  negative ratios: shrink to fit content and may not grow to full ratio
% Help: colspec: l/c/r columns do not grow
\longtabse[0.75]  % scale factor
{Nouns: consumption.}  % caption
{tbl:appendix-vocab-nouns-consumption}  % label
{}  % outer specification options
{
    colspec={X[-3,l]X[3,l]X[-3,l]},
    rowhead=1,
    % width=\linewidth,  % useful only with X columns
}  % inner specification options
{
    \toprule
    \textbf{Name} & \textbf{Meaning} & \textbf{Notes} \\
    \midrule
    \ruby{必要}{ひつ|よう} & necessity/need/requirement & \\
    % & & \\
    \midrule
    \midrule
    \ruby{値段}{ね|だん} & price/cost & \\

    [お]\ruby{金}{かね} & money & [polite] \\
    \ruby{買}{か}い\ruby{物}{もの} & the act of shopping/purchased goods & \\
    ビル & bill/invoice & \\
    かばん & bag/briefcase/basket & \\
    \ruby{袋}{ふくろ} & bag/sack/pouch & \\
    % & & \\
    \midrule
    \midrule
    \ruby{試験}{し|けん} & examination/test & also: テスト \\
    \ruby{試}{ため}し & trial/attempt/test & \\
    % & & \\
    \midrule
    \midrule
    \ruby{削除}{さく|じょ} & deletion/elimination/erasure & also a verb \\
    \ruby{電気}{でん|き} & electricity/electric lamp & \\
    ロック & lock & also a verb \\
    % & & \\
    \midrule
    \midrule
    おかわり & second serving & (お\ruby{代}{か}わり/お\ruby{替}{か}わり) \\
    % & & \\
    \bottomrule
}


\subsubsection{Interaction}
% Help: \SetCell[r=2,c=2]{c,m} <content>, \cmidrule[l]{3-4}
% Help: colspec: X[ratio, horizontal alignment] columns grow to fit width=\linewidth
%                  negative ratios: shrink to fit content and may not grow to full ratio
% Help: colspec: l/c/r columns do not grow
\longtabse[0.75]  % scale factor
{Nouns: interaction.}  % caption
{tbl:appendix-vocab-nouns-interaction}  % label
{}  % outer specification options
{
    colspec={X[-3,l]X[3,l]X[-3,l]},
    rowhead=1,
    % width=\linewidth,  % useful only with X columns
}  % inner specification options
{
    \toprule
    \textbf{Name} & \textbf{Meaning} & \textbf{Notes} \\
    \midrule
    いくら & how much (price) & (\ruby{幾}{いく}ら) \\
    \ruby{無料}{む|りょう} & free of charge & \\
    \ruby{有料}{ゆう|りょう} & fee-charging/paid/not free & \\
    % & & \\
    \midrule
    \midrule

    [お]\ruby{願}{ねが}い & wish/desire/hope; request/favour & [polite]; also an interjection \\
    〜\ruby{願}{ねがい} & written application & \suffix \\
    \ruby{感謝}{かん|しゃ} & thanks/gratitude/appreciation & also a verb \\
    % & & \\
    \midrule
    \ruby{命令}{めい|れい} & order/command/decree/directive; instruction/statement (computing) & also a verb \\
    % & & \\
    \midrule
    \ruby{優先}{ゆう|せん} & priority/precedence/preference & also a verb \\
    % & & \\
    \midrule
    \midrule
    \ruby{困難}{こん|なん} & difficulty/hardship/trouble/distress & \\
    % & & \\
    \midrule
    \midrule
    \ruby{名前}{な|まえ} & name & \\
    % & & \\
    \midrule
    \ruby{存在}{そん|ざい} & existence/presence & \\
    % & & \\
    \midrule
    \ruby{連絡}{れん|らく} & making contact/communication/call/message & also a verb \\
    \ruby{連絡先}{れん|らく|さき} & contact information (e.g.\ address/phone number) & \\
    \ruby{電話}{でん|わ} & phone/phone call & also a verb \\
    メールアドレス & email address & \\
    パソコン & personal computer (PC) & \\
    ノートパソコン & laptop computer (``notebook personal computer'') & \\
    % & & \\
    \midrule
    \midrule
    \ruby{登録}{とう|ろく} & presence in register/records; registration/subscription (YouTube) & also a verb; \href{https://dictionary.goo.ne.jp/word/\%e7\%99\%bb\%e9\%8c\%b2/}{[goo]} \\
    \ruby{入会}{にゅう|かい} & enrolment/admission into a club/society/mailing list & also a verb; \href{https://ja.hinative.com/questions/22502664}{[HN]} \\
    \ruby{加入}{か|にゅう} & becoming a member of (e.g.\ a group/project) & also a verb; \href{https://ja.hinative.com/questions/22502664}{[HN]} \\
    \ruby{退会}{たい|かい} & withdrawal/resignation from a club/society/mailing list & also a verb \\
    \midrule
    \midrule
    \ruby{告白}{こく|はく} & confession (of a crime/wrongdoing/romantic feelings) & also a verb \\
    % & & \\
    \midrule
    \ruby{逮捕}{たい|ほ} & arrest/capture & also a verb \\
    % & & \\
    \midrule
    \midrule
    \ruby{場合}{ば|あい} & case/occasion/situation/circumstances & \\
    \ruby{事故}{じ|こ} & accident/incident/trouble & \\
    % & & \\
    \bottomrule
}


\subsubsection{Society and culture}
% Help: \SetCell[r=2,c=2]{c,m} <content>, \cmidrule[l]{3-4}
% Help: colspec: X[ratio, horizontal alignment] columns grow to fit width=\linewidth
%                  negative ratios: shrink to fit content and may not grow to full ratio
% Help: colspec: l/c/r columns do not grow
\longtabse[0.75]  % scale factor
{Nouns: society and culture.}  % caption
{tbl:appendix-vocab-nouns-society-and-culture}  % label
{}  % outer specification options
{
    colspec={X[-3,l]X[3,l]X[-3,l]},
    rowhead=1,
    % width=\linewidth,  % useful only with X columns
}  % inner specification options
{
    \toprule
    \textbf{Name} & \textbf{Meaning} & \textbf{Notes} \\
    \midrule
    \ruby{経済}{けい|ざい} & economy & \\
    \ruby{地理}{ち|り} & geography & \\
    \ruby{歴史}{れき|し} & history & \\
    % & & \\
    \midrule
    \midrule
    \ruby{営業}{えい|ぎょう} & business/trade/operations & \\
    % & & \\
    \midrule
    \midrule
    \ruby{国際}{こく|さい} & international & \\
    \ruby{中国}{ちゅう|ごく} & China & \\
    \ruby{日本}{に|ほん}/\ruby{日本}{にっ|ぽん} & Japan & pronunciation: no consensus \\
    \ruby{日本国}{に|ほん|こく}/\ruby{日本国}{にっ|ぽん|こく} & Japan & pronunciation: no consensus \\
    \ruby{韓国}{かん|こく} & Korea & \\
    シンガポール & Singapore & \\
    \ruby{英国}{えい|こく} & United Kingdom/Britain & \\
    \ruby{米国}{べい|こく} & America/USA & also: アメリカ \\
    % & & \\
    \midrule
    \ruby{中国語}{ちゅう|ごく|ご} & Chinese language & \\
    \ruby{日本語}{に|ほん|ご} & Japanese language & \\
    \ruby{韓国語}{かん|こく|ご} & Korean language & \\
    \ruby{英語}{えい|ご} & English language & \\
    % & & \\
    \midrule
    \midrule
    \ruby{平和}{へい|わ} & peace/harmony & \\
    \ruby{平等}{びょう|どう} & equality/impartiality & also an adjective \\
    % & & \\
    \midrule
    \midrule
    \ruby{文化}{ぶん|か} & culture/civilisation & \\
    アニメ & animation/anime & \\
    % & & \\
    \bottomrule
}


\subsubsection{Health}
% Help: \SetCell[r=2,c=2]{c,m} <content>, \cmidrule[l]{3-4}
% Help: colspec: X[ratio, horizontal alignment] columns grow to fit width=\linewidth
%                  negative ratios: shrink to fit content and may not grow to full ratio
% Help: colspec: l/c/r columns do not grow
\longtabse[0.75]  % scale factor
{Nouns: health.}  % caption
{tbl:appendix-vocab-nouns-health}  % label
{}  % outer specification options
{
    colspec={X[-3,l]X[3,l]X[-3,l]},
    rowhead=1,
    % width=\linewidth,  % useful only with X columns
}  % inner specification options
{
    \toprule
    \textbf{Name} & \textbf{Meaning} & \textbf{Notes} \\
    \midrule
    \ruby{命}{いのち} & life/life force; most important thing/core & \\
    \ruby{生命}{せい|めい} & life/existence; one's working life/career; life/life force & \\
    % & & \\
    \midrule
    \ruby{精神}{せい|しん} & mind/spirit/soul/heart; attitude/mentality & \\
    \ruby{元気}{げん|き} & health/vigour & also an adjective \\
    \ruby{健康}{けん|こう} & health & also an adjective \\
    \ruby{疲}{つか}れ & tiredness/fatigue & \\
    \ruby{空}{す}き\ruby{腹}{はら} & empty stomach/hunger & \\
    \ruby{病気}{びょう|き} & illness/disease/sickness & also an adjective \\
    \ruby{苦痛}{く|つう} & pain/agony/suffering/distress/torment & \\
    % & & \\
    \midrule
    \ruby{安全}{あん|ぜん} & safety/security & also an adjective \\
    \ruby{危険}{き|けん} & danger/hazard/risk & also an adjective \\
    \ruby{非常}{ひ|じょう} & emergency & also an adjective \\
    まさか & the unexpected/emergency & (\ruby{真逆}{ま|さか}); also an interjection \\
    % & & \\
    \midrule
    \ruby{一酸化炭素中毒}{いっ|さん|か|たん|そ|ちゅう|どく} & carbon monoxide poisoning & \\
    % & & \\
    \midrule
    hospital & & \\
    % & & \\
    \midrule
    \midrule
    \ruby{休}{やす}み & rest/vacation & \\
    \ruby{昼寝}{ひる|ね} & afternoon nap & \\
    \ruby{暇}{ひま} & free time/time off/leisure & also an adjective \\
    % & & \\
    \midrule
    \ruby{夢}{ゆめ} & dream & \\
    \ruby{悪夢}{あく|む} & nightmare & \\
    % & & \\
    \midrule
    \ruby{風呂}{ふ|ろ} & bath/bathtub/bathroom/bathhouse; bathing & \\
    \ruby{日光浴}{にっ|こう|よく} & sunbathing/basking in the sun & also a verb \\
    % & & \\
    \midrule
    \ruby{春休}{はる|やす}み & spring break/vacation & \\
    \ruby{夏休}{なつ|やす}み & summer vacation & \\
    \ruby{秋休}{あき|やす}み & autumn break/vacation & also: \ruby{秋}{とき} \\
    \ruby{冬休}{ふゆ|やす}み & winter vacation & \\
    \ruby{月見}{つき|み} & Japanese equivalent of CN's mid-autumn festival (\ruby{同}{おな}じ\ruby{日}{ひ}) & \\
    % & & \\
    \midrule
    \midrule
    \ruby{寝袋}{ね|ぶくろ} & sleeping bag & \\
    % & & \\
    \bottomrule
}


\subsubsection{Disasters}
% Help: \SetCell[r=2,c=2]{c,m} <content>, \cmidrule[l]{3-4}
% Help: colspec: X[ratio, horizontal alignment] columns grow to fit width=\linewidth
%                  negative ratios: shrink to fit content and may not grow to full ratio
% Help: colspec: l/c/r columns do not grow
\longtabse[0.75]  % scale factor
{Nouns: disasters.}  % caption
{tbl:appendix-vocab-nouns-disasters}  % label
{}  % outer specification options
{
    colspec={X[-3,l]X[3,l]X[-3,l]},
    rowhead=1,
    % width=\linewidth,  % useful only with X columns
}  % inner specification options
{
    \toprule
    \textbf{Name} & \textbf{Meaning} & \textbf{Notes} \\
    \midrule
    \ruby{地震}{じ|しん} & earthquake & \\
    \ruby{台風}{たい|ふう} & typhoon & \\
    % & & \\
    \bottomrule
}


\subsubsection{Colours}
\emph{Read the main article on \href{https://cotoacademy.com/colors-japanese-use-japanese-color-words/}{CTA}.}

Colours are often used as labels, together with the の particle.

% Help: \SetCell[r=2,c=2]{c,m} <content>, \cmidrule[l]{3-4}
% Help: colspec: X[ratio, horizontal alignment] columns grow to fit width=\linewidth
%                  negative ratios: shrink to fit content and may not grow to full ratio
% Help: colspec: l/c/r columns do not grow
\longtabse[0.75]  % scale factor
{Nouns: colours.}  % caption
{tbl:appendix-vocab-nouns-colours}  % label
{}  % outer specification options
{
    colspec={X[-3,l]X[3,l]X[-3,l]},
    rowhead=1,
    % width=\linewidth,  % useful only with X columns
}  % inner specification options
{
    \toprule
    \textbf{Name} & \textbf{Meaning} & \textbf{Notes} \\
    \midrule
    \ruby{赤}{あか}[\ruby{色}{いろ}] & red (1, 0, 0) & also: レッド \\
    % \ruby{赤色}{せき|しょく} & red (1, 0, 0) & formal/literary [GMN] \\
    % \ruby{明赤色}{めい|せき|しょく} & light red & \\
    % \ruby{暗赤色}{あん|せき|しょく} & dark red & \\
    \ruby{橙}{だいだい}[\ruby{色}{いろ}] & orange (1, 0.64, 0) & also: オレンジ \\
    \ruby{黄}{き}[\ruby{色}{いろ}] & yellow (1, 1, 0)/amber (1, 0.75, 0) (midpoint of yellow and orange) & also: イエロー \\
    ベージュ & beige & \\
    % & & \\
    \midrule
    \ruby{黄緑}{き|みどり} & yellow-green & \\
    \ruby{緑}{みどり}[\ruby{色}{いろ}] & green (0, 1, 0) & also: グリーン \\
    \ruby{深緑色}{ふか|みどり|いろ} & dark green & \\

    % \ruby{緑色}{りょく|しょく} & green (0, 1, 0) & formal/literary [GMN] \\
    % \ruby{深緑}{しん|りょく} & dark green & formal/literary [GMN] \\
    % & & \\
    \midrule
    \ruby{青}{あお} & blue (0, 0, 1); green when used in compound words (fruits/plants/traffic lights) & also: ブルー \\
    \ruby{深青}{ふか|あお} & dark/navy blue & \\
    \ruby{抹茶色}{まっ|ちゃ|いろ} & tea green/soft yellow-green & \\
    \ruby{青色}{あお|いろ} & blue (0, 0, 1) & \\
    \ruby{水色}{みず|いろ} & light blue & \\
    \ruby{紺色}{こん|いろ} & navy/dark blue & also: ネイビー/ネービー \\
    % \ruby{青色}{せい|しょく} & blue (0, 0, 1) & formal/literary [GMN] \\
    \ruby{紫}{むらさき}[\ruby{色}{いろ}] & purple (0.5, 0, 0.5)/violet (0.5, 0, 1) & also: パープル \\
    % & & \\
    \midrule
    \midrule
    \ruby{白}{しろ} & white (1, 1, 1); innocent; blank (space) & \\
    \ruby{白色}{しろ|いろ} & white (1, 1, 1) & \\
    % \ruby{白色}{はく|しょく} & white (1, 1, 1) & formal/literary [GMN] \\
    \ruby{灰色}{はい|いろ} & grey & also: グレー/グレイ \\
    \ruby{黒}{くろ} & black (0, 0, 0); guilty & \\
    \ruby{黒色}{くろ|いろ} & black (0, 0, 0) & \\
    % \ruby{黒色}{こく|しょく} & black (0, 0, 0) & formal/literary [GMN] \\
    % & & \\
    \midrule
    \midrule
    \ruby{桃色}{もも|いろ} & pink & also: ピンク \\
    \ruby{茶}{ちゃ}[\ruby{色}{いろ}] & brown & also: ブラウン \\
    \ruby{赤茶色}{あか|ちゃ|いろ} & reddish brown & \\
    \ruby{薄茶色}{うす|ちゃ|いろ} & light/pale brown & \\
    \ruby{黒茶色}{くろ|ちゃ|いろ} & deep brown & \\
    こげ\ruby{茶色}{ちゃ|いろ} & dark/olive brown & (\ruby{焦}{こ}げ\ruby{茶色}{ちゃ|いろ}) \\
    \ruby{銀}{ぎん}[\ruby{色}{いろ}] & silver & also: シルバー \\
    \ruby{金}{きん}[\ruby{色}{いろ}] & gold & also: ゴールド \\
    \ruby{虹色}{にじ|いろ} & rainbow-coloured & \\
    \ruby{七色}{なな|いろ} & seven/prismatic colours (of the rainbow) & \\
    % & & \\
    \bottomrule
}


\subsubsection{Shapes}
% Help: \SetCell[r=2,c=2]{c,m} <content>, \cmidrule[l]{3-4}
% Help: colspec: X[ratio, horizontal alignment] columns grow to fit width=\linewidth
%                  negative ratios: shrink to fit content and may not grow to full ratio
% Help: colspec: l/c/r columns do not grow
\longtabse[0.75]  % scale factor
{Nouns: shapes.}  % caption
{tbl:appendix-vocab-nouns-shapes}  % label
{}  % outer specification options
{
    colspec={X[-3,l]X[3,l]X[-3,l]},
    rowhead=1,
    % width=\linewidth,  % useful only with X columns
}  % inner specification options
{
    \toprule
    \textbf{Name} & \textbf{Meaning} & \textbf{Notes} \\
    \midrule
    \ruby{丸}{まる} & circle & \\
    \ruby{三角形}{さん|かく|けい} & triangle & \href{https://ja.hinative.com/questions/3974177}{[HN]} \\
    バツ & cross & \\
    % & & \\
    \bottomrule
}


\subsubsection{Agreeability}
% Help: \SetCell[r=2,c=2]{c,m} <content>, \cmidrule[l]{3-4}
% Help: colspec: X[ratio, horizontal alignment] columns grow to fit width=\linewidth
%                  negative ratios: shrink to fit content and may not grow to full ratio
% Help: colspec: l/c/r columns do not grow
\longtabse[0.75]  % scale factor
{Nouns: agreeability.}  % caption
{tbl:appendix-vocab-nouns-agreeability}  % label
{}  % outer specification options
{
    colspec={X[-3,l]X[3,l]X[-3,l]},
    rowhead=1,
    % width=\linewidth,  % useful only with X columns
}  % inner specification options
{
    \toprule
    \textbf{Name} & \textbf{Meaning} & \textbf{Notes} \\
    \midrule
    \ruby{賛成}{さん|せい} & approval/agreement/support & also a verb\\
    \ruby{反対}{はん|たい} & objection/opposition/resistance/dissent & also a verb \\
    % & & \\
    \midrule
    \midrule
    \ruby{一流}{いち|りゅう} & first class/top-ranking & \\
    % & & \\
    \bottomrule
}


\subsubsection{Appearance and style}
% Help: \SetCell[r=2,c=2]{c,m} <content>, \cmidrule[l]{3-4}
% Help: colspec: X[ratio, horizontal alignment] columns grow to fit width=\linewidth
%                  negative ratios: shrink to fit content and may not grow to full ratio
% Help: colspec: l/c/r columns do not grow
\longtabse[0.75]  % scale factor
{Nouns: appearance and style.}  % caption
{tbl:appendix-vocab-nouns-appearance-and-style}  % label
{}  % outer specification options
{
    colspec={X[-3,l]X[3,l]X[-3,l]},
    rowhead=1,
    % width=\linewidth,  % useful only with X columns
}  % inner specification options
{
    \toprule
    \textbf{Name} & \textbf{Meaning} & \textbf{Notes} \\
    \midrule
    イケメン & good-looking/handsome/cool guy & \\
    \ruby{美少女}{び|しょう|じょ} & beautiful girl & \\
    % & & \\
    \midrule
    \midrule
    \ruby{魅力}{み|りょく} & charm/attraction/appeal & \\
    \ruby{味}{あじ} & charm/appeal/uniqueness/attractiveness & also in Table~\ref{tbl:appendix-vocab-nouns-taste-and-texture} \\
    \ruby{愛嬌}{あい|きょう} & charm/attractiveness; courtesy  & \\
    % & & \\
    \midrule
    \midrule
    \ruby{匂}{にお}い & smell/scent/flavour/mood & \\
    \ruby{臭}{にお}い & odour/stench & \\
    % & & \\
    \midrule
    \midrule
    \ruby{痩}{や}せた & thin/slim/skinny; barren/infertile/sterile & from \ruby{痩}{や}せる \\
    \ruby{太}{ふと}った & plump/fat/chubby & from \ruby{太}{ふと}る \\
    % & & \\
    \bottomrule
}


\subsubsection{Ability}
% Help: \SetCell[r=2,c=2]{c,m} <content>, \cmidrule[l]{3-4}
% Help: colspec: X[ratio, horizontal alignment] columns grow to fit width=\linewidth
%                  negative ratios: shrink to fit content and may not grow to full ratio
% Help: colspec: l/c/r columns do not grow
\longtabse[0.75]  % scale factor
{Nouns: ability.}  % caption
{tbl:appendix-vocab-nouns-ability}  % label
{}  % outer specification options
{
    colspec={X[-3,l]X[3,l]X[-3,l]},
    rowhead=1,
    % width=\linewidth,  % useful only with X columns
}  % inner specification options
{
    \toprule
    \textbf{Name} & \textbf{Meaning} & \textbf{Notes} \\
    \midrule
    \ruby{能力}{のう|りょく} & ability & \\
    \ruby{力}{ちから} & force/strength/power & \\
    \ruby{領域}{りょう|いき} & territory/domain/expertise; field/area/region & \\
    % & & \\
    \midrule
    \midrule
    \ruby{当}{あ}たり & success/hit & \\
    \ruby{成功}{せい|こう} & success/achievement & also a verb \\
    \ruby{失敗}{しっ|ぱい} & failure/mistake/blunder & also a verb \\
    % & & \\
    \midrule
    \midrule
    \ruby{簡易}{かん|い} & simplicity/ease/convenience & \\
    % & & \\
    \bottomrule
}


\subsubsection{Personalities}
% Help: \SetCell[r=2,c=2]{c,m} <content>, \cmidrule[l]{3-4}
% Help: colspec: X[ratio, horizontal alignment] columns grow to fit width=\linewidth
%                  negative ratios: shrink to fit content and may not grow to full ratio
% Help: colspec: l/c/r columns do not grow
\longtabse[0.75]  % scale factor
{Nouns: personalities.}  % caption
{tbl:appendix-vocab-nouns-personalities}  % label
{}  % outer specification options
{
    colspec={X[-3,l]X[3,l]X[-3,l]},
    rowhead=1,
    % width=\linewidth,  % useful only with X columns
}  % inner specification options
{
    \toprule
    \textbf{Name} & \textbf{Meaning} & \textbf{Notes} \\
    \midrule
    \ruby{人気}{にん|き} & popularity/public favour & \\
    \ruby{大人気}{だい|にん|き} & high popularity/public favour & \\
    % & & \\
    \midrule
    \ruby{無名}{む|めい} & anonymous/nameless; not famous & \\
    % & & \\
    \midrule
    \midrule
    \ruby{警戒}{けい|かい} & vigilance/caution/alertness/precaution/being on guard & also a verb \\
    % & & \\
    \midrule
    \midrule
    \ruby{自信}{じ|しん} & self-confidence & \\
    % & & \\
    \bottomrule
}


\subsubsection{Education and correctness}
% Help: \SetCell[r=2,c=2]{c,m} <content>, \cmidrule[l]{3-4}
% Help: colspec: X[ratio, horizontal alignment] columns grow to fit width=\linewidth
%                  negative ratios: shrink to fit content and may not grow to full ratio
% Help: colspec: l/c/r columns do not grow
\longtabse[0.75]  % scale factor
{Nouns: education and correctness.}  % caption
{tbl:appendix-vocab-nouns-education-and-correctness}  % label
{}  % outer specification options
{
    colspec={X[-3,l]X[3,l]X[-3,l]},
    rowhead=1,
    % width=\linewidth,  % useful only with X columns
}  % inner specification options
{
    \toprule
    \textbf{Name} & \textbf{Meaning} & \textbf{Notes} \\
    \midrule
    \ruby{教育}{きょう|いく} & education/training/upbringing & \\
    \ruby{授業}{じゅ|ぎょう} & lesson/class/teaching/instruction & \\
    % & & \\
    \midrule
    \ruby{勉強}{べん|きょう} & study/diligence/hard work & also a verb \\
    \ruby{練習}{れん|しゅう} & practice/train/drill & also a verb \\
    \ruby{自習}{じ|しゅう} & self-study & also a verb \\
    % & & \\
    \midrule
    \midrule
    \ruby{質問}{しつ|もん} & question/enquiry & also a verb \\
    \ruby{問題}{もん|だい} & problem/question & \\
    \ruby{宿題}{しゅく|だい} & homework/assignment & \\
    \ruby{質問者}{しつ|もん|しゃ} & questioner/interrogator & \\
    % & & \\
    \midrule
    \ruby{答}{こた}え & answer/reply/response & \\
    \ruby{説明}{せつ|めい} & explanation & also a verb \\
    \ruby{訳}{わけ} & reason/cause; meaning & \\
    \ruby{理由}{り|ゆう} & reason & \\
    \ruby{原因}{げん|いん} & cause & \\
    % & & \\
    \midrule
    \midrule
    \ruby{正解}{せい|かい} & correct answer/interpretation/decision/judgment & \\
    \ruby{了解}{りょう|かい} & understanding/comprehension/agreement & also an interjection, verb \\
    % & & \\
    \midrule
    \ruby{同}{おな}じ & similar/same/identical/equal/uniform/equivalent & \\
    \ruby{違}{ちが}い & difference/distinction/discrepency/miss (nuance: wrong) & \href{https://hinative.com/questions/16577683}{[HN]}, \href{https://japanese.stackexchange.com/a/30574}{[SE]} \\
    \ruby{間違}{ま|ちが}い & mistake/error/errata/blunder; accident/mishap & \\
    \ruby{別}{べつ} & distinction/difference/discrimination (nuance: another) & \href{https://japanese.stackexchange.com/a/30574}{[SE]} \\
    % & & \\
    \midrule
    \ruby{不正解}{ふ|せい|かい} & incorrect answer/solution/interpretation & \\
    \ruby{誤解}{ご|かい} & misunderstanding & \\
    % & & \\
    \midrule
    \midrule
    \ruby{部活動}{ぶ|かつ|どう} & club/extracurricular activities & \\
    \ruby{部活}{ぶ|かつ} & club/extracurricular activities (abbreviation) & \\
    \ruby{遠足}{えん|そく} & school trip/field trip/excursion/outing & \\
    % & & \\
    \bottomrule
}


\subsubsection{Academic fields}
% Help: \SetCell[r=2,c=2]{c,m} <content>, \cmidrule[l]{3-4}
% Help: colspec: X[ratio, horizontal alignment] columns grow to fit width=\linewidth
%                  negative ratios: shrink to fit content and may not grow to full ratio
% Help: colspec: l/c/r columns do not grow
\longtabse[0.75]  % scale factor
{Nouns: academic fields.}  % caption
{tbl:appendix-vocab-nouns-academic-fields}  % label
{}  % outer specification options
{
    colspec={X[-3,l]X[3,l]X[-3,l]},
    rowhead=1,
    % width=\linewidth,  % useful only with X columns
}  % inner specification options
{
    \toprule
    \textbf{Name} & \textbf{Meaning} & \textbf{Notes} \\
    \midrule
    \ruby{数学}{すう|がく} & mathematics & \\
    \ruby{科学}{か|がく} & science & \\
    \ruby{地理学}{ち|り|がく} & geography & \\
    \ruby{歴史学}{れき|し|がく} & history & \\
    \ruby{経済学}{けい|ざい|がく} & economics & \\
    \ruby{工学}{こう|がく} & engineering & \\
    \ruby{計算機科学}{けい|さん|き|か|がく} & computer science & \\
    \ruby{情報工学}{じょう|ほう|こう|がく} & information engineering & \\
    % & & \\
    \bottomrule
}


\subsubsection{Knowledge, truth and reality}
% Help: \SetCell[r=2,c=2]{c,m} <content>, \cmidrule[l]{3-4}
% Help: colspec: X[ratio, horizontal alignment] columns grow to fit width=\linewidth
%                  negative ratios: shrink to fit content and may not grow to full ratio
% Help: colspec: l/c/r columns do not grow
\longtabse[0.75]  % scale factor
{Nouns: knowledge, truth and reality.}  % caption
{tbl:appendix-vocab-nouns-knowledge-truth-and-reality}  % label
{}  % outer specification options
{
    colspec={X[-3,l]X[3,l]X[-3,l]},
    rowhead=1,
    % width=\linewidth,  % useful only with X columns
}  % inner specification options
{
    \toprule
    \textbf{Name} & \textbf{Meaning} & \textbf{Notes} \\
    \midrule
    \ruby{思}{おも}い & thought & \\
    % & & \\
    \midrule
    \midrule
    \ruby{情報}{じょう|ほう} & information/news/intelligence & \\
    \ruby{状況}{じょう|きょう} & state of affairs/situation/circumstances & \\
    \ruby{知識}{ち|しき} & knowledge/information & \\
    \ruby{意味}{い|み} & meaning/sense/significance & \\
    \ruby{未知}{み|ち} & the unknown & \\
    % & & \\
    \midrule
    \midrule
    \ruby{本物}{ほん|もの} & genuine article/real deal & \\
    \ruby{真実}{しん|じつ} & truth & \href{https://ja.hinative.com/questions/21280744}{[HN1]}, \href{https://ja.hinative.com/questions/23845869}{[HN2]} \\
    \ruby{事実}{じ|じつ} & fact & \href{https://ja.hinative.com/questions/23845869}{[HN]} \\
    \ruby{現実}{げん|じつ} & reality & \href{https://ja.hinative.com/questions/23845869}{[HN]} \\
    \ruby{理想}{り|そう} & ideal/ideals & \\
    \ruby{公式}{こう|しき} & official & \\
    % & & \\
    \midrule
    \ruby{偽物}{にせ|もの} & fake article/forgery/counterfeit/imiation  & \\
    \ruby{嘘}{うそ} & lie/fib/falsehood & also an interjection \\
    \ruby{非公式}{ひ|こう|しき} & unofficial & \\
    % & & \\
    \midrule
    \midrule
    \ruby{一般}{いっ|ぱん} & general/universal/ordinary/average/common & \\
    ただ & ordinary/common/usual & (\ruby{但}{ただ}); also an adverb \\
    \ruby{通常}{つう|じょう} & usual/ordinary/normal/regular/general/common & also an adverb \\
    % & & \\
    \midrule
    \midrule
    \ruby{観察}{かん|さつ} & observation/survey/watching & also a verb \\
    \ruby{研究}{けん|きゅう} & research/study/investigation & \\
    \ruby{実験}{じっ|けん} & experiment/experimentation & also a verb \\
    \ruby{報告}{ほう|こく} & report/information & also a verb \\
    \ruby{証明}{しょう|めい} & proof/testimony & also a verb \\
    % & & \\
    \midrule
    \midrule
    \ruby{内緒}{ない|しょ} & secret (in/out-group, personal level) & \href{https://ja.hinative.com/questions/6644230}{[HN]} \\
    \ruby{秘密}{ひ|みつ} & secret (official/corporate/country) & childish; \href{https://ja.hinative.com/questions/6644230}{[HN]} \\
    % & & \\
    \midrule
    \midrule
    \ruby{信用}{しん|よう} & trust/confidence/reputation (past) & also a verb; \href{https://japanese.stackexchange.com/q/24275}{[SE]} \\
    \ruby{信頼}{しん|らい} & trust/confidence/reliance/faith (future) & also a verb; \href{https://japanese.stackexchange.com/q/24275}{[SE]} \\
    % & & \\
    \bottomrule
}


\subsubsection{Courtesy}
% Help: \SetCell[r=2,c=2]{c,m} <content>, \cmidrule[l]{3-4}
% Help: colspec: X[ratio, horizontal alignment] columns grow to fit width=\linewidth
%                  negative ratios: shrink to fit content and may not grow to full ratio
% Help: colspec: l/c/r columns do not grow
\longtabse[0.75]  % scale factor
{Nouns: courtesy.}  % caption
{tbl:appendix-vocab-nouns-courtesy}  % label
{}  % outer specification options
{
    colspec={X[-3,l]X[3,l]X[-3,l]},
    rowhead=1,
    % width=\linewidth,  % useful only with X columns
}  % inner specification options
{
    \toprule
    \textbf{Name} & \textbf{Meaning} & \textbf{Notes} \\
    \midrule
    エチケット & etiquette/politeness/courtesy/good manners & \\
    teinei & & \\
    % & & \\
    \midrule
    \ruby{悪態}{あく|たい} & foul/abusive language & \\
    \ruby{失礼}{しつ|れい} & discourtesy/impoliteness & also a な-adjective \\
    \ruby{無礼}{ぶ|れい} & rudeness/discourtesy/insolence (stronger) & also a な-adjective \\
    % & & \\
    \midrule
    \midrule
    \ruby{最速}{さい|そく} & fastest & \\
    % & & \\
    \bottomrule
}


\subsubsection{Conflict and resolution}
% Help: \SetCell[r=2,c=2]{c,m} <content>, \cmidrule[l]{3-4}
% Help: colspec: X[ratio, horizontal alignment] columns grow to fit width=\linewidth
%                  negative ratios: shrink to fit content and may not grow to full ratio
% Help: colspec: l/c/r columns do not grow
\longtabse[0.75]  % scale factor
{Nouns: conflict and resolution.}  % caption
{tbl:appendix-vocab-nouns-conflict-and-resolution}  % label
{}  % outer specification options
{
    colspec={X[-3,l]X[3,l]X[-3,l]},
    rowhead=1,
    % width=\linewidth,  % useful only with X columns
}  % inner specification options
{
    \toprule
    \textbf{Name} & \textbf{Meaning} & \textbf{Notes} \\
    \midrule
    \ruby{作戦}{さく|せん} & tactics/strategy; military operation & \\
    % & & \\
    \midrule
    \midrule
    \ruby{勝}{か}ち & win/victory (personal) & \\
    \ruby{勝利}{しょう|り} & win/victory (larger scale) & also a verb \\
    % & & \\
    \midrule
    \ruby{負}{ま}け & loss/defeat (personal) & \\
    \ruby{敗北}{はい|ぼく} & loss/defeat (larger scale) & also a verb \\
    \ruby{全滅}{ぜん|めつ} & complete defeat/destruction/annihilation/failure & also a verb \\
    % & & \\
    \midrule
    \midrule
    \ruby{責任}{せき|にん} & duty/responsibility & \\
    <label>のせい & <label>'s fault; consequence of <label> & (\ruby[g]{所為}{せい}) \\
    \ruby{約束}{やく|そく} & promise/agreement/arrangement/contract/pact; appointment/date & also a verb \\
    % & & \\
    \midrule
    \ruby{犯罪}{はん|ざい} & crime/offence & \\
    \ruby{罰}{ばつ} & punishment/penalty & also a verb \\
    % & & \\
    \bottomrule
}


\subsubsection{Taste and texture}
% Help: \SetCell[r=2,c=2]{c,m} <content>, \cmidrule[l]{3-4}
% Help: colspec: X[ratio, horizontal alignment] columns grow to fit width=\linewidth
%                  negative ratios: shrink to fit content and may not grow to full ratio
% Help: colspec: l/c/r columns do not grow
\longtabse[0.75]  % scale factor
{Nouns: taste and texture.}  % caption
{tbl:appendix-vocab-nouns-taste-and-texture}  % label
{}  % outer specification options
{
    colspec={X[-3,l]X[3,l]X[-3,l]},
    rowhead=1,
    % width=\linewidth,  % useful only with X columns
}  % inner specification options
{
    \toprule
    \textbf{Name} & \textbf{Meaning} & \textbf{Notes} \\
    \midrule
    \ruby{味}{あじ} & flavour/taste (literal/abstract (e.g.\ victory)) & also in Table~\ref{tbl:appendix-vocab-nouns-appearance-and-style} \\
    もちもち & springy texture/elastic & \\
    % & & \\
    \bottomrule
}


\subsubsection{Amounts and sizes}
% Help: \SetCell[r=2,c=2]{c,m} <content>, \cmidrule[l]{3-4}
% Help: colspec: X[ratio, horizontal alignment] columns grow to fit width=\linewidth
%                  negative ratios: shrink to fit content and may not grow to full ratio
% Help: colspec: l/c/r columns do not grow
\longtabse[0.75]  % scale factor
{Nouns: amounts and sizes.}  % caption
{tbl:appendix-vocab-nouns-amounts-and-sizes}  % label
{}  % outer specification options
{
    colspec={X[-3,l]X[3,l]X[-3,l]},
    rowhead=1,
    % width=\linewidth,  % useful only with X columns
}  % inner specification options
{
    \toprule
    \textbf{Name} & \textbf{Meaning} & \textbf{Notes} \\
    \midrule
    \ruby{第}{だい}〜 & prefix for forming ordinal numbers & \prefix \\
    % & & \\
    \midrule
    \midrule
    \ruby{最小}{さい|しょう} & smallest/minimum & \\
    \ruby{最大}{さい|だい} & biggest/maximum & \\
    \ruby{百点}{ひゃく|てん} & hundred points/perfect mark & \\
    % & & \\
    \midrule
    \midrule
    \ruby{個別}{こ|べつ} & case-by-case/separate/individual & \\
    % & & \\
    \midrule
    \midrule
    \ruby{全員}{ぜん|いん} & all members/everyone & also an adverb \\
    % & & \\
    \bottomrule
}


\subsubsection{Creatures and divinity}
% Help: \SetCell[r=2,c=2]{c,m} <content>, \cmidrule[l]{3-4}
% Help: colspec: X[ratio, horizontal alignment] columns grow to fit width=\linewidth
%                  negative ratios: shrink to fit content and may not grow to full ratio
% Help: colspec: l/c/r columns do not grow
\longtabse[0.75]  % scale factor
{Nouns: creatures and divinity.}  % caption
{tbl:appendix-vocab-nouns-creatures-and-divinity}  % label
{}  % outer specification options
{
    colspec={X[-3,l]X[3,l]X[-3,l]},
    rowhead=1,
    % width=\linewidth,  % useful only with X columns
}  % inner specification options
{
    \toprule
    \textbf{Name} & \textbf{Meaning} & \textbf{Notes} \\
    \midrule
    \ruby{神}{かみ} & god/deity/divinity/spirit & \\
    \ruby{天使}{てん|し} & angel & \\
    \ruby{悪魔}{あく|ま} & devil/demon & \\
    \ruby{幽霊}{ゆう|れい} & ghost/spectre/apparition/phantom & \\
    % & & \\
    \midrule
    \ruby{巫女}{み|こ}/\ruby{神子}{み|こ} & shrine maiden & \href{https://detail.chiebukuro.yahoo.co.jp/qa/question_detail/q1424312974}{[YJ]} \\
    \ruby{御朱印}{ご|しゅ|いん} & seal stamp at shrines and temples & \\
    % & & \\
    \midrule
    \ruby{南無阿弥陀仏}{な|む|あ|み|だ|ぶつ} & hail Amitabha Buddha & \\
    % & & \\
    \bottomrule
}


\subsubsection{Nature}
% Help: \SetCell[r=2,c=2]{c,m} <content>, \cmidrule[l]{3-4}
% Help: colspec: X[ratio, horizontal alignment] columns grow to fit width=\linewidth
%                  negative ratios: shrink to fit content and may not grow to full ratio
% Help: colspec: l/c/r columns do not grow
\longtabse[0.75]  % scale factor
{Nouns: nature.}  % caption
{tbl:appendix-vocab-nouns-nature}  % label
{}  % outer specification options
{
    colspec={X[-3,l]X[3,l]X[-3,l]},
    rowhead=1,
    % width=\linewidth,  % useful only with X columns
}  % inner specification options
{
    \toprule
    \textbf{Name} & \textbf{Meaning} & \textbf{Notes} \\
    \midrule
    \ruby{山}{やま} & mountain/hill & \\
    \ruby{山々}{やま|やま} & mountains/hills & \\
    % & & \\
    \midrule
    \ruby{空気}{くう|き} & air/atmosphere & also in Table~\ref{tbl:appendix-vocab-nouns-emotions} \\
    \ruby{空}{そら} & the sky & \\
    \ruby{海}{うみ} & sea/ocean & \\
    % & & \\
    \midrule
    \midrule
    \ruby{光}{ひかり} & light & \\
    \ruby{日光}{にっ|こう} & sunlight/sunshine/sunbeams & \\
    \ruby{月光}{げっ|こう} & moonlight/moonbeam & \\
    \ruby{水}{みず} & water & \\
    \midrule
    \ruby{天気}{てん|き} & weather & \\
    \ruby{雨}{あめ} & rain & \\
    \ruby[g]{時雨}{しぐれ} & rain shower in late autumn/early winter; seasonal rain & \\
    \ruby[g]{時雨}{しぐれ} & seasonal rain/rain in late autumn--early winter & \\
    \ruby{雪}{ゆき} & snow & \\
    \ruby{虹}{にじ} & rainbow & \\
    % & & \\
    \midrule
    \midrule
    \ruby{花}{はな} & flower/blossom/bloom/petal & \\
    \ruby{桜}{さくら} & cherry tree/cherry blossom & \\
    \ruby{満開}{まん|かい} & full bloom (esp.\ of cherry blossom) & also a verb \\
    % & & \\
    \midrule
    \midrule
    \ruby{元素}{げん|そ} & element (chemical/classical (e.g.\ earth/water/air/fire)) & \\
    \ruby{水素}{すい|そ} & hydrogen & \\
    \ruby{炭素}{たん|そ} & carbon & \\
    \ruby{酸素}{さん|そ} & oxygen & \\
    \ruby{一酸化炭素}{いっ|さん|か|たん|そ} & carbon monoxide & \\
    \ruby{二酸化炭素}{に|さん|か|たん|そ} & carbon dioxide & \\
    % & & \\
    \midrule
    \ruby{炎}{ほのお} & flame/blaze; passion/flames (of intense emotion) & \\
    \ruby{雷}{かみなり} & lightning/thunder/thunderbolt; god of thunder/lightning & \\
    % & & \\
    \bottomrule
}


\subsubsection{Cosmic}
% Help: \SetCell[r=2,c=2]{c,m} <content>, \cmidrule[l]{3-4}
% Help: colspec: X[ratio, horizontal alignment] columns grow to fit width=\linewidth
%                  negative ratios: shrink to fit content and may not grow to full ratio
% Help: colspec: l/c/r columns do not grow
\longtabse[0.75]  % scale factor
{Nouns: cosmic.}  % caption
{tbl:appendix-vocab-nouns-cosmic}  % label
{}  % outer specification options
{
    colspec={X[-3,l]X[3,l]X[-3,l]},
    rowhead=1,
    % width=\linewidth,  % useful only with X columns
}  % inner specification options
{
    \toprule
    \textbf{Name} & \textbf{Meaning} & \textbf{Notes} \\
    \midrule
    \ruby{星}{ほし} & star (excluding the Sun); planet (excluding the Earth); heavenly body & \\
    \ruby{流星}{りゅう|せい} & meteor/shooting star & \\
    \ruby{太陽}{たい|よう} & the Sun & \\
    \ruby{日}{ひ} & the Sun & \\
    お\ruby{日様}{ひ|さま} & the Sun & children's language \\
    \ruby{月}{つき} & the Moon & \\
    お\ruby{月様}{つき|さま} & the Moon & children's language \\
    \ruby{月見}{つき|み} & moon viewing (eighth lunar month) & \\
    \ruby{火星}{か|せい} & Mars & \\
    \ruby{水星}{すい|せい} & Mercury & \\
    \ruby{木星}{もく|せい} & Jupiter & \\
    \ruby{金星}{きん|せい} & Venus & \\
    \ruby{土星}{ど|せい} & Saturn & \\
    % & & \\
    \midrule
    \midrule
    \ruby{世界}{せ|かい} & the world/the universe/society & \\
    \ruby{異世界}{い|せ|かい} & another world (esp.\ fiction)/parallel universe & \\
    % & & \\
    \midrule
    \midrule
    \ruby{運命}{うん|めい} & fate/destiny & \\
    % & & \\
    \bottomrule
}


\subsubsection{Physical units}
% Help: \SetCell[r=2,c=2]{c,m} <content>, \cmidrule[l]{3-4}
% Help: colspec: X[ratio, horizontal alignment] columns grow to fit width=\linewidth
%                  negative ratios: shrink to fit content and may not grow to full ratio
% Help: colspec: l/c/r columns do not grow
\longtabse[0.75]  % scale factor
{Nouns: physical units.}  % caption
{tbl:appendix-vocab-nouns-physical-units}  % label
{}  % outer specification options
{
    colspec={X[-3,l]X[3,l]X[-3,l]},
    rowhead=1,
    % width=\linewidth,  % useful only with X columns
}  % inner specification options
{
    \toprule
    \textbf{Name} & \textbf{Meaning} & \textbf{Notes} \\
    \midrule
    ページ/㌻ & page & \\
    ポイント/㌽ & point & \\
    パーセント/㌫ & percent & \\
    % & & \\
    \midrule
    \midrule
    オングストローム & angstrom (1 \AA = \SI{1e-10}{\metre}) & \\
    ミクロン/㍈ & micron (micrometre) & \\
    メートル/㍍ & metre & \\
    インチ/㌅ & inch & \\
    フィート/㌳ & feet (1 ft = 12 in) & \\
    ヤード/㍎ & yard (1 yd = 3 ft = 36 in) & \\
    キロメートル/㌖ & kilometre & \\
    マイル/㍄ & mile (1 mi = 1760 yd = 5280 ft) & \\
    \ruby{海里}{かい|り}/ノーティカルマイル & nautical mile (1 nmi = \SI{1852}{\metre}) & \\
    % & & \\
    \midrule
    \ruby{平方}{へい|ほう}メートル & square metre & \\
    ヘクタール/㌶ & hectare (1 ha = \SI{10000}{\square\metre} = \SI{0.01}{\square\kilo\metre}) & \\
    \ruby{平方}{へい|ほう}キロメートル & square kilometre (\SI{1}{\square\kilo\metre} = \SI{1000000}{\square\metre}) & \\
    エーカー/㌈ & acre (1 ac = 4840 sq yd = 43560 sq ft = \nicefrac{1}{640} sq mile) & \\
    % & & \\
    \midrule
    ヘルツ/㌹ & hertz (\SI{1}{\hertz} = \SI{1}{\per\second}) & \\
    ノット/㌩ & knot (1 kt = 1 nmi \SI{}{\per\hour}) & \\
    マッハ/㍅ & mach (multiple of speed of sound) & \\
    % & & \\
    \midrule
    グラム/㌘ & gram & \\
    キログラム/㌕ & kilogram & \\
    トン/㌧ & ton & \\
    % & & \\
    \midrule
    \ruby{立方}{りっ|ぽう}センチメートル & cubic centimetre & \\
    リットル/㍑ & litre (1 L = \SI{1000}{\cubic\centi\metre} = \SI{0.001}{\cubic\metre}) & \\
    ガロン/㌎ & gallon (there's a US one and a UK one\dots, both \lessapprox{} \SI{5}{\litre}) & \\
    \ruby{立方}{りっ|ぽう}メートル & cubic metre & \\
    % & & \\
    \midrule
    ジュール & joule (\SI{1}{\joule} = \SI{1}{\kilo\gram\metre\squared\per\second\squared}; $W = F\cdot s$; $F = ma$) & \\
    カロリー/㌍ & calorie (\approx{} \SI{4.184}{\joule}) & \\
    メガトン/㍌ & megaton (TNT equivalent) & \\
    キロワット\ruby{時}{じ} & kilowatt hour (\SI{}{\kilo\watt\hour}) & \\
    ワット/㍗ & watt (\SI{1}{\watt} = \SI{1}{\joule\per\second}; $P = \nicefrac{E}{t}$) & \\
    キロワット/㌗ & kilowatt & \\
    % & & \\
    \midrule
    アンペア/㌂ & ampere & \\
    クーロン & coulomb (\SI{1}{\coulomb} = \SI{1}{\ampere\second}) & \\
    ボルト/㌾ & volt (\SI{1}{\volt} = \SI{1}{\joule\per\coulomb}; $V = E/Q$) & \\
    オーム/おーむ & ohm (\SI{1}{\ohm} = \SI{1}{\volt\per\ampere}; $R = \nicefrac{V}{I}$) & \\
    ファラッド/㌲ & farad (capacitance; \SI{1}{\farad} = \SI{1}{\coulomb\per\volt}) & \\
    % & & \\
    \midrule
    \ruby{摂氏}{せっ|し} & Celsius/centigrade & also: セし \\
    \ruby{摂氏温度}{せっ|し|おん|ど} & degrees Celsius & also: セし\ruby{温度}{おん|ど} \\
    % & & \\
    \midrule
    \midrule
    \ruby{円}{えん}/¥ & Japanese yen & \\
    セント/㌣ & cents & \\
    ドル/㌦ & dollar & \\
    ユアン/㍐ & Chinese yuan & \\
    % & & \\
    \midrule
    \midrule
    ギガ/㌐ & giga- & \\
    メが/㍋ & mega- & \\
    キロ/㌔ & kilo- & \\
    デシ/㌥ & deci- & \\
    センチ/㌢ & centi- & \\
    ミリ/㍉ & milli- & \\
    マイクロ/㍃ & micro- & \\
    ナノ/㌨ & nano- & \\
    ピコ/㌰ & pico- & \\
    % & & \\
    \midrule
    \midrule
    アルファ/㌁ & alpha & \\
    ベータ/㌼ & beta & \\
    ガンマ/㌏ & gamma & \\
    % & & \\
    \bottomrule
}


\subsubsection{Hygiene}
% Help: \SetCell[r=2,c=2]{c,m} <content>, \cmidrule[l]{3-4}
% Help: colspec: X[ratio, horizontal alignment] columns grow to fit width=\linewidth
%                  negative ratios: shrink to fit content and may not grow to full ratio
% Help: colspec: l/c/r columns do not grow
\longtabse[0.75]  % scale factor
{Nouns: hygiene.}  % caption
{tbl:appendix-vocab-nouns-hygiene}  % label
{}  % outer specification options
{
    colspec={X[-3,l]X[3,l]X[-3,l]},
    rowhead=1,
    % width=\linewidth,  % useful only with X columns
}  % inner specification options
{
    \toprule
    \textbf{Name} & \textbf{Meaning} & \textbf{Notes} \\
    \midrule
    うんこ/ウンコ & poop & \\
    クソ & feces/excrement/dung/damned/blasted/stupid & (\ruby{糞}{くそ}) \\
    ごみ/ゴミ & trash/rubbish/garbage/refuse & (\ruby{塵}{ごみ}) \\
    % & & \\
    \bottomrule
}


\subsubsection{Common names}
% Help: \SetCell[r=2,c=2]{c,m} <content>, \cmidrule[l]{3-4}
% Help: colspec: X[ratio, horizontal alignment] columns grow to fit width=\linewidth
%                  negative ratios: shrink to fit content and may not grow to full ratio
% Help: colspec: l/c/r columns do not grow
\longtabse[0.75]  % scale factor
{Nouns: common names.}  % caption
{tbl:appendix-vocab-nouns-common-names}  % label
{}  % outer specification options
{
    colspec={X[-3,l]X[3,l]X[-3,l]},
    rowhead=1,
    % width=\linewidth,  % useful only with X columns
}  % inner specification options
{
    \toprule
    \textbf{Name} & \textbf{Meaning} & \textbf{Notes} \\
    \midrule
    \ruby{鈴木}{すず|き} & Suzuki (last name) & \\
    \ruby{田中}{た|なか} & Tanaka (last name) & \\
    \ruby{山田}{やま|だ} & Yamada (last name) & \\
    \ruby{加賀}{か|が} & Kaga (last name) & \\
    % & & \\
    \midrule
    \midrule
    \ruby{一郎}{いち|ろう} & Ichirou (first name) & \\
    \ruby{直子}{なお|こ} & Naoko (first name) & \\
    \ruby{美恵}{み|え} & Mie (first name) & \\
    \ruby[g]{智子}{ともこ} & Tomoko (first name; female) & \\
    \ruby[g]{洋介}{ようすけ} & Yousuke (first name; male) & \\
    % & & \\
    \bottomrule
}


\subsubsection{Character names}
% Help: \SetCell[r=2,c=2]{c,m} <content>, \cmidrule[l]{3-4}
% Help: colspec: X[ratio, horizontal alignment] columns grow to fit width=\linewidth
%                  negative ratios: shrink to fit content and may not grow to full ratio
% Help: colspec: l/c/r columns do not grow
\longtabse[0.75]  % scale factor
{Nouns: character names.}  % caption
{tbl:appendix-vocab-nouns-character-names}  % label
{}  % outer specification options
{
    colspec={X[-3,l]X[3,l]X[-3,l]},
    rowhead=1,
    % width=\linewidth,  % useful only with X columns
}  % inner specification options
{
    \toprule
    \textbf{Name} & \textbf{Meaning} & \textbf{Notes} \\
    \midrule
    \ruby[g]{胡桃}{フータオ} & Hu Tao & (くるみ) \\
    \ruby{神里綾華}{かみ|さと|あや|か} & Kamisato Ayaka & \\
    \ruby{八重神子}{や|え|み|こ} & Yae Miko & \\
    % & & \\
    \bottomrule
}
\end{document}
