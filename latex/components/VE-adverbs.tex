\documentclass[../nihongo-gakushuu-kyouzai-vocabulary.tex]{subfiles}
\begin{document}
\appendix
\setcounter{section}{4}

\section{\ruby{副詞}{ふく|し}と\ruby{接続詞}{せつ|ぞく|し} (adverbs and conjunctions)}
Adverbs modify both verbs and adjectives. They may also modify entire noun phrases or sentences.


\subsection{Grammatical}
% Help: \SetCell[r=2,c=2]{c,m} <content>, \cmidrule[l]{3-4}
% Help: colspec: X[ratio, horizontal alignment] columns grow to fit width=\linewidth
%                  negative ratios: shrink to fit content and may not grow to full ratio
% Help: colspec: l/c/r columns do not grow
\longtabse[0.75]  % scale factor
{Adverbs: grammatical.}  % caption
{tbl:appendix-vocab-adverbs-grammatical}  % label
{}  % outer specification options
{
    colspec={X[-3,l]X[3,l]X[-3,l]},
    rowhead=1,
    % width=\linewidth,  % useful only with X columns
}  % inner specification options
{
    \toprule
    \textbf{Modifier} & \textbf{Meaning} & \textbf{Notes} \\
    \midrule
    \ruby{例}{たと}えば & for example/for instance & \\
    つまり & in short/in other words & (\ruby{詰}{つ}まり) \\
    % & & \\
    \midrule
    まず & firstly & (\ruby{先}{ま}ず) \\
    \ruby{取}{と}りあえず & first of all/right away & also in Tables~\ref{tbl:appendix-vocab-adverbs-time}, \ref{tbl:appendix-vocab-adverbs-knowledge-truth-and-reality} \\
    \ruby{次}{つ}いで & secondly/next/subsequently & also a \conjunction \\
    {CだからE\\{}[CですからE]} & therefore (to speaker, E is a natural consequence of C; strong expression of speaker's attitude) & {\conjunction; also an expression; \href{https://www.youtube.com/watch?v=DSYc2BQrJEY}{[MCJ]}\\{}[polite]} \\
    CそれでE & (objective そ) therefore (general cause-and-effect; E must have certainly happened (either past or now)) & \conjunction; \href{https://dictionary.goo.ne.jp/thsrs/16809/meaning/m0u/}{[goo]}, \href{https://www.youtube.com/watch?v=DSYc2BQrJEY}{[MCJ]} \\
    CそこでE & (objective そ) therefore (C is problem/situation, E is action taken to solve/improve/advance) & \conjunction; \href{https://dictionary.goo.ne.jp/thsrs/16809/meaning/m0u/}{[goo]}, \href{https://www.youtube.com/watch?v=DSYc2BQrJEY}{[MCJ]} \\
    CするとE & thereupon (E happens \emph{immediately} after C) & \conjunction; \href{https://www.youtube.com/watch?v=DSYc2BQrJEY}{[MCJ]} \\
    しかし & however/but & \conjunction \\
    \ruby{但}{ただ} & only/merely/just/simply; but/however/nonetheless & also an noun \\
    ただし & but/however/provided that & (\ruby{但}{ただ}し) \\
    ために & for the sake of; because of/as a result of & (\ruby{為}{ため}に); \conjunction \\
    ちなみに & by the way/incidentally/in passing & (\ruby{因}{ちな}みに) \\
    っていう & meaning/called/said & (って\ruby{言}{い}う) slang of という \\
    <that>わけではない & it does not mean that/I don't mean that & (\ruby{訳}{わけ}ではない); technically an expression \\
    もし & if/in case/supposing & (\ruby{若}{も}し) \\
    % & & \\
    \bottomrule
}


\subsection{Directions}
% Help: \SetCell[r=2,c=2]{c,m} <content>, \cmidrule[l]{3-4}
% Help: colspec: X[ratio, horizontal alignment] columns grow to fit width=\linewidth
%                  negative ratios: shrink to fit content and may not grow to full ratio
% Help: colspec: l/c/r columns do not grow
\longtabse[0.75]  % scale factor
{Adverbs: directions.}  % caption
{tbl:appendix-vocab-adverbs-directions}  % label
{}  % outer specification options
{
    colspec={X[-3,l]X[3,l]X[-3,l]},
    rowhead=1,
    % width=\linewidth,  % useful only with X columns
}  % inner specification options
{
    \toprule
    \textbf{Modifier} & \textbf{Meaning} & \textbf{Notes} \\
    \midrule
    まっすぐ & straight (ahead)/directly/uprightly/erectly & (\ruby{真}{ま}っ\ruby{直}{す}ぐ) \\
    % & & \\
    \bottomrule
}


\subsection{Intensity modifiers}
% Help: \SetCell[r=2,c=2]{c,m} <content>, \cmidrule[l]{3-4}
% Help: colspec: X[ratio, horizontal alignment] columns grow to fit width=\linewidth
%                  negative ratios: shrink to fit content and may not grow to full ratio
% Help: colspec: l/c/r columns do not grow
\longtabse[0.75]  % scale factor
{Adverbs: intensity modifiers.}  % caption
{tbl:appendix-vocab-adverbs-intensity}  % label
{}  % outer specification options
{
    colspec={X[-3,l]X[3,l]X[-3,l]},
    rowhead=1,
    % width=\linewidth,  % useful only with X columns
}  % inner specification options
{
    \toprule
    \textbf{Modifier} & \textbf{Meaning} & \textbf{Notes} \\
    \midrule
    \ruby{全然}{ぜん|ぜん}<negative> & not at all & \\
    % & & \\
    \midrule
    とても<negative> & not at all/simply cannot & \\
    あまり<negative> & not very & (\ruby{余}{あま}り); slightly formal \href{https://hinative.com/questions/19606346}{[HN1]}, \href{https://ja.hinative.com/questions/19223174}{[HN2]} \\
    \ruby{別}{べつ}に<negative> & not particularly (nuance: not interested) & slightly informal, can be rude; \href{https://hinative.com/questions/19606346}{[HN1]}, \href{https://ja.hinative.com/questions/19223174}{[HN2]} \\
    % & & \\
    \midrule
    % <so>のあまり<verb>& so much <so> that you <verb> & 嬉しさのあまり\ruby{泣}{な}いた。\\
    \ruby{少}{すこ}し & somewhat/slightly/a little & \\
    ちょっと & a bit/slightly/somewhat/quite; just a minute & \\
    \ruby{一歩}{いっ|ぽ} & small degree/small amount (one step) & \\
    % & & \\
    \midrule
    かなり & quite/considerably/pretty & \\
    なかなか & very/considerably/fairly/quite/rather & (\ruby{中々}{なか|なか}) \\
    \ruby{相当}{そう|とう} & considerably/rather/quite/fairly/pretty & also an adjective, verb \\
    そこそこ & reasonably/fairly/all right/moderate & \onomatopoeic, also in Table~\ref{tbl:appendix-vocab-adverbs-amounts-and-sizes} \\
    いい\ruby{加減}{か|げん} & considerably/quite/rather/pretty enough (wanting something to end) & also an adjective \\
    \ruby{大}{だい}〜 & large/big/great/severe & \prefix. \htc; technically な-adj/noun \\
    すごく & very/immensely/awfully & (\ruby{凄}{すご}く) \\
    \ruby{全}{まった}く & really/truly/entirely/completely/perfectly; indeed & also a noun \\
    \ruby{随分}{ずい|ぶん} & surprisingly/very/extremely/fairly/quite/considerably/awfully/terribly & \\
    \ruby{大変}{たい|へん} & very/greatly/terribly/awfully & also an adjective \\
    そりゃ & very/extremely & \\
    とても & very/exceedingly/awfully & \\
    \ruby{特別}{とく|べつ} & especially/particularly/extraordinarily/exceptionally & also an adjective \\
    % & & \\
    \midrule
    \ruby{全然}{ぜん|ぜん} & extremely/very & e.g.\ 「\ruby{全然}{ぜん|ぜん}いいよ」 \\
    \ruby{全部}{ぜん|ぶ} & entirely/wholly/altogether & also a noun \\
    すべて & entirely/completely/wholly & (\ruby{全}{すべ}て) \\
    % & & \\
    \midrule
    \midrule
    たくさん & a lot/lots/plenty/much/a great deal; enough/too much & (\ruby{沢山}{たく|さん}); also an adjective \\
    いっぱい & fully/as much as possible; a lot/many; all of & (\ruby{一杯}{いっ|ぱい}); also a noun and adjective \\
    たっぷり & plentifully/with excess/amply/abundantly/copiously/generously/fully/a lot & \onomatopoeic \\
    % & & \\
    \midrule
    \midrule
    わざと & purposely/deliberately/intentionally & \\
    \ruby{偶然}{ぐう|ぜん} & coincidentally/by chance/unexpectedly/accidentally & also a noun \\
    % & & \\
    \midrule
    \midrule
    もしや & possibly/perhaps/by some chance & (\ruby{若}{も}しや) \\
    もしかし & maybe/perhaps/by some chance & (\ruby{若}{も}しかし) \\
    もしかして & perhaps/possibly/maybe/by any chance/if I'm not mistaken & (\ruby{若}{も}ししかして) \\
    \ruby{確}{たし}か & if I'm not mistaken/if I remember correctly & also a noun \\
    \ruby{多分}{た|ぶん} & probably/perhaps & \\
    まず & probably/most likely/almost certainly & (\ruby{先}{ま}ず) \\
    % & & \\
    \midrule
    \ruby{確}{たし}かに & certainly/for sure/indeed/really & \\
    % & & \\
    \midrule
    \midrule
    \ruby{一体}{いっ|たい} & (what) the heck/(why) in the world/(who) on earth (emphatic question prefix) & \\
    % & & \\
    \bottomrule
}


\subsection{Time}
% Help: \SetCell[r=2,c=2]{c,m} <content>, \cmidrule[l]{3-4}
% Help: colspec: X[ratio, horizontal alignment] columns grow to fit width=\linewidth
%                  negative ratios: shrink to fit content and may not grow to full ratio
% Help: colspec: l/c/r columns do not grow
\longtabse[0.75]  % scale factor
{Adverbs: time.}  % caption
{tbl:appendix-vocab-adverbs-time}  % label
{}  % outer specification options
{
    colspec={X[-3,l]X[3,l]X[-3,l]},
    rowhead=1,
    % width=\linewidth,  % useful only with X columns
}  % inner specification options
{
    \toprule
    \textbf{Modifier} & \textbf{Meaning} & \textbf{Notes} \\
    \midrule
    \ruby{一瞬}{いっ|しゅん} & momentarily/for an instant & \\
    \ruby{1日中}{いち|にち|じゅう} & all day long/throughout the day & note \ruby{中}{じゅう} \\
    \ruby{末永}{すえ|なが}く & everlastingly/forever/for many years to come & \\
    いつまでも & eternally/indefinitely/endlessly/forever/for a long time & (\ruby[g]{何時}{いつ}までも) \\
    % & & \\
    \midrule
    \midrule
    すぐ & immediately/at once/right away; soon; easily; right near/nearby & (\ruby{直}{す}ぐ) \\
    すぐに & immediately/at once/right away/instantly & (\ruby{直}{す}ぐに) \\
    \ruby{今}{いま}すぐ & immediately/at once/right now & (\ruby{今直}{いま|す}ぐ) \\
    さっさと & immediately/without delay/hurriedly/quickly & \\
    \ruby{早}{はや}く & early/soon/quickly/swiftly/rapidly & \\
    そろそろ & soon/it's about time/any time now (expresses impatience) & \\
    % & & \\
    \midrule
    \ruby{遅}{おそ}く & late/slowly & \\
    ゆっくり & slowly/unhurriedly/without haste/leisurely & \onomatopoeic \\
    % & & \\
    \midrule
    \midrule
    だんだん & gradually/little by little/more and more/increasingly & (\ruby{段々}{だん|だん}) \\
    % & & \\
    \midrule
    まだ & not yet/still & (\ruby{未}{ま}だ) \\
    もう & already; not any more/longer; again/another & again/another: used with counting 1 \\
    ようやく & finally/at last & \\
    % & & \\
    \midrule
    ちらっと & at a glance/by accident & \onomatopoeic \\
    % & & \\
    \midrule
    もともと & originally/from the start/from the onset & (\ruby{元々}{もと|もと}) \\
    % & & \\
    \midrule
    今のところ & at present/currently/so far/for now/for the time being & \\
    今のとこ & at present/currently/so far/for now/for the time being & slang, abbreviation \\
    \ruby{取}{と}りあえず & tentatively/for now/for the time being & also in Tables~\ref{tbl:appendix-vocab-adverbs-grammatical}, \ref{tbl:appendix-vocab-adverbs-knowledge-truth-and-reality} \\
    % & & \\
    \midrule
    \ruby{長}{なが}い\ruby{間}{あいだ} & for quite some time/a long time & also a noun \\
    ずっと & the whole time/continuously; much (more); (by) far & \\
    \ruby{引}{ひ}き\ruby{続}{つづ}き & continuously/continually/continued/without a break & \\
    いつも & always & (\ruby[g]{何時}{いつ}も) \\
    % & & \\
    \midrule
    \midrule
    これから & from now on/in the future; from here & also a noun \\
    % & & \\
    \bottomrule
}


\subsection{Attitude}
% Help: \SetCell[r=2,c=2]{c,m} <content>, \cmidrule[l]{3-4}
% Help: colspec: X[ratio, horizontal alignment] columns grow to fit width=\linewidth
%                  negative ratios: shrink to fit content and may not grow to full ratio
% Help: colspec: l/c/r columns do not grow
\longtabse[0.75]  % scale factor
{Adverbs: attitude.}  % caption
{tbl:appendix-vocab-adverbs-attitude}  % label
{}  % outer specification options
{
    colspec={X[-3,l]X[3,l]X[-3,l]},
    rowhead=1,
    % width=\linewidth,  % useful only with X columns
}  % inner specification options
{
    \toprule
    \textbf{Modifier} & \textbf{Meaning} & \textbf{Notes} \\
    \midrule
    ぶらぶら & (walking) leisurely/aimlessly & \onomatopoeic, also a verb \\
    \ruby{遠慮}{えん|りょ}なく & without reservation/freely & \\
    \ruby{静}{しず}かに & calmly/quietly/gently/peacefully & also an expression \\
    % & & \\
    \midrule
    ちゃんと & diligently/seriously/earnestly; properly/perfectly/exactly/regularly; quickly & \onomatopoeic \\
    \ruby{大切}{たい|せつ}に & carefully/with great care & also an adjective, verb \\
    \ruby{詳}{くわ}しく & in detail/fully/at length & \\
    よろしく & properly/well/suitably; please do & (\ruby{宜}{よろ}しく) \\
    \ruby{4649}{よ|ろ|し|く} & properly/well/suitably; please do & (\ruby{宜}{よろ}しく); slang \\
    <\dots>よろしく & just like/as though one were <\dots> & (\ruby{宜}{よろ}しく) \\
    % & & \\
    \midrule
    \ruby{絶対}{ぜっ|たい} & absolutely/definitely/unconditionally & \\
    よろしく<\dots>べし & by all means/of course do <\dots> & (\ruby{宜}{よろ}しく) \\
    お\ruby{腹}{なか}いっぱい & to one's heart's content & also a noun \\
    % & & \\
    \midrule
    \midrule
    \ruby{普通}{ふ|つう}に & normally/ordinarily/usually/generally/commonly & \\
    \ruby{通常}{つう|じょう} & usually/ordinarily/normally/regularly/generally/commonly & \\
    \ruby{大体}{だい|たい} & generally/on the whole/mostly/almost/nearly/approximately/roughly/about & \\
    % & & \\
    % & & \\
    \midrule
    \midrule
    \ruby{本当}{ほん|とう}に/\ruby{本当}{ほん|と}に & really/truly & \\
    \ruby{正直}{しょう|じき} & honestly/frankly & also an adjective \\
    % & & \\
    \midrule
    \midrule
    \ruby{別}{べつ}に & separately/additionally/extra & \\
    % & & \\
    \midrule
    \midrule
    ツンツン & aloof/cold/unfriendly/standoffish & \onomatopoeic; also a verb; also in Table~\ref{tbl:appendix-vocab-adverbs-taste-and-texture} \\
    デレデレ & flirting/philandering/being lovestruck/fawning & \onomatopoeic \\
    % & & \\
    \bottomrule
}


\subsection{Emotions}
% Help: \SetCell[r=2,c=2]{c,m} <content>, \cmidrule[l]{3-4}
% Help: colspec: X[ratio, horizontal alignment] columns grow to fit width=\linewidth
%                  negative ratios: shrink to fit content and may not grow to full ratio
% Help: colspec: l/c/r columns do not grow
\longtabse[0.75]  % scale factor
{Adverbs: emotions.}  % caption
{tbl:appendix-vocab-adverbs-emotions}  % label
{}  % outer specification options
{
    colspec={X[-3,l]X[3,l]X[-3,l]},
    rowhead=1,
    % width=\linewidth,  % useful only with X columns
}  % inner specification options
{
    \toprule
    \textbf{Modifier} & \textbf{Meaning} & \textbf{Notes} \\
    \midrule
    \ruby{喜}{よろこ}んで & with pleasure/gladly/willingly/certainly & \\
    ドキドキ & thump-thump/bang-bang/pit-a-pat/pitter-patter & \onomatopoeic; also a verb \\
    キュン & with a pitter-patter/heart-wringing/tightening of one's chest caused by powerful feelings (e.g.\ parting); 「\ruby{胸}{むね}がキュンとなる」 & \onomatopoeic \\
    ムカムカ & feeling sick/queasy/nauseated/disgusted & also a verb \\
    ソワソワ & restlessly/nervously/uneasily/in a fidget & \onomatopoeic; also a verb \\
    % & & \\
    \bottomrule
}


\subsection{Appearance and style}
% Help: \SetCell[r=2,c=2]{c,m} <content>, \cmidrule[l]{3-4}
% Help: colspec: X[ratio, horizontal alignment] columns grow to fit width=\linewidth
%                  negative ratios: shrink to fit content and may not grow to full ratio
% Help: colspec: l/c/r columns do not grow
\longtabse[0.75]  % scale factor
{Adverbs: appearance and style.}  % caption
{tbl:appendix-vocab-adverbs-appearance-and-style}  % label
{}  % outer specification options
{
    colspec={X[-3,l]X[3,l]X[-3,l]},
    rowhead=1,
    % width=\linewidth,  % useful only with X columns
}  % inner specification options
{
    \toprule
    \textbf{Modifier} & \textbf{Meaning} & \textbf{Notes} \\
    \midrule
    こう & in this way (closer to speaker) & \\
    そう & in that way (closer to listener) & also an interjection \\
    ああ & in that way (distant) & \\
    こんあふうに & approximately in this way (closer to speaker) & (こんな\ruby{風}{ふう}に) \\
    そんあふうに & approximately in that way (closer to listener) & (そんな\ruby{風}{ふう}に) \\
    あんあふうに & approximately in that way (distant) & (あんな\ruby{風}{ふう}に) \\
    % & & \\
    \midrule
    \midrule
    キラキラ & glittering/sparkling/glistening/twinkling & \onomatopoeic \\
    ぴょんぴょん & hopping/skipping/lightly and repeatedly jumping & \onomatopoeic \\
    % & & \\
    \bottomrule
}


\subsection{Interaction}
% Help: \SetCell[r=2,c=2]{c,m} <content>, \cmidrule[l]{3-4}
% Help: colspec: X[ratio, horizontal alignment] columns grow to fit width=\linewidth
%                  negative ratios: shrink to fit content and may not grow to full ratio
% Help: colspec: l/c/r columns do not grow
\longtabse[0.75]  % scale factor
{Adverbs: interaction.}  % caption
{tbl:appendix-vocab-adverbs-interaction}  % label
{}  % outer specification options
{
    colspec={X[-3,l]X[3,l]X[-3,l]},
    rowhead=1,
    % width=\linewidth,  % useful only with X columns
}  % inner specification options
{
    \toprule
    \textbf{Modifier} & \textbf{Meaning} & \textbf{Notes} \\
    \midrule
    \ruby{久}{ひさ}しぶりに & for the first time in a while/after a long time & \\
    % & & \\
    \midrule
    \ruby{一緒}{いっ|しょ} & together/at the same time; identical & \\
    <with>と\ruby{一緒}{いっ|しょ}に<verb> & together with & \\
    % & & \\
    \midrule
    \midrule
    どうか & please/if you would/would you mind (sentence starter) & polite; also a pronoun \\
    どうぞ & please/by all means/certainly/of course/go ahead/feel free to; here you are (passing something) & \\
    \ruby{是非}{ぜ|ひ} & certainly/without fail/by all means & \\
    ぜひぜひ & certainly/by all means & (\ruby{是非是非}{ぜ|ひ|ぜ|ひ}) \\
    % & & \\
    \bottomrule
}


\subsection{Knowledge, truth and reality}
% Help: \SetCell[r=2,c=2]{c,m} <content>, \cmidrule[l]{3-4}
% Help: colspec: X[ratio, horizontal alignment] columns grow to fit width=\linewidth
%                  negative ratios: shrink to fit content and may not grow to full ratio
% Help: colspec: l/c/r columns do not grow
\longtabse[0.75]  % scale factor
{Adverbs: knowledge, truth and reality.}  % caption
{tbl:appendix-vocab-adverbs-knowledge-truth-and-reality}  % label
{}  % outer specification options
{
    colspec={X[-3,l]X[3,l]X[-3,l]},
    rowhead=1,
    % width=\linewidth,  % useful only with X columns
}  % inner specification options
{
    \toprule
    \textbf{Modifier} & \textbf{Meaning} & \textbf{Notes} \\
    \midrule
    \ruby{実}{じつ}は & to be honest/frankly/to tell you the truth & \\
    % & & \\
    \midrule
    \midrule
    やはり & as expected/sure enough; in any case/after all/in the end & \\
    やっぱり & as expected/sure enough; in any case/after all/in the end & \\
    \ruby{当然}{とう|ぜん} & naturally/rightly/deservedly/justly & also an adjective \\
    さすが & just as you'd expect from & (\ruby[g]{流石}{さすが}) \\
    さすがに & as one would expect/naturally/indeed & (\ruby[g]{流石}{さすが}に) \\
    もちろん & of course/certainly/naturally/definitely & (\ruby{勿論}{もち|ろん}) \\
    % & & \\
    \midrule
    とにかく & anyway/in any case & (\ruby{兎}{と}に\ruby{角}{かく}) \\
    \ruby{取}{と}りあえず & anyway & also in Tables~\ref{tbl:appendix-vocab-adverbs-grammatical}, \ref{tbl:appendix-vocab-adverbs-time} \\
    % & & \\
    \bottomrule
}


\subsection{Ability}
% Help: \SetCell[r=2,c=2]{c,m} <content>, \cmidrule[l]{3-4}
% Help: colspec: X[ratio, horizontal alignment] columns grow to fit width=\linewidth
%                  negative ratios: shrink to fit content and may not grow to full ratio
% Help: colspec: l/c/r columns do not grow
\longtabse[0.75]  % scale factor
{Adverbs: ability.}  % caption
{tbl:appendix-vocab-adverbs-ability}  % label
{}  % outer specification options
{
    colspec={X[-3,l]X[3,l]X[-3,l]},
    rowhead=1,
    % width=\linewidth,  % useful only with X columns
}  % inner specification options
{
    \toprule
    \textbf{Modifier} & \textbf{Meaning} & \textbf{Notes} \\
    \midrule
    ペラペラ & fluently (speaking a foreign language) & also an adjective \\
    % & & \\
    \bottomrule
}


\subsection{Taste and texture}
% Help: \SetCell[r=2,c=2]{c,m} <content>, \cmidrule[l]{3-4}
% Help: colspec: X[ratio, horizontal alignment] columns grow to fit width=\linewidth
%                  negative ratios: shrink to fit content and may not grow to full ratio
% Help: colspec: l/c/r columns do not grow
\longtabse[0.75]  % scale factor
{Adverbs: taste and texture.}  % caption
{tbl:appendix-vocab-adverbs-taste-and-texture}  % label
{}  % outer specification options
{
    colspec={X[-3,l]X[3,l]X[-3,l]},
    rowhead=1,
    % width=\linewidth,  % useful only with X columns
}  % inner specification options
{
    \toprule
    \textbf{Modifier} & \textbf{Meaning} & \textbf{Notes} \\
    \midrule
    ペロペロ & licking/lapping up; gobbling up & \\
    % & & \\
    \midrule
    \midrule
    ふわふわ & lightly/buoyantly & \onomatopoeic, also an adjective \\
    % & & \\
    \midrule
    \midrule
    ぷにぷに & squishy/springy/bouncy (chubby when used on person) & \onomatopoeic \\
    ガリガリ & hard/crunchy (of muscles, when used on person) & \onomatopoeic \\
    ツンツン & spiky (hairstyle); sticking up straight (e.g.\ plant stems) & \onomatopoeic; also in Table~\ref{tbl:appendix-vocab-adverbs-attitude} \\
    % & & \\
    \bottomrule
}


\subsection{Amounts and sizes}
% Help: \SetCell[r=2,c=2]{c,m} <content>, \cmidrule[l]{3-4}
% Help: colspec: X[ratio, horizontal alignment] columns grow to fit width=\linewidth
%                  negative ratios: shrink to fit content and may not grow to full ratio
% Help: colspec: l/c/r columns do not grow
\longtabse[0.75]  % scale factor
{Adverbs: amounts and sizes.}  % caption
{tbl:appendix-vocab-adverbs-amounts-and-sizes}  % label
{}  % outer specification options
{
    colspec={X[-3,l]X[3,l]X[-3,l]},
    rowhead=1,
    % width=\linewidth,  % useful only with X columns
}  % inner specification options
{
    \toprule
    \textbf{Modifier} & \textbf{Meaning} & \textbf{Notes} \\
    \midrule
    どんなに & to what extent/amount & \\
    こんなに & to this extent/amount & \\
    そんなに & to that extent/amount & \\
    あんなに & to that extent/amount (distant memory) & \\
    どこまで & how far/to what extent/up to what point & \\
    \midrule
    もっと & some more/even more/longer/further & \\
    % & & \\
    \midrule
    ギリギリ & just barely/only just/at the very limit/at the last moment & (\ruby{限}{ぎ}り\ruby{限}{ぎ}り) \\
    ピッタリ & tightly/closely; exactly/precisely; perfectly suited/in an ideal manner & \\
    % & & \\
    \midrule
    \ruby{半分}{はん|ぶん} & half & also a noun \\
    \ruby{十分}{じゅう|ぶん} & sufficiently/fully/thoroughly/well/perfectly & also an adjective \\
    % & & \\
    \midrule
    ほぼ & almost/roughly/approximately/about/around & \\
    そこそこ & approximately/about/or so & also in Table~\ref{adverb-tbl:appendix-vocab-adverbs-intensity} \\
    % & & \\
    \midrule
    \ruby{初}{はじ}めて & for the first time & \\
    また & again/once again/another time/some other time; also; on the other hand & (\ruby{又}{また}) \\
    もう\ruby{一回}{いっ|かい} & one more time/once again & \\
    % & & \\
    \midrule
    <v te>\ruby{初}{はじ}めて<\dots> & only after <v te> is it/do you <\dots> & \\
    % & & \\
    \midrule
    \midrule
    \ruby{全員}{ぜん|いん} & all members/everyone & also a noun \\
    % & & \\
    \bottomrule
}


\subsection{Onomatopoeia}
% Help: \SetCell[r=2,c=2]{c,m} <content>, \cmidrule[l]{3-4}
% Help: colspec: X[ratio, horizontal alignment] columns grow to fit width=\linewidth
%                  negative ratios: shrink to fit content and may not grow to full ratio
% Help: colspec: l/c/r columns do not grow
\longtabse[0.75]  % scale factor
{Adverbs: onomatopoeia.}  % caption
{tbl:appendix-vocab-adverbs-onomatopoeia}  % label
{}  % outer specification options
{
    colspec={X[-3,l]X[3,l]X[-3,l]},
    rowhead=1,
    % width=\linewidth,  % useful only with X columns
}  % inner specification options
{
    \toprule
    \textbf{Modifier} & \textbf{Meaning} & \textbf{Notes} \\
    \midrule
    ピンポン & ding-dong (doorbell/intercom) & \onomatopoeic \\
    ピンポン & ding ding ding!/correct!/right answer! & \onomatopoeic, slang \\
    % & & \\
    \bottomrule
}

\end{document}
