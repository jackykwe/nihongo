\documentclass[../nihongo-gakushuu-kyouzai-vocabulary.tex]{subfiles}
\begin{document}
\appendix
\setcounter{section}{0}

\section{Interjections and expressions}


\subsection{Greetings}
% Help: \SetCell[r=2,c=2]{c,m} <content>, \cmidrule[l]{3-4}
% Help: colspec: X[ratio, horizontal alignment] columns grow to fit width=\linewidth
%                  negative ratios: shrink to fit content and may not grow to full ratio
% Help: colspec: l/c/r columns do not grow
\longtabse[0.5]  % scale factor
{Interections and expressions: greetings. \hl{TO ORGANISE}}  % caption
{tbl:appendix-vocab-interjections-and-expressions-greetings}  % label
{}  % outer specification options
{
    colspec={X[-5,l]X[10,l]X[-5,l]},
    rowhead=1,
    % width=\linewidth,  % useful only with X columns
}  % inner specification options
{
    \toprule
    \textbf{Interjection} & \textbf{Meaning} & \textbf{Notes} \\
    \midrule
    おはよ & good morning & colloquial \\
    おはよう[ございます] & good morning & (お\ruby{早}{はよ}う); [polite] \\
    こんにちは & hello/good afternoon/good day & (\ruby{今日}{こん|にち}は) \\
    こんばんは & good evening & (\ruby{今晩}{こん|ばん}は) \\
    お\ruby{休}{やす}み[なさい] & good night & \\
    % & & \\
    \midrule
    お\ruby{疲}{つか}れ & thanks for coming/glad you could make it; thanks for helping & \\
    % & & \\
    \midrule
    はじめまして & nice to meet you/glad to make your acquaintance/how do you do & (\ruby{始}{はじ}めまして) \\
    お\ruby{変}{か}わりありませんか & (after some time apart) how have you been?/nothing untoward has happened, has it? & polite \\
    {よろしくお\ruby{願}{ねが}いします\\よろしくお\ruby{願}{ねが}いいたします} & I look forward to working with you/please remember me/please treat me favourably/please help me & (\ruby{宜}{よろ}しくお\ruby{願}{ねが}い\{します/\ruby{致}{いた}します\}); polite; also in Table~\ref{tbl:appendix-vocab-interjections-and-expressions-exclamations} \\
    % & & \\
    \midrule
    \midrule
    ただいま & I'm home/I'm back & (\ruby{只今}{ただ|いま}) \\
    お\ruby{帰}{かえ}り[なさい] & welcome home & \\
    % & & \\
    \midrule
    \midrule
    ちょっと & excuse me/hey & also an adverb \\
    もしもし & hello (on phone)/excuse me (calling out to someone) & (\ruby{申}{もう}し\ruby{申}{もう}し) \\
    お\ruby{久}{ひさ}しぶり & long time no see & polite \\
    こら & hey! (calling out to someone); hey! (to scold) & \\
    こいつ & hey, you!/you bastard!/damn you! & derogatory \\
    % & & \\
    \midrule
    お\ruby{邪魔}{じゃ|ま}します & please excuse my intrusion/I'm coming in (``I'm intruding'') & \\
    \ruby{失礼}{しつ|れい}します & excuse me/I'm sorry/I'm coming in (``I'm being rude') & polite \\
    % & & \\
    \midrule
    これまで & that's enough for today & also a noun \\
    お\ruby{疲}{つか}れ\ruby{様}{さま} & thank you for your hard work/good work/see you/goodbye/goodnight & \\
    また\ruby[g]{明日}{あした} & see you tomorrow & \\
    またね & bye/see you later & \\
    \ruby{行}{い}ってくる & I'm off/see you later & \\
    \ruby{行}{い}ってきます & I'm off/see you later & \\
    いってらっしゃい & have a good day/take care/see you (often in response to \ruby{行}{い}ってきます) & (\ruby{行}{い}ってらっしゃい) \\
    お\ruby{邪魔}{じゃ|ま}しました & please excuse my intrusion/I'm leaving (``I intruded'') & \\
    \ruby{失礼}{しつ|れい}しました & excuse me/I'm sorry/I'm leaving (``I was rude'') & polite \\
    お\ruby{先}{さき}に\ruby{失礼}{しつ|れい}します & pardon me for leaving first (before everyone else still present) & \\
    % & & \\
    \midrule
    \midrule
    ようこそ & welcome & \\
    いらっしゃい & welcome & honorific; also in Table~\ref{tbl:appendix-vocab-interjections-and-expressions-exclamations} \\
    いらっしゃいませ & welcome (in shops and restaurants) & \\
    お\ruby{待}{ま}たせしました & thank you for waiting/sorry to have kept you waiting & polite \\
    いただきます & thank you for the meal (just served); I receive (this meal) & (\ruby{頂}{いただ}きます) \\
    ごちそうさま[でした] & thank you for the meal (consumed) & (ご\ruby{馳走様}{ち|そう|さま}[でした]); [polite] \\
    % & & \\
    \bottomrule
}


\subsection{Exclamations}
\emph{Read the main article for apologies on \href{https://www.clozemaster.com/blog/sorry-in-japanese/}{CM}.}

% Help: \SetCell[r=2,c=2]{c,m} <content>, \cmidrule[l]{3-4}
% Help: colspec: X[ratio, horizontal alignment] columns grow to fit width=\linewidth
%                  negative ratios: shrink to fit content and may not grow to full ratio
% Help: colspec: l/c/r columns do not grow
\longtabse[0.5]  % scale factor
{Interections and expressions: exclamations. \hl{TO ORGANISE}}  % caption
{tbl:appendix-vocab-interjections-and-expressions-exclamations}  % label
{}  % outer specification options
{
    colspec={X[-5,l]X[10,l]X[-5,l]},
    rowhead=1,
    % width=\linewidth,  % useful only with X columns
}  % inner specification options
{
    \toprule
    \textbf{Interjection} & \textbf{Meaning} & \textbf{Notes} \\
    \midrule
    はい & yes/that is correct/I'm here/pardon? & \\
    うん & yes/yeah/mhmm & \\
    イエス & yes & \\
    そう/そうだ/[そうです/そうでございます] & that's right/indeed (reference to something that was said/done) & e.g.\ 「そうだよ」、「そうです」, [polite] \\
    そうそう[\dots] & that's right/indeed/that's it (reference to something that was said/done) & casual \\
    そうそう & oh, yes!/that's it/indeed/I remember & \\
    そういえば & now that you mention it/that reminds me/speaking of which & \\
    % & & \\
    \midrule
    \ruby{大丈夫}{だい|じょう|ぶ} & no thanks/I'm good & (そう\ruby{言}{い}えば) \\
    いいえ/いえ & no & (\ruby{否}{いいえ}) \\
    ううん/うーん & um/well/no & \\
    ノー & no/not needed/not allowed & \\
    もういい & skip it/drop it; I've had enough/that's enough & \\
    ご\ruby{遠慮}{えん|りょ}いたします & no thanks/I will refrain & (ご\ruby{遠慮}{えん|りょ}\ruby{致}{いた}します); \ruby{遠慮}{えん|りょ}の例文から \\
    % & & \\
    \midrule
    \midrule
    お\ruby{願}{ねが}い[します] & please & [humble] \\
    \ruby{頼}{たの}む & please/please do & slang, also a verb \\
    {よろしくお\ruby{願}{ねが}いします\\よろしくお\ruby{願}{ねが}いいたします} & please do/please take care of & (\ruby{宜}{よろ}しくお\ruby{願}{ねが}い\{します/\ruby{致}{いた}します\}); polite; also in Table~\ref{tbl:appendix-vocab-interjections-and-expressions-greetings} \\
    (お/ご<nn>)/<v-te>ください & please do for me & honorific \\
    ください & please give me (imperative of くださる) & (\ruby{下}{くだ}さい); honorific \\
    いらっしゃい & please come/go/stay (polite imperative) & also in Table~\ref{tbl:appendix-vocab-interjections-and-expressions-greetings} \\
    % & & \\
    \midrule
    ちゃった & finished doing/did completely & slang of しまった/しまいました \\
    % & & \\
    \midrule
    ご\ruby{遠慮}{えん|りょ}ください & please refrain from & \\
    % & & \\
    \midrule
    \midrule
    \ruby{気持}{き|も}ちいい & feels good & also an adjective \\
    \midrule
    うめぇ & delicious/skilled/good & colloquial \\
    % & & \\
    \midrule
    \midrule
    その/あの[ー] & um/er/well/say & \\
    どれ & well/now & also a pronoun \\
    \ruby{何}{なに}か & what (are you trying to say/do you mean)? & \\
    え & eh? what? oh? & \\
    あれ[っ/え/ー] & huh? eh? what? look! listen! & \\
    あら & oh!/ah!/oh no & feminine \\
    しまった & darn it!/oops!/oh dear!/oh no! & \\
    \ruby{何}{なん}だと & what did you just say (to me)?/what's that? & \\
    なんでよ & why? why not? what's wrong? & \\
    % & & \\
    \midrule
    かもしれない & perhaps/possibly (sentence ender) & (かも\ruby{知}{し}れない) \\
    % & & \\
    \midrule
    ここだけの\ruby{話}{はなし} & confidential talk/conversation between you and me & \\
    % & & \\
    \midrule
    \midrule
    \ruby{助}{たす}けて & help! & \\
    \ruby{危}{あぶ}ない & watch out!/look out!/be careful! \\
    \ruby{気}{き}をつけて & take care/be careful & \\
    % & & \\
    \midrule
    がんばれ & hang in there/go for it/keep at it/do your best & (\ruby{頑張}{がん|ば}れ) \\
    % & & \\
    \midrule
    \midrule
    \ruby{悪}{わる}い & my bad/sorry & casual; also an adjective \\
    \ruby{悪}{わる}かった & my bad/sorry (for past mistake) & casual; also an adjective \\
    ごめん[ね] & I'm sorry/excuse me/pardon me & (\ruby{御免}{ご|めん}[ね]); casual \\
    ごめんなさい & I'm sorry/excuse me/pardon me & (\ruby{御免}{ご|めん}なさい); semi-formal \\
    \ruby{許}{ゆり}してください & please forgive me & (\ruby{許}{ゆり}して\ruby{下}{くだ}さい); honorific, semi-formal \\
    お\ruby{許}{ゆり}しください & please forgive me & (お\ruby{許}{ゆり}し\ruby{下}{くだ}さい); honorific, formal \\
    \{\ruby{反省}{はん|せい}/\ruby{後悔}{こう|かい}\}しています & I'm sorry (``I am regretful'') & (\{\ruby{反省}{はん|せい},\ruby{後悔}{こう|かい}\}して\ruby{居}{い}ます); semi-formal \\
    \{\ruby{反省}{はん|せい}/\ruby{後悔}{こう|かい}\}しております & I'm sorry (``I am regretful'') & (\{\ruby{反省}{はん|せい},\ruby{後悔}{こう|かい}\}して\ruby{居}{お}ります); humble, formal \\
    \ruby{謝罪}{しゃ|ざい}いたします & I'm sorry (esp.\ in written apology) (``I perform apology'') & (\ruby{謝罪}{しゃ|ざい}\ruby{致}{いた}します); humble, formal \\
    お\ruby{詫}{わ}びいたします & I'm sorry (``I perform apology'') & (お\ruby{詫}{わ}び\ruby{致}{いた}します); humble, formal \\
    お\ruby{詫}{わ}び\ruby{申}{もう}し\ruby{上}{あ}げます & I'm sorry (``I offer my apology'') & humble, formal \\
    \ruby{申}{もう}し\ruby{訳}{わけ}ない & I'm sorry/it's inexcusable (``excuse does not exist'') & semi-polite \\
    \ruby{申}{もう}し\ruby{訳}{わけ}\{ございません/ありません\} & I'm sorry/it's inexcusable (``excuse does not exist'') & polite \\
    % & & \\
    \midrule
    すまん/すいません/[すみません] & excuse me/pardon me/I'm sorry (general/for the inconvenience) & (\ruby{済}{す}みません); [polite] \\
    \ruby{済}{す}まない & excuse me/I'm sorry & also an adjective \\
    \ruby{失礼}{しつ|れい}しました & I'm sorry/excuse me/my apologies (general/for the inconvenience) & polite \\
    ご\ruby{迷惑}{めい|わく}をおかけして\ruby{申}{もう}し\ruby{訳}{わけ}ございません & We apologise for any inconvenience this may cause (common email/announcement-end greeting) (``I have no excuse for causing you this trouble.'') & polite, formal \\
    % & & \\
    \midrule
    \midrule
    どうも & thanks (abbreviation) & \\
    どうもありがとう & thank you very much & (どうも\ruby[g]{有[り]難}{ありがと}う) \\
    ありがとう[ございます] & thanks/thank you & (\ruby[g]{有[り]難}{ありがと}う[ございます]); [polite] \\
    ありがとうございました & thank you (for past action) & (\ruby[g]{有[り]難}{ありがと}うございました); polite \\
    \ruby{済}{す}まない & thank you & also an adjective \\
    % & & \\
    \midrule
    どういたしまして & you're welcome/don't mention it/not at all/my pleasure & (どう\ruby{致}{いた}しまして) \\
    とんでもない & it was no bother at all/not at all/don't mention it & \\
    こちらこそ & it is I who should say so & also in Table~\ref{tbl:appendix-vocab-nouns-pronouns-and-question-words} \\
    \ruby{気}{き}にするな & don't worry about it/nevermind & \\
    \ruby{気}{き}にしないで & don't worry about it/forget about it & \\
    お\ruby{安}{や}い\ruby{御用}{ご|よう} & no problem/easy task & \\
    \ruby{問題}{もん|だい}ない & い & no problem/not an issue/all right & also an adjective \\
    \ruby{構}{かま}わない & no problem/it doesn't matter & \\
    \ruby{構}{かま}いません & no problem/it doesn't matter & \\
    % & & \\
    \midrule
    \midrule
    わかる? & Do you know?/do you think so too? & (\ruby{分}{わ}かる?); also a verb \\
    わかる & I know/I think so too & (\ruby{分}{わ}かる); also a verb \\
    だろ[う] & seems/I think/I guess/I wonder/I hope; right?/don't you agree?/I thought you'd say that! & conjectural form of copula だ \\
    % & & \\
    \midrule
    なるほど & I see/that's right/indeed & (\ruby{成}{な}る\ruby{程}{ほど}) \\
    そうか & is that so? (rhetorical); I see/right/oh/OK & \\
    そっか & oh, right/I see/OK/gotcha & \\
    そういうことか & I got it/I see/I now know/so that's the reason & (そういう\ruby{事}{こと}か) \\
    \ruby{了解}{りょう|かい} & OK/roger & also a noun, verb \\
    % & & \\
    \midrule
    そういうことなら & in that case/that being the case & \\
    % & & \\
    \midrule
    \midrule
    \ruby{嘘}{うそ} & no way! really!? unbelieveable! & colloquial \\
    それはそれは & my goodness (surprise/wonder) & \\
    まさか & by no means/never!/no way! & (\ruby{真逆}{ま|さか}); also a noun\\
    まさかの\ruby{時}{とき} & in case of emergency/for a rainy day/in time of need & \\
    とんでもない & absolutely not!/far from it!/impossible!/what a thing to say!/no way! & \\
    % & & \\
    \midrule
    だから & like I said/I told you already & also a conjunction \\
    % & & \\
    \midrule
    お[お/ー]い/オ[オ/ー]イ & oi! hey! come on! & \\
    ほら & look! see! hey! & \\
    もう & jeez/come on & \\
    まったくもう & good grief & (\ruby{全}{まった}くもう) \\
    やだ/ヤダ & no way/not a chance & feminine/childish \\
    そんな & no way!/never! & \\
    ふざけんな & stop messing around! get real! screw you! & slang \\
    いい\ruby{加減}{か|げん}にしろ & that's enough!/cut it out!/get a life! & \\
    いい\ruby{加減}{か|げん}にしなさい & shape up!/act properly! & \\
    % & & \\
    \midrule
    クソ & damn/damn it/shit/crap & (\ruby{糞}{くそ}) \\
    この\ruby{野郎}{や|ろう} & you bastard/son of a bitch! & derogatory \\
    ちくしょう & damn it/son of a bitch/god damn it & (\ruby{畜生}{ちく|しょう}) \\
    % & & \\
    \midrule
    \midrule
    うぜえ & annoying/noisy & colloquial \\
    うるさい/うるせ\{え/ー\}/うっせ\{え/ー\} & shut up!/be quiet! & (\ruby{煩}{うるさ}い)\\
    \ruby{静}{しず}かに & be quiet! & also an adverb \\
    % & & \\
    \midrule
    \midrule
    \ruby{以上}{い|じょう} & that's all & \\
    % & & \\
    \midrule
    \midrule
    つまらない\ruby{物}{もの}ですが & it's not much, but\dots (when giving a gift) & humble; \href{https://www.youtube.com/shorts/HV4GmjgQZHw}{[KK]} \\
    % & & \\
    \bottomrule
}


\subsection{Sentence builders}
% Help: \SetCell[r=2,c=2]{c,m} <content>, \cmidrule[l]{3-4}
% Help: colspec: X[ratio, horizontal alignment] columns grow to fit width=\linewidth
%                  negative ratios: shrink to fit content and may not grow to full ratio
% Help: colspec: l/c/r columns do not grow
\longtabse[0.5]  % scale factor
{Interections and expressions: sentence builders.}  % caption
{tbl:appendix-vocab-interjections-and-expressions-sentence-builders}  % label
{}  % outer specification options
{
    colspec={X[-5,l]X[10,l]X[-5,l]},
    rowhead=1,
    % width=\linewidth,  % useful only with X columns
}  % inner specification options
{
    \toprule
    \textbf{Interjection} & \textbf{Meaning} & \textbf{Notes} \\
    \midrule
    <basis>からすると & judging from/on the basis of/from the point of view of & \\
    <basis>からしたら & judging from/on the basis of/from the point of view of & \\
    <basis>からすれば & judging from/on the basis of/from the point of view of & \\
    % & & \\
    \midrule
    \midrule
    <about>ついて & about/concerning/regarding/as for <about> & (\ruby{就}{つ}いて) \\
    <about>について & concerning/regarding <about> & (に\ruby{就}{つ}いて) \\
    % & & \\
    \midrule
    \midrule
    % & & \\
    \bottomrule
}

\end{document}
